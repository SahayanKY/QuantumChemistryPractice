%
%
%-----------------------------数学関係のマクロ----------------------------------
%
%
%微小量
%下にドットを打つマクロと重複
\renewcommand{\d}{{\ensuremath{\rm d}}}


%偏微分
\newcommand{\pard}[3][1]{{%
	\ifnum #1<1%
		\text{エラー}%
	\else{%
		\ifnum #1=1%
			\ensuremath{%
				\frac{\partial #3}{\partial #2}%
				%\partial_{#2} #3%
			}%
		\else{%
			\ensuremath{%
				\frac{\partial^{#1} #3}{\partial #2^{#1}}%
				%\partial_{#2}^{#1} #3
			}%
		}\fi%
	}\fi%
}}

\newcommand{\operator}[3]{%
	%#1:演算子をかける対象を括弧で囲むかどうか
	%	0->囲まない
	%	1->()で囲む
	%	2->{}で囲む
	%#2:演算子
	%#3:演算子をかける対象
	{\ensuremath{#2%
		\ifnum #1=0%
			#3%
		\else{%
			\ifnum #1=1%
				\left(#3\right)%
			\else{%
				\ifnum #1=2%
					\left\{#3\right\}%
				\else{%
					\text{エラー}%
				}\fi%
			}\fi%
		}\fi%
	}}%
}

\newcommand{\rightoperator}[3]{%
	%#1:演算子をかける対象を括弧で囲むかどうか
	%	0->囲まない
	%	1->()で囲む
	%	2->{}で囲む
	%#2:演算子
	%#3:演算子をかける対象
	{\ensuremath{%
		\ifnum #1=0%
			#3%
		\else{%
			\ifnum #1=1%
				\left(#3\right)%
			\else{%
				\ifnum #1=2%
					\left\{#3\right\}%
				\else{%
					\text{エラー}%
				}\fi%
			}\fi%
		}\fi%
		#2%
	}}%
}


%転置行列
%transposed
\newcommand{\tp}[2][0]{\rightoperator{#1}{{}^t}{#2}}

%共役行列、随伴行列
%adjoint
\newcommand{\adj}[2][0]{\rightoperator{#1}{{}^{\dagger}}{#2}}

%複素共役行列
%conjugate
\newcommand{\conju}[2][0]{\rightoperator{#1}{{}^{\ast}}{#2}}

%トレース
%trace
\newcommand{\tr}[2][0]{\operator{#1}{{\rm tr}}{#2}}

%単位行列
%identity matrix
\newcommand{\identity}{{\text{{\bf \textit{1}}}}}

%回転
\newcommand{\rot}[2][0]{\operator{#1}{\nabla\times}{#2}}


%発散
%割り算の記号÷と重複
\renewcommand{\div}[2][0]{\operator{#1}{\nabla\cdot}{#2}}


%勾配
\newcommand{\grad}[2][0]{\operator{#1}{\nabla}{#2}}


%ラプラシアン
\newcommand{\laplacian}[2][0]{\operator{#1}{\nabla^2}{#2}}


%基底ベクトル
\newcommand{\e}[2][]{{\ensuremath{\bm{e}#1_{#2}}}}


%法線ベクトル
\newcommand{\n}{{\ensuremath{\bm{n}}}}


%接線ベクトル
%ハットみたいなもの(ただしカーブ)と重複
%\hat,\widehatとは被らない
\renewcommand{\t}{{\ensuremath{\bm{t}}}}

%面ベクトル、Poyntingベクトル
%セクションの記号§と重複
\renewcommand{\S}{{\ensuremath{\bm{S}}}}


%位置ベクトル
%\angstromが内部で\rを呼び出しているので若干汚い
%
%位置ベクトルのrを定義
\def\r@PositionVector{{\ensuremath{\bm{r}}}}
%元の\rの定義を\r@Originにコピーしておく
\let\r@Origin = \r
%siunitxの\angstromが展開されて出てくるコマンドをコピー
%(\angstromそのものはコピーすることはできなかった)
\let\angstrom@Origin = \SIUnitSymbolAngstrom
%位置ベクトル\rを定義
\def\r{\r@PositionVector}
%現在の定義
\let\r@current = \r
%\SIUnitSymbolAngstromの展開の前後に\rの定義のすり替えを行う
\def\SIUnitSymbolAngstrom{%
	%現在の定義を取得しておく
	\let\r@current = \r%
	%元の定義に直しておく
	\let\r = \r@Origin%
	%オングストロームを出力
	\angstrom@Origin%
	%自分が定義していたものに戻しておく
	\let\r = \r@current%
}




%位置ベクトル2
\newcommand{\x}{{\ensuremath{\bm{x}}}}


%平均
\newcommand{\average}[1]{{\ensuremath{\left<#1\right>}}}

%符号関数
\newcommand{\sgn}{{\rm sgn}}

%無次元量
\newcommand{\dimless}[1]{{\ensuremath{{#1}^{\ast}}}}

%真の値
\newcommand{\truevalue}[1]{{\ensuremath{\tilde{#1}}}}

%二項係数
\newcommand{\bicoeff}[2]{{\ensuremath{%
% 	\left(
% 	\begin{array}{c}
% 		#1 \\ #2
% 	\end{array}
% 	\right)
	{}_{#1}C_{#2}
}}}

%物理学者の記法
\newcommand{\phystwo}[2]{{\ensuremath{\braket{#1|#2}}}}

%反対称化された2電子積分
\newcommand{\antitwo}[2]{{\ensuremath{\braket{#1||#2}}}}

%化学者の記法(2電子積分)
\newcommand{\chemtwo}[2]{{\ensuremath{[#1|#2]}}}

%化学者の記法(1電子積分)
\newcommand{\chemone}[3]{{\ensuremath{[#1|#2|#3]}}}

%
%
%------------------------------数字の体裁関係のマクロ--------------------------------
%
%
%
%ローマ数字の出力に使う
\newcommand{\ONE}{{\rm I}}
\newcommand{\TWO}{{\rm I\hspace{-.1em}I}}
\newcommand{\THREE}{{\rm I\hspace{-.1em}I\hspace{-.1em}I}}
\newcommand{\FOUR}{{\rm I\hspace{-.1em}V}}
\newcommand{\FIVE}{{\rm V}}



%〇数字などに使う
\newcommand{\circled}[1]{{\textcircled{\scriptsize #1}}}



