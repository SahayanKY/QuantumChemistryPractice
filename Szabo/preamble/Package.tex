%
%
%-------------------------------パッケージ--------------------------------------
%
%
%--------------------------------数学関係---------------------------------------
	\usepackage{amssymb}
	\usepackage{amsmath}
	\usepackage{bm}
	\usepackage{braket}



%--------------------------------graphicx---------------------------------------
	\usepackage[dvipdfmx]{graphicx}


%--------------------------------multirow---------------------------------------
	%表組みで縦セルを結合
	\usepackage{multirow}


%---------------------------------color-----------------------------------------
	\usepackage[dvipdfmx]{color}


%--------------------------------caption----------------------------------------
	%キャプションの体裁を調整
	\usepackage{caption}

	%表4.1「.」と最後にピリオドをつけて区切る
	\captionsetup{labelsep=period}

	%キャプション文を太字にし、一回り文字を小さくする
	\captionsetup{font = {bf,small}}

	%キャプション文を改行する際は、図X.Y.の分だけ文字を下げる
	\captionsetup{format = hang}

	%キャプション文全体の横幅を調整。大きくするほど横幅は小さくなる
	\captionsetup{margin = 60pt}


%----------------------------------tikz-----------------------------------------
	%tikz
	\usepackage{tikz}
	%矢印関連のライブラリ、ドキュメント16章
	\usetikzlibrary{arrows.meta}
	%座標の計算
	\usetikzlibrary{calc}
	%\usetikzlibrary{calc,patterns,intersections}


%---------------------------------mhchem----------------------------------------
	%化学式記述用のパッケージ
	\usepackage[version=3]{mhchem}


%---------------------------------siunitx---------------------------------------
	%単位、数値フォーマット
	\usepackage{siunitx}

	%物理量のフォーマット
	%数値と単位の間を半角スペースに設定
	\sisetup{number-unit-product=\ }

	%誤差表示のフォーマット
	%有効にすると誤差を±で表示
	%\sisetup{separate-uncertainty}

	%表組において、S指定のカラムで文字列を中央に寄せる
	\sisetup{table-number-alignment = center}

	%プラスを明示的に書いた場合はプラスも出力される
	%書かなかった場合は出力されない
	\sisetup{retain-explicit-plus}

%-----------------------------------url-----------------------------------------
	%参考文献でurlを正しく表示するのに使う
	\usepackage{url}

%-----------------------------------listing-----------------------------------------
	%ソースコードを埋め込むためのもの
	%lstlisting環境を用いる
	\usepackage{listings,jlisting}

	\lstset{
	    basicstyle={\ttfamily\small}, %書体の指定
	    frame=tRBl, %フレームの指定
	    framesep=10pt, %フレームと中身(コード)の間隔
	    breaklines=true, %行が長くなった場合の改行
	    lineskip=-0.5ex, %行間の調整
	    tabsize=2, %Tabを何文字幅にするかの指定
	 	captionpos = t, %キャプションの場所("tb"ならば上下両方に記載)
	 	keywordstyle = {\bfseries \color[cmyk]{0,1,0,0}}, %キーワード(int, ifなど)の書体
	 	stringstyle = {\ttfamily \color[rgb]{0,0,1}}, %""で囲まれたなどの"文字"の書体
	 	commentstyle = {\itshape \color[cmyk]{1,0.5,0,0}}, %コメントの書体
	}

%-----------------------------------longtable-----------------------------------------
	%複数ページにまたがる表組に利用する
	\usepackage{longtable}


%-----------------------------------fancy-----------------------------------------
	%ページスタイルの設定
	\usepackage{fancyhdr}


%----------------------------------ascmac----------------------------------------
	%複数行の囲む枠
	%要約で使う
	\usepackage{ascmac}

