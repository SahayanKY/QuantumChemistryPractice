%ファイルID
%2020/04/22 06:28
%->2004220628(10進数)->X59EJ8(ファイル作成者)
%->X59EJ8SahayanKY
\subsection{問}
行列$A$が
\begin{align}
	A
&=
	\left[
	\begin{array}{cc}
		a & b \\
		b & a
	\end{array}
	\right]
\end{align}
であるとき、
\begin{align}
	f(A)
&=
	\left[
	\begin{array}{cc}
		\frac{1}{2}(f(a+b)+f(a-b)) &
		%
		\frac{1}{2}(f(a+b)-f(a-b)) \\
	%
		\frac{1}{2}(f(a+b)-f(a-b)) &
		%
		\frac{1}{2}(f(a+b)+f(a-b))
	\end{array}
	\right]
\end{align}
であることを示せ。

\subsection{解}
まず$A$を対角化する。固有ベクトルと対応する固有値は
\begin{align}
	\bm{c}^{1}
&=
	\frac{1}{\sqrt{2}}
		\left[
		\begin{array}{c}
			1 \\ 1
		\end{array}
		\right] &
%
%
	\omega_1
&=
	a+b \\
%
%
	\bm{c}^2
&=
	\frac{1}{\sqrt{2}}
		\left[
		\begin{array}{c}
			1 \\ -1
		\end{array}
		\right] &
%
%
	\omega_2
&=
	a-b
\end{align}
である。従って、ユニタリー行列$U,\adj{U}$と対角行列$\bm{a}$は
\begin{align}
	U
&=
	\frac{1}{\sqrt{2}}
	\left[
	\begin{array}{cc}
		1 & 1 \\
		1 & -1
	\end{array}
	\right]
=
	\adj{U} &
%
%
	\bm{a}
&=
	\left[
	\begin{array}{cc}
		a+b & 0 \\
		0 & a-b
	\end{array}
	\right]
\end{align}
である。
従って、
\begin{align}
	f(A)
&=
	U f(\bm{a}) \adj{U} \\
%
%
&=
	\frac{1}{2}
		\left[
		\begin{array}{cc}
			1 & 1 \\
			1 & -1
		\end{array}
		\right]
		\left[
		\begin{array}{cc}
			f(a+b) & 0 \\
			0 & f(a-b)
		\end{array}
		\right]
		\left[
		\begin{array}{cc}
			1 & 1 \\
			1 & -1
		\end{array}
		\right] \\
%
%
&=
	\frac{1}{2}
		\left[
		\begin{array}{cc}
			f(a+b) & f(a-b) \\
			f(a+b) & -f(a-b)
		\end{array}
		\right]
		\left[
		\begin{array}{cc}
			1 & 1 \\
			1 & -1
		\end{array}
		\right] \\
%
%
&=
	\frac{1}{2}
		\left[
		\begin{array}{cc}
			f(a+b)+f(a-b) &
			%
			f(a+b)-f(a-b) \\
		%
			f(a+b)-f(a-b) &
			%
			f(a+b)+f(a-b)
		\end{array}
		\right]
\end{align}
となる。
