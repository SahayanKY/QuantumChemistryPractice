%ファイルID
%2020/04/23 23:04
%->2004232304(10進数)->X59NJK(ファイル作成者)
%->X59NJKSahayanKY
\subsection{問}
ポテンシャル$-\delta(x)$のもとに1次元運動する
1つの電子の\Schrodinger 方程式は
\begin{align}
	\left(
		-
		\frac{1}{2}\frac{\d^2}{\d x^2}
		-
		\delta(x)
	\right)\ket{\Phi}
&=
	\mathscr{E}\ket{\Phi}
\end{align}
である。

試行関数
\begin{align}
	\ket{\tilde{\Phi}}
&=
	N \exp(-\alpha x^2)
\end{align}
で変分法による計算を行い、
得られるエネルギーが$-\pi^{-1}$であることを示せ。
また、それが正確な基底状態のエネルギーの
$-0.5$より大きいことを示せ。

なお、積分公式として
\begin{align}
	\int_{-\infty}^{\infty} \d x\
		x^{2m} \exp(-\alpha x^2)
&=
	\frac{
		(2m)! \pi^{\frac{1}{2}}
	}{
		2^{2m} m! \alpha^{m+\frac{1}{2}}
	}
\end{align}
を用いてもよい。

\subsection{解}
エネルギーの期待値は
\begin{align}
	\braket{\tilde{\Phi}|\H|\tilde{\Phi}}
&=
	\int_{-\infty}^{\infty} \d x\
		\conju{\tilde{\Phi}}(x) \H \tilde{\Phi}(x) \\
%
%
&=
	\int_{-\infty}^{\infty} \d x\
		\left(
			\conju{N} \exp(-\alpha x^2)
		\right)
			\H
			\left(
				N \exp(-\alpha x^2)
			\right) \\
%
%
&=
	|N|^2
		\int_{-\infty}^{\infty} \d x\
			\exp(-\alpha x^2)
				\left(
					-
					\frac{1}{2}\frac{\d^2}{\d x^2}
					-
					\delta(x)
				\right)
				\exp(-\alpha x^2) \\
%
%
&=
	|N|^2
		\left\{
		\begin{array}{>{\displaystyle}l}
			-
			\frac{1}{2}
				\int_{-\infty}^{\infty} \d x\
					\exp(-\alpha x^2)
						\frac{\d}{\d x}\left(
							\exp(-\alpha x^2)(-2\alpha x)
						\right) \\\quad
			-
			\int_{-\infty}^{\infty} \d x\
				\delta(x) \exp(-2\alpha x^2)
		\end{array}
		\right\} \\
%
%
&=
	|N|^2
		\left\{
		\begin{array}{>{\displaystyle}l}
			-
			\frac{1}{2}
				\int_{-\infty}^{\infty} \d x\
					\exp(-2\alpha x^2) (-2\alpha x)^2 \\\quad
			-
			\frac{1}{2}
				\int_{-\infty}^{\infty} \d x\
					\exp(-2\alpha x^2) (-2\alpha) \\\quad
			-
			1
		\end{array}
		\right\} \\
%
%
&=
	|N|^2
		\left\{
			-
			2\alpha^2 \cdot
				\frac{
					2!\ \pi^{\frac{1}{2}}
				}{
					2^2\ 1!\ (2\alpha)^{\frac{3}{2}}
				}
			+
			\alpha \cdot
				\frac{
					0!\ \pi^{\frac{1}{2}}
				}{
					2^0\ 0!\ (2\alpha)^{\frac{1}{2}}
				}
			-
			1
		\right\} \\
%
%
&=
	|N|^2
		\left\{
			-
			2\alpha^2 \cdot
				\frac{
					2 \pi^{\frac{1}{2}}
				}{
					4 \cdot 2^{\frac{3}{2}} \cdot \alpha^{\frac{3}{2}}
				}
			+
			\alpha \cdot
				\frac{
					\pi^{\frac{1}{2}}
				}{
					2^{\frac{1}{2}} \cdot \alpha^{\frac{1}{2}}
				}
			-
			1
		\right\} \\
%
%
&=
	|N|^2
		\left\{
			-
			2^{-\frac{3}{2}} \alpha^{\frac{1}{2}} \pi^{\frac{1}{2}}
			+
			2^{-\frac{1}{2}} \alpha^{\frac{1}{2}} \pi^{\frac{1}{2}}
			-
			1
		\right\} \\
%
%
&=
	|N|^2
		\left\{
			2^{-\frac{3}{2}} \alpha^{\frac{1}{2}} \pi^{\frac{1}{2}}
				\left(
					-1
					+2
				\right)
			-
			1
		\right\} \\
%
%
&=
	|N|^2
		\left(
			2^{-\frac{3}{2}} \alpha^{\frac{1}{2}} \pi^{\frac{1}{2}}
			-
			1
		\right)
\end{align}
である。また、規格化条件により、
\begin{align}
	\braket{\tilde{\Phi}|\tilde{\Phi}}
&=
	1 \\
%
%
	|N|^2
		\int_{-\infty}^{\infty} \d x\
			\exp(-2\alpha x^2)
&=
	1 \\
%
%
	|N|^2 \cdot
		\frac{
			\pi^{\frac{1}{2}}
		}{
			2^{\frac{1}{2}} \alpha^{\frac{1}{2}}
		}
&=
	1 \\
%
%
	|N|^2
&=
	\frac{
		2^{\frac{1}{2}} \alpha^{\frac{1}{2}}
	}{
		\pi^{\frac{1}{2}}
	}
\end{align}
である。従って、期待値の極小値は、
$\alpha$で微分してゼロになるときにとるため
\begin{align}
	\braket{\tilde{\Phi}|\H|\tilde{\Phi}}
&=
	\frac{
		2^{\frac{1}{2}} \alpha^{\frac{1}{2}}
	}{
		\pi^{\frac{1}{2}}
	}
		\left(
			2^{-\frac{3}{2}} \alpha^{\frac{1}{2}} \pi^{\frac{1}{2}}
			-
			1
		\right) \\
%
%
&=
	2^{-1} \alpha
	-
	2^{\frac{1}{2}} \pi^{-\frac{1}{2}} \alpha^{\frac{1}{2}} \\
%
%
	\frac{\d}{\d \alpha}\left(
		\braket{\tilde{\Phi}|\H|\tilde{\Phi}}
	\right)
&=
	0 \\
%
%
	2^{-1}
	-
	2^{-\frac{1}{2}} \pi^{-\frac{1}{2}} \alpha^{-\frac{1}{2}}
&=
	0 \\
%
%
	\alpha^{-\frac{1}{2}}
&=
	2^{-\frac{1}{2}} \pi^{\frac{1}{2}} \\
%
%
	\alpha
&=
	2 \pi^{-1}
\end{align}
\begin{align}
	\min\left(
		\braket{\tilde{\Phi}|\H|\tilde{\Phi}}
	\right)
&=
	2^{-1} \cdot 2\pi^{-1}
	-
	2^{\frac{1}{2}} \pi^{-\frac{1}{2}} \cdot 2^{\frac{1}{2}} \pi^{-\frac{1}{2}} \\
%
%
&=
	\pi^{-1}
	-
	2 \pi^{-1} \\
%
%
&=
	-\pi^{-1}
\end{align}
である。更にこの値は
\begin{align}
	2
&<
	\pi \\
%
%
	\frac{1}{\pi}
&<
	\frac{1}{2}
=
	0.5 \\
%
%
	-0.5
&<
	-\pi^{-1}
\end{align}
であるから、確かに基底状態の厳密なエネルギーよりも大きくなることが分かる。


