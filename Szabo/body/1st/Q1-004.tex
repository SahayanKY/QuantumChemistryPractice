%ファイルID
%2020/04/18 14:19
%->2004181419(10進数)->X58KA3(ファイル作成者)
%->X58KA3SahayanKY
次の関係を示せ。

\subsection{(a)問}
\begin{align}
	\tr[1]{AB}
&=
	\tr[1]{BA}
\end{align}

\subsection{(a)解}
\begin{align}
	\tr{C}
&=
	C_{ii}
\end{align}
であり、
\begin{align}
	C
&=
	AB \\
%
%
	C_{ij}
&=
	A_{ik} B_{kj}
\end{align}
である。従って、
\begin{align}
	\tr[1]{AB}
&=
	A_{ik} B_{ki} \\
%
%
&=
	B_{ki} A_{ik} \\
%
%
&=
	B_{ik} A_{ki} \\
%
%
&=
	\tr[1]{BA}
\end{align}
である。


\subsection{(b)問}
\begin{align}
	(AB)^{-1}
&=
	B^{-1} A^{-1}
\end{align}

\subsection{(b)解}
$\identity$を単位行列とすると、
\begin{align}
	(AB) (AB)^{-1}
&=
	\identity \\
%
%
	A^{-1} AB (AB)^{-1}
&=
	A^{-1} \identity \\
%
%
	\identity B (AB)^{-1}
&=
	A^{-1} \\
%
%
	B (AB)^{-1}
&=
	A^{-1} \\
%
%
	B^{-1} B (AB)^{-1}
&=
	B^{-1} A^{-1} \\
%
%
	\identity (AB)^{-1}
&=
	B^{-1} A^{-1} \\
%
%
	(AB)^{-1}
&=
	B^{-1} A^{-1}
\end{align}
である。

\subsection{(c)問}
$U$はユニタリー行列、
即ち$U^{-1}=\adj{U}$とする。
$B=\adj{U}AU$のとき、
$A=UB\adj{U}$であることを示せ。

\subsection{(c)解}
\begin{align}
	B
&=
	\adj{U}AU \\
%
%
&=
	U^{-1}A(\adj{U})^{-1} \\
%
%
	UB\adj{U}
&=
	U U^{-1}A(\adj{U})^{-1} \adj{U} \\
%
%
	UB\adj{U}
&=
	\identity A \identity \\
%
%
	\therefore
	A
&=
	UB\adj{U}
\end{align}
である。

\subsection{(d)問}
エルミート行列$A$と$B$の積、
$C=AB$もまたエルミート行列ならば、
$A$と$B$は可換であることを示せ。

\subsection{(d)解}
$A$と$B$がエルミート行列であることから、
\begin{align}
	A
&=
	\adj{A} &
%
%
	B
&=
	\adj{B}
\end{align}
である。更に、$C$がエルミート行列であることから
\begin{align}
	C
&=
	\adj{C} \\
%
%
	AB
&=
	\adj[1]{AB} \\
%
%
&=
	\adj{B} \adj{A} \\
%
%
&=
	BA
\end{align}
である。よって、$A$と$B$は可換である。


\subsection{(e)問}
$A$がエルミート行列であり、逆行列$A^{-1}$が存在する場合、
$A^{-1}$もまたエルミート行列であることを示せ。

\subsection{(e)解}
\begin{align}
	A A^{-1}
&=
	\identity \\
%
%
	\adj[1]{AA^{-1}}
&=
	\adj[1]{\identity}
=
	\tp[1]{\conju[1]{\identity}}
=
	\identity \\
%
%
	\adj[1]{A^{-1}}
		\adj{A}
&=
	\identity \\
%
%
	\adj[1]{A^{-1}}
		A
&=
	\identity
	%
	\qquad
	(\because \adj{A}=A) \\
%
%
	\adj[1]{A^{-1}}
&=
	A^{-1}
\end{align}
従って、$A^{-1}$もまたエルミート行列である。


\subsection{(f)問}
\begin{align}
	A
&=
	\left[
	\begin{array}{cc}
		A_{11} & A_{12} \\
		A_{21} & A_{22}
	\end{array}
	\right]
\end{align}
のとき、
\begin{align}
	A^{-1}
&=
	\frac{
		1
	}{
		A_{11}A_{22}-A_{12}A_{21}
	}
		\left[
		\begin{array}{cc}
			 A_{22} & -A_{12} \\
			-A_{21} &  A_{11}
		\end{array}
		\right]
\end{align}
であることを示せ。


\subsection{(f)解}
行列$B$を
\begin{align}
	B
&=
	\frac{
		1
	}{
		A_{11}A_{22}-A_{12}A_{21}
	}
		\left[
		\begin{array}{cc}
			 A_{22} & -A_{12} \\
			-A_{21} &  A_{11}
		\end{array}
		\right]
\end{align}
と置く。このとき、
\begin{align}
	AB
&=
	\left[
	\begin{array}{cc}
		A_{11} & A_{12} \\
		A_{21} & A_{22}
	\end{array}
	\right]
		\frac{
			1
		}{
			A_{11}A_{22}-A_{12}A_{21}
		}
			\left[
			\begin{array}{cc}
				 A_{22} & -A_{12} \\
				-A_{21} &  A_{11}
			\end{array}
			\right] \\
%
%
&=
	\frac{
		1
	}{
		A_{11}A_{22}-A_{12}A_{21}
	}
		\left[
		\begin{array}{cc}
			A_{11}A_{22}-A_{12}A_{21} & -A_{11}A_{12}+A_{12}A_{11} \\
			A_{21}A_{22}-A_{22}A_{21} & -A_{21}A_{12}+A_{22}A_{11}
		\end{array}
		\right] \\
%
%
&=
	\left[
	\begin{array}{cc}
		1 & 0 \\
		0 & 1
	\end{array}
	\right]
=
	\identity
\end{align}
よって、$B=A^{-1}$である。
