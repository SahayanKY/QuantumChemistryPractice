%ファイルID
%2020/04/21 22:07
%->2004212207(10進数)->X5981B(ファイル作成者)
%->X5981BSahayanKY
次の関係を与える。
\begin{align}
	\adj{U} AU
=
	\bm{a}
=
	\left[
	\begin{array}{cc}
		\begin{array}{cc}
			a_1 &  \\
			 & a_2
		\end{array} &
		%
		\bm{0} \\
	%
		\bm{0} &
		%
		\begin{array}{cc}
			\ddots &  \\
			 & a_N
		\end{array}
	\end{array}
	\right]
\end{align}
もしくは
\begin{align}
	A\bm{c}^{\alpha}
&=
	a_{\alpha}\bm{c}^{\alpha}
	%
	\qquad
	(\alpha=1,2,\cdots,N)
\end{align}

\subsection{(a)問}
次の等式を証明せよ。
\begin{align}
	|A^n|
&=
	a_1^n a_2^n \cdots a_N^n
\end{align}

\subsection{(a)解}
\begin{align}
	\left(\adj{U}AU\right)^n
&=
	\bm{a}^n \\
%
%
	\adj{U} A^n U
&=
	\bm{a}^n
=
	\left[
	\begin{array}{cc}
		\begin{array}{cc}
			a_1^n & \\
			& a_2^n
		\end{array} &
		%
		\bm{0} \\
	%
		\bm{0} &
		%
		\begin{array}{cc}
			\ddots &  \\
			 & a_N^n
		\end{array}
	\end{array}
	\right]
\end{align}
\begin{align}
	|\adj{U} A^n U|
&=
	|\bm{a}^n| \\
%
%
	|\adj{U}| |A^n| |U|
&=
	a_1^n a_2^n \cdots a_N^n \\
%
%
	|\identity| |A^n|
&= \\
%
%
	|A^n|
&=
	a_1^n a_2^n \cdots a_N^n
\end{align}


\subsection{(b)問}
次の等式を証明せよ。
\begin{align}
	\tr[1]{A^n}
&=
	\sum_{\alpha=1}^N
		a_{\alpha}^n
\end{align}


\subsection{(b)解}
\begin{align}
	\adj{U} A^n U
&=
	\bm{a}^n \\
%
%
	\tr[1]{\adj{U} A^n U}
&=
	\tr[1]{\bm{a}^n} \\
%
%
	\tr[1]{A^n U \adj{U}}
&=
	\tr[1]{{\rm diag}(a_1^n,a_2^n,\cdots,a_N^n)}
	%
	\qquad
	(\because \tr[1]{AB}=\tr[1]{BA}) \\
%
%
	\tr[1]{A^n}
&=
	\sum_{\alpha=1}^N
		a_{\alpha}^n
\end{align}


\subsection{(c)問}
$G(\omega)=(\omega\identity-A)^{-1}$のとき、
\begin{align}
	(G(\omega))_{ij}
=
	\sum_{\alpha=1}^N
		\frac{
			U_{i\alpha} \conju{U_{j\alpha}}
		}{
			\omega-a_{\alpha}
		}
=
	\sum_{\alpha=1}^N
		\frac{
			c_{i}^{\alpha} \conju{c_{j}^{\alpha}}
		}{
			\omega-a_{\alpha}
		}
\end{align}
であることを示せ。加えて、
\begin{align}
	(G(\omega))_{ij}
=
	\Braket{i|\mathscr{G}(\omega)|j}
=
	\sum_{\alpha=1}^N
		\frac{
			\Braket{i|\alpha}
				\Braket{\alpha|j}
		}{
			\omega-a_{\alpha}
		}
\end{align}
であることも示せ。

\subsection{(c)解}
$\adj{U}AU=\bm{a}$より、$A=U\bm{a}\adj{U}$である。
従って、$B=\omega\identity-A$の逆行列$B^{-1}=G$は
\begin{align}
	BB^{-1}
&=
	\identity \\
%
%
	(\omega\identity-U\bm{a}\adj{U})
		G
&=
	\identity \\
%
%
	U(\omega\identity-\bm{a})\adj{U}
		G
&=
	\identity
\end{align}
$\omega\identity-\bm{a}$の逆行列が存在する場合、
\begin{align}
	U(\omega\identity-\bm{a})\adj{U}
		G
&=
	\identity \\
%
%
	(\omega\identity-\bm{a})\adj{U}
		G
&=
	\adj{U} \identity
=
	\adj{U} \\
%
%
	\adj{U}
		G
&=
	(\omega\identity-\bm{a})^{-1} \adj{U} \\
%
%
	G
&=
	U (\omega\identity-\bm{a})^{-1} \adj{U}
\end{align}
である。$\omega\identity-\bm{a}$は対角行列であるので、
その逆行列は
\begin{align}
	(\omega\identity-\bm{a})^{-1}
&=
	\left[
	\begin{array}{cc}
		\begin{array}{cc}
			(\omega-a_1)^{-1} & \\
			 & (\omega-a_2)^{-1}
		\end{array} &
		%
		\bm{0} \\
	%
		\bm{0} &
		%
		\begin{array}{cc}
			\ddots & \\
			 & (\omega-a_N)^{-1}
		\end{array}
	\end{array}
	\right]
\end{align}
従って、
\begin{align}
	(G(\omega))_{ij}
&=
	\sum_{\alpha}
		U_{i\alpha} (\omega-a_\alpha)^{-1} \adj{U}_{\alpha j} \\
%
%
&=
	\sum_{\alpha}
		\frac{
			U_{i\alpha} \conju{U_{j\alpha}}
		}{
			\omega-a_{\alpha}
		}
\end{align}
である。更に、
$U=[\bm{c}^1\ \bm{c}^2\ \cdots\ \bm{c}^N]$より、
$U_{ij}=c_{i}^{j}$であるから、
\begin{align}
	(G(\omega))_{ij}
&=
	\sum_{\alpha}
		\frac{
			c_{i}^{\alpha} \conju{c_{j}^{\alpha}}
		}{
			\omega-a_{\alpha}
		}
\end{align}
となる。

次に2つ目の式の証明に移る。
$\mathscr{G}$の固有ケットを$\ket{\alpha}$とするとき、
行列$G$を対角化して得られる対角行列が
$(\omega\identity-\bm{a})^{-1}$であることから、
\begin{align}
	\mathscr{G}(\omega) \ket{\alpha}
&=
	(\omega-a_{\alpha})^{-1} \ket{\alpha}
\end{align}
となる。(もしくは$\ket{\alpha}$をこのように定義する)
従って、
\begin{align}
	(G(\omega))_{ij}
&=
	\Braket{i|\mathscr{G}|j} \\
%
%
&=
	\sum_{\alpha}
		\Braket{i|\mathscr{G}|\alpha}
			\Braket{\alpha|j} \\
%
%
&=
	\sum_{\alpha}
		\Braket{i|(\omega-a_{\alpha})^{-1}|\alpha}
			\Braket{\alpha|j} \\
%
%
&=
	\sum_{\alpha}
		\frac{
			\Braket{i|\alpha} \Braket{\alpha|j}
		}{
			\omega-a_{\alpha}
		}
\end{align}



