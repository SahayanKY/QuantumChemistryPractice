%ファイルID
%2020/04/28 06:35
%->2004280635(10進数)->X5AOU3(ファイル作成者)
%->X5AOU3SahayanKY
\subsection{問}
$z$軸方向に均一な電場$F$がかかった状態での
水素原子の\Schrodinger 方程式は
\begin{align}
	\left(
		-
		\frac{1}{2}\laplacian{}
		-
		\frac{1}{r}
		+
		Fr\cos\theta
	\right)\ket{\Phi}
=
	\left(
		\mathscr{H}_0
		+
		Fr\cos\theta
	\right)\ket{\Phi}
=
	\mathscr{E}(F)\ket{\Phi}
\end{align}
である。

試行関数$\ket{\tilde{\Phi}}$として、
\begin{align}
	\ket{\tilde{\Phi}}
&=
	c_1 \ket{1s}
	+
	c_2 \ket{2p_z}
\end{align}
を用いる。ここで$\ket{1s}$と$\ket{2p_z}$は
$\mathscr{H}_0$の規格化固有関数であり、
\begin{align}
	\ket{1s}
&=
	\pi^{-\frac{1}{2}} \exp(-r) &
%
%
	\mathscr{H}_0 \ket{1s}
&=
	-
	\frac{1}{2} \ket{1s} \\
%
%
	\ket{2p_z}
&=
	(32\pi)^{-\frac{1}{2}} r \exp\left(-\frac{r}{2}\right) \cos\theta &
%
%
	\mathscr{H}_0 \ket{2p_z}
&=
	-\frac{1}{8} \ket{2p_z}
\end{align}
である。$\mathscr{E}(F)$の上限$E(F)$を求めよ。

また、$E(F)$をテイラー展開$(1+x)^{\frac{1}{2}}\simeq 1+\frac{1}{2}x$を用いて
$F$の多項式に書き換え、
\begin{align}
	E(F)
&=
	E(0)
	-
	\frac{1}{2} \alpha F^2
	+
	\cdots
\end{align}
と比較することにより、近似的な双極子分極率$\alpha$を求めよ。

\subsection{解}
試行関数$\ket{\tilde{\Phi}}$での最良近似は、
\begin{align}
	\left[
	\begin{array}{cc}
		\braket{1s|\mathscr{H}|1s} &
		%
		\braket{1s|\mathscr{H}|2p_z} \\
	%
		\braket{2p_z|\mathscr{H}|1s} &
		%
		\braket{2p_z|\mathscr{H}|2p_z}
	\end{array}
	\right]
		\left[
		\begin{array}{c}
			c_1 \\ c_2
		\end{array}
		\right]
&=
	E(F)
		\left[
		\begin{array}{c}
			c_1 \\ c_2
		\end{array}
		\right]
\end{align}
を満たす$(c_1,c_2)$である。
左辺の係数行列の各要素の値を求めていく。
\begin{align}
	\braket{1s|\mathscr{H}|1s}
&=
	\braket{1s|(\mathscr{H}_0+Fr\cos\theta)|1s} \\
%
%
&=
	\bra{1s}\left(
		\mathscr{H}_0\ket{1s}
		+
		Fr\cos\theta\ket{1s}
	\right) \\
%
%
&=
	-
	\frac{1}{2}
		\braket{1s|1s}
	+
	F \braket{1s|r\cos\theta|1s} \\
%
%
&=
	-
	\frac{1}{2} \cdot 1
	+
	F
		\int_0^{\infty} \d r \int_0^\pi \d\theta \int_0^{2\pi} \d\phi\
			r^2\sin\theta \cdot
				\conju[1]{\pi^{-\frac{1}{2}}\exp(-r)} \cdot
				r \cos\theta \cdot
				\pi^{-\frac{1}{2}}\exp(-r) \\
%
%
&=
	-
	\frac{1}{2}
	+
	F \cdot 2\pi \cdot \pi^{-1}
		\int_0^\infty \d r\
			r^3 \exp(-2r)
		\int_0^\pi \d\theta\
			\sin\theta\cos\theta \\
%
%
&=
	-
	\frac{1}{2}
	+
	F
		\int_0^\infty 2t\d t\cdot %r=t^2, t:0->infty, dr=2t dt
			t^6 \exp(-2t^2)
		\int_0^\pi \d\theta\
			\sin 2\theta
	%
	\qquad (r=t^2) \\
%
%
&=
	-
	\frac{1}{2}
	+
	2F
		\int_0^\infty \d t\
			t^7 \exp(-2t^2)
		\left[
			-\frac{1}{2}\cos 2\theta
		\right]_0^\pi \\
%
%
&=
	-
	\frac{1}{2}
\end{align}
\begin{align}
	\braket{1s|\mathscr{H}|2p_z}
&=
	\braket{1s|\mathscr{H}_0|2p_z}
	+
	F
		\braket{1s|r\cos\theta|2p_z} \\
%
%
&=
	-
	\frac{1}{8} \braket{1s|2p_z} \nonumber\\&\qquad
	+
	F
		\int_0^\infty \d r  \int_0^\pi \d\theta  \int_0^{2\pi} \d\phi\
			r^2\sin\theta \cdot
			\conju[1]{\pi^{-\frac{1}{2}}\exp(-r)} \cdot
			r\cos\theta \cdot
			(32\pi)^{-\frac{1}{2}} r\exp\left(-\frac{r}{2}\right)\cos\theta \\
%
%
&=
	-
	\frac{1}{8} \cdot 0
	+
	F \cdot 2\pi \cdot 32^{-\frac{1}{2}}\pi^{-1}
		\int_0^\infty \d r\
			r^4 \exp\left(-\frac{3}{2}r\right)
		\int_0^\pi \d\theta\
			\sin\theta\cos^2\theta \\
%
%
&=
	F\cdot 2^{1-\frac{5}{2}}
		\int_0^\infty 2t\d t\
			t^8 \exp\left(-\frac{3}{2}t^2\right)
		(-1)
		\int_0^\pi \d(\cos\theta)\
			\cos^2\theta \\
%
%
&=
	F \cdot 2^{-\frac{1}{2}}
		\int_0^\infty \d t\
			t^9 \exp\left(-\frac{3}{2}t^2\right)
		(-1)
		\left[
			\frac{1}{3}\cos^3\theta
		\right]_0^\pi \\
%
%
&=
	F \cdot 2^{-\frac{1}{2}}
		%m=4, alpha=3/2
		\frac{
			4!
		}{
			2\left(\frac{3}{2}\right)^{5}
		}
		\left(
			-\frac{1}{3}
		\right)
		\left(
			(-1)^3
			-
			1^3
		\right) \\
%
%
&=
	-
	F \cdot
		2^{-\frac{1}{2}} \cdot
		\frac{
			2^3 \cdot 3
		}{
			2^{-4} \cdot 3^5
		} \cdot
		3^{-1} \cdot
		(-2) \\
%
%
&=
	F \cdot
		2^{-\frac{1}{2}+3+4+1} \cdot
		3^{1-5-1}
%
%
=
	F \cdot
		2^{\frac{15}{2}} \cdot 3^{-5}
\end{align}
\begin{align}
	\braket{2p_z|\mathscr{H}|1s}
&=
	\conju{\braket{1s|\mathscr{H}|2p_z}} \\
%
%
&=
	F \cdot 2^{\frac{15}{2}} \cdot 3^{-5}
\end{align}
\begin{align}
	&\quad
	\braket{2p_z|\mathscr{H}|2p_z} \\
&=
	\braket{2p_z|\mathscr{H}_0|2p_z}
	+
	F
		\braket{2p_z|r\cos\theta|2p_z} \\
%
%
&=
	-
	\frac{1}{8} \braket{2p_z|2p_z} \nonumber\\&\quad
	+
	F
		\int_0^\infty \d r  \int_0^\pi \d\theta  \int_0^{2\pi} \d\phi\
			r^2\sin\theta \cdot
				\conju[1]{
					(32\pi)^{-\frac{1}{2}} r\exp\left(-\frac{r}{2}\right) \cos\theta
				} \cdot
				r\cos\theta \cdot
				(32\pi)^{-\frac{1}{2}} r\exp\left(-\frac{r}{2}\right) \cos\theta \\
%
%
&=
	-
	\frac{1}{8}
	+
	F \cdot 2\pi \cdot (32\pi)^{-1}
		\int_0^\infty \d r\
			r^5 \exp(-r)
		\int_0^\pi \d\theta\
			\sin\theta \cos^3\theta \\
%
%
&=
	-
	\frac{1}{8}
	+
	F \cdot 2^{1-5}
		\int_0^\infty \d r\
			r^5 \exp(-r)
		(-1)
		\int_0^\pi \d(\cos\theta)
			\cos^3\theta \\
%
%
&=
	-
	\frac{1}{8}
	-
	F \cdot 2^{-4}
		\int_0^\infty \d r\
			r^5 \exp(-r)
		\left[
			\frac{1}{4} \cos^4\theta
		\right]_0^\pi \\
%
%
&=
	-
	\frac{1}{8}
\end{align}
である。従って、固有値方程式は
\begin{align}
	\left[
	\begin{array}{*2{>{\displaystyle}c}}
		-\frac{1}{2} &
		%
		2^{\frac{15}{2}} 3^{-5} F \\[3mm]
	%
		2^{\frac{15}{2}} 3^{-5} F &
		%
		-\frac{1}{8}
	\end{array}
	\right]
		\left[
		\begin{array}{c}
			c_1 \\[4mm] c_2
		\end{array}
		\right]
&=
	E(F)
		\left[
		\begin{array}{c}
			c_1 \\[4mm] c_2
		\end{array}
		\right]
\end{align}
となる。よって、固有値($\mathscr{E}(F)$の上限)は
\begin{align}
	E_1(F)
&=
	\frac{1}{2}
		\left(
			-
			\frac{1}{2}
			-
			\frac{1}{8}
			-
			\sqrt{
				\left(
					-
					\frac{1}{8}
					+
					\frac{1}{2}
				\right)^2
				+
				4 \cdot \left(2^{\frac{15}{2}}3^{-5}F\right)^2
			}
		\right) \\
%
%
&=
	\frac{1}{2}
		\left(
			-
			\frac{5}{8}
			-
			\sqrt{
				2^{-6} 3^2
				+
				2^{2+15} 3^{-10} F^2
			}
		\right) \\
%
%
&=
	\frac{1}{2}
		\left(
			-
			\frac{5}{8}
			-
			2^{-3} 3^1
				\sqrt{
					1
					+
					2^{23} 3^{-12} F^2
				}
		\right)
\end{align}
である。$|2^{23} 3^{-12} F^2|<<1$であるならば、
テイラー展開により
\begin{align}
	E_1(F)
&\simeq
	\frac{1}{2}
		\left(
			-
			\frac{5}{8}
			-
			2^{-3} 3^1
				\left(
					1
					+
					\frac{1}{2} 2^{23} 3^{-12} F^2
				\right)
		\right) \\
%
%
&=
	\frac{1}{2}
		\left(
			-
			1
			-
			2^{19} 3^{-11} F^2
		\right)
%
%
=
	E_1(0)
	-
	\frac{1}{2}
		2^{19} 3^{-11} F^2
\end{align}
従って、
\begin{align}
	E(F)
&=
	E(0)
	-
	\frac{1}{2} \alpha F^2
	+
	\cdots
\end{align}
と比較することにより、$\alpha$は
\begin{align}
	\alpha
&=
	2^{19} 3^{-11} \\
%
%
&=
	2.959\cdots
=
	2.96
\end{align}
と求まる。



