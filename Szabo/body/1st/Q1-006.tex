%ファイルID
%2020/04/18 17:07
%->2004181707(10進数)->X58KI3(ファイル作成者)
%->X58KI3SahayanKY
問題1.5で示した性質を利用して以下の性質を証明せよ。
\subsection{(6)問}
ある2つの行(または列)が同じであるならば、
行列式の値はゼロである。

\subsection{(6)解}
そのような行列を$A$とおく。
該当する行(または列)同士を入れ替えた行列$B$は
同一の行列$A$である。($A=B$)
一方で、行列式の性質により、
行(または列)を入れ替えると
行列式の符号が反転することから、
\begin{align}
	|A|
&=
	-|B|
=
	-|A| \\
%
%
	2|A|
&=
	0 \\
%
%
	|A|
&=
	0
\end{align}
従って、同一の行または列をもつ行列では
行列式はゼロになる。


\subsection{(7)問}
\begin{align}
	|A^{-1}|
&=
	(|A|)^{-1}
\end{align}

\subsection{(7)解}
単位行列$\identity$の行列式は、
対角行列であることから
\begin{align}
	|\identity|
=
	1
\end{align}
である。従って、
\begin{align}
	AA^{-1}
&=
	\identity \\
%
%
	|AA^{-1}|
&=
	|\identity| \\
%
%
	|A||A^{-1}|
&=
	1 \\
%
%
	|A^{-1}|
&=
	(|A|)^{-1}
\end{align}
である。


\subsection{(8)問}
$A\adj{A}=\identity$ならば
$|A|\conju[1]{|A|}=1$
である。

\subsection{(8)解}
\begin{align}
	A \adj{A}
&=
	\identity \\
%
%
	|A \adj{A}|
&=
	|\identity| \\
%
%
	|A| |\adj{A}|
&=
	1 \\
%
%
	|A| \conju[1]{|A|}
&=
	1
	%
	\qquad
	(\because |A|=\conju[1]{|\adj{A}|})
\end{align}


\subsection{(9)問}
$\adj{U}OU=\Omega$かつ$U^{-1}=\adj{U}$ならば
$|O|=|\Omega|$である。

\subsection{(9)解}
\begin{align}
	|\adj{U} O U|
&=
	|\adj{U}| |O| |U| \\
%
%
&=
	|\adj{U}| |U| |O| \\
%
%
&=
	|\adj{U}U| |O| \\
%
%
&=
	|\identity| |O| \\
%
%
&=
	|O|
\end{align}
\begin{align}
	\therefore
	|\adj{U}OU|
=
	|O|
=
	|\Omega|
\end{align}
である。

