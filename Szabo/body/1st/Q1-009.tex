%ファイルID
%2020/04/19 11:17
%->2004191117(10進数)->X58RRH(ファイル作成者)
%->X58RRHSahayanKY

\subsection{問}
次の式を考える。
\begin{align}
	OU
&=
	U
		\left[
		\begin{array}{cc}
				\begin{array}{cc}
					\omega_1 &  \\
					 & \omega_2
				\end{array}
				&
				\bm{0} \\
			%
				\bm{0}
				&
				\begin{array}{cc}
					\ddots &    \\
					 & \omega_N
				\end{array}
		\end{array}
		\right] &
%
%
	U
&=
	\left[
	\begin{array}{cccc}
		\bm{c}^1 & \bm{c}^2 & \cdots & \bm{c}^N
	\end{array}
	\right]
\end{align}
この式が$\alpha=1,2,\cdots,N$についての
下の式を含むことを示せ。
\begin{align}
	O \bm{c}^{\alpha}
&=
	\omega_{\alpha} \bm{c}^{\alpha}
\end{align}


\subsection{解}
\begin{align}
	OU
&=
	O
		\left[
		\begin{array}{cccc}
			\bm{c}^1 & \bm{c}^2 & \cdots & \bm{c}^N
		\end{array}
		\right] \\
%
%
&=
	\left[
	\begin{array}{cccc}
		O\bm{c}^1 & O\bm{c}^2 & \cdots & O\bm{c}^N
	\end{array}
	\right]
\end{align}
\begin{align}
	U {\rm diag}(\omega_1,\omega_2,\cdots,\omega_N)
&=
	\left[
	\begin{array}{cccc}
		\omega_1\bm{c}^1 & \omega_2\bm{c}^2 & \cdots & \omega_N\bm{c}^N
	\end{array}
	\right]
\end{align}
である。従って、それぞれの行列の列を比較することで、
\begin{align}
	O\bm{c}^{\alpha}
&=
	\omega_{\alpha}\bm{c}^{\alpha}
\end{align}
となる。
