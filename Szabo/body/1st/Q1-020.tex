%ファイルID
%2020/04/24 11:06
%->2004241106(10進数)->X59UC2(ファイル作成者)
%->X59UC2SahayanKY
\subsection{問}
変分原理を行列の固有値問題に適用する。
2次対称行列
\begin{align}
	O
&=
	\left[
	\begin{array}{cc}
		O_{11} & O_{12} \\
		O_{12} & O_{22}
	\end{array}
	\right]
\end{align}
に対して試行ベクトル
\begin{align}
	\bm{c}
&=
	\left[
	\begin{array}{c}
		\cos\theta \\ \sin \theta
	\end{array}
	\right]
\end{align}
を考える。$\omega(\theta)=\adj{\bm{c}}O\bm{c}$を
極小にする$\theta$の値$\theta_0$を求め、
そのときに丁度$O$の最小固有値になることを示せ。

\subsection{解}
\begin{align}
	\omega(\theta)
&=
	\adj{\bm{c}}O\bm{c} \\
%
%
&=
	\left[
	\begin{array}{cc}
		\cos\theta & \sin\theta
	\end{array}
	\right]
		\left[
		\begin{array}{cc}
			O_{11} & O_{12} \\
			O_{12} & O_{22}
		\end{array}
		\right]
		\left[
		\begin{array}{c}
			\cos\theta \\ \sin\theta
		\end{array}
		\right] \\
%
%
&=
	\cos\theta
		\left(
			O_{11}\cos\theta
			+
			O_{12}\sin\theta
		\right)
	+
	\sin\theta
		\left(
			O_{12}\cos\theta
			+
			O_{22}\sin\theta
		\right) \\
%
%
&=
	O_{11}\cos^2\theta
	+
	2O_{12}\cos\theta\sin\theta
	+
	O_{22}\sin^2\theta
\end{align}
従って、$\omega(\theta)$を極小にする$\theta$は
\begin{align}
	\frac{\d\omega(\theta)}{\d\theta}
&=
	0 \\
%
%
	-
	2O_{11}\cos\theta_0 \sin\theta_0
	+
	2O_{12}\cos 2\theta_0
	+
	2O_{22}\sin\theta_0 \cos\theta_0
&=
	0 \\
%
%
	(O_{22}-O_{11})\sin 2\theta_0
	+
	2O_{12} \cos 2\theta_0
&=
	0 \\
%
%
	\tan 2\theta_0
&=
	\frac{2O_{12}}{O_{11}-O_{22}} \\
%
%
	\theta_0
&=
	\frac{1}{2}
		\tan^{-1}\left(
			\frac{2O_{12}}{O_{11}-O_{22}}
		\right)
\end{align}
であり、そのとき$\omega(\theta)$は
\begin{align}
	\omega(\theta_0)
&=
	O_{11}\cos^2 \theta_0
	+
	O_{22}\sin^2 \theta_0
	+
	O_{12}\sin 2\theta_0
\end{align}
となる。これは既にみた通りに行列$O$の固有値の1つである。




