%ファイルID
%2020/04/18 15:32
%->2004181532(10進数)->X58KD8(ファイル作成者)
%->X58KD8SahayanKY
次の性質を$2\times 2$行列に対して確かめよ。
\subsection{(1)問}
ある行、あるいはある列の要素がすべてゼロならば、
行列式はゼロである。

\subsection{(1)解}
次の4つの行列の行列式を考える。
\begin{align}
	\left[
	\begin{array}{cc}
		0 & 0 \\
		a & b
	\end{array}
	\right] \qquad
%
%
	\left[
	\begin{array}{cc}
		a & b \\
		0 & 0
	\end{array}
	\right] \qquad
%
%
	\left[
	\begin{array}{cc}
		0 & a \\
		0 & b
	\end{array}
	\right] \qquad
%
%
	\left[
	\begin{array}{cc}
		a & 0 \\
		b & 0
	\end{array}
	\right]
\end{align}
1つ目の行列は
\begin{align}
	\left|
	\begin{array}{cc}
		0 & 0 \\
		a & b
	\end{array}
	\right|
=
	0 \cdot b - b \cdot a
=
	0
\end{align}
である。2つ目の行列は
\begin{align}
	\left|
	\begin{array}{cc}
		a & b \\
		0 & 0
	\end{array}
	\right|
=
	a \cdot 0 - b \cdot 0
=
	0
\end{align}
である。3つ目の行列は
\begin{align}
	\left|
	\begin{array}{cc}
		0 & a \\
		0 & b
	\end{array}
	\right|
=
	0 \cdot b - a \cdot 0
=
	0
\end{align}
である。4つ目の行列は
\begin{align}
	\left|
	\begin{array}{cc}
		a & 0 \\
		b & 0
	\end{array}
	\right|
=
	a \cdot 0 - 0 \cdot b
=
	0
\end{align}
である。よって、確かに行列式がゼロになることが分かる。


\subsection{(2)問}
$A_{ij}=A_{ii}\delta_{ij}$ならば、
$|A|=\Pi_{i}A_{ii}=A_{11}A_{22}\cdots A_{NN}$である。

\subsection{(2)解}
次の行列の行列式で確かめる。
\begin{align}
	\left|
	\begin{array}{cc}
		a & 0 \\
		0 & b
	\end{array}
	\right|
=
	a \cdot b - 0 \cdot 0
=
	ab
\end{align}
よって、確かに対角要素の総積になっていることが分かる。


\subsection{(3)問}
2つの行、あるいは2つの列を入れ替えると
行列式の符号が変わる。

\subsection{(3)解}
次の3つの行列の行列式で考える。
\begin{align}
	A
&=
	\left[
	\begin{array}{cc}
		a & b \\
		c & d
	\end{array}
	\right] &
%
%
	B
&=
	\left[
	\begin{array}{cc}
		c & d \\
		a & b
	\end{array}
	\right] &
%
%
	C
&=
	\left[
	\begin{array}{cc}
		b & a \\
		d & c
	\end{array}
	\right]
\end{align}
それぞれの行列式は
\begin{align}
	|A|
&=
	ad-bc &
%
%
	|B|
&=
	cb-da
=
	-|A| &
%
%
	|C|
&=
	bc-ad
=
	-|A|
\end{align}
よって、確かに行、列を入れ替えると符号は変わる。


\subsection{(4)問}
\begin{align}
	|A|
&=
	\conju[1]{|\adj{A}|}
\end{align}


\subsection{(4)解}
行列$A$を次のように置く。
\begin{align}
	A
&=
	\left[
	\begin{array}{cc}
		a & b \\
		c & d
	\end{array}
	\right]
\end{align}
このとき、行列式$|A|$と$|\adj{A}|$は
\begin{align}
	|A|
&=
	ad-bc
\end{align}
\begin{align}
	\adj{A}
&=
	\left[
	\begin{array}{cc}
		\conju{a} & \conju{c} \\
		\conju{b} & \conju{d}
	\end{array}
	\right] \\
%
%
	|\adj{A}|
&=
	\conju{a}\conju{d}-\conju{c}\conju{b} \\
%
%
&=
	\conju[1]{
		ad-bc
	} \\
%
%
&=
	\conju[1]{|A|} \\
%
%
	\conju[1]{|\adj{A}|}
&=
	\conju{\conju[1]{|A|}}
=
	|A| \\
%
%
	\therefore
	|A|
&=
	\conju[1]{|\adj{A}|}
\end{align}


\subsection{(5)問}
\begin{align}
	|AB|
&=
	|A| |B|
\end{align}


\subsection{(5)解}
行列$A,B$を次の通りに置く。
\begin{align}
	A
&=
	\left[
	\begin{array}{cc}
		a & b \\
		c & d
	\end{array}
	\right] &
%
%
	B
&=
	\left[
	\begin{array}{cc}
		e & f \\
		g & h
	\end{array}
	\right]
\end{align}
このとき、行列式$|A|,|B|,|AB|$は、
\begin{align}
	|A|
&=
	ad-bc &
%
%
	|B|
&=
	eh-fg
\end{align}
\begin{align}
	AB
&=
	\left[
	\begin{array}{cc}
		a & b \\
		c & d
	\end{array}
	\right]
		\left[
		\begin{array}{cc}
			e & f \\
			g & h
		\end{array}
		\right] \\
%
%
&=
	\left[
	\begin{array}{cc}
		ae+bg & af+bh \\
		ce+dg & cf+dh
	\end{array}
	\right] \\
%
%
	|AB|
&=
	(ae+bg)(cf+dh)
	-
	(af+bh)(ce+dg) \\
%
%
&=
	(
		acef
		+
		adeh
		+
		bcgf
		+
		bdgh
	)
	-
	(
		acfe
		+
		adfg
		+
		bche
		+
		bdhg
	) \\
%
%
&=
	adeh
	-
	adfg
	+
	bcgf
	-
	bche \\
%
%
&=
	ad(eh-fg)
	-
	bc(eh-fg) \\
%
%
&=
	(ad-bc)(eh-fg) \\
%
%
&=
	|A||B|
\end{align}
となる。