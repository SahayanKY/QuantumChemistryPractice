%ファイルID
%2020/04/24 16:57
%->2004241657(10進数)->X59URD(ファイル作成者)
%->X59URDSahayanKY
固有方程式の厳密解を$\ket{\Phi_{\alpha}}(\alpha=0,1,\cdots)$とする。
基底状態の厳密解の波動関数$\ket{\Phi_0}$と
直交する規格化された試行関数$\ket{\tilde{\Phi}'}$を考える。
つまり、$\braket{\tilde{\Phi}'|\Phi_0}=0$とする。

\subsection{(a)問}
基底状態に関する変分原理の証明と同様にして
\begin{align}
	\braket{\tilde{\Phi}'|\mathscr{H}|\tilde{\Phi}'}
&\geq
	\mathscr{E}_1
\end{align}
であることを示せ。

\subsection{(a)解}
$\ket{\tilde{\Phi}'}$を、基底関数を$\ket{\Phi_{\alpha}}$として展開すると、
\begin{align}
	\ket{\tilde{\Phi}'}
&=
	\sum_{\alpha} \ket{\Phi_{\alpha}} \braket{\Phi_{\alpha}|\tilde{\Phi}'} \\
%
%
&=
	\sum_{\alpha>0} \ket{\Phi_{\alpha}} \braket{\Phi_{\alpha}|\tilde{\Phi}'}
\end{align}
である。従って、この試行関数でのエネルギーの期待値は
\begin{align}
	\braket{\tilde{\Phi}'|\mathscr{H}|\tilde{\Phi}'}
&=
	\left(
		\sum_{\alpha>0}
			\braket{\tilde{\Phi}'|\Phi_{\alpha}} \bra{\Phi_{\alpha}}
	\right)
		\mathscr{H}
		\left(
			\sum_{\beta>0}
				\ket{\Phi_{\beta}} \braket{\Phi_{\beta}|\tilde{\Phi}'}
		\right) \\
%
%
&=
	\sum_{\alpha>0,\beta>0}
		\braket{\tilde{\Phi}'|\Phi_{\alpha}}
			\braket{\Phi_{\alpha}|\mathscr{H}|\Phi_{\beta}}
			\braket{\Phi_{\beta}|\tilde{\Phi}'} \\
%
%
&=
	\sum_{\alpha>0,\beta>0}
		\braket{\tilde{\Phi}'|\Phi_{\alpha}}
			\mathscr{E}_{\beta} \braket{\Phi_{\alpha}|\Phi_{\beta}}
			\braket{\Phi_{\beta}|\tilde{\Phi}'} \\
%
%
&=
	\sum_{\alpha>0,\beta>0}
		\braket{\tilde{\Phi}'|\Phi_{\alpha}}
			\mathscr{E}_{\beta} \delta_{\alpha\beta}
			\braket{\Phi_{\beta}|\tilde{\Phi}'} \\
%
%
&=
	\sum_{\alpha>0}
		\braket{\tilde{\Phi}'|\Phi_{\alpha}}
			\mathscr{E}_{\alpha}
			\braket{\Phi_{\alpha}|\tilde{\Phi}'} \\
%
%
&=
	\sum_{\alpha>0}
		\mathscr{E}_{\alpha}
			\left|\braket{\Phi_{\alpha}|\tilde{\Phi}'}\right|^2
\end{align}
各項について、$\mathscr{E}_{\alpha}\geq\mathscr{E}_{1}(\alpha>0)$より
\begin{align}
	\mathscr{E}_{\alpha}
		\left|\braket{\Phi_{\alpha}|\tilde{\Phi}'}\right|^2
\geq
	\mathscr{E}_{1}
		\left|\braket{\Phi_{\alpha}|\tilde{\Phi}'}\right|^2
	%
	\qquad
	(\alpha>0)
\end{align}
であるので、
\begin{align}
	\braket{\tilde{\Phi}'|\mathscr{H}|\tilde{\Phi}'}
\geq
	\sum_{\alpha>0}
		\mathscr{E}_{1}
			\left|
				\braket{\Phi_{\alpha}|\tilde{\Phi}'}
			\right|^2
=
	\mathscr{E}_{1}
		\sum_{\alpha>0}
			\left|
				\braket{\Phi_{\alpha}|\tilde{\Phi}'}
			\right|^2
\end{align}
更に、
\begin{align}
	\sum_{\alpha>0}
		\left|
			\braket{\Phi_{\alpha}|\tilde{\Phi}'}
		\right|^2
&=
	\sum_{\alpha\geq 0}
		\left|
			\braket{\Phi_{\alpha}|\tilde{\Phi}'}
		\right|^2
	%
	\qquad
	(\because\braket{\Phi_{0}|\tilde{\Phi}'}=0) \\
%
%
&=
	\sum_{\alpha}
		\braket{\tilde{\Phi}'|\Phi_{\alpha}}
			\braket{\Phi_{\alpha}|\tilde{\Phi}'} \\
%
%
&=
	\bra{\tilde{\Phi}'} 1 \ket{\tilde{\Phi}'}
=
	\braket{\tilde{\Phi}'|\tilde{\Phi}'} \\
%
%
&=
	1
\end{align}
であることから、
\begin{align}
	\braket{\tilde{\Phi}'|\mathscr{H}|\tilde{\Phi}'}
&\geq
	\mathscr{E}_1
		\sum_{\alpha>0}
			\left|
				\braket{\Phi_{\alpha}|\tilde{\Phi}'}
			\right|^2 \\
%
%
&\geq
	\mathscr{E}_1
\end{align}
