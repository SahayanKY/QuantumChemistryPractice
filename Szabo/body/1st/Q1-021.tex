%ファイルID
%2020/04/24 16:57
%->2004241657(10進数)->X59URD(ファイル作成者)
%->X59URDSahayanKY
固有方程式の厳密解を$\ket{\Phi_{\alpha}}(\alpha=0,1,\cdots)$とする。
基底状態の厳密解の波動関数$\ket{\Phi_0}$と
直交する規格化された試行関数$\ket{\tilde{\Phi}'}$を考える。
つまり、$\braket{\tilde{\Phi}'|\Phi_0}=0$とする。

\subsection{(a)問}
基底状態に関する変分原理の証明と同様にして
\begin{align}
	\braket{\tilde{\Phi}'|\mathscr{H}|\tilde{\Phi}'}
&\geq
	\mathscr{E}_1
\end{align}
であることを示せ。

\subsection{(a)解}
$\ket{\tilde{\Phi}'}$を、基底関数を$\ket{\Phi_{\alpha}}$として展開すると、
\begin{align}
	\ket{\tilde{\Phi}'}
&=
	\sum_{\alpha} \ket{\Phi_{\alpha}} \braket{\Phi_{\alpha}|\tilde{\Phi}'} \\
%
%
&=
	\sum_{\alpha>0} \ket{\Phi_{\alpha}} \braket{\Phi_{\alpha}|\tilde{\Phi}'}
\end{align}
である。従って、この試行関数でのエネルギーの期待値は
\begin{align}
	\braket{\tilde{\Phi}'|\mathscr{H}|\tilde{\Phi}'}
&=
	\left(
		\sum_{\alpha>0}
			\braket{\tilde{\Phi}'|\Phi_{\alpha}} \bra{\Phi_{\alpha}}
	\right)
		\mathscr{H}
		\left(
			\sum_{\beta>0}
				\ket{\Phi_{\beta}} \braket{\Phi_{\beta}|\tilde{\Phi}'}
		\right) \\
%
%
&=
	\sum_{\alpha>0,\beta>0}
		\braket{\tilde{\Phi}'|\Phi_{\alpha}}
			\braket{\Phi_{\alpha}|\mathscr{H}|\Phi_{\beta}}
			\braket{\Phi_{\beta}|\tilde{\Phi}'} \\
%
%
&=
	\sum_{\alpha>0,\beta>0}
		\braket{\tilde{\Phi}'|\Phi_{\alpha}}
			\mathscr{E}_{\beta} \braket{\Phi_{\alpha}|\Phi_{\beta}}
			\braket{\Phi_{\beta}|\tilde{\Phi}'} \\
%
%
&=
	\sum_{\alpha>0,\beta>0}
		\braket{\tilde{\Phi}'|\Phi_{\alpha}}
			\mathscr{E}_{\beta} \delta_{\alpha\beta}
			\braket{\Phi_{\beta}|\tilde{\Phi}'} \\
%
%
&=
	\sum_{\alpha>0}
		\braket{\tilde{\Phi}'|\Phi_{\alpha}}
			\mathscr{E}_{\alpha}
			\braket{\Phi_{\alpha}|\tilde{\Phi}'} \\
%
%
&=
	\sum_{\alpha>0}
		\mathscr{E}_{\alpha}
			\left|\braket{\Phi_{\alpha}|\tilde{\Phi}'}\right|^2
\end{align}
各項について、$\mathscr{E}_{\alpha}\geq\mathscr{E}_{1}(\alpha>0)$より
\begin{align}
	\mathscr{E}_{\alpha}
		\left|\braket{\Phi_{\alpha}|\tilde{\Phi}'}\right|^2
\geq
	\mathscr{E}_{1}
		\left|\braket{\Phi_{\alpha}|\tilde{\Phi}'}\right|^2
	%
	\qquad
	(\alpha>0)
\end{align}
であるので、
\begin{align}
	\braket{\tilde{\Phi}'|\mathscr{H}|\tilde{\Phi}'}
\geq
	\sum_{\alpha>0}
		\mathscr{E}_{1}
			\left|
				\braket{\Phi_{\alpha}|\tilde{\Phi}'}
			\right|^2
=
	\mathscr{E}_{1}
		\sum_{\alpha>0}
			\left|
				\braket{\Phi_{\alpha}|\tilde{\Phi}'}
			\right|^2
\end{align}
更に、
\begin{align}
	\sum_{\alpha>0}
		\left|
			\braket{\Phi_{\alpha}|\tilde{\Phi}'}
		\right|^2
&=
	\sum_{\alpha\geq 0}
		\left|
			\braket{\Phi_{\alpha}|\tilde{\Phi}'}
		\right|^2
	%
	\qquad
	(\because\braket{\Phi_{0}|\tilde{\Phi}'}=0) \\
%
%
&=
	\sum_{\alpha}
		\braket{\tilde{\Phi}'|\Phi_{\alpha}}
			\braket{\Phi_{\alpha}|\tilde{\Phi}'} \\
%
%
&=
	\bra{\tilde{\Phi}'} 1 \ket{\tilde{\Phi}'}
=
	\braket{\tilde{\Phi}'|\tilde{\Phi}'} \\
%
%
&=
	1
\end{align}
であることから、
\begin{align}
	\braket{\tilde{\Phi}'|\mathscr{H}|\tilde{\Phi}'}
&\geq
	\mathscr{E}_1
		\sum_{\alpha>0}
			\left|
				\braket{\Phi_{\alpha}|\tilde{\Phi}'}
			\right|^2 \\
%
%
&\geq
	\mathscr{E}_1
\end{align}
となる。


\subsection{(b)問}
関数$\ket{\tilde{\Phi}'}$を
基底状態と第1励起状態の試行関数
$\ket{\tilde{\Phi}_0}$と$\ket{\tilde{\Phi}_1}$によって、
\begin{align}
	\ket{\tilde{\Phi}'}
&=
	x \ket{\tilde{\Phi}_0}
	+
	y \ket{\tilde{\Phi}_1}
\end{align}
と置く。$\ket{\tilde{\Phi}'}$の規格化条件が
\begin{align}
	|x|^2
	+
	|y|^2
&=
	1
\end{align}
であることを示せ。

\subsection{(b)解}
\begin{align}
	\braket{\tilde{\Phi}'|\tilde{\Phi}'}
&=
	1 \\
%
%
	\left(
		\conju{x}\bra{\tilde{\Phi}_0}
		+
		\conju{y}\bra{\tilde{\Phi}_1}
	\right)
		\left(
			x\ket{\tilde{\Phi}_0}
			+
			y\ket{\tilde{\Phi}_1}
		\right)
&=
	1 \\
%
%
	|x|^2 \braket{\tilde{\Phi}_0|\tilde{\Phi}_0}
	+
	\conju{x}y \braket{\tilde{\Phi}_0|\tilde{\Phi}_1}
	+
	\conju{y}x \braket{\tilde{\Phi}_1|\tilde{\Phi}_0}
	+
	|y|^2 \braket{\tilde{\Phi}_1|\tilde{\Phi}_1}
&=
	1
\end{align}
ここで、$\braket{\tilde{\Phi}_0|\tilde{\Phi}_1}$は、
$\ket{\Psi_i}$を試行関数の基底関数とすると
\begin{align}
	\braket{\tilde{\Phi}_0|\tilde{\Phi}_1}
&=
	\left(
		\sum_i \conju{c_i^0} \bra{\Psi_i}
	\right)
		\left(
			\sum_j c_j^1 \ket{\Psi_j}
		\right) \\
%
%
&=
	\sum_{i,j}
		\conju{c_i^0} c_j^1 \braket{\Psi_i|\Psi_j} \\
%
%
&=
	\sum_{i,j}
		\conju{c_i^0} c_j^1 \delta_{ij} \\
%
%
&=
	\sum_{i}
		\conju{c_i^0} c_i^1 \\
%
%
&=
	\adj{\bm{c}^0}\bm{c}^1 \\
%
%
&=
	0
\end{align}
より直交するためにゼロである。従って、規格化条件に戻すと、
\begin{align}
	|x|^2 \cdot 1
	+
	\conju{x}y \cdot 0
	+
	\conju{y}x \cdot 0
	+
	|y|^2 \cdot 1
&=
	1 \\
%
%
	|x|^2
	+
	|y|^2
&=
	1
\end{align}
である。

\subsection{(c)問}
$x$と$y$が、
$\ket{\tilde{\Phi}'}$が規格化され、
かつ、$\braket{\tilde{\Phi}'|\Phi_0}=0$とする。
\begin{align}
	\braket{\tilde{\Phi}'|\mathscr{H}|\tilde{\Phi}'}
&=
	E_1
	-
	|x|^2 (E_1-E_0)
\end{align}
となることを示せ。

(補足)
$E_1\geq E_0$より、
\begin{align}
	E_1
\geq
	E_1-|x|^2(E_1-E_0)
=
	\braket{\tilde{\Phi}'|\mathscr{H}|\tilde{\Phi}'}
\end{align}
である。
さらに、(a)より
$\braket{\tilde{\Phi}'|\mathscr{H}|\tilde{\Phi}'}\geq\mathscr{E}_1$
であるので、
\begin{align}
	E_1
\geq
	\braket{\tilde{\Phi}'|\mathscr{H}|\tilde{\Phi}'}
\geq
	\mathscr{E}_1
\end{align}
となる。よって、試行関数$\tilde{\Phi}_1$に関する
ハミルトニアンの期待値$E_1$は
第1励起状態のエネルギーの真の値$\mathscr{E}_1$の上限となることが言える。


\subsection{(c)解}
\begin{align}
	\braket{\tilde{\Phi}_{\beta}|\mathscr{H}|\tilde{\Phi}_{\alpha}}
&=
	\sum_{i,j}
		\conju{c_i^\beta} \braket{\Psi_i|\mathscr{H}|\Psi_j} c_j^\alpha \\
%
%
&=
	\sum_{i,j}
		\conju{c_i^\beta} H_{ij} c_j^\alpha \\
%
%
&=
	\sum_{i}
		\conju{c_i^\beta} E_{\alpha} c_i^\alpha \\
%
%
&=
	E_{\alpha} \adj{\bm{c}^\beta}\bm{c}^{\alpha} \\
%
%
&=
	E_{\alpha} \delta_{\alpha\beta}
\end{align}
である。したがって、
\begin{align}
	\braket{\tilde{\Phi}'|\mathscr{H}|\tilde{\Phi}'}
&=
	\left(
		\conju{x}\bra{\tilde{\Phi}_0}
		+
		\conju{y}\bra{\tilde{\Phi}_1}
	\right)
		\mathscr{H}
		\left(
			x\ket{\tilde{\Phi}_0}
			+
			y\ket{\tilde{\Phi}_1}
		\right) \\
%
%
&=
	|x|^2 \braket{\tilde{\Phi}_0|\mathscr{H}|\tilde{\Phi}_0}
	+
	\conju{x}y \braket{\tilde{\Phi}_0|\mathscr{H}|\tilde{\Phi}_1}
	+
	\conju{y}x \braket{\tilde{\Phi}_1|\mathscr{H}|\tilde{\Phi}_0}
	+
	|y|^2 \braket{\tilde{\Phi}_1|\mathscr{H}|\tilde{\Phi}_1} \\
%
%
&=
	|x|^2 E_0
	+
	\conju{x}y \cdot 0
	+
	\conju{y}x \cdot 0
	+
	|y|^2 E_1 \\
%
%
&=
	|x|^2 E_0
	+
	(1-|x|^2) E_1
	%
	\qquad
	(\because\text{$\ket{\tilde{\Phi}'}$の規格化条件}) \\
%
%
&=
	E_1
	-
	|x|^2 (E_1-E_0)
\end{align}
となる。



