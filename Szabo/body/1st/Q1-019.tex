%ファイルID
%2020/04/24 07:30
%->2004240730(10進数)->X59U1M(ファイル作成者)
%->X59U1MSahayanKY
\subsection{問}
水素原子の\Schrodinger 方程式は
\begin{align}
	\left(
		-
		\frac{1}{2} \laplacian{}
		-
		\frac{1}{r}
	\right) \ket{\Phi}
&=
	\mathscr{E} \ket{\Phi}
\end{align}
である。

試行関数として
\begin{align}
	\ket{\tilde{\Phi}}
&=
	N \exp(-\alpha r^2)
\end{align}
を用いて変分計算を行い、
得られるエネルギーが$-\frac{4}{3\pi}$であることを示せ。
また、それが厳密なエネルギー$-0.5$よりも大きいことを示せ。

なお、公式として
\begin{align}
	\laplacian{f(r)}
&=
	r^{-2}
		\frac{\d}{\d r}\left(
			r^2 \frac{\d f}{\d r}
		\right) \\
%
%
	\int_0^{\infty} \d r\
		r^{2m} \exp(-\alpha r^2)
&=
	\frac{
		(2m)! \pi^{\frac{1}{2}}
	}{
		2^{2m+1} m! \alpha^{m+\frac{1}{2}}
	} \\
%
%
	\int_0^{\infty} \d r\
		r^{2m+1} \exp(-\alpha r^2)
&=
	\frac{m!}{2\alpha^{m+1}}
\end{align}
を用いてもよい。


\subsection{解}
まず規格化条件から、
\begin{align}
	\braket{\tilde{\Phi}|\tilde{\Phi}}
&=
	1 \\
%
%
	\int_0^{\infty} \d r\
		\int_0^\pi \d\theta\
		\int_0^{2\pi} \d\phi\
		r^2\sin\theta \conju[1]{N\exp(-\alpha r^2)} N\exp(-\alpha r^2)
&=
	1 \\
%
%
	4\pi |N|^2
		\int_0^{\infty} \d r\
			r^2 \exp(-2\alpha r^2)
&=
	1 \\
%
%
	4\pi |N|^2 \cdot
		\frac{
			2!\ \pi^{\frac{1}{2}}
		}{
			2^3\ 1!\ (2\alpha)^{\frac{3}{2}}
		}
&=
	1 \\
%
%
	|N|^2
&=
	2^{\frac{3}{2}} \pi^{-\frac{3}{2}} \alpha^{\frac{3}{2}}
\end{align}
である。この下で、エネルギーの期待値を求めると、
\begin{align}
	\braket{\tilde{\Phi}|\H|\tilde{\Phi}}
&=
	\int_0^{\infty} \d r\
		\int_0^\pi \d\theta\
		\int_0^{2\pi} \d\phi\
		r^2\sin\theta
			\conju[1]{N\exp(-\alpha r^2)}
			\left(
				-
				\frac{1}{2}\laplacian{}
				-
				\frac{1}{r}
			\right)
			N\exp(-\alpha r^2) \\
%
%
&=
	4\pi |N|^2
		\int_0^{\infty} \d r\
			r^2
				\exp(-\alpha r^2)
				\left(
					-
					\frac{1}{2}r^{-2}
						\frac{\d}{\d r}\left(
							r^2\frac{\d}{\d r}
						\right)
					-
					\frac{1}{r}
				\right) \exp(-\alpha r^2) \\
%
%
&=
	4\pi |N|^2
		\left\{
		\begin{array}{>{\displaystyle}l}
			-
			\frac{1}{2}
				\int_0^{\infty} \d r\
					\exp(-\alpha r^2)
						\frac{\d}{\d r}\left(
							r^2 \cdot \exp(-\alpha r^2) \cdot (-2\alpha r)
						\right) \\
			-
			\int_0^{\infty} \d r\
				r \exp(-2\alpha r^2)
		\end{array}
		\right\} \\
%
%
&=
	4\pi |N|^2
		\left\{
		\begin{array}{>{\displaystyle}l}
			\alpha
				\int_0^{\infty} \d r\Bigl(
					3r^2 \exp(-2\alpha r^2)
					+
					r^3 \exp(-2\alpha r^2) (-2\alpha r)
				\Bigr) \\
			-
			\frac{0!}{2(2\alpha)^{1}}
		\end{array}
		\right\} \\
%
%
&=
	4\pi |N|^2
		\left\{
			3\alpha
				\frac{%m=1,alpha=2alpha
					2!\ \pi^{\frac{1}{2}}
				}{
					2^3\ 1!\ (2\alpha)^{\frac{3}{2}}
				}
			-
			2\alpha^2
				\frac{%m=2,alpha=2alpha
					4!\ \pi^{\frac{1}{2}}
				}{
					2^5\ 2!\ (2\alpha)^{\frac{5}{2}}
				}
			-
			2^{-2} \alpha^{-1}
		\right\} \\
%
%
&=
	4\pi \cdot
		2^{\frac{3}{2}} \pi^{-\frac{3}{2}} \alpha^{\frac{3}{2}}
		\left\{
			2^{-\frac{7}{2}} 3 \pi^{\frac{1}{2}} \alpha^{-\frac{1}{2}}
			-
			2^{-\frac{9}{2}} 3 \pi^{\frac{1}{2}} \alpha^{-\frac{1}{2}}
			-
			2^{-2} \alpha^{-1}
		\right\} \\
%
%
&=
	2^{\frac{7}{2}} \pi^{-\frac{1}{2}}
		\left\{
			2^{-\frac{9}{2}} 3 \pi^{\frac{1}{2}} \alpha^{1}
			-
			2^{-2} \alpha^{\frac{1}{2}}
		\right\}
\end{align}
となる。これを極小化する$\alpha$は
\begin{align}
	\frac{\d}{\d\alpha}\left(
		\braket{\tilde{\Phi}|\H|\tilde{\Phi}}
	\right)
&=
	0 \\
%
%
	2^{\frac{7}{2}} \pi^{-\frac{1}{2}}
		\left\{
			2^{-\frac{9}{2}} 3 \pi^{\frac{1}{2}}
			-
			2^{-3} \alpha^{-\frac{1}{2}}
		\right\}
&=
	0 \\
%
%
	\alpha^{-\frac{1}{2}}
&=
	2^{-\frac{3}{2}} 3 \pi^{\frac{1}{2}} \\
%
%
	\alpha
&=
	2^{3} 3^{-2} \pi^{-1}
=
	\alpha_0
\end{align}
であるから、期待値の極小値は
\begin{align}
	\min\left(
		\braket{\tilde{\Phi}|\H|\tilde{\Phi}}
	\right)
&=
	2^{\frac{7}{2}} \pi^{-\frac{1}{2}}
		\left(
			2^{-\frac{9}{2}} 3 \pi^{\frac{1}{2}} \alpha_0
			-
			2^{-2} \alpha_0^{\frac{1}{2}}
		\right) \\
%
%
&=
	2^{\frac{7}{2}} \pi^{-\frac{1}{2}}
		\left(
			2^{-\frac{9}{2}} 3 \pi^{\frac{1}{2}} \cdot
				2^{3} 3^{-2} \pi^{-1}
			-
			2^{-2} 2^{\frac{3}{2}} 3^{-1} \pi^{-\frac{1}{2}}
		\right) \\
%
%
&=
	2^{\frac{7}{2}} \pi^{-\frac{1}{2}}
		\left(
			2^{-\frac{3}{2}} 3^{-1} \pi^{-\frac{1}{2}}
			-
			2^{-\frac{1}{2}} 3^{-1} \pi^{-\frac{1}{2}}
		\right) \\
%
%
&=
	-
	2^{\frac{7}{2}} \pi^{-\frac{1}{2}} \cdot
		2^{-\frac{3}{2}} 3^{-1} \pi^{-\frac{1}{2}} \\
%
%
&=
	-
	\frac{4}{3\pi}
\end{align}
である。
\begin{align}
	3
&<
	\pi \\
%
%
	\frac{4}{3\pi}
&<
	\frac{4}{9} \\
%
%
	-0.5
<
	-0.\dot{4}
=
	-\frac{4}{9}
&<
	-\frac{4}{3\pi}
\end{align}
であることから、得られた値は厳密解よりも大きいことが分かる。
