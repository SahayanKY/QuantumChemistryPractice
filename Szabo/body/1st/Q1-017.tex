%ファイルID
%2020/04/23 09:06
%->2004230906(10進数)->X59MGQ(ファイル作成者)
%->X59MGQSahayanKY
番号付けが可能な(離散的な)無限個の完全規格直交基底は
\begin{align}
	\sum_i \ket{i}\bra{i}
&=
	1
	%
	\label{eqX59MGQSahayanKY_CompletenessRelationOfDiscreteBasis} \\
%
%
	\braket{i|j}
&=
	\delta_{ij}
\end{align}
となる。

一方で、連続無限の完全基底$\ket{x}$は、対応するように
\begin{align}
	\int \d x
		\ket{x}\bra{x}
&=
	1
	%
	\label{eqX59MGQSahayanKY_CompletenessRelationOfContinuousBasis}
\end{align}
となる。これに左から$\bra{a}$、
右から$\ket{b}$をかけると、
\begin{align}
	\int \d x
		\braket{a|x} \braket{x|b}
=
	\braket{a|b}
=
	\int \d x\
		\conju{a}(x) b(x)
\end{align}
となることから、
\begin{align}
	\conju{a}(x)
&=
	\braket{a|x} &
%
%
	b(x)
&=
	\braket{x|b}
\end{align}
である。


\subsection{(a)問}
式\ref{eqX59MGQSahayanKY_CompletenessRelationOfContinuousBasis}に、
左から$\bra{i}$、右から$\ket{j}$をかける。
すると、
\begin{align}
	\int \d x\
		\conju{\psi_i}(x) \psi_j(x)
&=
	\delta_{ij}
\end{align}
に等しいことを示せ。

\subsection{(a)解}
\begin{align}
	\int \d x
		\ket{x}\bra{x}
&=
	1 \\
%
%
	\int \d x
		\braket{i|x} \braket{x|j}
&=
	\braket{i|j} \\
%
%
	\int \d x\
		\conju{\psi_i}(x) \psi_{j}(x)
&=
	\delta_{ij}
\end{align}
である。

\subsection{(b)問}
式\ref{eqX59MGQSahayanKY_CompletenessRelationOfDiscreteBasis}に、
左から$\bra{x}$、右から$\ket{x'}$をかける。すると、
$\braket{x|x'}=\delta(x-x')$であれば
\begin{align}
	\sum_i
		\psi_i(x)\conju{\psi_i}(x')
&=
	\delta(x-x')
\end{align}
となることを示せ。


\subsection{(b)解}
\begin{align}
	\sum_i
		\ket{i}\bra{i}
&=
	1 \\
%
%
	\sum_i
		\braket{x|i}\braket{i|x'}
&=
	\braket{x|x'} \\
%
%
	\sum_i
		\psi_i(x) \conju{\psi_i}(x')
&=
	\delta(x-x')
\end{align}
である。


\subsection{(c)問}
式\ref{eqX59MGQSahayanKY_CompletenessRelationOfContinuousBasis}に、
左から$\bra{x'}$、右から$\ket{a}$をかけると
\begin{align}
	a(x)
&=
	\int \d x'
		\delta(x-x') a(x')
\end{align}
が得られることを示せ。

\subsection{(c)解}
\begin{align}
	\int \d x
		\ket{x}\bra{x}
&=
	1 \\
%
%
	\int \d x
		\braket{x'|x}\bra{x|a}
&=
	\braket{x'|a} \\
%
%
	\int \d x
		\delta(x-x')a(x)
&=
	a(x') \\
%
%
	\int \d x'
		\delta(x'-x)a(x')
&=
	a(x)
\end{align}
である。


\subsection{(d)問}
ある演算子$\mathscr{O}$の
連続基底$\ket{x}$における行列要素は
\begin{align}
	\braket{x|\mathscr{O}|x'}
&=
	O(x,x')
\end{align}
である。また、$\mathscr{O}\ket{a}=\ket{b}$とする。これを変形すると、
\begin{align}
	\mathscr{O}\ket{a}
&=
	\ket{b} \\
%
%
	\mathscr{O}1\ket{a}
&=
	\\
%
%
	\int \d x\
		\mathscr{O}\ket{x} \braket{x|a}
&=
	\ket{b}
\end{align}
となる。この式に$\bra{x'}$をかけると
\begin{align}
	b(x)
=
	\mathscr{O} a(x)
=
	\int \d x'\
		O(x,x') a(x')
\end{align}
が得られることを示せ。

\subsection{(d)解}
\begin{align}
	\int \d x\
		\braket{x'|\mathscr{O}|x} \braket{x|a}
&=
	\braket{x'|b} \\
%
%
	\int \d x\
		O(x',x) a(x)
&=
	b(x') \\
%
%
	b(x)
&=
	\int \d x'\
		O(x,x') a(x')
\end{align}
である。


\subsection{(e)問}
\begin{align}
	O_{ij}
=
	\braket{i|\mathscr{O}|j}
%
\Rightarrow
%
	O(x,x')
=
	\sum_{i,j}
		\psi_i(x)O_{ij}\conju{\psi_j}(x')
\end{align}
であることを示せ。

\subsection{(e)解}
\begin{align}
	O(x,x')
&=
	\braket{x|\mathscr{O}|x'} \\
%
%
&=
	\bra{x}
		\left(
			\sum_i \ket{i}\bra{i}
		\right)
		\mathscr{O}
		\left(
			\sum_j \ket{j}\bra{j}
		\right)
		\ket{x'} \\
%
%
&=
	\sum_{i,j}
		\braket{x|i}
			\braket{i|\mathscr{O}|j}
			\braket{j|x'} \\
%
%
&=
	\sum_{i,j}
		\psi_i(x)
			O_{ij}
			\conju{\psi_j}(x')
\end{align}
である。