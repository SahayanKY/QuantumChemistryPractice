%ファイルID
%2020/04/17 17:51
%->2004171751(10進数)->X58CTJ(ファイル作成者)
%->X58CTJSahayanKY
\subsection{問}
\begin{align}
	\adj[1]{AB}
=
	\adj{B} \adj{A}
\end{align}
を示せ。なお、$\adj{A}$は共役(adjoint)行列であり、
\begin{align}
	(\adj{A})_{ij}
=
	(\conju{A})_{ji}
=
	(\tp[1]{\conju{A}})_{ij}
=
	(\conju[1]{\tp{A}})_{ij}
\end{align}
である。即ち、$A$の複素共役をとったものを転置したものである。

\subsection{解}
\begin{align}
	\left(\adj[1]{AB}\right)_{ij}
&=
	\left(\conju[1]{AB}\right)_{ji} \\
%
%
&=
	\conju[1]{
		(AB)_{ji}
	} \\
%
%
&=
	\conju[1]{
		A_{jk} B_{ki}
	} \\
%
%
&=
	\conju{A_{jk}} \conju{B_{ki}} \\
%
%
&=
	\conju[1]{
		(\tp{A})_{kj}
	}
		\conju[1]{
			(\tp{B})_{ik}
		} \\
%
%
&=
	(\conju[1]{\tp{A}})_{kj}
		(\conju[1]{\tp{B}})_{ik} \\
%
%
&=
	(\adj{A})_{kj}
		(\adj{B})_{ik} \\
%
%
&=
	(\adj{B}\adj{A})_{ij}
\end{align}
よって、
\begin{align}
	\adj[1]{AB}
&=
	\adj{B} \adj{A}
\end{align}
である。




