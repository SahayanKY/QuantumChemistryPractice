%ファイルID
%2020/04/20 08:27
%->2004200827(10進数)->X58Z97(ファイル作成者)
%->X58Z97SahayanKY
次の2つの行列について、
固有値、固有ベクトルを指定の方法で求めよ。
\begin{align}
	A
&=
	\left[
	\begin{array}{cc}
		3 & 1 \\
		1 & 3
	\end{array}
	\right] &
%
%
	B
&=
	\left[
	\begin{array}{cc}
		3 & 1 \\
		1 & 2
	\end{array}
	\right]
\end{align}

\subsection{(a)問}
永年行列式を利用して求めよ。

\subsection{(a)解}
まず行列$A$について考える。
固有値を$\omega$として、
永年方程式及び固有値は
\begin{align}
	|A-\omega\identity|
&=
	0 \\
%
%
	\left|
	\begin{array}{cc}
		3-\omega & 1 \\
		1 & 3-\omega
	\end{array}
	\right|
&=
	0 \\
%
%
	\omega^2
	-
	6\omega
	+
	8
&=
	0 \\
%
%
	\omega
&=
	3
	\pm
	1
=
	4,\ 2
\end{align}
である。

固有値$\omega$が4のときは、
\begin{align}
	A\bm{c}
&=
	4\bm{c} \\
%
%
	\left[
	\begin{array}{cc}
		-1 & 1 \\
		1 & -1
	\end{array}
	\right] \bm{c}
&=
	\bm{0} \\
%
%
	\bm{c}
&=
	C
		\left[
		\begin{array}{c}
			1 \\ 1
		\end{array}
		\right]
	%
	\qquad
	(C\in\mathbb{R})
\end{align}
一方で$\omega$が2のときには、
\begin{align}
	A\bm{c}
&=
	2\bm{c} \\
%
%
	\left[
	\begin{array}{cc}
		1 & 1 \\
		1 & 1
	\end{array}
	\right] \bm{c}
&=
	\bm{0} \\
%
%
	\bm{c}
&=
	C
		\left[
		\begin{array}{c}
			1 \\ -1
		\end{array}
		\right]
	%
	\qquad
	(C\in\mathbb{R})
\end{align}
である。

次に行列$B$について考える。
同様に永年方程式とその固有値$\omega$は
\begin{align}
	|B-\omega\identity|
&=
	0 \\
%
%
	\left|
	\begin{array}{cc}
		3-\omega & 1 \\
		1 & 2-\omega
	\end{array}
	\right|
&=
	0 \\
%
%
	\omega^2
	-
	5\omega
	+
	5
&=
	0 \\
%
%
	\omega
&=
	\frac{1}{2}
		\left(
			5
			\pm
			\sqrt{5}
		\right)
\end{align}
である。$\omega=\frac{5}{2}+\frac{\sqrt{5}}{2}$のとき
固有ベクトルは
\begin{align}
	\left[
	\begin{array}{cc}
		\frac{1}{2}-\frac{\sqrt{5}}{2} & 1 \\
		1 & -\frac{1}{2}-\frac{\sqrt{5}}{2}
	\end{array}
	\right]
		\bm{c}
&=
	\bm{0} \\
%
%
	\left[
	\begin{array}{cc}
		\frac{1}{2}-\frac{\sqrt{5}}{2} & 1 \\
		\frac{1}{2}-\frac{\sqrt{5}}{2} & 1
	\end{array}
	\right]
		\bm{c}
&=
	\bm{0} \\
%
%
	\bm{c}
&=
	C
		\left[
		\begin{array}{c}
			1 \\ -\frac{1}{2}+\frac{\sqrt{5}}{2}
		\end{array}
		\right]
	%
	\qquad
	(C\in\mathbb{R})
\end{align}
$\omega=\frac{5}{2}-\frac{\sqrt{5}}{2}$のとき
固有ベクトルは
\begin{align}
	\left[
	\begin{array}{cc}
		\frac{1}{2}+\frac{\sqrt{5}}{2} & 1 \\
		1 & -\frac{1}{2}+\frac{\sqrt{5}}{2}
	\end{array}
	\right]
		\bm{c}
&=
	\bm{0} \\
%
%
	\left[
	\begin{array}{cc}
		\frac{1}{2}+\frac{\sqrt{5}}{2} & 1 \\
		\frac{1}{2}+\frac{\sqrt{5}}{2} & 1
	\end{array}
	\right]
		\bm{c}
&=
	\bm{0} \\
%
%
	\bm{c}
&=
	C
		\left[
		\begin{array}{c}
			1 \\ -\frac{1}{2}-\frac{\sqrt{5}}{2}
		\end{array}
		\right]
	%
	\qquad
	(C\in\mathbb{R})
\end{align}

\subsection{(b)問}
ユニタリー変換を使う方法で求めよ。

\subsection{(b)解}
まず、行列$A$について考える。
行列$U$を次の通りに置く。
\begin{align}
	U
&=
	\left[
	\begin{array}{cc}
		\cos\theta & \sin\theta \\
		\sin\theta & -\cos\theta
	\end{array}
	\right]
\end{align}
このとき、$\adj{U}AU$が対角行列となる$\theta$は
\begin{align}
	\frac{1}{2}
		(A_{11}-A_{22})
		\sin 2\theta
	-
	A_{12}
		\cos 2\theta
&=
	0 \\
%
%
	\cos 2\theta
&=
	0 \\
%
%
	\theta
&=
	\frac{\pi}{4}
\end{align}
である。更に、固有値は
\begin{align}
	\omega_1
&=
	A_{11} \cos^2\theta
	+
	A_{22} \sin^2\theta
	+
	A_{12} \sin 2\theta \\
%
%
&=
	3 \cdot \left(\frac{1}{\sqrt{2}}\right)^2
	+
	3 \cdot \left(\frac{1}{\sqrt{2}}\right)^2
	+
	1 \cdot 1 \\
%
%
&=
	4
\end{align}
\begin{align}
	\omega_2
&=
	A_{11} \sin^2\theta
	+
	A_{22} \cos^2\theta
	-
	A_{12} \sin 2\theta \\
%
%
&=
	3 \cdot \left(\frac{1}{\sqrt{2}}\right)^2
	+
	3 \cdot \left(\frac{1}{\sqrt{2}}\right)^2
	-
	1 \cdot 1 \\
%
%
&=
	2
\end{align}
である。また、それぞれの固有ベクトルは
\begin{align}
	\left[
	\begin{array}{c}
		\cos\theta \\ \sin\theta
	\end{array}
	\right]
&=
	\frac{1}{\sqrt{2}}
		\left[
		\begin{array}{c}
			1 \\ 1
		\end{array}
		\right] &
%
%
	\left[
	\begin{array}{c}
		\sin\theta \\ -\cos\theta
	\end{array}
	\right]
&=
	\frac{1}{\sqrt{2}}
		\left[
		\begin{array}{c}
			1 \\ -1
		\end{array}
		\right]
\end{align}
である。

次に行列$B$について考える。
同様に行列$U$を置くと、
行列$\adj{U}BU$が対角行列となる$\theta$は
\begin{align}
	\frac{1}{2}
		(B_{11}-B_{22})
		\sin 2\theta
	-
	B_{12}
		\cos 2\theta
&=
	0 \\
%
%
	\frac{1}{2}
		\sin 2\theta
	-
	\cos 2\theta
&=
	0 \\
%
%
	\tan 2\theta
&=
	2 \\
%
%
	\frac{
		\sin^2 2\theta
	}{
		1
		-
		\sin^2 2\theta
	}
&=
	4 \\
%
%
	\sin^2 2\theta
&=
	\frac{4}{5} \\
%
%
	\cos^2 2\theta
&=
	\frac{1}{5}
\end{align}
\begin{align}
	\cos 2\theta
&=
	\frac{1}{\sqrt{5}} \\
%
%
	2\cos^2 \theta
	-
	1
&=
	\frac{\sqrt{5}}{5} \\
%
%
	\cos^2 \theta
&=
	\frac{5+\sqrt{5}}{10} \\
%
%
	\sin^2 \theta
&=
	\frac{5-\sqrt{5}}{10}
\end{align}
\begin{align}
	\sin 2\theta
&=
	\sqrt{\frac{4}{5}}
=
	\frac{2\sqrt{5}}{5}
\end{align}
である。よって、固有値は
\begin{align}
	\omega_1
&=
	B_{11} \cos^2 \theta
	+
	B_{22} \sin^2 \theta
	+
	B_{12} \sin 2\theta \\
%
%
&=
	3 \cdot \frac{5+\sqrt{5}}{10}
	+
	2 \cdot \frac{5-\sqrt{5}}{10}
	+
	1 \cdot \frac{2\sqrt{5}}{5} \\
%
%
&=
	\frac{
		25
		+
		5\sqrt{5}
	}{
		10
	} \\
%
%
&=
	\frac{5+\sqrt{5}}{2} \\
%
%
	\omega_2
&=
	B_{11} \sin^2 \theta
	+
	B_{22} \cos^2 \theta
	-
	B_{12} \sin 2\theta \\
%
%
&=
	3 \cdot \frac{5-\sqrt{5}}{10}
	+
	2 \cdot \frac{5+\sqrt{5}}{10}
	-
	1 \cdot \frac{2\sqrt{5}}{5} \\
%
%
&=
	\frac{
		25
		-
		5\sqrt{5}
	}{
		10
	} \\
%
%
&=
	\frac{5-\sqrt{5}}{2}
\end{align}
である。また、
\begin{align}
	\tan^2 \theta
&=
	\frac{
		\frac{5-\sqrt{5}}{10}
	}{
		\frac{5+\sqrt{5}}{10}
	} \\
%
%
&=
	\frac{
		(5-\sqrt{5})^2
	}{
		20
	} \\
%
%
	\tan \theta
&=
	\frac{
		5-\sqrt{5}
	}{
		2\sqrt{5}
	} \\
%
%
&=
	\frac{
		-1+\sqrt{5}
	}{
		2
	}
\end{align}
であることから、それぞれの固有ベクトルは
\begin{align}
	\left[
	\begin{array}{c}
		1 \\ \tan\theta
	\end{array}
	\right]
&=
	\left[
	\begin{array}{c}
		1 \\ \frac{-1+\sqrt{5}}{2}
	\end{array}
	\right] &
%
%
	\left[
	\begin{array}{c}
		\tan\theta \\ -1
	\end{array}
	\right]
&=
	\left[
	\begin{array}{c}
		\frac{-1+\sqrt{5}}{2} \\ -1
	\end{array}
	\right]
\end{align}
である。