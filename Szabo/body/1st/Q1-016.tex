%ファイルID
%2020/04/22 22:44
%->2004222244(10進数)->X59FS4(ファイル作成者)
%->X59FS4SahayanKY
\subsection{問}
固有値問題
\begin{align}
	\mathscr{O} \phi
&=
	\omega \phi
\end{align}
を考える。完全系$\psi_i$で$\phi$を
\begin{align}
	\phi
&=
	\sum_i c_i \psi_i
\end{align}
と展開すると、この問題は行列の固有値問題
\begin{align}
	O\bm{c}
&=
	\omega\bm{c}
\end{align}
と等価になることを示せ。
その証明の方法として、
ブラケット記法を使う方法と
使わない方法の2通りを示せ。


\subsection{解}
まず、ブラケット記法を使わずに示す。
\begin{align}
	\mathscr{O}\phi
&=
	\omega\phi \\
%
%
	\mathscr{O}\left(
		\sum_j c_j \psi_j
	\right)
&=
	\omega\left(
		\sum_i c_i \psi_i
	\right) \\
%
%
	\int \d x\
		\conju{\psi_k}
			\mathscr{O}\left(
				\sum_j c_j \psi_j
			\right)
&=
	\int \d x\
		\conju{\psi_k}
			\omega\left(
				\sum_i c_i \psi_i
			\right) \\
%
%
	\sum_j
		c_j\left(
			\int \d x\
				\conju{\psi_k}\mathscr{O}\psi_j
		\right)
&=
	\omega
		\sum_i
			c_i\left(
				\int \d x\
					\conju{\psi_k}\psi_i
			\right) \\
%
%
	\sum_j
		c_j O_{kj}
&=
	\omega
		\sum_i
			c_i \delta_{ki} \\
%
%
	\sum_j
		O_{ij} c_j
&=
	\omega c_i
\end{align}
である。これは即ち行列の固有値問題に他ならない。
従って、確かに関数の固有値問題は行列の固有値問題に
書き換えることが可能である。

次にブラケット記法を用いて示す。
\begin{align}
	\mathscr{O}\ket{\phi}
&=
	\omega\ket{\phi} \\
%
%
	\mathscr{O}\left(
		\sum_j c_j \ket{j}
	\right)
&=
	\omega\left(
		\sum_i c_i \ket{i}
	\right) \\
%
%
	\sum_j
		c_j \mathscr{O}\ket{j}
&=
	\sum_i
		\omega c_i \ket{i} \\
%
%
	\sum_j
		c_j \Braket{k|\mathscr{O}|j}
&=
	\sum_i
		\omega c_i \Braket{k|i} \\
%
%
	\sum_j
		c_j O_{kj}
&=
	\sum_i
		\omega c_i \delta_{ki} \\
%
%
	\sum_j
		O_{ij} c_j
&=
	\omega c_i
\end{align}
従って、ブラケット記法でも同様である。



