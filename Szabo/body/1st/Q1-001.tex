%ファイルID
%2020/04/17 10:55
%->2004171055(10進数)->X58CA7(ファイル作成者)
%->X58CA7SahayanKY
\begin{align}
	\mathscr{O}\e{i}
&=
	\e{j} O_{ji}
\end{align}
とする。また、$\e{i}$は正規直交基底である。

\subsection{(a)問}
\begin{align}
	O_{ij}
&=
	\e{i}\cdot \mathscr{O} \e{j}
\end{align}
を示せ。

\subsection{(a)解}
\begin{align}
	\mathscr{O}\e{j}
&=
	\e{k} O_{kj} \\
%
%
	\e{i}\cdot \left(\mathscr{O}\e{j}\right)
&=
	\e{i}\cdot \left(\e{k} O_{kj}\right) \\
%
%
	\e{i}\cdot \mathscr{O}\e{j}
&=
	\delta_{ik} O_{kj} \\
%
%
&=
	O_{ij} \\
%
%
	\therefore
	O_{ij}
&=
	\e{i}\cdot \mathscr{O}\e{j}
\end{align}


\subsection{(b)問}
\begin{align}
	\bm{b}
&=
	\mathscr{O}\bm{a}
\end{align}
とするとき、
\begin{align}
	b_i
&=
	O_{ij} a_{j}
\end{align}
であることを示せ。

\subsection{(b)解}
\begin{align}
	\bm{b}
&=
	\mathscr{O}\bm{a} \\
%
%
	b_i \e{i}
&=
	\mathscr{O}(a_k \e{k}) \\
%
%
&=
	a_k \mathscr{O} \e{k} \\
%
%
&=
	a_k \e{j} O_{jk} \\
%
%
	b_i \e{i}\cdot \e{l}
&=
	a_k O_{jk} \e{j}\cdot \e{l} \\
%
%
	b_i \delta_{il}
&=
	a_k O_{jk} \delta_{jl} \\
%
%
	b_l
&=
	a_k O_{lk} \\
%
%
	\therefore
	b_i
&=
	O_{ij} a_j
\end{align}

