%ファイルID
%2020/04/22 16:33
%->2004221633(10進数)->X59FB5(ファイル作成者)
%->X59FB5SahayanKY
\subsection{問}
デルタ関数$\delta(x)$は次のように書くことができる。
\begin{align}
	\delta(x)
&=
	\lim_{\epsilon \rightarrow +0}
		\delta_{\epsilon}(x) &
%
%
	\delta_{\epsilon}(x)
&=
	\left\{
	\begin{array}{ll}
		\frac{1}{2\epsilon} &
		%
		(-\epsilon\leq x\leq \epsilon) \\
	%
		0 &
		%
		(\text{otherwise})
	\end{array}
	\right.
\end{align}
このとき、次の式を示せ。
\begin{align}
	a(0)
&=
	\int_{-\infty}^{\infty} \d x\
		a(x) \delta(x)
\end{align}


\subsection{解}
極限と積分が可換であることを仮定すれば、
\begin{align}
	\int_{-\infty}^{\infty} \d x\
		a(x) \delta(x)
%
%
&=
	\int_{-\infty}^{\infty} \d x\
		a(x)
			\lim_{\epsilon\rightarrow +0}
				\delta_{\epsilon}(x) \\
%
%
&=
	\lim_{\epsilon\rightarrow +0}
		\int_{-\infty}^{\infty} \d x\
			a(x) \delta_{\epsilon}(x) \\
%
%
&=
	\lim_{\epsilon\rightarrow +0}
		\int_{-\epsilon}^{\epsilon} \d x\
			a(x) \frac{1}{2\epsilon} \\
%
%
&=
	\lim_{\epsilon\rightarrow +0}
		\frac{1}{2\epsilon}
			\int_{-\epsilon}^{\epsilon} \d x\
				a(x)
\end{align}
ここで、$A(x)$を$a(x)$の原始関数とすると、
\begin{align}
&=
	\lim_{\epsilon\rightarrow +0}
		\frac{1}{2\epsilon}
			\left[A(x)\right]_{-\epsilon}^{\epsilon} \\
%
%
&=
	\lim_{\epsilon\rightarrow +0}
		\frac{
			A(\epsilon)
			-
			A(-\epsilon)
		}{
			2\epsilon
		} \\
%
%
&=
	\frac{1}{2}
		\lim_{\epsilon\rightarrow +0}\left(
			\frac{A(\epsilon)-A(0)}{\epsilon}
			+
			\frac{A(0)-A(-\epsilon)}{\epsilon}
		\right) \\
%
%
&=
	\frac{1}{2}
		\left(
			\frac{\d A}{\d x}(0)
			+
			\frac{\d A}{\d x}(0)
		\right) \\
%
%
&=
	a(0)
\end{align}
従って、
\begin{align}
	a(0)
&=
	\int_{-\infty}^{\infty} \d x\
		a(x) \delta(x)
\end{align}
である。


