%ファイルID
%2020/04/18 19:53
%->2004181953(10進数)->X58KOX(ファイル作成者)
%->X58KOXSahayanKY
\subsection{問}
$|A|=0$のとき$A^{-1}$は存在しない。
$\bm{c}$に関する方程式
\begin{align}
	A\bm{c}
&=
	\bm{0}
\end{align}
が自明でない解($\bm{c}\neq\bm{0}$)をもつのは
$|A|=0$のときだけであることを示せ。

\subsection{解}
問は次のように読み替えることができる。
即ち、
自明でない解をもち、かつ
$|A|\neq 0$であることはあり得ないことを
示す。

$|A|\neq 0$であるとき、
$A$の逆行列$A^{-1}$が存在する。
従って、方程式の両辺にかけると、
\begin{align}
	A\bm{c}
&=
	\bm{0} \\
%
%
	A^{-1} A\bm{c}
&=
	A^{-1} \bm{0} \\
%
%
	\bm{c}
&=
	\bm{0}
\end{align}
となる。従って、このときは
自明解のみが存在する。
よって、自明でない解をもちながら
$|A|\neq 0$はあり得ないため、
自明でない解をもつのは
$|A|=0$のときのみである。




