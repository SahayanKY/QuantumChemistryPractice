%ファイルID
%2020/04/20 08:04
%->2004200804(10進数)->X58Z8K(ファイル作成者)
%->X58Z8KSahayanKY

\subsection{問}
固有値問題
\begin{align}
	\left[
	\begin{array}{cc}
		O_{11} & O_{12} \\
		O_{21} & O_{22}
	\end{array}
	\right]
		\left[
		\begin{array}{c}
			c_1 \\ c_2
		\end{array}
		\right]
&=
	\omega
		\left[
		\begin{array}{c}
			c_1 \\ c_2
		\end{array}
		\right]
\end{align}
では、固有ベクトルの成分の比例関係のみが求まり、
個々の成分の値(ベクトルのノルム)には任意性がある。
$c_1=1,c_2=c$と置くことで、
\begin{align}
	\left\{
	\begin{array}{rl}
			O_{11} +O_{12}c
		&=
			\omega \\
		%
		%
			O_{21} +O_{22}c
		&=
			\omega c
	\end{array}
	\right.
\end{align}
となる。この方程式から$c$を消して
得られる2次方程式の解$\omega$が、
永年方程式を解いて得られる固有値に一致することを示せ。
その固有値は次の通りである。
\begin{align}
	\omega_1
&=
	\frac{1}{2}
		\left(
			O_{11}
			+
			O_{22}
			-
			\sqrt{
				(O_{22}-O_{11})^2
				+
				4 O_{12} O_{21}
			}
		\right) \\
%
%
	\omega_2
&=
	\frac{1}{2}
		\left(
			O_{11}
			+
			O_{22}
			+
			\sqrt{
				(O_{22}-O_{11})^2
				+
				4 O_{12} O_{21}
			}
		\right)
\end{align}


\subsection{解}
\begin{align}
	O_{11}
	+
	O_{12} c
&=
	\omega \\
%
%
	c
&=
	\frac{\omega-O_{11}}{O_{12}}
\end{align}
\begin{align}
	O_{21}
	+
	O_{22} c
&=
	\omega c \\
%
%
	O_{21}
	+
	O_{22} \cdot \frac{\omega-O_{11}}{O_{12}}
&=
	\omega \cdot \frac{\omega-O_{11}}{O_{12}} \\
%
%
	O_{21} O_{12}
	+
	O_{22} \omega
	-
	O_{22} O_{11}
&=
	\omega^2
	-
	\omega O_{11} \\
%
%
	\omega^2
	-
	(O_{11}+O_{22}) \omega
	+
	O_{11} O_{22}
	-
	O_{12} O_{21}
&=
	0
\end{align}
\begin{align}
	\omega
&=
	\frac{1}{2}
		\left(
			O_{11}
			+
			O_{22}
			\pm
			\sqrt{
				(O_{11}+O_{22})^2
				-
				4 (O_{11}O_{22}-O_{12}O_{21})
			}
		\right) \\
%
%
&=
	\frac{1}{2}
		\left(
			O_{11}
			+
			O_{22}
			\pm
			\sqrt{
				(O_{22}-O_{11})^2
				+
				4 O_{12} O_{21}
			}
		\right)
\end{align}
従って、確かに永年方程式で得られた固有値と
等しい固有値が得られることが言える。





