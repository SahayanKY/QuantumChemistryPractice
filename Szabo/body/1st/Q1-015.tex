%ファイルID
%2020/04/22 22:07
%->2004222207(10進数)->X59FR3(ファイル作成者)
%->X59FR3SahayanKY
\subsection{問}
基底関数$\{\psi_i(x)\}$における
演算子$\mathscr{O}$の表現行列$O_{ij}$を考える。
つまり、
\begin{align}
	\mathscr{O} \psi_i(x)
&=
	\sum_j \psi_j(x) O_{ji}
	%
	\label{eqX59FR3SahayanKY_expressionMatrix}
\end{align}
とするときに、
\begin{align}
	O_{ji}
&=
	\int \d x\
		\conju{\psi_j}(x) \mathscr{O} \psi_i(x)
\end{align}
であることを示せ。

また、式\ref{eqX59FR3SahayanKY_expressionMatrix}を
ブラケット記法に書き換えると
\begin{align}
	\mathscr{O}\ket{i}
&=
	\sum_j
		\ket{j}\braket{j|\mathscr{O}|i}
\end{align}
となることも示せ。


\subsection{解}
$\psi_i(x)$が正規直交基底であることから、
\begin{align}
	\int \d x\
		\conju{\psi_k} \mathscr{O} \psi_i
&=
	\int \d x\
		\conju{\psi_k}
			\left(
				\sum_j
					\psi_j O_{ji}
			\right) \\
%
%
&=
	\sum_j
		\int \d x\
			\conju{\psi_k} \psi_j O_{ji} \\
%
%
&=
	\sum_j
		\delta_{kj} O_{ji} \\
%
%
&=
	O_{ki}
\end{align}
従って、
\begin{align}
	O_{ji}
&=
	\int \d x\
		\conju{\psi_j} \mathscr{O} \psi_i
\end{align}
となる。


また、この右辺はブラケット記法により、
\begin{align}
	O_{ji}
&=
	\braket{j|\mathscr{O}|i}
\end{align}
となるため、
\begin{align}
	\mathscr{O} \ket{i}
&=
	\sum_j \ket{j} O_{ji} \\
%
%
&=
	\sum_j \ket{j} \braket{j|\mathscr{O}|i}
\end{align}
である。
