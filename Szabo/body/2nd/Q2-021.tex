%ファイルID
%2020/06/08 06:01
%->2006080601(10進数)->(36進数)(ファイル作成者)
%->X6D9P5SahayanKY
\subsection{問}
最小基底関数系における\ce{H2}の完全CI行列は、
問題2.17より
\begin{align}
	H
&=
	\left[
	\begin{array}{cc}
		2\spaceone{1}{h}{1}
		+
		\spacetwo{11}{11} &
		%
		\spacetwo{12}{12} \\
	%
		\spacetwo{21}{21} &
		%
		2\spaceone{2}{h}{2}
		+
		\spacetwo{22}{22}
	\end{array}
	\right]
\end{align}
である。空間分子軌道として実関数を用いた場合、
クーロン積分$J_{ij}$と交換積分$K_{ij}$を使って、
\begin{align}
	H
&=
	\left[
	\begin{array}{cc}
		2h_{11}
		+
		J_{11} &
		%
		K_{12} \\
	%
		K_{12} &
		%
		2h_{22}
		+
		J_{22}
	\end{array}
	\right]
\end{align}
となることを示せ。


\subsection{解}
まず、1電子積分については
\begin{align}
	\spaceone{i}{h}{j}
&=
	h_{ij}
\end{align}
より
\begin{align}
	\spaceone{1}{h}{1}
&=
	h_{11} &
%
	\spaceone{2}{h}{2}
&=
	h_{22}
\end{align}
である。

次に、2電子積分について考える。
\begin{align}
	\spacetwo{ii}{jj}
&=
	J_{ij} &
%
	\spacetwo{ij}{ji}
&=
	\spacetwo{ij}{ij}
=
	\spacetwo{ji}{ji}
=
	K_{ij}
\end{align}
である。ここで後者の関係については問題2.20で導いたことを利用した。
従って、
\begin{align}
	\spacetwo{11}{11}
&=
	J_{11} &
%
	\spacetwo{22}{22}
&=
	J_{22}
\end{align}
であり、
\begin{align}
	\spacetwo{12}{12}
&=
	K_{12} &
%
	\spacetwo{21}{21}
&=
	K_{21}
=
	K_{12}
\end{align}
である。

従って、完全CI行列は
\begin{align}
	H
&=
	\left[
	\begin{array}{cc}
		2h_{11}
		+
		J_{11} &
		%
		K_{12} \\
	%
		K_{12} &
		%
		2h_{22}
		+
		J_{22}
	\end{array}
	\right]
\end{align}
となる。
