%ファイルID
%2020/04/30 02:07
%->2004300207(10進数)->X5B3XR(ファイル作成者)
%->X5B3XRSahayanKY
\subsection{問}
問題2.2同様に
ハミルトニアン$\mathscr{H}$を1電子演算子$h(i')$で
\begin{align}
	\mathscr{H}
=
	\sum_{i'=1}^{2} h(i')
=
	h(1)
	+
	h(2)
\end{align}
とする。また、スピン軌道$\chi_i,\chi_j$がそれらの固有関数であり、
\begin{align}
	h(1) \chi_i(\x_1)
&=
	\epsilon_i \chi_i(\x_1) &
%
%
	h(1) \chi_j(\x_1)
&=
	\epsilon_j \chi_j(\x_1) \\
%
%
	h(2) \chi_i(\x_2)
&=
	\epsilon_i \chi_i(\x_2) &
%
%
	h(2) \chi_j(\x_2)
&=
	\epsilon_j \chi_j(\x_2)
\end{align}
とする。

以下のHartree積とその反対称化された波動関数
\begin{align}
	\Psi^{\rm HP}_{12}(\x_1,\x_2)
&=
	\chi_i(\x_1) \chi_j(\x_2) \\
%
%
	\Psi^{\rm HP}_{21}(\x_1,\x_2)
&=
	\chi_i(\x_2) \chi_j(\x_1) \\
%
%
	\Psi(\x_1,\x_2)
&=
	\frac{1}{\sqrt{2}}
		\left(
			\Psi^{\rm HP}_{12}(\x_1,\x_2)
			-
			\Psi^{\rm HP}_{21}(\x_1,\x_2)
		\right)
\end{align}
がハミルトニアン$\mathscr{H}$の固有関数であり、
同じ固有値$\epsilon_i+\epsilon_j$を持つことを示せ。


\subsection{解}
$\Psi^{\rm HP}_{12}(\x_1,\x_2)$については
既に問題2.2にて示した通りであるが、
\begin{align}
	\mathscr{H} \Psi^{\rm HP}_{12}(\x_1,\x_2)
&=
	h(1) \chi_i(\x_1) \chi_j(\x_2)
	+
	h(2) \chi_i(\x_1) \chi_j(\x_2) \\
%
%
&=
	(h(1) \chi_i(\x_1)) \chi_j(\x_2)
	+
	\chi_i(\x_1) (h(2) \chi_j(\x_2)) \\
%
%
&=
	(\epsilon_i \chi_i(\x_1)) \chi_j(\x_2)
	+
	\chi_i(\x_1) (\epsilon_j \chi_j(\x_2)) \\
%
%
&=
	(\epsilon_i+\epsilon_j) \Psi^{\rm HP}_{12}(\x_1,\x_2)
\end{align}
であるので、固有関数であり、
固有値は$\epsilon_i+\epsilon_j$である。

次に$\Psi^{\rm HP}_{21}(\x_1,\x_2)$について見る。
\begin{align}
	\mathscr{H}\Psi^{\rm HP}_{21}(\x_1,\x_2)
&=
	h(1) \chi_i(\x_2) \chi_j(\x_1)
	+
	h(2) \chi_i(\x_2) \chi_j(\x_1) \\
%
%
&=
	\chi_i(\x_2) (h(1) \chi_j(\x_1))
	+
	(h(2) \chi_i(\x_2)) \chi_j(\x_1) \\
%
%
&=
	\chi_i(\x_2) (\epsilon_j \chi_j(\x_1))
	+
	(\epsilon_i \chi_i(\x_2)) \chi_j(\x_1) \\
%
%
&=
	(\epsilon_i+\epsilon_j) \Psi^{\rm HP}_{21}(\x_1,\x_2)
\end{align}
であるので、同様である。

最後に$\Psi(\x_1,\x_2)$について見る。
\begin{align}
	\mathscr{H} \Psi(\x_1,\x_2)
&=
	\mathscr{H}\left\{
		\frac{1}{\sqrt{2}}
			\left(
				\Psi^{\rm HP}_{12}(\x_1,\x_2)
				-
				\Psi^{\rm HP}_{21}(\x_1,\x_2)
			\right)
	\right\} \\
%
%
&=
	\frac{1}{\sqrt{2}}
		\left(
			\mathscr{H} \Psi^{\rm HP}_{12}(\x_1,\x_2)
			-
			\mathscr{H} \Psi^{\rm HP}_{21}(\x_1,\x_2)
		\right) \\
%
%
&=
	\frac{1}{\sqrt{2}}
		\left(
			(\epsilon_i+\epsilon_j) \Psi^{\rm HP}_{12}(\x_1,\x_2)
			-
			(\epsilon_i+\epsilon_j) \Psi^{\rm HP}_{21}(\x_1,\x_2)
		\right) \\
%
%
&=
	(\epsilon_i+\epsilon_j) \Psi(\x_1,\x_2)
\end{align}
であるので、同様である。


