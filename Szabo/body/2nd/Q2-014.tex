%ファイルID
%2020/05/21 16:24
%->2005211624(10進数)->X5UN6W(ファイル作成者)
%->X5UN6WSahayanKY
\subsection{問}
$N$電子系のHartree-Fock基底状態$\ket{\Psi_0}$を
\begin{align}
	\ket{\Psi_0}
&=
	\ket{\chi_1 \cdots \chi_{a-1} \chi_a \chi_{a+1} \cdots \chi_N}
\end{align}
とする。また、$\chi_a$から1つ電子が除かれたイオン化状態$\ket{^{N-1}\Psi_a}$を
\begin{align}
	\ket{^{N-1}\Psi_a}
&=
	\ket{\chi_1 \cdots \chi_{a-1} \chi_{a+1} \cdots \chi_N}
\end{align}
とする。それぞれの状態のエネルギーを
\begin{align}
	^{N}E_0
&=
	\braket{^{N}\Psi_0|\mathscr{H}|^{N}\Psi_0} &
%
%
	^{N-1}E_a
&=
	\braket{^{N-1}\Psi_a|\mathscr{H}|^{N-1}\Psi_a}
\end{align}
とする。
	%
	\footnote{
		厳密には後者のハミルトニアン$\mathscr{H}$は
		$^{N-1}\mathscr{H}$などと区別されるべきだと思う。
	}
	%

このとき、イオン化過程に必要なエネルギーが
\begin{align}
	^{N}E_0
	-
	^{N-1}E_a
&=
	\braket{a|h|a}
	+
	\sum_b^N \antitwo{ab}{ab}
\end{align}
であることを示せ。



