%ファイルID
%2020/06/10 12:55
%->2006101255SahayanKY(ファイル作成者)
\subsection{問}
$\ket{K}=\ket{\chi_1\chi_2\dots\chi_N}=\adj{a_1}\adj{a_2}\dots\adj{a_N}\ket{}$とする。
以下の等式を示せ。
\begin{align}
	\braket{K|\adj{a_i}a_j|K}
&=
	\left\{
	\begin{array}{ll}
		1 &
		%
		(i=j,\ i\in\{1,2,\dots,N\}) \\
	%
		0 &
		%
		(\text{otherwise})
	\end{array}
	\right.
\end{align}


\subsection{解}
$j\notin\{1,2,\dots,N\}$のとき、
$a_j\ket{K}=0$である。
また、$i\notin\{1,2,\dots,N\}$のとき、
$\bra{K}\adj{a_i}=0$である。
さらに、$i\in\{1,2,\dots,N\}\setminus\{j\}$のとき、
$\adj{a_i}a_j\ket{K}=0$である。
従って、$\braket{K|\adj{a_i}a_j|K}$が非ゼロであるためには
$i\in\{1,2,\dots,N\}$かつ$j\in\{1,2,\dots,N\}$かつ、
$i\notin\{1,2,\dots,N\}\setminus\{j\}$である必要がある。
1つ目と3つ目の条件を組合わせると$i=j$と同値であるので、
この必要条件は$i=j$かつ$i\in\{1,2,\dots,N\}$に読み替えることができる。

逆に、この必要条件を満たすとき、
\begin{align}
	\braket{K|\adj{a_i}a_i|K}
&=
	\braket{K|(1-a_i\adj{a_i})|K} \\
%
%
&=
	\braket{K|K}
	-
	\braket{K|a_i\adj{a_i}|K} \\
%
%
&=
	1
	-
	\bra{K}a_i 0 \\
%
%
&=
	1
\end{align}
となることから、これは必要十分条件である。
従って、上記の通りとなる。


