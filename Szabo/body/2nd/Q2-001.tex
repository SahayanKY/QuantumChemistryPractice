%ファイルID
%2020/04/29 08:30
%->2004290830(10進数)->X5AWPA(ファイル作成者)
%->X5AWPASahayanKY
\subsection{問}
$K$個の規格直交空間関数$\{\psi_i^\alpha(\r)\}$と、
もう一つの$K$個の規格直交空間関数$\{\psi_i^\beta(\r)\}$を考える。
これらは互いに直交しておらず、
\begin{align}
	\int \d\r\
		\psi_i^\alpha(\r) \psi_j^\beta(\r)
=
	S_{ij}
\neq
	0
\end{align}
であるとする。

$\{\psi_i^\alpha(\r)\}$に$\alpha$スピン関数を、
$\{\psi_i^\beta(\r)\}$に$\beta$スピン関数をかけて得られる
$2K$個のスピン軌道$\chi_i(\x)$
\begin{align}
	\begin{array}{rl}
			\chi_{2i-1}(\x)
		&=
			\psi_i^\alpha(\r) \alpha(\omega) \\
		%
			\chi_{2i}(\x)
		&=
			\psi_i^\beta(\r) \beta(\omega)
	\end{array}
	\qquad
	(i=1,2,\cdots,K)
\end{align}
が規格直交系であることを示せ。


\subsection{解}
$\chi_i(\x)$が規格直交系であることを示すためには
次の内積を示せばよい。
\begin{align}
	\braket{\chi_{2i-1}|\chi_{2j-1}}
&=
	\delta_{ij} &
%
	\braket{\chi_{2i}|\chi_{2j}}
&=
	\delta_{ij} &
%
	\braket{\chi_{2i-1}|\chi_{2j}}
&=
	0
\end{align}

1つ目の関係については
\begin{align}
	\braket{\chi_{2i-1}|\chi_{2j-1}}
&=
	\int \d\r \int \d\omega\
		\conju{\psi_i^\alpha}(\r) \conju{\alpha}(\omega)
		\psi_j^\alpha(\r) \alpha(\omega) \\
%
%
&=
	\int \d\r\
		\conju{\psi_i^\alpha}(\r) \psi_j^\alpha(\r)
	\int \d\omega\
		\conju{\alpha}(\omega) \alpha(\omega) \\
%
%
&=
	\braket{\psi_i^\alpha|\psi_j^\alpha}
		\braket{\alpha|\alpha} \\
%
%
&=
	\delta_{ij}
\end{align}
である。

2つ目の関係については、
\begin{align}
	\braket{\chi_{2i}|\chi_{2j}}
&=
	\int \d\r \int \d\omega\
		\conju{\psi_i^\beta}(\r) \conju{\beta}(\omega)
		\psi_j^\beta(\r) \beta(\omega) \\
%
%
&=
	\int \d\r\
		\conju{\psi_i^\beta}(\r) \psi_j^\beta(\r)
	\int \d\omega\
		\conju{\beta}(\omega) \beta(\omega) \\
%
%
&=
	\braket{\psi_i^\beta|\psi_j^\beta}
		\braket{\beta|\beta} \\
%
%
&=
	\delta_{ij}
\end{align}
である。

3つ目の関係については
\begin{align}
	\braket{\chi_{2i-1}|\chi_{2j-1}}
&=
	\int \d\r \int \d\omega\
		\conju{\psi_i^\alpha}(\r) \conju{\alpha}(\omega)
		\psi_j^\beta(\r) \beta(\omega) \\
%
%
&=
	\int \d\r\
		\conju{\psi_i^\alpha}(\r) \psi_j^\beta(\r)
	\int \d\omega\
		\conju{\alpha}(\omega) \beta(\omega) \\
%
%
&=
	\braket{\psi_i^\alpha|\psi_j^\beta}
		\braket{\alpha|\beta} \\
%
%
&=
	0
\end{align}
である。

よって、確かに$\chi_i$は規格直交系である。


