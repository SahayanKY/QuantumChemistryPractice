%ファイルID
%2020/05/05 10:36
%->2005051036(10進数)->X5R7A4(ファイル作成者)
%->X5R7A4SahayanKY
\subsection{問}
最小基底での\ce{H2}モデルの完全CI行列
($\ket{\Psi_0}$と$\ket{\Psi_{12}^{34}}$のハミルトニアン行列)が
\begin{align}
	H
&=
	\left[
	\begin{array}{cc}
		\braket{1|h|1}
		+
		\braket{2|h|2}
		+
		\braket{12|12}
		-
		\braket{12|21} &
		%
		\braket{12|34}
		-
		\braket{12|43} \\
	%
		\braket{34|12}
		-
		\braket{34|21} &
		%
		\braket{3|h|3}
		+
		\braket{4|h|4}
		+
		\braket{34|34}
		-
		\braket{34|43}
	\end{array}
	\right]
\end{align}
であることを示せ。また、これがエルミート行列であることも示せ。

ここで、$\braket{i|h|j}$と$\braket{ij|kl}$は
\begin{align}
	\braket{i|h|j}
&=
	\int \d\x_1
		\conju{\chi_i}(\x_1) h(\r_1) \chi_j(\x_1) \\
%
%
	\braket{ij|kl}
&=
	\int \d\x_1 \int \d\x_2
		\conju{\chi_i}(\x_1) \conju{\chi_j}(\x_2)
		r_{12}^{-1}
		\chi_k(\x_1) \chi_l(\x_2)
\end{align}
である。


\subsection{解}
ハミルトニアン行列$H$は
\begin{align}
	H
&=
	\left[
	\begin{array}{cc}
		\braket{12|\mathscr{H}|12} &
		%
		\braket{12|\mathscr{H}|34} \\
	%
		\braket{34|\mathscr{H}|12} &
		%
		\braket{34|\mathscr{H}|34}
	\end{array}
	\right]
\end{align}
である。$\braket{12|\mathscr{H}|12}$については、
既に本文中に解説があった通りに、
\begin{align}
	\braket{12|\mathscr{H}|12}
&=
	\braket{1|h|1}
	+
	\braket{2|h|2}
	+
	\braket{12|12}
	-
	\braket{12|21}
\end{align}
である。

$\braket{12|\mathscr{H}|34}$は
\begin{align}
	\braket{12|\mathscr{H}|34}
&=
	\braket{12|\mathscr{O}_1|34}
	+
	\braket{12|\mathscr{O}_2|34} \\
%
%
&=
	\braket{12|\mathscr{O}_2|34}
	%
	\qquad
	(\because \braket{12|\mathscr{O}_1|34}=0)
\end{align}
であり、
\begin{align}
	\braket{12|\mathscr{O}_2|34}
&=
	\int \d\x_1 \int \d\x_2
		\conju[1]{
			\frac{1}{\sqrt{2}}
			\left(
				\chi_1(\x_1) \chi_2(\x_2)
				-
				\chi_2(\x_1) \chi_1(\x_2)
			\right)
		}
		r_{12}^{-1}
		\frac{1}{\sqrt{2}}
		\left(
			\chi_3(\x_1) \chi_4(\x_2)
			-
			\chi_4(\x_1) \chi_3(\x_2)
		\right) \\
%
%
&=
	\frac{1}{2}
	\biggl(
		\braket{12|34}
		-
		\braket{12|43}
		-
		\braket{21|34}
		+
		\braket{21|43}
	\biggr) \\
%
%
&=
	\frac{1}{2}
	\biggl(
		\braket{12|34}
		-
		\braket{12|43}
		-
		\braket{12|43}
		+
		\braket{12|34}
	\biggr)
	%
	\qquad
	(\braket{ij|kl}=\braket{ji|lk} ) \\
%
%
&=
	\braket{12|34}
	-
	\braket{12|43}
\end{align}
従って、
\begin{align}
	\braket{12|\mathscr{H}|34}
&=
	\braket{12|34}
	-
	\braket{12|43}
\end{align}
である。

次に$\braket{34|\mathscr{H}|12}$について見る。同様に、
\begin{align}
	\braket{34|\mathscr{H}|12}
&=
	\braket{34|\mathscr{O}_1|12}
	+
	\braket{34|\mathscr{O}_2|12} \\
%
%
&=
	\braket{34|\mathscr{O}_2|12} \\
%
%
&=
	\int \d\x_1 \int \d\x_2
		\conju[1]{
			\frac{1}{\sqrt{2}}
			\left(
				\chi_3(\x_1) \chi_4(\x_2)
				-
				\chi_4(\x_1) \chi_3(\x_2)
			\right)
		}
		r_{12}^{-1}
		\frac{1}{\sqrt{2}}
		\left(
			\chi_1(\x_1) \chi_2(\x_2)
			-
			\chi_2(\x_1) \chi_1(\x_2)
		\right) \\
%
%
&=
	\frac{1}{2}
	\biggl(
		\braket{34|12}
		-
		\braket{34|21}
		-
		\braket{43|12}
		+
		\braket{43|21}
	\biggr) \\
%
%
&=
	\frac{1}{2}
	\biggl(
		\braket{34|12}
		-
		\braket{34|21}
		-
		\braket{34|21}
		+
		\braket{34|12}
	\biggr) \\
%
%
&=
	\braket{34|12}
	-
	\braket{34|21}
\end{align}
である。

最後に$\braket{34|\mathscr{H}|34}$について見る。
\begin{align}
	\braket{34|\mathscr{H}|34}
&=
	\braket{34|\mathscr{O}_1|34}
	+
	\braket{34|\mathscr{O}_2|34}
\end{align}
\begin{align}
&\quad
	\braket{34|\mathscr{O}_1|34} \nonumber\\
&=
	\int \d\x_1 \int \d\x_2
		\conju[1]{
			\frac{1}{\sqrt{2}}
			\left(
				\chi_3(\x_1) \chi_4(\x_2)
				-
				\chi_4(\x_1) \chi_3(\x_2)
			\right)
		}
		(h(\r_1) +h(\r_2))
		\frac{1}{\sqrt{2}}
		\left(
			\chi_3(\x_1) \chi_4(\x_2)
			-
			\chi_4(\x_1) \chi_3(\x_2)
		\right) \\
%
%
&=
	\frac{1}{2}
	\left(
	\begin{array}{l}
		\braket{3|h|3} \braket{4|4}
		-
		\braket{3|h|4} \braket{4|3}
		+
		\braket{3|3} \braket{4|h|4}
		-
		\braket{3|4} \braket{4|h|3} \\
		-
		\braket{4|h|3} \braket{3|4}
		+
		\braket{4|h|4} \braket{3|3}
		-
		\braket{4|3} \braket{3|h|4}
		+
		\braket{4|4} \braket{3|h|3}
	\end{array}
	\right) \\
%
%
&=
	\braket{3|h|3}
	+
	\braket{4|h|4}
\end{align}
\begin{align}
	\braket{34|\mathscr{O}_2|34}
&=
	\int \d\x_1 \int \d\x_2
		\conju[1]{
			\frac{1}{\sqrt{2}}
			\left(
				\chi_3(\x_1) \chi_4(\x_2)
				-
				\chi_4(\x_1) \chi_3(\x_2)
			\right)
		}
		r_{12}^{-1}
		\frac{1}{\sqrt{2}}
		\left(
			\chi_3(\x_1) \chi_4(\x_2)
			-
			\chi_4(\x_1) \chi_3(\x_2)
		\right) \\
%
%
&=
	\frac{1}{2}
	\biggl(
		\braket{34|34}
		-
		\braket{34|43}
		-
		\braket{43|34}
		+
		\braket{43|43}
	\biggr) \\
%
%
&=
	\frac{1}{2}
	\biggl(
		\braket{34|34}
		-
		\braket{34|43}
		-
		\braket{34|43}
		+
		\braket{34|34}
	\biggr) \\
%
%
&=
	\braket{34|34}
	-
	\braket{34|43}
\end{align}
従って、
\begin{align}
	\braket{34|\mathscr{H}|34}
&=
	\braket{3|h|3}
	+
	\braket{4|h|4}
	+
	\braket{34|34}
	-
	\braket{34|43}
\end{align}
である。

よってこれらから、ハミルトニアン行列は
\begin{align}
	H
&=
	\left[
	\begin{array}{cc}
		\braket{1|h|1}
		+
		\braket{2|h|2}
		+
		\braket{12|12}
		-
		\braket{12|21} &
		%
		\braket{12|34}
		-
		\braket{12|43} \\
	%
		\braket{34|12}
		-
		\braket{34|21} &
		%
		\braket{3|h|3}
		+
		\braket{4|h|4}
		+
		\braket{34|34}
		-
		\braket{34|43}
	\end{array}
	\right]
\end{align}
である。

また、$H$がエルミート行列であること、すなわち$H=\adj{H}$であることは
\begin{align}
	\conju{H_{11}}
&=
	\conju{\braket{1|h|1}}
	+
	\conju{\braket{2|h|2}}
	+
	\conju{\braket{12|12}}
	-
	\conju{\braket{12|21}} \\
%
%
&=
	\braket{1|h|1}
	+
	\braket{2|h|2}
	+
	\braket{12|12}
	-
	\braket{21|12}
	%
	\qquad
	(
		\conju{\braket{i|h|i}}=\braket{i|h|i},
		\conju{\braket{ij|kl}}=\braket{kl|ij}
	) \\
%
%
&=
	\braket{1|h|1}
	+
	\braket{2|h|2}
	+
	\braket{12|12}
	-
	\braket{12|21} \\
%
%
&=
	H_{11}
\end{align}
\begin{align}
	\conju{H_{12}}
&=
	\conju{\braket{12|34}}
	-
	\conju{\braket{12|43}} \\
%
%
&=
	\braket{34|12}
	-
	\braket{43|12} \\
%
%
&=
	\braket{34|12}
	-
	\braket{34|21} \\
%
%
&=
	H_{21}
\end{align}
\begin{align}
	\conju{H_{22}}
&=
	\conju{\braket{3|h|3}}
	+
	\conju{\braket{4|h|4}}
	+
	\conju{\braket{34|34}}
	-
	\conju{\braket{34|43}} \\
%
%
&=
	\braket{3|h|3}
	+
	\braket{4|h|4}
	+
	\braket{34|34}
	-
	\braket{43|34} \\
%
%
&=
	\braket{3|h|3}
	+
	\braket{4|h|4}
	+
	\braket{34|34}
	-
	\braket{34|43} \\
%
%
&=
	H_{22}
\end{align}
であることから、$\conju{H_{ij}}=H_{ji}$となり、
明らかである。