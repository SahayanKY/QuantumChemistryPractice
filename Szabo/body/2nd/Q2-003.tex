%ファイルID
%2020/04/30 00:11
%->2004300011(10進数)->X5B3SB(ファイル作成者)
%->X5B3SBSahayanKY
\subsection{問}
規格直交するスピン軌道$\chi_i(\x)$を用いて得られる
次の波動関数$\Psi(\x_1,\x_2)$を考える。
\begin{align}
	\Psi(\x_1,\x_2)
&=
	\frac{1}{\sqrt{2}}
		\left(
			\chi_i(\x_1) \chi_j(\x_2)
			-
			\chi_j(\x_1) \chi_i(\x_2)
		\right)
\end{align}
これが規格化されていることを示せ。

\subsection{解}
\begin{align}
	\int \d\x_1 \int \d\x_2
		\conju{\Psi}\Psi
&=
	\frac{1}{2}
		\int \d\x_1 \int \d\x_2
			\left(
				\conju{\chi_i}(\x_1) \conju{\chi_j}(\x_2)
				-
				\conju{\chi_j}(\x_1) \conju{\chi_i}(\x_2)
			\right)
				\left(
					\chi_i(\x_1) \chi_j(\x_2)
					-
					\chi_j(\x_1) \chi_i(\x_2)
				\right) \\
%
%
&=
	\frac{1}{2}
		\left\{
		\begin{array}{>{\displaystyle}l}
			\int \d\x_1 \int \d\x_2
				\conju{\chi_i}(\x_1) \conju{\chi_j}(\x_2) \chi_i(\x_1) \chi_j(\x_2) \\[3mm]
			-
			\int \d\x_1 \int \d\x_2
				\conju{\chi_i}(\x_1) \conju{\chi_j}(\x_2) \chi_j(\x_1) \chi_i(\x_2) \\[3mm]
			-
			\int \d\x_1 \int \d\x_2
				\conju{\chi_j}(\x_1) \conju{\chi_i}(\x_2) \chi_i(\x_1) \chi_j(\x_2) \\[3mm]
			+
			\int \d\x_1 \int \d\x_2
				\conju{\chi_j}(\x_1) \conju{\chi_i}(\x_2) \chi_j(\x_1) \chi_i(\x_2)
		\end{array}
		\right\} \\
%
%
&=
	\frac{1}{2}
		\left\{
			\delta_{ii} \delta_{jj}
			-
			\delta_{ij} \delta_{ji}
			-
			\delta_{ji} \delta_{ij}
			+
			\delta_{jj} \delta_{ii}
		\right\} \\
%
%
&=
	\frac{1}{2}
		\left\{
			1
			-0
			-0
			+1
		\right\} \\
%
%
&=
	1
\end{align}
従って、規格化されているといえる。
