%ファイルID
%2020/06/15 13:07
%->2006151307SahayanKY(ファイル作成者)
\subsection{問}
基底$\ket{\alpha},\ket{\beta}$における
$s^2,s_z,s_+,s_-$の表現行列を求めよ。
また、それら行列に対して、
問2.32(b)で求めた関係式が成立することを示せ。

\subsection{解}
$s^2,s_z,s_+,s_-$の表現行列をそれぞれ
$A_{s^2}, A_{s_z}, A_{s_-}, A_{s_+}$とする。

まず$s^2$の表現行列$A_{s^2}$は
\begin{align}
	A_{s^2}
&=
	\left[
	\begin{array}{cc}
		\braket{\alpha|s^2|\alpha} &
		%
		\braket{\alpha|s^2|\beta} \\
	%
		\braket{\beta|s^2|\alpha} &
		%
		\braket{\beta|s^2|\beta}
	\end{array}
	\right] \\
%
%
&=
	\left[
	\begin{array}{cc}
		\frac{3}{4}\braket{\alpha|\alpha} &
		%
		\frac{3}{4}\braket{\alpha|\beta} \\[2mm]
	%
		\frac{3}{4}\braket{\beta|\alpha} &
		%
		\frac{3}{4}\braket{\beta|\beta}
	\end{array}
	\right] \\
%
%
&=
	\frac{3}{4}
	\left[
	\begin{array}{cc}
		1 & 0 \\
		0 & 1
	\end{array}
	\right]
\end{align}
である。


次に、$s_z$の表現行列$A_{s_z}$は
\begin{align}
	A_{s_z}
&=
	\left[
	\begin{array}{cc}
		\braket{\alpha|s_z|\alpha} &
		%
		\braket{\alpha|s_z|\beta} \\
	%
		\braket{\beta|s_z|\alpha} &
		%
		\braket{\beta|s_z|\beta}
	\end{array}
	\right] \\
%
%
&=
	\left[
	\begin{array}{cc}
		\frac{1}{2} \braket{\alpha|\alpha} &
		%
		-\frac{1}{2} \braket{\alpha|\beta} \\[2mm]
	%
		\frac{1}{2} \braket{\beta|\alpha} &
		%
		-\frac{1}{2} \braket{\beta|\beta}
	\end{array}
	\right] \\
%
%
&=
	\frac{1}{2}
	\left[
	\begin{array}{cc}
		1 & 0 \\
		0 & -1
	\end{array}
	\right]
\end{align}
である。


次に$s_+$の表現行列$A_{s_+}$は
\begin{align}
	A_{s_+}
&=
	\left[
	\begin{array}{cc}
		\braket{\alpha|s_+|\alpha} &
		%
		\braket{\alpha|s_+|\beta} \\
	%
		\braket{\beta|s_+|\alpha} &
		%
		\braket{\beta|s_+|\beta}
	\end{array}
	\right] \\
%
%
&=
	\left[
	\begin{array}{cc}
		\braket{\alpha|0} &
		%
		\braket{\alpha|\alpha} \\
	%
		\braket{\beta|0} &
		%
		\braket{\beta|\alpha}
	\end{array}
	\right] \\
%
%
&=
	\left[
	\begin{array}{cc}
		0 & 1 \\
		0 & 0
	\end{array}
	\right]
\end{align}
である。

最後に$s_-$の表現行列$A_{s_-}$は
\begin{align}
	A_{s_-}
&=
	\left[
	\begin{array}{cc}
		\braket{\alpha|s_-|\alpha} &
		%
		\braket{\alpha|s_-|\beta} \\
	%
		\braket{\beta|s_-|\alpha} &
		%
		\braket{\beta|s_-|\beta}
	\end{array}
	\right] \\
%
%
&=
	\left[
	\begin{array}{cc}
		\braket{\alpha|\beta} &
		%
		\braket{\alpha|0} \\
	%
		\braket{\beta|\beta} &
		%
		\braket{\beta|0}
	\end{array}
	\right] \\
%
%
&=
	\left[
	\begin{array}{cc}
		0 & 0 \\
		1 & 0
	\end{array}
	\right]
\end{align}
である。

これら表現行列の間に、$s^2,s_z,s_+,s_-$の関係が成立するか調べる。
まず、1つ目の
\begin{align}
	s^2
&=
	s_+ s_-
	-
	s_z
	+
	s_z^2
\end{align}
について見る。
\begin{align}
&\quad
	A_{s_+} A_{s_-}
	-
	A_{s_z}
	+
	A_{s_z}^2 \\
&=
	\left[
	\begin{array}{cc}
		0 & 1 \\
		0 & 0
	\end{array}
	\right]
	\left[
	\begin{array}{cc}
		0 & 0 \\
		1 & 0
	\end{array}
	\right]
	-
	\frac{1}{2}
	\left[
	\begin{array}{cc}
		1 & 0 \\
		0 & -1
	\end{array}
	\right]
	+
	\left(
		\frac{1}{2}
		\left[
		\begin{array}{cc}
			1 & 0 \\
			0 & -1
		\end{array}
		\right]
	\right)
	\left(
		\frac{1}{2}
		\left[
		\begin{array}{cc}
			1 & 0 \\
			0 & -1
		\end{array}
		\right]
	\right) \\
%
%
&=
	\left[
	\begin{array}{cc}
		1 & 0 \\
		0 & 0
	\end{array}
	\right]
	-
	\frac{1}{2}
	\left[
	\begin{array}{cc}
		1 & 0 \\
		0 & -1
	\end{array}
	\right]
	+
	\frac{1}{4}
	\left[
	\begin{array}{cc}
		1 & 0 \\
		0 & 1
	\end{array}
	\right] \\
%
%
&=
	\left[
	\begin{array}{cc}
		\frac{3}{4} & 0 \\[2mm]
		0 & \frac{3}{4}
	\end{array}
	\right] \\
%
%
&=
	A_{s^2}
\end{align}
である。従って、確かに同様の式が成立する。

次に2つ目の
\begin{align}
	s^2
&=
	s_- s_+
	+
	s_z
	+
	s_z^2
\end{align}
について見る。
\begin{align}
&\quad
	A_{s_-} A_{s_+}
	+
	A_{s_z}
	+
	A_{s_z}^2 \\
&=
	\left[
	\begin{array}{cc}
		0 & 0 \\
		1 & 0
	\end{array}
	\right]
	\left[
	\begin{array}{cc}
		0 & 1 \\
		0 & 0
	\end{array}
	\right]
	+
	\frac{1}{2}
	\left[
	\begin{array}{cc}
		1 & 0 \\
		0 & -1
	\end{array}
	\right]
	+
	\left(
		\frac{1}{2}
		\left[
		\begin{array}{cc}
			1 & 0 \\
			0 & -1
		\end{array}
		\right]
	\right)
	\left(
		\frac{1}{2}
		\left[
		\begin{array}{cc}
			1 & 0 \\
			0 & -1
		\end{array}
		\right]
	\right) \\
%
%
&=
	\left[
	\begin{array}{cc}
		0 & 0 \\
		0 & 1
	\end{array}
	\right]
	+
	\frac{1}{2}
	\left[
	\begin{array}{cc}
		1 & 0 \\
		0 & -1
	\end{array}
	\right]
	+
	\frac{1}{4}
	\left[
	\begin{array}{cc}
		1 & 0 \\
		0 & 1
	\end{array}
	\right] \\
%
%
&=
	\left[
	\begin{array}{cc}
		\frac{3}{4} & 0 \\[2mm]
		0 & \frac{3}{4}
	\end{array}
	\right] \\
%
%
&=
	A_{s^2}
\end{align}
である。従って、確かに同様の式が成立する。





