%ファイルID
%2020/04/30 19:18
%->2004301918(10進数)->X5B59A(ファイル作成者)
%->X5B59ASahayanKY
\subsection{問}
規格化された原子軌道$\phi_1,\phi_2$から
線形結合によって分子軌道$\psi_1,\psi_2$をつくる。
\begin{align}
	\psi_1
&=
	\frac{1}{\sqrt{2(1+S_{12})}}
		(\phi_1+\phi_2) &
%
%
	\psi_2
&=
	\frac{1}{\sqrt{2(1-S_{12})}}
		(\phi_1-\phi_2)
\end{align}
ここで、$S_{12}$は
\begin{align}
	S_{12}
&=
	\int \d\r\
		\conju{\phi_1}(\r) \phi_2(\r)
\end{align}
である。また、$\phi_1,\phi_2$は互いに直交しておらず、等しくもない。
このとき、$\psi_1,\psi_2$が規格直交系であることを示せ。


\subsection{解}
$\phi_1,\phi_2$は実関数であるとする。
即ち、$S_{12}$は実数であるとする。
\begin{align}
	\braket{\psi_1|\psi_1}
&=
	\int \d\r\
		\conju{\psi_1} \psi_1 \\
%
%
&=
	\frac{1}{2(1+S_{12})}
		\int \d\r\
			(\conju{\phi_1}+\conju{\phi_2}) (\phi_1+\phi_2) \\
%
%
&=
	\frac{1}{2(1+S_{12})}
		\left(
			1 +2S_{12} +1
		\right) \\
%
%
&=
	1
\end{align}
より、$\psi_1$は規格化されている。

\begin{align}
	\braket{\psi_2|\psi_2}
&=
	\int \d\r\
		\conju{\psi_2} \psi_2 \\
%
%
&=
	\frac{1}{2(1-S_{12})}
		\int \d\r\
			(\conju{\phi_1}-\conju{\phi}_2)(\phi_1-\phi_2) \\
%
%
&=
	\frac{1}{2(1-S_{12})}
		\left(
			1 -2S_{12} +1
		\right) \\
%
%
&=
	1
\end{align}
より、$\psi_2$は規格化されている。

\begin{align}
	\braket{\psi_1|\psi_2}
&=
	\int \d\r\
		\conju{\psi_1} \psi_2 \\
%
%
&=
	\frac{1}{\sqrt{4(1-S_{12}^2)}}
		\int \d\r\
			(\conju{\phi}_1+\conju{\phi}_2) (\phi_1-\phi_2) \\
%
%
&=
	\frac{1}{2\sqrt{1-S_{12}^2}}
		\left(
			1 -S_{12} +S_{12} -1
		\right) \\
%
%
&=
	0
\end{align}
より、$\psi_1$と$\psi_2$は直交している。
