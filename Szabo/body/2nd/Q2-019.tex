%ファイルID
%2020/06/07 14:19
%->2006071419(10進数)->(36進数)(ファイル作成者)
%->X6D2M3SahayanKY
\subsection{問}
クーロン積分$J_{ij}$と交換積分$K_{ij}$の間に成立する以下の性質を示せ。
\begin{align}
	J_{ii}
&=
	K_{ii}
	%
	\label{eqX6D2M3SahayanKY_1} \\
%
%
	\conju{J_{ij}}
&=
	J_{ij}
	%
	\label{eqX6D2M3SahayanKY_2} \\
%
%
	\conju{K_{ij}}
&=
	K_{ij}
	%
	\label{eqX6D2M3SahayanKY_3} \\
%
%
	J_{ij}
&=
	J_{ji}
	%
	\label{eqX6D2M3SahayanKY_4} \\
%
%
	K_{ij}
&=
	K_{ji}
	%
	\label{eqX6D2M3SahayanKY_5}
\end{align}

ただし、クーロン積分と交換積分はそれぞれ次の通りである。
\begin{align}
	J_{ij}
&=
	\spacetwo{ii}{jj}
=
	\phystwo{ij}{ij} \\
%
%
	K_{ij}
&=
	\spacetwo{ij}{ji}
=
	\phystwo{ij}{ji}
\end{align}

\subsection{解}
まず式\ref{eqX6D2M3SahayanKY_1}から見ていく。
\begin{align}
	J_{ii}
&=
	\spacetwo{ii}{ii} &
%
	K_{ii}
&=
	\spacetwo{ii}{ii}
\end{align}
であるので、
\begin{align}
	J_{ii}
&=
	K_{ii}
\end{align}
である。

次に式\ref{eqX6D2M3SahayanKY_2}について見ていく。
\begin{align}
	\conju{J_{ij}}
&=
	\conju{\spacetwo{ii}{jj}} \\
%
%
&=
	\int\d\r_1 \d\r_2
		\psi_i(\r_1) \conju{\psi_i}(\r_1)
		r_{12}^{-1}
		\psi_j(\r_2) \conju{\psi_j}(\r_2) \\
%
%
&=
	\int\d\r_1 \d\r_2
		\conju{\psi_i}(\r_1) \psi_i(\r_1)
		r_{12}^{-1}
		\conju{\psi_j}(\r_2) \psi_j(\r_2) \\
%
%
&=
	\spacetwo{ii}{jj} \\
%
%
&=
	J_{ij}
\end{align}
である。

次に式\ref{eqX6D2M3SahayanKY_3}について見ていく。
\begin{align}
	\conju{K_{ij}}
&=
	\conju{\spacetwo{ij}{ji}} \\
%
%
&=
	\int\d\r_1 \d\r_2
		\psi_i(\r_1) \conju{\psi_j}(\r_1)
		r_{12}^{-1}
		\psi_j(\r_2) \conju{\psi_i}(\r_2) \\
%
%
&=
	\int\d\r_1 \d\r_2
		\conju{\psi_i}(\r_2) \psi_j(\r_2)
		r_{12}^{-1}
		\conju{\psi_j}(\r_1) \psi_i(\r_1) \\
%
%
&=
	\int\d\r_2 \d\r_1
		\conju{\psi_i}(\r_1) \psi_j(\r_1)
		r_{21}^{-1}
		\conju{\psi_j}(\r_2) \psi_i(\r_2) \\
%
%
&=
	\spacetwo{ij}{ji} \\
%
%
&=
	K_{ij}
\end{align}
である。

次に、式\ref{eqX6D2M3SahayanKY_4}について見ていく。
\begin{align}
	J_{ij}
&=
	\spacetwo{ii}{jj} \\
%
%
&=
	\int\d\r_1 \d\r_2
		\conju{\psi_i}(\r_1) \psi_i(\r_1)
		r_{12}^{-1}
		\conju{\psi_j}(\r_2) \psi_j(\r_2) \\
%
%
&=
	\int\d\r_1 \d\r_2
		\conju{\psi_j}(\r_2) \psi_j(\r_2)
		r_{12}^{-1}
		\conju{\psi_i}(\r_1) \psi_i(\r_1) \\
%
%
&=
	\int\d\r_2 \d\r_1
		\conju{\psi_j}(\r_1) \psi_j(\r_1)
		r_{21}^{-1}
		\conju{\psi_i}(\r_2) \psi_i(\r_2) \\
%
%
&=
	\spacetwo{jj}{ii} \\
%
%
&=
	J_{ji}
\end{align}
である。

最後に、式\ref{eqX6D2M3SahayanKY_5}について見ていく。
\begin{align}
	K_{ij}
&=
	\spacetwo{ij}{ji} \\
%
%
&=
	\int\d\r_1 \d\r_2
		\conju{\psi_i}(\r_1) \psi_j(\r_1)
		r_{12}^{-1}
		\conju{\psi_j}(\r_2) \psi_i(\r_2) \\
%
%
&=
	\int\d\r_1 \d\r_2
		\conju{\psi_j}(\r_2) \psi_i(\r_2)
		r_{12}^{-1}
		\conju{\psi_i}(\r_1) \psi_j(\r_1) \\
%
%
&=
	\int\d\r_2 \d\r_1
		\conju{\psi_j}(\r_1) \psi_i(\r_1)
		r_{21}^{-1}
		\conju{\psi_i}(\r_2) \psi_j(\r_2) \\
%
%
&=
	\spacetwo{ji}{ij} \\
%
%
&=
	K_{ji}
\end{align}
である。




