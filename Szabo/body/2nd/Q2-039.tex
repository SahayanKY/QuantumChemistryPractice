%ファイルID
%2020/06/23 11:01
%->2006231101SahayanKY(ファイル作成者)
\subsection{問}
$\mathscr{S}^2=\mathscr{S}_-\mathscr{S}_+ +\mathscr{S}_z +\mathscr{S}_z^2$を用いて
$\ket{^1\Psi_1^2}$が一重項で、
$\ket{^3\Psi_1^2},\ket{\Psi_{\aorb{1}}^{\borb{2}}},\ket{\Psi_{\borb{1}}^{\aorb{2}}}$が三重項であることを示せ。

ただし、$\ket{^1\Psi_1^2},\ket{^3\Psi_1^2}$は
次の通りである。
\begin{align}
	\ket{^1\Psi_1^2}
&=
	\frac{1}{\sqrt{2}}
	\left(
		\ket{\Psi_{\borb{1}}^{\borb{2}}}
		+
		\ket{\Psi_{\aorb{1}}^{\aorb{2}}}
	\right) \\
%
%
&=
	\frac{1}{\sqrt{2}}
	\left(
		\ket{\aorb{1}\borb{2}}
		+
		\ket{\aorb{2}\borb{1}}
	\right)
\end{align}
\begin{align}
	\ket{^3\Psi_1^2}
&=
	\frac{1}{\sqrt{2}}
	\left(
		\ket{\Psi_{\borb{1}}^{\borb{2}}}
		-
		\ket{\Psi_{\aorb{1}}^{\aorb{2}}}
	\right) \\
%
%
&=
	\frac{1}{\sqrt{2}}
	\left(
		\ket{\aorb{1}\borb{2}}
		-
		\ket{\aorb{2}\borb{1}}
	\right)
\end{align}
である。

\subsection{解}
まず、$\ket{^1\Psi_1^2}$について考える。
$\mathscr{S}_- \mathscr{S}_+\ket{^1\Psi_1^2}$は、前問を参考にすると
\begin{align}
	\mathscr{S}_- \mathscr{S}_+ \ket{^1\Psi_1^2}
&=
	\frac{1}{\sqrt{2}}
	\biggl(
		\mathscr{S}_- \mathscr{S}_+ \ket{\aorb{1}\borb{2}}
		+
		\mathscr{S}_- \mathscr{S}_+ \ket{\aorb{2}\borb{1}}
	\biggr) \\
%
%
&=
	\frac{1}{\sqrt{2}}
	\biggl(
		\mathscr{S}_-\left(
			0 +\ket{\aorb{1}\aorb{2}}
		\right)
		+
		\mathscr{S}_-\left(
			0 +\ket{\aorb{2}\aorb{1}}
		\right)
	\biggr) \\
%
%
&=
	\frac{1}{\sqrt{2}}
	\biggl(
		\left(
			\ket{\borb{1}\aorb{2}}
			+
			\ket{\aorb{1}\borb{2}}
		\right)
		+
		\left(
			\ket{\borb{2}\aorb{1}}
			+
			\ket{\aorb{2}\borb{1}}
		\right)
	\biggr) \\
%
%
&=
	\frac{1}{\sqrt{2}}
	\left(
		-\ket{\aorb{2}\borb{1}}
		+\ket{\aorb{1}\borb{2}}
		-\ket{\aorb{1}\borb{2}}
		+\ket{\aorb{2}\borb{1}}
	\right) \\
%
%
&=
	0
\end{align}
である。また、$\mathscr{S}_z\ket{^1\Psi_1^2}$と$\mathscr{S}_z^2\ket{^1\Psi_1^2}$については
問2.37より
\begin{align}
	\mathscr{S}_z\ket{^1\Psi_1^2}
&=
	\frac{1}{\sqrt{2}}
	\biggl(
		\mathscr{S}_z \ket{\aorb{1}\borb{2}}
		+
		\mathscr{S}_z \ket{\aorb{2}\borb{1}}
	\biggr) \\
%
%
&=
	\frac{1}{\sqrt{2}}
	\left(
		\frac{1}{2} (1-1) \ket{\aorb{1}\borb{2}}
		+
		\frac{1}{2} (1-1) \ket{\aorb{2}\borb{1}}
	\right) \\
%
%
&=
	0 \\[2mm]
%
%
	\mathscr{S}_z^2 \ket{^1\Psi_1^2}
&=
	0
\end{align}
である。従って、
\begin{align}
	\mathscr{S}^2 \ket{^1\Psi_1^2}
&=
	\left(
		\mathscr{S}_- \mathscr{S}_+
		+
		\mathscr{S}_z
		+
		\mathscr{S}_z^2
	\right) \ket{^1\Psi_1^2} \\
%
%
&=
	0 \\
%
%
&=
	0 (0+1) \ket{^1\Psi_1^2}
\end{align}
となる。よって、$\ket{^1\Psi_1^2}$は$\mathscr{S}^2$の固有関数であり、
$2\cdot 0+1=1$重項状態である。

次に、$\ket{^3\Psi_1^2}$について考える。
まず、$\mathscr{S}_- \mathscr{S}_+ \ket{^3\Psi_1^2}$は
\begin{align}
	\mathscr{S}_- \mathscr{S}_+ \ket{^3\Psi_1^2}
&=
	\frac{1}{\sqrt{2}}
	\biggl(
		\mathscr{S}_- \mathscr{S}_+ \ket{\aorb{1}\borb{2}}
		-
		\mathscr{S}_- \mathscr{S}_+ \ket{\aorb{2}\borb{1}}
	\biggr) \\
%
%
&=
	\frac{1}{\sqrt{2}}
	\biggl(
		\mathscr{S}_-\left(
			0 +\ket{\aorb{1}\aorb{2}}
		\right)
		-
		\mathscr{S}_-\left(
			0 +\ket{\aorb{2}\aorb{1}}
		\right)
	\biggr) \\
%
%
&=
	\frac{1}{\sqrt{2}}
	\biggl(
		\left(
			\ket{\borb{1}\aorb{2}}
			+
			\ket{\aorb{1}\borb{2}}
		\right)
		-
		\left(
			\ket{\borb{2}\aorb{1}}
			+
			\ket{\aorb{2}\borb{1}}
		\right)
	\biggr) \\
%
%
&=
	\frac{1}{\sqrt{2}}
	\left(
		-\ket{\aorb{2}\borb{1}}
		+\ket{\aorb{1}\borb{2}}
		+\ket{\aorb{1}\borb{2}}
		-\ket{\aorb{2}\borb{1}}
	\right) \\
%
%
&=
	2 \cdot
	\frac{1}{\sqrt{2}}
	\left(
		\ket{\aorb{1}\borb{2}}
		-
		\ket{\aorb{2}\borb{1}}
	\right) \\
%
%
&=
	2 \ket{^3\Psi_1^2}
\end{align}
である。また、$\mathscr{S}_z \ket{^3\Psi_1^2}$と
$\mathscr{S}_z^2 \ket{^3\Psi_1^2}$については
\begin{align}
	\mathscr{S}_z \ket{^3\Psi_1^2}
&=
	\frac{1}{\sqrt{2}}
	\biggl(
		\mathscr{S}_z \ket{\aorb{1}\borb{2}}
		-
		\mathscr{S}_z \ket{\aorb{2}\borb{1}}
	\biggr) \\
%
%
&=
	\frac{1}{\sqrt{2}}
	\left(
		0 \ket{\aorb{1}\borb{2}}
		-
		0 \ket{\aorb{2}\borb{1}}
	\right) \\
%
%
&=
	0 \\[2mm]
%
%
	\mathscr{S}_z^2 \ket{^3\Psi_1^2}
&=
	0
\end{align}
となる。従って、
\begin{align}
	\mathscr{S}^2 \ket{^3\Psi_1^2}
&=
	\left(
		\mathscr{S}_- \mathscr{S}_+
		+
		\mathscr{S}_z
		+
		\mathscr{S}_z^2
	\right) \ket{^3\Psi_1^2} \\
%
%
&=
	2 \ket{^3\Psi_1^2} \\
%
%
&=
	1 (1+1) \ket{^3\Psi_1^2}
\end{align}
となり、$\ket{^3\Psi_1^2}$は$2\cdot 1+1=3$重項状態である。

次に$\ket{\Psi_{\aorb{1}}^{\borb{2}}}$について考える。
まず、$\mathscr{S}_- \mathscr{S}_+ \ket{\Psi_{\aorb{1}}^{\borb{2}}}$は
\begin{align}
	\mathscr{S}_- \mathscr{S}_+ \ket{\Psi_{\aorb{1}}^{\borb{2}}}
&=
	\mathscr{S}_- \mathscr{S}_+ \ket{\borb{2}\borb{1}} \\
%
%
&=
	\mathscr{S}_-\left(
		\ket{\aorb{2}\borb{1}}
		+
		\ket{\borb{2}\aorb{1}}
	\right) \\
%
%
&=
	\ket{\borb{2}\borb{1}}
	+
	0
	+
	0
	+
	\ket{\borb{2}\borb{1}} \\
%
%
&=
	2 \ket{\borb{2}\borb{1}} \\
%
%
&=
	2 \ket{\Psi_{\aorb{1}}^{\borb{2}}}
\end{align}
となる。また、$\mathscr{S}_z \ket{\Psi_{\aorb{1}}^{\borb{2}}}$と
$\mathscr{S}_z^2 \ket{\Psi_{\aorb{1}}^{\borb{2}}}$については
\begin{align}
	\mathscr{S}_z \ket{\Psi_{\aorb{1}}^{\borb{2}}}
&=
	\mathscr{S}_z \ket{\borb{2}\borb{1}} \\
%
%
&=
	\frac{1}{2} (0-2)
	\ket{\borb{2}\borb{1}} \\
%
%
&=
	- \ket{\Psi_{\aorb{1}}^{\borb{2}}} \\[2mm]
%
%
	\mathscr{S}_z^2 \ket{\Psi_{\aorb{1}}^{\borb{2}}}
&=
	(-1)^2 \ket{\Psi_{\aorb{1}}^{\borb{2}}}
=
	\ket{\Psi_{\aorb{1}}^{\borb{2}}}
\end{align}
である。従って、
\begin{align}
	\mathscr{S}^2 \ket{\Psi_{\aorb{1}}^{\borb{2}}}
&=
	\left(
		\mathscr{S}_- \mathscr{S}_+
		+
		\mathscr{S}_z
		+
		\mathscr{S}_z^2
	\right) \ket{\Psi_{\aorb{1}}^{\borb{2}}} \\
%
%
&=
	(2-1+1) \ket{\Psi_{\aorb{1}}^{\borb{2}}} \\
%
%
&=
	2 \ket{\Psi_{\aorb{1}}^{\borb{2}}} \\
%
%
&=
	1 (1+1) \ket{\Psi_{\aorb{1}}^{\borb{2}}}
\end{align}
となるので、$\ket{\Psi_{\aorb{1}}^{\borb{2}}}$は
$2\cdot 1+1=3$重項状態である。

最後に$\ket{\Psi_{\borb{1}}^{\aorb{2}}}$について考える。
まず$\mathscr{S}_- \mathscr{S}_+ \ket{\Psi_{\borb{1}}^{\aorb{2}}}$は
\begin{align}
	\mathscr{S}_- \mathscr{S}_+ \ket{\Psi_{\borb{1}}^{\aorb{2}}}
&=
	\mathscr{S}_- \mathscr{S}_+ \ket{\aorb{1}\aorb{2}} \\
%
%
&=
	\mathscr{S}_-\left(
		0 +0
	\right) \\
%
%
&=
	0
\end{align}
である。また、$\mathscr{S}_z \ket{\Psi_{\borb{1}}^{\aorb{2}}}$と
$\mathscr{S}_z^2 \ket{\Psi_{\borb{1}}^{\aorb{2}}}$については
\begin{align}
	\mathscr{S}_z \ket{\Psi_{\borb{1}}^{\aorb{2}}}
&=
	\mathscr{S}_z \ket{\aorb{1}\aorb{2}} \\
%
%
&=
	\frac{1}{2} (2-0) \ket{\aorb{1}\aorb{2}} \\
%
%
&=
	\ket{\Psi_{\borb{1}}^{\aorb{2}}} \\[2mm]
%
%
	\mathscr{S}_z^2 \ket{\Psi_{\borb{1}}^{\aorb{2}}}
&=
	1^2 \ket{\Psi_{\borb{1}}^{\aorb{2}}}
=
	\ket{\Psi_{\borb{1}}^{\aorb{2}}}
\end{align}
となる。従って、
\begin{align}
	\mathscr{S}^2 \ket{\Psi_{\borb{1}}^{\aorb{2}}}
&=
	\left(
		\mathscr{S}_- \mathscr{S}_+
		+
		\mathscr{S}_z
		+
		\mathscr{S}_z^2
	\right) \ket{\Psi_{\borb{1}}^{\aorb{2}}} \\
%
%
&=
	(0+1+1) \ket{\Psi_{\borb{1}}^{\aorb{2}}} \\
%
%
&=
	1 (1+1) \ket{\Psi_{\borb{1}}^{\aorb{2}}}
\end{align}
となるので、$\ket{\Psi_{\borb{1}}^{\aorb{2}}}$は
$2\cdot 1+1=3$重項状態である。

