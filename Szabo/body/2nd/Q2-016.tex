%ファイルID
%2020/06/02 17:11
%->2006021711(10進数)->X6C09B(ファイル作成者)
%->X6C09BSahayanKY
\subsection{問}
$\ket{K}$は$N$電子Slater行列式であり、
$\ket{K^{HP}}$は$\ket{K}$に対するHartree積である。
即ち、
\begin{align}
	\ket{K}
&=
	\ket{\chi_m(\x_1)\chi_n(\x_2)\dots} &
%
	\ket{K^{HP}}
&=
	\chi_m(\x_1)\chi_n(\x_2)\dots
\end{align}
である。このとき、
\begin{align}
	\braket{K|\H|L}
&=
	\sqrt{N!}
	\braket{K^{HP}|\H|L}
\end{align}
が成立することを証明せよ。


\subsection{解}
$\ket{L}$を次の通りに置く。
\begin{align}
	\ket{L}
&=
	\frac{1}{\sqrt{N!}}
	\sum_{i}^{N!}
		\sgn(P_i)\
		P_i\Bigl\{
			\chi_m'(\x_1) \chi_n'(\x_2) \dots
		\Bigr\}
\end{align}
ここで$\sgn(P_i)$は$P_i$の符号であり、$(-1)^{p_i}$のことである。

従って、$\braket{K|\H|L}$は
\begin{align}
&\quad
	\braket{K|\H|L} \nonumber \\
&=
	\left(
		\frac{1}{\sqrt{N!}}
	\right)^2
	\sum_i^{N!}
	\sum_j^{N!}
		\sgn(P_i)
		\sgn(P_j) \nonumber \\ &\qquad\qquad\qquad\qquad \times
		\int \d\x_1 \d\x_2 \dots \d\x_N
			P_i\Bigl\{
				\chi_m^\ast(\x_1) \chi_n^\ast(\x_2) \dots
			\Bigr\}
			\H
			P_j\Bigl\{
				\chi_m'(\x_1) \chi_n'(\x_2) \dots
			\Bigr\} \\
%
%
&=
	\left(
		\frac{1}{\sqrt{N!}}
	\right)^2
	\sum_i^{N!}
	\sum_j^{N!}
		\sgn(P_i)
		\sgn(P_j) \nonumber \\ &\qquad\qquad\qquad\qquad \times
		\int \d\x_1 \d\x_2 \dots \d\x_N
			P_i\Bigl\{
				\chi_m^\ast(\x_1) \chi_n^\ast(\x_2) \dots
			\Bigr\}
			P_i\Bigl\{
				\H
			\Bigr\}
			(P_iP_i^{-1}P_j)\Bigl\{
				\chi_m'(\x_1) \chi_n'(\x_2) \dots
			\Bigr\} \\
	&\qquad
	\left(
		\because P_i\{\H\} = \H
	\right) \nonumber \\
%
%
&=
	\left(
		\frac{1}{\sqrt{N!}}
	\right)^2
	\sum_i^{N!}
	\sum_j^{N!}
		\sgn(P_i)
		\sgn(P_j) \nonumber \\ &\qquad\qquad\qquad\qquad \times
		\int \d\x_1 \d\x_2 \dots \d\x_N
			P_i\Bigl\{
				\chi_m^\ast(\x_1) \chi_n^\ast(\x_2) \dots
				\H
				(P_i^{-1}P_j)\Bigl\{
					\chi_m'(\x_1) \chi_n'(\x_2) \dots
				\Bigr\}
			\Bigr\} \\
%
%
&=
	\left(
		\frac{1}{\sqrt{N!}}
	\right)^2
	\sum_i^{N!}
	\sum_j^{N!}
		\sgn(P_i)
		\sgn(P_j) \nonumber \\ &\qquad\qquad\qquad\qquad \times
		\int \d\x_1 \d\x_2 \dots \d\x_N
			\chi_m^\ast(\x_1) \chi_n^\ast(\x_2) \dots
			\H
			(P_i^{-1}P_j)\Bigl\{
				\chi_m'(\x_1) \chi_n'(\x_2) \dots
			\Bigr\} \\
	&\qquad
	\left(
		\because \int\d\x_1\d\x_2\ f(\x_2,\x_1) = \int\d\x_2\d\x_1\ f(\x_1,\x_2) = \int\d\x_1\d\x_2\ f(\x_1,\x_2)
	\right) \nonumber
\end{align}
ここで、$P_i^{-1}P_j$はまた別の置換を表しており、これを$P_k$とおく。
また、$\sgn(P_i)=\sgn(P_i^{-1})$であること、
$\sgn(P_k)=\sgn(P_i^{-1})\sgn(P_j)$であることから、
\begin{align}
	\braket{K|\H|L}
&=
	\left(
		\frac{1}{\sqrt{N!}}
	\right)^2
	\sum_i^{N!}
	\sum_j^{N!}
		\sgn(P_k)
		\int \d\x_1 \d\x_2 \dots \d\x_N
			\chi_m^\ast(\x_1) \chi_n^\ast(\x_2) \dots
			\H
			P_k\Bigl\{
				\chi_m'(\x_1) \chi_n'(\x_2) \dots
			\Bigr\}
\end{align}
ここで、異なる$(i,j)$の組合せであっても$P_k=P_i^{-1}P_j$が等しくなる場合がある。
$(i,j)$の組は$N!^2$通りあるが、$P_k$は全部で$N!$個までである。
このため、$N!^2/N!=N!$組が等しい$P_k$を与えることになる。
従って、
\begin{align}
	\braket{K|\H|L}
&=
	\left(
		\frac{1}{\sqrt{N!}}
	\right)^2
	N!
	\sum_k^{N!}
		\sgn(P_k)
		\int \d\x_1\d\x_2\dots\d\x_N
			\chi_m^\ast(\x_1) \chi_n^\ast(\x_2) \dots
			\H
			P_k\Bigl\{
				\chi_m'(\x_1) \chi_n'(\x_2) \dots
			\Bigr\} \\
%
%
&=
	\frac{1}{\sqrt{N!}}
	N!
	\braket{K^{HP}|\H|L} \\
%
%
&=
	\sqrt{N!}
	\braket{K^{HP}|\H|L}
\end{align}
となる。



