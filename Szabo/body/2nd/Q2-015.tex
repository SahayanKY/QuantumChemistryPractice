%ファイルID
%2020/05/27 16:00
%->2005271600(10進数)->X5VXGW(ファイル作成者)
%->X5VXGWSahayanKY
\subsection{問}
問題2.4を一般化させる。
スピン軌道$\chi_i,\chi_j,\dots,\chi_k$を1電子演算子$h$の固有関数とし、
それぞれの固有値を$\epsilon_i,\epsilon_j,\dots,\epsilon_k$とする。
これから得られる$N$電子Slater行列式$\ket{\chi_i\chi_j \dots\chi_k}$が
独立電子のハミルトニアン$\mathscr{H}=\sum_i^N h(i)$の固有関数であること、
その固有値が$\epsilon_i+\epsilon_j+\dots+\epsilon_k$であることを示せ。


\subsection{解}
$N$電子Slater行列式は次の通りである。
\begin{align}
	\ket{\chi_i\chi_j\dots\chi_k}
&=
	\frac{1}{\sqrt{N!}}
	\sum_{n=1}^{N!}
		(-1)^{p_n}
		\mathscr{P}_n\left\{
			\chi_i(\x_1) \chi_j(\x_2) \dots \chi_k(\x_N)
		\right\} \\
%
%
&=
	\frac{1}{\sqrt{N!}}
	\sum_{n=1}^{N!}
		(-1)^{p_n}
		\chi_i(\x_{\mathscr{P}_n(1)})
		\chi_j(\x_{\mathscr{P}_n(2)})
		\dots
		\chi_k(\x_{\mathscr{P}_n(N)})
\end{align}
ここで、$\mathscr{P}_n$は電子の添え字$1,2,\dots,N$の置換演算子であり、
$n$は置換の仕方を区別するためのラベルである。
また、$p_n$は置換$\mathscr{P}_n$における置換の回数である。
また、$\mathscr{P}_n(i)$は添え字$i$を置換$\mathscr{P}_n$に従って置き換えた添え字である。

ハミルトニアン$\mathscr{H}$に置換演算を作用させると、
\begin{align}
	\mathscr{P}_n\left\{
		\mathscr{H}
	\right\}
&=
	\mathscr{P}_n\left\{
		\sum_{i=1}^{N} h(\x_i)
	\right\} \\
%
%
&=
	\sum_{i=1}^{N} h(\x_{\mathscr{P}_n(i)}) \\
%
%
&=
	\sum_{j=\mathscr{P}_n(1),\dots,\mathscr{P}_n(N)} h(\x_j) \\
%
%
&=
	\sum_{j=1,\dots,N} h(\x_j)
%
%
=
	\sum_{i=1}^{N} h(\x_i) \\
%
%
&=
	\mathscr{H}
\end{align}
となる。従って、Slater行列式にハミルトニアン$\mathscr{H}$を作用させると、
\begin{align}
	\mathscr{H}\ket{\chi_i\chi_j\dots\chi_k}
&=
	\frac{1}{\sqrt{N!}}
	\sum_{n=1}^{N!}
		(-1)^{p_n}
		\mathscr{H}
		\mathscr{P}_n\left\{
			\chi_i(\x_1) \chi_j(\x_2) \dots \chi_k(\x_N)
		\right\} \\
%
%
&=
	\frac{1}{\sqrt{N!}}
	\sum_{n=1}^{N!}
		(-1)^{p_n}
		\mathscr{P}_n\left\{
			\mathscr{H}
		\right\}
		\mathscr{P}_n\left\{
			\chi_i(\x_1) \chi_j(\x_2) \dots \chi_k(\x_N)
		\right\} \\
%
%
&=
	\frac{1}{\sqrt{N!}}
	\sum_{n=1}^{N!}
		(-1)^{p_n}
		\mathscr{P}_n\left\{
			\mathscr{H}
			\chi_i(\x_1) \chi_j(\x_2) \dots \chi_k(\x_N)
		\right\}
\end{align}
となる。置換演算の中身について注目すると、
これはHartree積にハミルトニアンを作用させており、
\begin{align}
	\mathscr{H}
	\chi_i(\x_1) \chi_j(\x_2) \dots \chi_k(\x_N)
&=
	(\epsilon_i+\epsilon_j+\dots+\epsilon_k)
	\chi_i(\x_1) \chi_j(\x_2) \dots \chi_k(\x_N)
\end{align}
であるため、
\begin{align}
	\mathscr{H}\ket{\chi_i\chi_j\dots\chi_k}
&=
	\frac{1}{\sqrt{N!}}
	\sum_{n=1}^{N!}
		(-1)^{p_n}
		\mathscr{P}_n\left\{
			(\epsilon_i+\epsilon_j+\dots+\epsilon_k)
			\chi_i(\x_1) \chi_j(\x_2) \dots \chi_k(\x_N)
		\right\} \\
%
%
&=
	(\epsilon_i+\epsilon_j+\dots+\epsilon_k)
	\frac{1}{\sqrt{N!}}
	\sum_{n=1}^{N!}
		(-1)^{p_n}
		\mathscr{P}_n\left\{
			\chi_i(\x_1) \chi_j(\x_2) \dots \chi_k(\x_N)
		\right\} \\
%
%
&=
	(\epsilon_i+\epsilon_j+\dots+\epsilon_k)
	\ket{\chi_i\chi_j\dots\chi_k}
\end{align}
となる。従って、Slater行列式は$\mathscr{H}$の固有関数で、
固有値は$\epsilon_i+\epsilon_j+\dots+\epsilon_k$であることが言える。









