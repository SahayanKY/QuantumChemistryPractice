%ファイルID
%2020/06/18 15:14
%->2006181514SahayanKY(ファイル作成者)
\subsection{問}
閉殻行列式$\ket{\aorb{\psi}_i\borb{\psi}_i\aorb{\psi}_j\borb{\psi}_j\dots}$について、
\begin{align}
	\mathscr{S}^2 \ket{\aorb{\psi}_i\borb{\psi}_i\aorb{\psi}_j\borb{\psi}_j\dots}
=
	0 (0+1) \ket{\aorb{\psi}_i\borb{\psi}_i\aorb{\psi}_j\borb{\psi_j}\dots}
=
	0
\end{align}
であることを示せ。

\subsection{解}
まず、スピン演算子$\mathscr{S}^2$は
\begin{align}
	\mathscr{S}^2
&=
	\mathscr{S}_- \mathscr{S}_+
	+
	\mathscr{S}_z
	+
	\mathscr{S}_z^2
\end{align}
に変形できる。

閉殻行列式では$\alpha$スピン軌道の数$N^{\alpha}$と
$\beta$スピン軌道の数$N^{\beta}$が等しいので
\begin{align}
	\mathscr{S}_z \ket{\aorb{\psi}_i\borb{\psi}_i\aorb{\psi}_j\borb{\psi}_j\dots}
&=
	\frac{1}{2}
	(N^{\alpha} -N^{\beta})
	\ket{\aorb{\psi}_i\borb{\psi}_i\aorb{\psi}_j\borb{\psi}_j\dots} \\
%
%
&=
	0
\end{align}
となる。また、
\begin{align}
	\mathscr{S}_z^2 \ket{\aorb{\psi}_i\borb{\psi}_i\aorb{\psi}_j\borb{\psi}_j\dots}
&=
	\mathscr{S}_z \mathscr{S}_z
		\ket{\aorb{\psi}_i\borb{\psi}_i\aorb{\psi}_j\borb{\psi}_j\dots} \\
%
%
&=
	0
\end{align}
となる。従って、
\begin{align}
	\mathscr{S}^2 \ket{\aorb{\psi}_i\borb{\psi}_i\aorb{\psi}_j\borb{\psi}_j\dots}
&=
	\mathscr{S}_- \mathscr{S}_+
		\ket{\aorb{\psi}_i\borb{\psi}_i\aorb{\psi}_j\borb{\psi}_j\dots}
\end{align}
である。

$\mathscr{S}_+ \ket{\aorb{\psi}_i\borb{\psi}_i\aorb{\psi}_j\borb{\psi}_j\dots}$について考える。
前問と同様に考えると、$\mathscr{S}_+$は$\mathscr{P}_n$に関して不変であるので、
\begin{align}
	\mathscr{S}_+ \ket{\aorb{\psi}_i\borb{\psi}_i\aorb{\psi}_j\borb{\psi}_j\dots}
&=
	\mathscr{S}_+\left\{
		\frac{1}{\sqrt{N!}}
		\sum_{n=1}^{N!}
			(-1)^{p_n}
			\mathscr{P}_n\{
				\aorb{\psi}_i(1) \borb{\psi}_i(2) \aorb{\psi}_j(3) \borb{\psi}_j(4) \dots
			\}
	\right\} \\
%
%
&=
	\frac{1}{\sqrt{N!}}
	\sum_{n=1}^{N!}
		(-1)^{p_n}
		\mathscr{P}_n\Bigl\{
				\mathscr{S}_+\{
					\aorb{\psi}_i(1) \borb{\psi}_i(2) \aorb{\psi}_j(3) \borb{\psi}_j(4) \dots
				\}
		\Bigr\}
\end{align}
となる。更に、$s_+\ket{\alpha}=0, s_+\ket{\beta}=\ket{\alpha}$であるので、
\begin{align}
	\mathscr{S}_+\{
		\aorb{\psi}_i(1) \borb{\psi}_i(2) \aorb{\psi}_j(3) \borb{\psi}_j(4) \dots
	\}
&=
	(s_+(1) \aorb{\psi}_i(1)) \borb{\psi}_i(2) \aorb{\psi}_j(3) \borb{\psi}_j(4) \dots \nonumber \\ &\quad
	+
	\aorb{\psi}_i(1) (s_+(2) \borb{\psi}_i(2)) \aorb{\psi}_j(3) \borb{\psi}_j(4) \dots \nonumber \\ &\quad
	+
	\aorb{\psi}_i(1) \borb{\psi}_i(2) (s_+(3) \aorb{\psi}_j(3)) \borb{\psi}_j(4) \dots \nonumber \\ &\quad
	+
	\aorb{\psi}_i(1) \borb{\psi}_i(2) \aorb{\psi}_j(3) (s_+(4) \borb{\psi}_j(4)) \dots \nonumber \\ &\quad
	+
	\dots \\
%
%
&=
	0 \cdot \borb{\psi}_i(2) \aorb{\psi}_j(3) \borb{\psi}_j(4) \dots \nonumber \\ &\quad
	+
	\aorb{\psi}_i(1) \aorb{\psi}_i(2) \aorb{\psi}_j(3) \borb{\psi}_j(4) \dots \nonumber \\ &\quad
	+
	\aorb{\psi}_i(1) \borb{\psi}_i(2) \cdot 0 \cdot \borb{\psi}_j(4) \dots \nonumber \\ &\quad
	+
	\aorb{\psi}_i(1) \borb{\psi}_i(2) \aorb{\psi}_j(3) \aorb{\psi}_j(4) \dots \nonumber \\ &\quad
	+
	\dots \\
%
%
&=
	\aorb{\psi}_i(1) \aorb{\psi}_i(2) \aorb{\psi}_j(3) \borb{\psi}_j(4) \dots \nonumber \\ &\quad
	+
	\aorb{\psi}_i(1) \borb{\psi}_i(2) \aorb{\psi}_j(3) \aorb{\psi}_j(4) \dots \nonumber \\ &\quad
	+
	\dots
\end{align}
従って、
\begin{align}
	\mathscr{S}_+ \ket{\aorb{\psi}_i\borb{\psi}_i\aorb{\psi}_j\borb{\psi}_j\dots}
&=
	\frac{1}{\sqrt{N!}}
	\sum_{n=1}^{N!}
		(-1)^{p_n}
		\mathscr{P}_n\left\{
		\begin{array}{l}
			\aorb{\psi}_i(1) \aorb{\psi}_i(2) \aorb{\psi}_j(3) \borb{\psi}_j(4) \dots \\
			+
			\aorb{\psi}_i(1) \borb{\psi}_i(2) \aorb{\psi}_j(3) \aorb{\psi}_j(4) \dots \\
			+
			\dots
		\end{array}
		\right\} \\
%
%
&=
	\frac{1}{\sqrt{N!}}
	\sum_{n=1}^{N!}
		(-1)^{p_n}
		\mathscr{P}_n\left\{
			\aorb{\psi}_i(1) \aorb{\psi}_i(2) \aorb{\psi}_j(3) \borb{\psi}_j(4) \dots
		\right\} \nonumber \\ &\quad
	+
	\frac{1}{\sqrt{N!}}
	\sum_{n=1}^{N!}
		(-1)^{p_n}
		\mathscr{P}_n\left\{
			\aorb{\psi}_i(1) \borb{\psi}_i(2) \aorb{\psi}_j(3) \aorb{\psi}_j(4) \dots
		\right\} \nonumber \\ &\quad
	+
	\dots \\
%
%
&=
	\ket{\aorb{\psi}_i\aorb{\psi}_i\aorb{\psi}_j\borb{\psi}_j\dots}
	+
	\ket{\aorb{\psi}_i\borb{\psi}_i\aorb{\psi}_j\aorb{\psi}_j\dots}
	+
	\dots \\
%
%
&=
	0
\end{align}
である。最後、$\ket{\aorb{\psi}_i\aorb{\psi}_i\aorb{\psi}_j\borb{\psi}_j\dots}=0$としたのは
Slater行列式の反対称性による。

従って、$\mathscr{S}^2\ket{\aorb{\psi}_i\borb{\psi}_i\aorb{\psi}_j\borb{\psi}_j\dots}$は
\begin{align}
	\mathscr{S}^2 \ket{\aorb{\psi}_i\borb{\psi}_i\aorb{\psi}_j\borb{\psi}_j\dots}
&=
	\mathscr{S}_- \mathscr{S}_+
		\ket{\aorb{\psi}_i\borb{\psi}_i\aorb{\psi}_j\borb{\psi}_j\dots} \\
%
%
&=
	0
\end{align}
となる。


