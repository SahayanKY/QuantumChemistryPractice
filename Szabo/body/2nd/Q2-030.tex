%ファイルID
%2020/06/11 19:54
%->2006111954SahayanKY(ファイル作成者)
\subsection{問}
\begin{align}
	\braket{\Psi_a^r|\Oone|\Psi_0}
&=
	\sum_i^{2K} \sum_j^{2K}
		\braket{i|h|j}
		\braket{\Psi_0|\adj{a_a} a_r \adj{a_i} a_j|\Psi_0} \\
%
%
&=
	\braket{r|h|a}
\end{align}
となることを示せ。


\subsection{解}
\begin{align}
	\braket{\Psi_a^r|\Oone|\Psi_0}
&=
	\sum_i^{2K} \sum_j^{2K}
		\braket{i|h|j}
		\braket{\Psi_a^r|\adj{a_i} a_j|\Psi_0} \\
%
%
&=
	\sum_i^{2K} \sum_j^{2K}
		\braket{i|h|j}
		\braket{\Psi_0|\adj{a_a} a_r \adj{a_i} a_j|\Psi_0}
	%
	\qquad
	(\because \ket{\Psi_a^r}=\adj{a_r} a_a\ket{\Psi_0})
\end{align}
である。
ここで、
\begin{align}
	\adj{a_a} a_r \adj{a_i} a_j
&=
	\adj{a_a} (\delta_{ri} -\adj{a_i} a_r) a_j \\
%
%
&=
	\delta_{ri} \adj{a_a} a_j
	-
	\adj{a_a} \adj{a_i} a_r a_j \\
%
%
&=
	\delta_{ri} (\delta_{aj} -a_j \adj{a_a})
	-
	\adj{a_a} \adj{a_i}(-a_j a_r) \\
%
%
&=
	\delta_{ri} \delta_{aj}
	-
	\delta_{ri} a_j \adj{a_a}
	+
	\adj{a_a} \adj{a_i} a_j a_r \\
%
%
	\adj{a_a} a_r \adj{a_i} a_j \ket{\Psi_0}
&=
	\delta_{ri} \delta_{aj} \ket{\Psi_0}
\end{align}
であることから、
\begin{align}
	\braket{\Psi_a^r|\Oone|\Psi_0}
&=
	\sum_i^{2K} \sum_j^{2K}
		\braket{i|h|j}
		\delta_{ri}
		\delta_{aj}
		\braket{\Psi_0|\Psi_0} \\
%
%
&=
	\sum_i^{2K} \sum_j^{2K}
		\braket{i|h|j}
		\delta_{ri}
		\delta_{aj} \\
%
%
&=
	\braket{r|h|a}
\end{align}
となる。

