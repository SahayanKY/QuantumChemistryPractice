%ファイルID
%2020/06/09 17:25
%->2006091725SahayanKY(ファイル作成者)
\subsection{問}
生成演算子$\adj{a_i}$は次の通りに定義される。
\begin{align}
	\adj{a_i}\ket{\chi_k\dots\chi_l}
&=
	\ket{\chi_i\chi_k\dots\chi_l}
\end{align}

このとき、
\begin{align}
	(\adj{a_1}\adj{a_2} +\adj{a_2}\adj{a_1}) \ket{K}
&=
	0
\end{align}
を、$\ket{K}=\ket{\chi_1\chi_2}, \ket{\chi_1\chi_3}, \ket{\chi_1\chi_4},
\ket{\chi_2\chi_3}, \ket{\chi_2\chi_4}, \ket{\chi_3\chi_4}$について示せ。


\subsection{解}
$\ket{\chi_1\chi_2}$について考える。
\begin{align}
	(\adj{a_1}\adj{a_2}+\adj{a_2}\adj{a_1}) \ket{\chi_1\chi_2}
&=
	\adj{a_1} \ket{\chi_2\chi_1\chi_2}
	+
	\adj{a_2} \ket{\chi_1\chi_1\chi_2} \\
%
%
&=
	\adj{a_1} 0
	+
	\adj{a_2} 0 \\
%
%
&=
	0
\end{align}
である。

次に、$\ket{\chi_1\chi_3}$について考える。
\begin{align}
	(\adj{a_1}\adj{a_2}+\adj{a_2}\adj{a_1}) \ket{\chi_1\chi_3}
&=
	\ket{\chi_1\chi_2\chi_1\chi_3}
	+
	\ket{\chi_2\chi_1\chi_1\chi_3} \\
%
%
&=
	0
	+
	0 \\
%
%
&=
	0
\end{align}
である。

次に、$\ket{\chi_1\chi_4}$について考える。
\begin{align}
	(\adj{a_1}\adj{a_2}+\adj{a_2}\adj{a_1}) \ket{\chi_1\chi_4}
&=
	\ket{\chi_1\chi_2\chi_1\chi_4}
	+
	\ket{\chi_2\chi_1\chi_1\chi_4} \\
%
%
&=
	0
	+
	0 \\
%
%
&=
	0
\end{align}
である。

次に、$\ket{\chi_2\chi_3}$について考える。
\begin{align}
	(\adj{a_1}\adj{a_2}+\adj{a_2}\adj{a_1}) \ket{\chi_2\chi_3}
&=
	\ket{\chi_1\chi_2\chi_2\chi_3}
	+
	\ket{\chi_2\chi_1\chi_2\chi_3} \\
%
%
&=
	0
	+
	0 \\
%
%
&=
	0
\end{align}
である。

次に、$\ket{\chi_2\chi_4}$について考える。
\begin{align}
	(\adj{a_1}\adj{a_2}+\adj{a_2}\adj{a_1}) \ket{\chi_2\chi_4}
&=
	\ket{\chi_1\chi_2\chi_2\chi_4}
	+
	\ket{\chi_2\chi_1\chi_2\chi_4} \\
%
%
&=
	0
	+
	0 \\
%
%
&=
	0
\end{align}
である。

最後に、$\ket{\chi_3\chi_4}$について考える。
\begin{align}
	(\adj{a_1}\adj{a_2}+\adj{a_2}\adj{a_1}) \ket{\chi_3\chi_4}
&=
	\ket{\chi_1\chi_2\chi_3\chi_4}
	+
	\ket{\chi_2\chi_1\chi_3\chi_4} \\
%
%
&=
	\ket{\chi_1\chi_2\chi_3\chi_4}
	-
	\ket{\chi_1\chi_2\chi_3\chi_4} \\
%
%
&=
	0
\end{align}
である。


