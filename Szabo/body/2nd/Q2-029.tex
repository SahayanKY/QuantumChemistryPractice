%ファイルID
%2020/06/11 17:53
%->2006111753SahayanKY(ファイル作成者)
\subsection{問}
最小基底関数系を使った\ce{H2}のHartree-Fock基底状態
$\ket{\Psi_0}=\ket{\chi_1\chi_2}=\adj{a_1}\adj{a_2}\ket{}$とする。
この時、
\begin{align}
	\braket{\Psi_0|\Oone|\Psi_0}
&=
	\sum_i \sum_j
		\braket{i|h|j}
		\braket{|a_2a_1\adj{a_i}a_j\adj{a_1}\adj{a_2}|} \\
%
%
&=
	\braket{1|h|1}
	+
	\braket{2|h|2}
\end{align}
であることを示せ。
ただし、
\begin{align}
	\Oone
&=
	\sum_i \sum_j
		\braket{i|h|j} \adj{a_i} a_j
\end{align}
である。また、$i,j$に関する総和はスピン軌道の組
($\chi_1,\chi_2,\chi_3,\chi_4$)についてとる。

\subsection{解}
$\Oone$の両側から$\bra{\Psi_0}$と$\ket{\Psi_0}$を作用させると、
($\bra{\Psi_0}=\bra{\chi_1\chi_2}$とする)
\begin{align}
	\Oone
&=
	\sum_i \sum_j
		\braket{i|h|j} \adj{a_i} a_j \\
%
%
	\braket{\Psi_0|\Oone|\Psi_0}
&=
	\sum_i \sum_j
		\braket{i|h|j}
		\braket{\Psi_0|\adj{a_i}a_j|\Psi_0} \\
%
%
&=
	\sum_i \sum_j
		\braket{i|h|j}
		\braket{\chi_1\chi_2|\adj{a_i}a_j|\chi_1\chi_2} \\
%
%
&=
	\sum_i \sum_j
		\braket{i|h|j}
		\braket{\chi_1\chi_2|(\delta_{ij}-a_j\adj{a_i})|\chi_1\chi_2} \\
%
%
&=
	\sum_i \sum_j
		\braket{i|h|j}
		\delta_{ij}
		\braket{\chi_1\chi_2|\chi_1\chi_2}
	-
	\sum_i \sum_j
		\braket{i|h|j}
		\braket{\chi_j\chi_1\chi_2|\chi_i\chi_1\chi_2} \\
%
%
&=
	\sum_{i=1,2,3,4}
		\braket{i|h|i}
	-
	\sum_{i=3,4} \sum_{j=3,4}
		\braket{i|h|j}
		\braket{\chi_j\chi_1\chi_2|\chi_i\chi_1\chi_2} \nonumber \\&
	%
	\qquad
	(\because
		\braket{\chi_1\chi_2|\chi_1\chi_2}=1,\
		\ket{\chi_1\chi_1\chi_2} = \ket{\chi_2\chi_1\chi_2} = 0
	) \\
%
%
&=
	\sum_{i=1,2,3,4}
		\braket{i|h|i}
	-
	\sum_{i=3,4} \sum_{j=3,4}
		\braket{i|h|j}
		\delta_{ij}
	%
	\qquad
	(\because \braket{\chi_3\chi_1\chi_2|\chi_4\chi_1\chi_2}=0) \\
%
%
&=
	\sum_{i=1,2}
		\braket{i|h|i} \\
%
%
&=
	\braket{1|h|1}
	+
	\braket{2|h|2}
\end{align}
となる。








