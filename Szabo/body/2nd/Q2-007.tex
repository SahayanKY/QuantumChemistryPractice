%ファイルID
%2020/05/03 19:22
%->2005031922(10進数)->X5QSJ6(ファイル作成者)
%->X5B59ASahayanKY
\subsection{問}
最小基底によるベンゼンの計算では72個のスピン軌道が得られる。
完全CI行列の次元を求めよ。
また、1電子励起行列式の数と2電子励起行列式の数を求めよ。

\subsection{解}
完全CI行列の次元数は
$N$電子行列式の数に等しい。
ベンゼンでは$2K=72$であり、
\ce{C6H6}より$N=6\cdot 6 +1\cdot 6=42$であるので、
\begin{align}
	\bicoeff{2K}{N}
&=
	\frac{72!}{(72-42)!42!} \\
%
%
&=
	164,307,576,757,973,059,488
\end{align}
である。

1電子励起行列式の数は、
Hartree-Fock基底状態$\ket{\chi_1\chi_2\cdots\chi_N}$における
$N$通りの占有軌道から1つを選び、
$2K-N$通りの非占有軌道から1つを選ぶ組み合わせだけ存在するため、
$N(2K-N)=42\cdot 30=1,260$個存在する。

2電子励起行列式の数についても同様に、
$N$通りの占有軌道から2つを選ぶ組み合わせは$\bicoeff{N}{2}=\frac{1}{2}N(N-1)$、
$2K-N$通りの非占有軌道から2つを選ぶ組み合わせは$\bicoeff{2K-N}{2}=\frac{1}{2}(2K-N)(2K-N-1)$であるため、
$\frac{1}{4}N(N-1)(2K-N)(2K-N-1)=374,535$個存在する。
