%ファイルID
%2020/06/23 17:11
%->2006231711SahayanKY(ファイル作成者)
非直交の空間軌道$\psi_1^{\alpha}, \psi_1^{\beta}$から成る
行列式$\ket{K}=\ket{\aorb{\psi_1^{\alpha}}\borb{\psi_1^{\beta}}}$を考える。
なお、$\braket{\psi_1^{\alpha}|\psi_1^{\beta}}=S_{11}^{\alpha\beta}$とする。
\subsection{(a)問}
$\psi_1^{\alpha}=\psi_1^{\beta}$のときだけ
$\ket{K}$は$\mathscr{S}^2$の固有関数であることを示せ。


\subsection{(a)解}
まず、問題2.37にて示した等式は
制限付き行列式であっても非制限行列式であっても適用できるため、
\begin{align}
	\mathscr{S}_z \ket{K}
&=
	\frac{1}{2}(1-1) \ket{K} \\
%
%
&=
	0
\end{align}
である。
従って、
\begin{align}
	\mathscr{S}^2 \ket{K}
&=
	(
		\mathscr{S}_- \mathscr{S}_+
		+
		\mathscr{S}_z
		+
		\mathscr{S}_z^2
	) \ket{K} \\
%
%
&=
	\mathscr{S}_- \mathscr{S}_+ \ket{K} \\
%
%
&=
	\mathscr{S}_-
	(s_+(1) +s_+(2))
	\left(
		\frac{1}{\sqrt{2}}
		\left(
			\psi_1^{\alpha}(\r_1) \alpha(\omega_1)
			\psi_1^{\beta}(\r_2) \beta(\omega_2)
			-
			\psi_1^{\alpha}(\r_2) \alpha(\omega_2)
			\psi_1^{\beta}(\r_1) \beta(\omega_1)
		\right)
	\right) \\
%
%
&=
	\frac{1}{\sqrt{2}}
	\mathscr{S}_-
	\left\{
	\begin{array}{l}
		\psi_1^{\alpha}(\r_1) s_+(1) \alpha(\omega_1) \psi_1^{\beta}(\r_2) \beta(\omega_2)
		-
		\psi_1^{\alpha}(\r_2) \alpha(\omega_2) \psi_1^{\beta}(\r_1) s_+(1) \beta(\omega_1) \\[2mm]
		+
		\psi_1^{\alpha}(\r_1) \alpha(\omega_1) \psi_1^{\beta}(\r_2) s_+(2) \beta(\omega_2)
		-
		\psi_1^{\alpha}(\r_2) s_+(2) \alpha(\omega_2) \psi_1^{\beta}(\r_1) \beta(\omega_1)
	\end{array}
	\right\} \\
%
%
&=
	\frac{1}{\sqrt{2}}
	\mathscr{S}_-
	\left\{
		-
		\psi_1^{\alpha}(\r_2) \alpha(\omega_2) \psi_1^{\beta}(\r_1) \alpha(\omega_1)
		+
		\psi_1^{\alpha}(\r_1) \alpha(\omega_1) \psi_1^{\beta}(\r_2) \alpha(\omega_2)
	\right\} \\
%
%
&=
	\frac{1}{\sqrt{2}}
	\left\{
	\begin{array}{l}
		-
		\psi_1^{\alpha}(\r_2) \alpha(\omega_2) \psi_1^{\beta}(\r_1) s_-(1) \alpha(\omega_1)
		+
		\psi_1^{\alpha}(\r_1) s_-(1) \alpha(\omega_1) \psi_1^{\beta}(\r_2) \alpha(\omega_2) \\[2mm]
		-
		\psi_1^{\alpha}(\r_2) s_-(2) \alpha(\omega_2) \psi_1^{\beta}(\r_1) \alpha(\omega_1)
		+
		\psi_1^{\alpha}(\r_1) \alpha(\omega_1) \psi_1^{\beta}(\r_2) s_-(2) \alpha(\omega_2)
	\end{array}
	\right\} \\
%
%
&=
	\frac{1}{\sqrt{2}}
	\left\{
	\begin{array}{l}
		-
		\psi_1^{\alpha}(\r_2) \alpha(\omega_2) \psi_1^{\beta}(\r_1) \beta(\omega_1)
		+
		\psi_1^{\alpha}(\r_1) \beta(\omega_1) \psi_1^{\beta}(\r_2) \alpha(\omega_2) \\[2mm]
		-
		\psi_1^{\alpha}(\r_2) \beta(\omega_2) \psi_1^{\beta}(\r_1) \alpha(\omega_1)
		+
		\psi_1^{\alpha}(\r_1) \alpha(\omega_1) \psi_1^{\beta}(\r_2) \beta(\omega_2)
	\end{array}
	\right\} \\
%
%
&=
	\frac{1}{\sqrt{2}}
	\left(
		\psi_1^{\alpha}(\r_1) \alpha(\omega_1) \psi_1^{\beta}(\r_2) \beta(\omega_2)
		-
		\psi_1^{\alpha}(\r_2) \alpha(\omega_2) \psi_1^{\beta}(\r_1) \beta(\omega_1)
	\right) \nonumber \\ &\quad
	+
	\frac{1}{\sqrt{2}}
	\left(
		\psi_1^{\alpha}(\r_1) \beta(\omega_1) \psi_1^{\beta}(\r_2) \alpha(\omega_2)
		-
		\psi_1^{\alpha}(\r_2) \beta(\omega_2) \psi_1^{\beta}(\r_1) \alpha(\omega_1)
	\right) \\
%
%
&=
	\ket{\aorb{\psi_1^{\alpha}}\borb{\psi_1^{\beta}}}
	+
	\ket{\borb{\psi_1^{\alpha}}\aorb{\psi_1^{\beta}}} \\
%
%
&=
	\ket{K}
	+
	\ket{L}
	%
	\qquad
	(
		\ket{L}=\ket{\borb{\psi_1^{\alpha}}\aorb{\psi_1^{\beta}}}
	)
\end{align}
となる。

$\ket{K}$が$\mathscr{S}^2$の固有関数であるためには
$\ket{K}$と$\ket{L}$が線形従属、
つまりは定数倍の関係になければならない。
一方で、$\ket{K},\ket{L}$は規格化されている。
従って、
\begin{align}
	\ket{K}
&=
	c \ket{L} \\
%
%
	\braket{K|K}
&=
	\bra{L} \conju{c} c \ket{L} \\
%
%
	1
&=
	\conju{c} c \\
%
%
	|c|
&=
	1
\end{align}


\subsection{(b)問}



\subsection{(b)解}

