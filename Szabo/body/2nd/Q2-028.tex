%ファイルID
%2020/06/10 21:42
%->2006102142SahayanKY(ファイル作成者)
\subsection{問}
$\ket{\Psi_0}=\ket{\chi_1\dots\chi_a\chi_b\dots\chi_N}$を
Hartree-Fock基底状態波動関数とする。
このとき、以下の等式を示せ。
\begin{align}
	\begin{array}{>{\rm}ll}
		(a) &
		a_r\ket{\Psi_0} = 0, \quad
		\bra{\Psi_0}\adj{a_r} = 0 \\
	%
		(b) &
		\adj{a_a}\ket{\Psi_0} = 0, \quad
		\bra{\Psi_0}a_a = 0 \\
	%
		(c) &
		\ket{\Psi_a^r} = \adj{a_r} a_a \ket{\Psi_0} \\
	%
		(d) &
		\bra{\Psi_a^r} = \bra{\Psi_0} \adj{a_a} a_r \\
	%
		(e) &
		\ket{\Psi_{ab}^{rs}} = \adj{a_s} a_b \adj{a_r} a_a \ket{\Psi_0} = \adj{a_r} \adj{a_s} a_b a_a \ket{\Psi_0} \\
	%
		(f) &
		\bra{\Psi_{ab}^{rs}} = \bra{\Psi_0} \adj{a_a} a_r \adj{a_b} a_s = \bra{\Psi_0} \adj{a_a} \adj{a_b} a_s a_r
	\end{array}
\end{align}


\subsection{解}
(a)について考える。$\ket{\Psi_0}$では、$\chi_r$を電子は占有していない。
従って、消滅演算子$a_r$を作用さるとゼロになる。従って、
$a_r\ket{\Psi_0}=0$である。
また、この共役形を考えると、$\bra{\Psi_0}\adj{a_r}=0$となる。

(b)について考える。$\ket{\Psi_0}$では、既に$\chi_a$を電子は占有している。
従って、生成演算子$\adj{a_a}$を作用させるとゼロになる。
従って、$\adj{a_a}\ket{\Psi_0}=0$である。
また、この共役形を考えると、$\bra{\Psi_0}a_a=0$となる。

(c)について考える。
\begin{align}
	\adj{a_r} a_a \ket{\Psi_0}
&=
	\adj{a_r} a_a \ket{\chi_1\dots\chi_a\chi_b\dots\chi_N} \\
%
%
&=
	-
	\adj{a_r} a_a \ket{\chi_a\dots\chi_1\chi_b\dots\chi_N} \\
%
%
&=
	-
	\adj{a_r} \ket{\dots\chi_1\chi_b\dots\chi_N} \\
%
%
&=
	-
	\ket{\chi_r\dots\chi_1\chi_b\dots\chi_N} \\
%
%
&=
	\ket{\chi_1\dots\chi_r\chi_b\dots\chi_N} \\
%
%
&=
	\ket{\Psi_a^r}
\end{align}
である。

(d)について考える。
これは、(c)の共役形を考えればよく、
\begin{align}
	\bra{\Psi_a^r}
&=
	\bra{\Psi_0} \adj[1]{\adj{a_r} a_a} \\
%
%
&=
	\bra{\Psi_0} \adj{a_a} \adj{\adj{a_r}} \\
%
%
&=
	\bra{\Psi_0} \adj{a_a} a_r
\end{align}
となる。

(e)について考える。
\begin{align}
	\adj{a_s} a_b \adj{a_r} a_a \ket{\Psi_0}
&=
	\adj{a_s} a_b \ket{\Psi_a^r} \\
%
%
&=
	\adj{a_s} a_b \ket{\chi_1\dots\chi_r\chi_b\dots\chi_N} \\
%
%
&=
	-
	\adj{a_s} a_b \ket{\chi_b\dots\chi_r\chi_1\dots\chi_N} \\
%
%
&=
	-
	\adj{a_s} \ket{\dots\chi_r\chi_1\dots\chi_N} \\
%
%
&=
	-
	\ket{\chi_s\dots\chi_r\chi_1\dots\chi_N} \\
%
%
&=
	\ket{\chi_1\dots\chi_r\chi_s\dots\chi_N} \\
%
%
&=
	\ket{\Psi_{ab}^{rs}}
\end{align}
となる。2つ目の等号関係は$a_i\adj{a_j}+\adj{a_j}a_i=\delta_{ij}$と
$\adj{a_i}\adj{a_j}+\adj{a_j}\adj{a_i}=0$から導かれる。

(f)について考える。
これは(e)の共役形を考えればよく、
\begin{align}
	\bra{\Psi_{ab}^{rs}}
&=
	\bra{\Psi_0} \adj[1]{\adj{a_s}a_b\adj{a_r}a_a} \\
%
%
&=
	\bra{\Psi_0} \adj{a_a} \adj[1]{\adj{a_s}a_b\adj{a_r}} \\
%
%
&=
	\bra{\Psi_0} \adj{a_a} \adj{\adj{a_r}} \adj[1]{\adj{a_s}a_b} \\
%
%
&=
	\bra{\Psi_0} \adj{a_a} a_r \adj{a_b} a_s
\end{align}
となる。2つ目の等号関係についても同様である。


