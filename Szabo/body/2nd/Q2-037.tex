%ファイルID
%2020/06/18 08:53
%->2006180853SahayanKY(ファイル作成者)
\subsection{問}
多電子系におけるスピン角運動量の$z$成分の演算子は
\begin{align}
	\mathscr{S}_z
&=
	\sum_i^N
		s_z(i)
\end{align}
である。
任意の$N$電子Slater行列式$\ket{\chi_i\chi_j\dots\chi_k}$に対して
\begin{align}
	\mathscr{S}_z \ket{\chi_i\chi_j\dots\chi_k}
&=
	\frac{1}{2}
	(N^{\alpha} -N^{\beta})
	\ket{\chi_i\chi_j\dots\chi_k}
\end{align}
となることを示せ。
ここで$N^{\alpha}$は$\alpha$スピンをもつスピン軌道の数であり、
$N^{\beta}$は$\beta$スピンをもつスピン軌道の数である。

\subsection{解}
$N$電子Slater行列式は
\begin{align}
	\ket{\chi_i\chi_j\dots\chi_k}
&=
	\frac{1}{\sqrt{N!}}
	\sum_{n=1}^{N!}
		(-1)^{p_n}
		\mathscr{P}_n\{
			\chi_i(1) \chi_j(2) \dots \chi_k(N)
		\}
\end{align}
である。
これに$\mathscr{S}_z$を作用させる前に、
$\mathscr{S}_z$と$\mathscr{P}_n$の関係について考える。
\begin{align}
	\mathscr{P}_n\{\mathscr{S}_z\}
&=
	\mathscr{P}_n\left\{
		\sum_i^N s_z(i)
	\right\} \\
%
%
&=
	\sum_i^N s_z(\mathscr{P}_n(i)) \\
%
%
&=
	\sum_i^N s_z(i) \\
%
%
&=
	\mathscr{S}_z
\end{align}
である。つまり、$\mathscr{S}_z$は$\mathscr{P}_n$に関して不変である。
よって、
\begin{align}
	\mathscr{S}_z \ket{\chi_i\chi_j\dots\chi_k}
&=
	\mathscr{S}_z\left\{
		\frac{1}{\sqrt{N!}}
		\sum_{n=1}^{N!}
			(-1)^{p_n}
			\mathscr{P}_n\{
				\chi_i(1) \chi_j(2) \dots \chi_k(N)
			\}
	\right\} \\
%
%
&=
	\frac{1}{\sqrt{N!}}
	\sum_{n=1}^{N!}
		(-1)^{p_n}
		\mathscr{S}_z
		\mathscr{P}_n\{
			\chi_i(1) \chi_j(2) \dots \chi_k(N)
		\} \\
%
%
&=
	\frac{1}{\sqrt{N!}}
	\sum_{n=1}^{N!}
		(-1)^{p_n}
		\mathscr{P}_n\{\mathscr{S}_z\}
		\mathscr{P}_n\{
			\chi_i(1) \chi_j(2) \dots \chi_k(N)
		\} \\
%
%
&=
	\frac{1}{\sqrt{N!}}
	\sum_{n=1}^{N!}
		(-1)^{p_n}
		\mathscr{P}_n\Bigl\{
			\mathscr{S}_z\{
				\chi_i(1) \chi_j(2) \dots \chi_k(N)
			\}
		\Bigr\}
\end{align}
となる。

$\mathscr{S}_z\{\chi_i(1) \chi_j(2) \dots \chi_k(N)\}$については
\begin{align}
	\mathscr{S}_z\{
		\chi_i(1) \chi_j(2) \dots \chi_k(N)
	\}
&=
	\sum_n^N
		s_z(n)\{
			\chi_i(1) \chi_j(2) \dots \chi_k(N)
		\} \\
%
%
&=
	(s_z(1) \chi_i(1)) \chi_j(2) \dots \chi_k(N) \nonumber \\ &\quad
	+
	\chi_i(1) (s_z(2) \chi_j(2)) \dots \chi_k(N) \nonumber \\ &\quad
	+
	\dots \nonumber \\ &\quad
	+
	\chi_i(1) \chi_j(2) \dots (s_z(N) \chi_k(N))
\end{align}
であり、
\begin{align}
	\chi_i(\x_n)
=
	\psi_{i'}(\r_n) \alpha(\omega_n)
\Rightarrow
	s_z(\omega_n) \chi_i(\x_n)
=
	\psi_{i'}(\r_n) s_z(\omega_n) \alpha(\omega_n)
=
	\frac{1}{2} \psi_{i'}(\r_n) \alpha(\omega_n)
=
	\frac{1}{2} \chi_i(\x_n) \\
%
%
	\chi_i(\x_n)
=
	\psi_{i'}(\r_n) \beta(\omega_n)
\Rightarrow
	s_z(\omega_n) \chi_i(\x_n)
=
	\psi_{i'}(\r_n) s_z(\omega_n) \beta(\omega_n)
=
	-
	\frac{1}{2} \psi_{i'}(\r_n) \beta(\omega_n)
=
	-
	\frac{1}{2} \chi_i(\x_n)
\end{align}
であることから、
\begin{align}
	\mathscr{S}_z\{
		\chi_i(1) \chi_j(2) \dots \chi_k(N)
	\}
&=
	\frac{1}{2} N^{\alpha}
	\chi_i(1) \chi_j(2) \dots \chi_k(N)
	-
	\frac{1}{2} N^{\beta}
	\chi_i(1) \chi_j(2) \dots \chi_k(N) \\
%
%
&=
	\frac{1}{2}
	(N^{\alpha} -N^{\beta})
	\chi_i(1) \chi_j(2) \dots \chi_k(N)
\end{align}
となる。

従って、
\begin{align}
	\mathscr{S}_z\ket{\chi_i\chi_j\dots\chi_k}
&=
	\frac{1}{\sqrt{N!}}
	\sum_{n=1}^{N!}
		(-1)^{p_n}
		\mathscr{P}_n\Bigl\{
			\mathscr{S}_z\{
				\chi_i(1) \chi_j(2) \dots \chi_k(N)
			\}
		\Bigr\} \\
%
%
&=
	\frac{1}{\sqrt{N!}}
	\sum_{n=1}^{N!}
		(-1)^{p_n}
		\mathscr{P}_n\left\{
			\frac{1}{2}
			(N^{\alpha} -N^{\beta})
			\chi_i(1) \chi_j(2) \dots \chi_k(N)
		\right\} \\
%
%
&=
	\frac{1}{2}
	(N^{\alpha} -N^{\beta}) \cdot
	\frac{1}{\sqrt{N!}}
	\sum_{n=1}^{N!}
		(-1)^{p_n}
		\mathscr{P}_n\{
			\chi_i(1) \chi_j(2) \dots \chi_k(N)
		\} \\
%
%
&=
	\frac{1}{2}
	(N^{\alpha} -N^{\beta})
	\ket{\chi_i\chi_j\dots\chi_k}
\end{align}
となる。


