%ファイルID
%2020/05/05 08:18
%->2005050818(10進数)->X5R742(ファイル作成者)
%->X5R742SahayanKY
\subsection{問}
以下の2式を導出せよ。
\begin{align}
	\braket{\Psi_{12}^{34}|\Oone|\Psi_{12}^{34}}
&=
	\braket{3|h|3}
	+
	\braket{4|h|4} \\
%
%
	\braket{\Psi_0|\Oone|\Psi_{12}^{34}}
&=
	\braket{\Psi_{12}^{34}|\Oone|\Psi_0}
=
	0
\end{align}
ここで、$\Oone$は
\begin{align}
	\Oone
&=
	h(1)
	+
	h(2)
=
	\sum_{i=1}^{2}\left(
		-
		\frac{1}{2} \laplacian{}_{i}
		-
		\sum_A \frac{Z_A}{r_{iA}}
	\right)
\end{align}
である。


\subsection{解}
まず1式目を導出する。
\begin{align}
&\quad
	\braket{\Psi_{12}^{34}|\Oone|\Psi_{12}^{34}} \nonumber \\
&=
	\braket{\chi_3\chi_4|\Oone|\chi_3\chi_4} \\
%
%
&=
	\int \d\x_1 \int \d\x_2
		\conju[1]{
			\frac{1}{\sqrt{2}}\left(
				\chi_3(\x_1) \chi_4(\x_2)
				-
				\chi_4(\x_1) \chi_3(\x_2)
			\right)
		}
			(h(\r_1)+h(\r_2)) %\nonumber \\ &\qquad\qquad\qquad\qquad \cdot
			\frac{1}{\sqrt{2}}\left(
				\chi_3(\x_1) \chi_4(\x_2)
				-
				\chi_4(\x_1) \chi_3(\x_2)
			\right) \\
%
%
&=
	\frac{1}{2}
		\left(
		\begin{array}{>{\displaystyle}l}
			\int \d\x_1
				\conju{\chi_3}(\x_1) h(\r_1) \chi_3(\x_1)
			\int \d\x_2
				\conju{\chi_4}(\x_2) \chi_4(\x_2)
			-
			\int \d\x_1
				\conju{\chi_3}(\x_1) h(\r_1) \chi_4(\x_1)
			\int \d\x_2
				\conju{\chi_4}(\x_2) \chi_3(\x_2) \\
			+
			\int \d\x_1
				\conju{\chi_3}(\x_1) \chi_3(\x_1)
			\int \d\x_2
				\conju{\chi_4}(\x_2) h(\r_2) \chi_4(\x_2)
			-
			\int \d\x_1
				\conju{\chi_3}(\x_1) \chi_4(\x_1)
			\int \d\x_2
				\conju{\chi_4}(\x_2) h(\r_2) \chi_3(\x_2) \\
			-
			\int \d\x_1
				\conju{\chi_4}(\x_1) h(\r_1) \chi_3(\x_1)
			\int \d\x_2
				\conju{\chi_3}(\x_2) \chi_4(\x_2)
			+
			\int \d\x_1
				\conju{\chi_4}(\x_1) h(\r_1) \chi_4(\x_1)
			\int \d\x_2
				\conju{\chi_3}(\x_2) \chi_3(\x_2) \\
			-
			\int \d\x_1
				\conju{\chi_4}(\x_1) \chi_3(\x_1)
			\int \d\x_2
				\conju{\chi_3}(\x_2) h(\r_2) \chi_4(\x_2)
			+
			\int \d\x_1
				\conju{\chi_4}(\x_1) \chi_4(\x_1)
			\int \d\x_2
				\conju{\chi_3}(\x_2) h(\r_2) \chi_3(\x_2)
		\end{array}
		\right) \\
%
%
&=
	\frac{1}{2}
		\left(
		\begin{array}{>{\displaystyle}l}
			\braket{\chi_3|h|\chi_3} \cdot 1
			-
			\braket{\chi_3|h|\chi_4} \cdot 0 \\
			+
			1 \cdot \braket{\chi_4|h|\chi_4}
			-
			0 \cdot \braket{\chi_4|h|\chi_3} \\
			-
			\braket{\chi_4|h|\chi_3} \cdot 0
			+
			\braket{\chi_4|h|\chi_4} \cdot 1 \\
			-
			0 \cdot \braket{\chi_3|h|\chi_4}
			+
			1 \cdot \braket{\chi_3|h|\chi_3}
		\end{array}
		\right) \\
%
%
&=
	\braket{\chi_3|h|\chi_3}
	+
	\braket{\chi_4|h|\chi_4} \\
%
%
&=
	\braket{3|h|3}
	+
	\braket{4|h|4}
\end{align}
である。

2式目について導出する。
\begin{align}
&\quad
	\braket{\Psi_0|\Oone|\Psi_{12}^{34}} \nonumber \\
%
%
&=
	\braket{12|\Oone|34} \\
%
%
&=
	\int \d\x_1 \int \d\x_2
		\conju[1]{
			\frac{1}{\sqrt{2}}
			\left(
				\chi_1(\x_1) \chi_2(\x_2)
				-
				\chi_2(\x_1) \chi_1(\x_2)
			\right)
		}
			(h(\r_1)+h(\r_2))
			\frac{1}{\sqrt{2}}
			\left(
				\chi_3(\x_1) \chi_4(\x_2)
				-
				\chi_4(\x_1) \chi_3(\x_2)
			\right) \\
%
%
&=
	\frac{1}{2}
		\left(
		\begin{array}{>{\displaystyle}l}
			\braket{1|h|3} \braket{2|4}
			-
			\braket{1|h|4} \braket{2|3}
			+
			\braket{1|3} \braket{2|h|4}
			-
			\braket{1|4} \braket{2|h|3} \\
			-
			\braket{2|h|3} \braket{1|4}
			+
			\braket{2|h|4} \braket{1|3}
			-
			\braket{2|3} \braket{1|h|4}
			+
			\braket{2|4} \braket{1|h|3}
		\end{array}
		\right) \\
%
%
&=
	\frac{1}{2}
		\left(
		\begin{array}{>{\displaystyle}l}
			\braket{1|h|3} \cdot 0
			-
			\braket{1|h|4} \cdot 0
			+
			0 \cdot \braket{2|h|4}
			-
			0 \cdot \braket{2|h|3} \\
			-
			\braket{2|h|3} \cdot 0
			+
			\braket{2|h|4} \cdot 0
			-
			0 \cdot \braket{1|h|4}
			+
			0 \cdot \braket{1|h|3}
		\end{array}
		\right) \\
%
%
&=
	0
\end{align}
また、$\Oone$はエルミート演算子であるので、
\begin{align}
	\braket{\Psi_{12}^{34}|\Oone|\Psi_0}
&=
	\conju{
		\braket{\Psi_0|\Oone|\Psi_{12}^{34}}
	} \\
%
%
&=
	0
\end{align}
である。
