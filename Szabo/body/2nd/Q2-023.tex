%ファイルID
%2020/06/09 17:01
%->2006091701SahayanKY(ファイル作成者)
\subsection{問}
次のSlater行列式が示しているエネルギーになることを確かめよ。
\begin{align}
	\begin{array}{>{\rm}ccc}
		(a) &
		%
		\ket{\aorb{1}\aorb{2}} &
		%
		h_{11} +h_{22} +J_{12} -K_{12} \\
	%
		(b) &
		%
		\ket{\borb{1}\aorb{2}} &
		%
		h_{11} +h_{22} +J_{12} \\
	%
		(c) &
		%
		\ket{\aorb{1}\borb{1}} &
		%
		2h_{11} +J_{11} \\
	%
		(d) &
		%
		\ket{\aorb{2}\borb{2}} &
		%
		2h_{22} +J_{22} \\
	%
		(e) &
		%
		\ket{\aorb{1}\borb{1}\aorb{2}} &
		%
		2h_{11} +h_{22} +J_{11} +2J_{12} -K_{12} \\
	%
		(f) &
		%
		\ket{\aorb{1}\aorb{2}\borb{2}} &
		%
		2h_{22} +h_{11} +J_{22} +2J_{12} -K_{12} \\
	%
		(g) &
		%
		\ket{\aorb{1}\borb{1}\aorb{2}\borb{2}} &
		%
		2h_{11} +2h_{22} +J_{11} +J_{22} +4J_{12} -2K_{12}
	\end{array}
	\nonumber
\end{align}

\subsection{解}
自明なので省略
