%ファイルID
%2020/04/29 11:22
%->2004291122(10進数)->X5AWXE(ファイル作成者)
%->X5AWXESahayanKY
\subsection{問}
多電子系において
電子間の相互作用を無視(もしくは平均化)するとき、
ハミルトニアン$\mathscr{H}$は
\begin{align}
	\mathscr{H}
&=
	\sum_{i=1}^N h(i)
\end{align}
と置ける。ここで$h(i)$は電子$i$の運動エネルギーとポテンシャルを
表す演算子である。$h(i)$の固有関数をスピン軌道$\chi_j(\x_i)$とすると、
\begin{align}
	h(i) \chi_j(\x_i)
&=
	\epsilon_j \chi_j(\x_i)
\end{align}
となる。

多電子の波動関数$\Psi^{\rm HP}(\x_1,\x_2,\cdots,\x_N)$を
次の通りに置く。
\begin{align}
	\Psi^{\rm HP}(\x_1,\x_2,\cdots,\x_N)
&=
	\chi_i(\x_1)\chi_j(\x_2)\cdots\chi_k(\x_N)
\end{align}
これはHartree積と呼ばれる。
これが$\mathscr{H}$の固有関数であり、
その固有値は$E=\epsilon_i+\epsilon_j+\cdots+\epsilon_k$であることを示せ。

\subsection{解}
\begin{align}
	\mathscr{H}\Psi^{\rm HP}
&=
	\sum_{i'=1}^N
		h(i') \chi_i(\x_1) \chi_j(\x_2) \cdots \chi_k(\x_N) \\
%
%
&=
	h(1) \chi_i(\x_1) \chi_j(\x_2) \cdots \chi_k(\x_N) \nonumber\\&\quad
	+
	h(2) \chi_i(\x_1) \chi_j(\x_2) \cdots \chi_k(\x_N) \nonumber\\&\quad
	+
	\cdots
	+
	h(N) \chi_i(\x_1) \chi_j(\x_2) \cdots \chi_k(\x_N) \\[2mm]
%
%
&=
	(h(1) \chi_i(\x_1)) \chi_j(\x_2) \cdots \chi_k(\x_N) \nonumber\\&\quad
	+
	\chi_i(\x_1) (h(2)\chi_j(\x_2)) \cdots \chi_k(\x_N) \nonumber\\&\quad
	+
	\cdots
	+
	\chi_i(\x_1) \chi_j(\x_2) \cdots (h(N)\chi_k(\x_N)) \\[2mm]
%
%
&=
	(\epsilon_1 \chi_i(\x_1)) \chi_j(\x_2) \cdots \chi_k(\x_N) \nonumber\\&\quad
	+
	\chi_i(\x_1) (\epsilon_2\chi_j(\x_2)) \cdots \chi_k(\x_N) \nonumber\\&\quad
	+
	\cdots
	+
	\chi_i(\x_1) \chi_j(\x_2) \cdots (\epsilon_N\chi_k(\x_N)) \\[2mm]
%
%
&=
	(\epsilon_1+\epsilon_2+\cdots+\epsilon_N)
		\chi_i(\x_1)\chi_j(\x_2)\cdots\chi_k(\x_N) \\
%
%
&=
	E \Psi^{\rm HP}
\end{align}
である。したがって、
$\Psi^{\rm HP}$は$\mathscr{H}$の固有関数であり、
固有値は$E=\epsilon_1+\epsilon_2+\cdots+\epsilon_N$である。

