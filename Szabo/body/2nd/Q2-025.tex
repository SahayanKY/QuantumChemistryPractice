%ファイルID
%2020/06/09 23:57
%->2006092357SahayanKY(ファイル作成者)
\subsection{問}
生成演算子$\adj{a_i}$と消滅演算子$a_i$がもつ次の性質
\begin{align}
	\left\{
	\begin{array}{l}
		(a_1\adj{a_2} +\adj{a_2}a_1) \ket{K}
		=
		0 \\
	%
		(a_1\adj{a_1} +\adj{a_1}a_1) \ket{K}
		=
		\ket{K}
	\end{array}
	\right.
\end{align}
を、
$\ket{K}=\ket{\chi_1\chi_2}, \ket{\chi_1\chi_3}, \ket{\chi_1\chi_4},
\ket{\chi_2\chi_3}, \ket{\chi_2\chi_4}, \ket{\chi_3\chi_4}$
について示せ。

\subsection{解}
1つ目の性質から見ていく。まず、$\ket{\chi_1\chi_2}$は
\begin{align}
	(a_1\adj{a_2} +\adj{a_2}a_1) \ket{\chi_1\chi_2}
&=
	a_1 \ket{\chi_2\chi_1\chi_2}
	+
	\adj{a_2} \ket{\chi_2} \\
%
%
&=
	0
	+
	\ket{\chi_2\chi_2} \\
%
%
&=
	0
\end{align}
である。次に、$\ket{\chi_1\chi_3}$では
\begin{align}
	(a_1\adj{a_2} +\adj{a_2}a_1) \ket{\chi_1\chi_3}
&=
	a_1 \ket{\chi_2\chi_1\chi_3}
	+
	\adj{a_2} \ket{\chi_3} \\
%
%
&=
	-
	\ket{\chi_2\chi_3}
	+
	\ket{\chi_2\chi_3} \\
%
%
&=
	0
\end{align}
となる。次に、$\ket{\chi_1\chi_4}$では
\begin{align}
	(a_1\adj{a_2} +\adj{a_2}a_1) \ket{\chi_1\chi_4}
&=
	a_1 \ket{\chi_2\chi_1\chi_4}
	+
	\adj{a_2} \ket{\chi_4} \\
%
%
&=
	-
	\ket{\chi_2\chi_4}
	+
	\ket{\chi_2\chi_4} \\
%
%
&=
	0
\end{align}
となる。次に、$\ket{\chi_2\chi_3}$では
\begin{align}
	(a_1\adj{a_2} +\adj{a_2}a_1) \ket{\chi_2\chi_3}
&=
	a_1 \ket{\chi_2\chi_2\chi_3}
	+
	\adj{a_2} 0 \\
%
%
&=
	0
\end{align}
となる。次に、$\ket{\chi_2\chi_4}$では
\begin{align}
	(a_1\adj{a_2} +\adj{a_2}a_1) \ket{\chi_2\chi_4}
&=
	a_1 \ket{\chi_2\chi_2\chi_4}
	+
	\adj{a_2} 0 \\
%
%
&=
	0
\end{align}
となる。最後に、$\ket{\chi_3\chi_4}$では
\begin{align}
	(a_1\adj{a_2} +\adj{a_2}a_1) \ket{\chi_3\chi_4}
&=
	a_1 \ket{\chi_2\chi_3\chi_4}
	+
	\adj{a_2} 0 \\
%
%
&=
	0
\end{align}
となる。

次に、2つ目の性質を見ていく。まず、$\ket{\chi_1\chi_2}$では
\begin{align}
	(a_1\adj{a_1} +\adj{a_1}a_1) \ket{\chi_1\chi_2}
&=
	a_1 \ket{\chi_1\chi_1\chi_2}
	+
	\adj{a_1} \ket{\chi_2} \\
%
%
&=
	0
	+
	\ket{\chi_1\chi_2} \\
%
%
&=
	\ket{\chi_1\chi_2}
\end{align}
となる。次に、$\ket{\chi_1\chi_3}$では
\begin{align}
	(a_1\adj{a_1} +\adj{a_1}a_1) \ket{\chi_1\chi_3}
&=
	a_1 \ket{\chi_1\chi_1\chi_3}
	+
	\adj{a_1} \ket{\chi_3} \\
%
%
&=
	0
	+
	\ket{\chi_1\chi_3} \\
%
%
&=
	\ket{\chi_1\chi_3}
\end{align}
となる。次に、$\ket{\chi_1\chi_4}$では
\begin{align}
	(a_1\adj{a_1} +\adj{a_1}a_1) \ket{\chi_1\chi_4}
&=
	a_1 \ket{\chi_1\chi_1\chi_4}
	+
	\adj{a_1} \ket{\chi_4} \\
%
%
&=
	0
	+
	\ket{\chi_1\chi_4} \\
%
%
&=
	\ket{\chi_1\chi_4}
\end{align}
となる。次に、$\ket{\chi_2\chi_3}$では
\begin{align}
	(a_1\adj{a_1} +\adj{a_1}a_1) \ket{\chi_2\chi_3}
&=
	a_1 \ket{\chi_1\chi_2\chi_3}
	+
	\adj{a_1} 0 \\
%
%
&=
	\ket{\chi_2\chi_3}
\end{align}
となる。次に、$\ket{\chi_2\chi_4}$では
\begin{align}
	(a_1\adj{a_1} +\adj{a_1}a_1) \ket{\chi_2\chi_4}
&=
	a_1 \ket{\chi_1\chi_2\chi_4}
	+
	\adj{a_1} 0 \\
%
%
&=
	\ket{\chi_2\chi_4}
\end{align}
となる。最後に、$\ket{\chi_3\chi_4}$では
\begin{align}
	(a_1\adj{a_1} +\adj{a_1}a_1) \ket{\chi_3\chi_4}
&=
	a_1 \ket{\chi_1\chi_3\chi_4}
	+
	\adj{a_1} 0 \\
%
%
&=
	\ket{\chi_3\chi_4}
\end{align}
となる。

従って、確かに上記の性質をもっている。



