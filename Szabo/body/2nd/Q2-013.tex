%ファイルID
%2020/05/21 15:15
%->2005211515(10進数)->X5UN3V(ファイル作成者)
%->X5UN3VSahayanKY
\subsection{問}
次式を示せ。
\begin{align}
	\braket{\Psi_{a}^{r}|\mathscr{O}_1|\Psi_{b}^{s}}
&=
	\left\{
	\begin{array}{>{\displaystyle}ll}
		0 					& (a\neq b, r\neq s) \\[1mm]
		%
		\braket{r|h|s}		& (a=b, r\neq s) \\[1mm]
		%
		-\braket{b|h|a}		& (a\neq b, r=s) \\[1mm]
		%
		\sum_c^N \braket{c|h|c}
		-
		\braket{a|h|a}
		+
		\braket{r|h|r}		& (a=b, r=s)
	\end{array}
	\right.
\end{align}


\subsection{解}
まず、$a\neq b,r\neq s$について考える。
この時、$\ket{\Psi_{a}^{r}}, \ket{\Psi_{b}^{s}}$は、
Hartree-Fock基底状態$\ket{\Psi_0}$に対して
\begin{align}
	\ket{\Psi_0}
&=
	\ket{\cdots a\cdots b\cdots} \\
%
%
	\ket{\Psi_{a}^{r}}
&=
	\ket{\cdots r\cdots b\cdots}
=
	(-1)^n \ket{\cdots rb\cdots} \\
%
%
	\ket{\Psi_{b}^{s}}
&=
	\ket{\cdots a\cdots s\cdots}
=
	(-1)^n \ket{\cdots as\cdots}
\end{align}
となる。ここで$n$は$b$を$a$の隣まで移動させるために行った置換の回数である。
これは表2.3におけるケース3に相当する。
$\ket{\cdots a\cdots b\cdots}$に対応するブラベクトルを
$\bra{\cdots a\cdots b\cdots}$とすると、
\begin{align}
	\braket{\Psi_{a}^{r}|\mathscr{O}_1|\Psi_{b}^{s}}
&=
	\braket{\cdots r\cdots b\cdots|\mathscr{O}_1|\cdots a\cdots s\cdots} \\
%
%
&=
	(-1)^{2n}
	\braket{\cdots rb\cdots|\mathscr{O}_1|\cdots as\cdots} \\
%
%
&=
	0
\end{align}
となる。

次に、$a=b, r\neq s$の場合について考える。
$\ket{\Psi_0},\ket{\Psi_{a}^{r}},\ket{\Psi_{b}^{s}}$は
\begin{align}
	\ket{\Psi_0}
&=
	\ket{\cdots a\cdots}
=
	\ket{\cdots b\cdots} \\
%
%
	\ket{\Psi_{a}^{r}}
&=
	\ket{\cdots r\cdots} \\
%
%
	\ket{\Psi_{b}^{s}}
&=
	\ket{\cdots s\cdots}
\end{align}
となる。これは表2.3のケース2に相当する。
従って、
\begin{align}
	\braket{\Psi_{a}^{r}|\mathscr{O}_1|\Psi_{b}^{s}}
&=
	\braket{\cdots r\cdots|\mathscr{O}_1|\cdots s\cdots} \\
%
%
&=
	\braket{r|h|s}
\end{align}
となる。

次に、$a\neq b, r=s$の場合について考える。
$\ket{\Psi_0},\ket{\Psi_{a}^{r}},\ket{\Psi_{b}^{s}}$は
\begin{align}
	\ket{\Psi_0}
&=
	\ket{\cdots a\cdots b\cdots} \\
%
%
	\ket{\Psi_{a}^{r}}
&=
	\ket{\cdots r\cdots b\cdots} \\
%
&=
	(-1)^n \ket{\cdots rb\cdots}
%
%
	\ket{\Psi_{b}^{s}}
&=
	\ket{\cdots a\cdots s\cdots}
=
	\ket{\cdots a\cdots r\cdots} \\
%
&=
	(-1)^n \ket{\cdots ar\cdots} \\
%
&=
	(-1)^{n+1} \ket{\cdots ra\cdots}
\end{align}
となる。これは表2.3のケース2に相当する。
従って、
\begin{align}
	\braket{\Psi_{a}^{r}|\mathscr{O}_1|\Psi_{b}^{s}}
&=
	\braket{\cdots r\cdots b\cdots|\mathscr{O}_1|\cdots a\cdots s\cdots} \\
%
%
&=
	(-1)^{2n+1}
	\braket{\cdots rb\cdots|\mathscr{O}_1|\cdots ra\cdots} \\
%
%
&=
	-\braket{b|h|a}
\end{align}
となる。

最後に$a=b, r=s$の場合について考える。
$\ket{\Psi_0},\ket{\Psi_{a}^{r}},\ket{\Psi_{b}^{s}}$は
\begin{align}
	\ket{\Psi_0}
&=
	\ket{\cdots a\cdots} \\
%
%
	\ket{\Psi_{a}^{r}}
&=
	\ket{\cdots r\cdots} \\
%
%
	\ket{\Psi_{b}^{s}}
&=
	\ket{\Psi_{a}^{r}}
=
	\ket{\cdots r\cdots}
\end{align}
となる。これは表2.3のケース1に相当する。
従って、
\begin{align}
	\braket{\Psi_{a}^{r}|\mathscr{O}_1|\Psi_{b}^{s}}
&=
	\braket{\cdots r\cdots|\mathscr{O}_1|\cdots r\cdots} \\
%
%
&=
	\sum_{c=\cdots,r,\cdots}
		\braket{c|h|c} \\
%
%
&=
	\sum_{c=\cdots,a,\cdots}
		\braket{c|h|c}
	+
	\braket{r|h|r}
	-
	\braket{a|h|a}
\end{align}
となる。

