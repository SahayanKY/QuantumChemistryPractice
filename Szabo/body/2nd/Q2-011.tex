%ファイルID
%2020/05/08 10:28
%->2005081028(10進数)->X5RUF8(ファイル作成者)
%->X5RUF8SahayanKY
\subsection{問}
$\ket{K}=\ket{\chi_1\chi_2\chi_3}$として、
\begin{align}
	\braket{K|\mathscr{H}|K}
&=
	\braket{1|h|1}
	+
	\braket{2|h|2}
	+
	\braket{3|h|3}
	+
	\antitwo{12}{12}
	+
	\antitwo{12}{13}
	+
	\antitwo{23}{23}
\end{align}
であることを示せ。
\begin{align}
	\braket{K|\mathscr{H}|K}
&=
	\braket{K|\mathscr{O}_1|K}
	+
	\braket{K|\mathscr{O}_2|K}
\end{align}
である。第1項については、
\begin{align}
	\braket{K|\mathscr{O}_1|K}
&=
	\sum_{m=1,2,3} \braket{m|h|m} \\
%
%
&=
	\braket{1|h|1}
	+
	\braket{2|h|2}
	+
	\braket{3|h|3}
\end{align}
である。第2項については
\begin{align}
	\braket{K|\mathscr{O}_2|K}
&=
	\frac{1}{2}
	\sum_{m=1,2,3}\ \sum_{n=1,2,3}
		\antitwo{mn}{mn} \\
%
%
&=
	\frac{1}{2}
	\left(
	\begin{array}{l}
		\antitwo{11}{11}
		+
		\antitwo{12}{12}
		+
		\antitwo{13}{13}
		+
		\antitwo{21}{21}
		+
		\antitwo{22}{22}
		+
		\antitwo{23}{23} \\ \quad
		+
		\antitwo{31}{31}
		+
		\antitwo{32}{32}
		+
		\antitwo{33}{33}
	\end{array}
	\right) \\
%
%
&=
	\frac{1}{2}
	\left(
		0
		+
		\antitwo{12}{12}
		+
		\antitwo{13}{13}
		+
		\antitwo{12}{12}
		+
		0
		+
		\antitwo{23}{23}
		+
		\antitwo{13}{13}
		+
		\antitwo{23}{23}
		+
		0
	\right) \\
%
%
&=
	\antitwo{12}{12}
	+
	\antitwo{13}{13}
	+
	\antitwo{23}{23}
\end{align}
従って、
\begin{align}
	\braket{K|\mathscr{H}|K}
&=
	\braket{1|h|1}
	+
	\braket{2|h|2}
	+
	\braket{3|h|3}
	+
	\antitwo{12}{12}
	+
	\antitwo{13}{13}
	+
	\antitwo{23}{23}
\end{align}
となる。


