%ファイルID
%2020/06/12 18:51
%->2006121851SahayanKY(ファイル作成者)
\subsection{(a)問}
スピン角運動量演算子を$\bm{s}$、
その各成分を$s_x,s_y,s_z$、
また、大きさの2乗を$s^2=s_x^2+s_y^2+s_z^2$とする。
このとき
\begin{align}
	s^2 \ket{\alpha}
&=
	\frac{3}{4} \ket{\alpha} &
%
	s^2 \ket{\beta}
&=
	\frac{3}{4} \ket{\beta} \\
%
%
	s_z \ket{\alpha}
&=
	\frac{1}{2} \ket{\alpha} &
%
	s_z \ket{\beta}
&=
	-\frac{1}{2} \ket{\beta} \\
%
%
	s_x \ket{\alpha}
&=
	\frac{1}{2} \ket{\beta} &
%
	s_x \ket{\beta}
&=
	\frac{1}{2} \ket{\alpha} \\
%
%
	s_y \ket{\alpha}
&=
	\frac{i}{2} \ket{\beta} &
%
	s_y \ket{\beta}
&=
	-\frac{i}{2} \ket{\alpha}
\end{align}
を利用して、
\begin{align}
	s_+ \ket{\alpha}
&=
	0 &
%
	s_+ \ket{\beta}
&=
	\ket{\alpha} \\
%
%
	s_- \ket{\alpha}
&=
	\ket{\beta} &
%
	s_- \ket{\beta}
&=
	0
\end{align}
であることを示せ。

ここで、$s_+=s_x+is_y, s_-=s_x-is_y$であり、
$\ket{\alpha}=\ket{\frac{1}{2},\frac{1}{2}},\
\ket{\beta}=\ket{\frac{1}{2},-\frac{1}{2}}$である。


\subsection{(a)解}
まず1つ目は
\begin{align}
	s_+ \ket{\alpha}
&=
	(s_x +is_y) \ket{\alpha} \\
%
%
&=
	s_x \ket{\alpha}
	+
	i s_y \ket{\alpha} \\
%
%
&=
	\frac{1}{2} \ket{\beta}
	+
	i \cdot\frac{i}{2} \ket{\beta} \\
%
%
&=
	0
\end{align}
である。

次に2つ目は
\begin{align}
	s_+ \ket{\beta}
&=
	(s_x +is_y) \ket{\beta} \\
%
%
&=
	s_x \ket{\beta}
	+
	i s_y \ket{\beta} \\
%
%
&=
	\frac{1}{2} \ket{\alpha}
	+
	i \cdot\left(-\frac{i}{2}\ket{\alpha}\right) \\
%
%
&=
	\ket{\alpha}
\end{align}
である。

次に3つ目については
\begin{align}
	s_- \ket{\alpha}
&=
	(s_x -i s_y) \ket{\alpha} \\
%
%
&=
	s_x \ket{\alpha}
	-
	i s_y \ket{\alpha} \\
%
%
&=
	\frac{1}{2} \ket{\beta}
	-
	i \cdot \frac{i}{2} \ket{\beta} \\
%
%
&=
	\ket{\beta}
\end{align}
である。

最後に4つ目については
\begin{align}
	s_- \ket{\beta}
&=
	(s_x -i s_y) \ket{\beta} \\
%
%
&=
	s_x \ket{\beta}
	-
	i s_y \ket{\beta} \\
%
%
&=
	\frac{1}{2} \ket{\alpha}
	-
	i \cdot\left(-\frac{i}{2}\ket{\alpha}\right) \\
%
%
&=
	0
\end{align}
である。


\subsection{(b)問}
次の式を導出せよ。
\begin{align}
	s^2
&=
	s_+ s_-
	-
	s_z
	+
	s_z^2 \\
%
%
	s^2
&=
	s_- s_+
	+
	s_z
	+
	s_z^2
\end{align}


\subsection{(b)解}
まず、1式目については
\begin{align}
	s_+ s_-
	-
	s_z
	+
	s_z^2
&=
	(s_x +is_y)
	(s_x -is_y)
	-
	s_z
	+
	s_z^2 \\
%
%
&=
	s_x^2
	-
	i s_x s_y
	+
	i s_y s_x
	+
	s_y^2
	-
	s_z
	+
	s_z^2 \\
%
%
&=
	s^2
	-
	i [s_x,s_y]
	-
	s_z \\
%
%
&=
	s^2
	-
	i \cdot is_z
	-
	s_z
	%
	\qquad
	(\because [s_x,s_y]=is_z) \\
%
%
&=
	s^2
\end{align}
である。

同様に、2式目についても
\begin{align}
	s_- s_+
	+
	s_z
	+
	s_z^2
%
%
&=
	(s_x -is_y)
	(s_x +is_y)
	+
	s_z
	+
	s_z^2 \\
%
%
&=
	s_x^2
	+
	i s_x s_y
	-
	i s_y s_x
	+
	s_y^2
	+
	s_z
	+
	s_z^2 \\
%
%
&=
	s^2
	+
	i [s_x,s_y]
	+
	s_z \\
%
%
&=
	s^2
	+
	i \cdot i s_z
	+
	s_z \\
%
%
&=
	s^2
\end{align}
である。



