%ファイルID
%2020/06/03 13:17
%->2006031317(10進数)->(36進数)(ファイル作成者)
%->X6C7O5SahayanKY
\subsection{問}
最小基底関数系を用いた\ce{H2}の完全CI行列$H$は
\begin{align}
	H
&=
	\left[
	\begin{array}{cc}
		\braket{1|h|1}
		+
		\braket{2|h|2}
		+
		\antitwo{12}{12} &
		%
		\antitwo{12}{34} \\
	%
		\antitwo{34}{12} &
		%
		\braket{3|h|3}
		+
		\braket{4|h|4}
		+
		\antitwo{34}{34}
	\end{array}
	\right]
\end{align}
である。これの各要素について、
スピンについて積分すると
\begin{align}
	H
&=
	\left[
	\begin{array}{cc}
		2\spaceone{1}{h}{1}
		+
		\spacetwo{11}{11} &
		%
		\spacetwo{12}{12} \\
	%
		\spacetwo{21}{21} &
		%
		2\spaceone{2}{h}{2}
		+
		\spacetwo{22}{22}
	\end{array}
	\right]
\end{align}
となることを示せ。

ここで、$\spaceone{i}{h}{j},\spacetwo{ij}{kl}$は
\begin{align}
	\spaceone{i}{h}{j}
&=
	\int\d\r_1
		\conju{\psi_i}(\r_1) h(\r_1) \psi_j(\r_1) \\
%
%
	\spacetwo{ij}{kl}
&=
	\int\d\r_1\d\r_2
		\conju{\psi_i}(\r_1) \psi_j(\r_1) r_{12}^{-1} \conju{\psi_k}(\r_2) \psi_l(\r_2)
\end{align}
である。


\subsection{解}
まず、スピン軌道のラベリングから、
空間軌道のラベリングに変換すると次の通りになる。
\begin{align}
	H
&=
	\left[
	\begin{array}{cc}
		\braket{\aorb{1}|h|\aorb{1}}
		+
		\braket{\borb{1}|h|\borb{1}}
		+
		\antitwo{\aorb{1}\borb{1}}{\aorb{1}\borb{1}} &
		%
		\antitwo{\aorb{1}\borb{1}}{\aorb{2}\borb{2}} \\
	%
		\antitwo{\aorb{2}\borb{2}}{\aorb{1}\borb{1}} &
		%
		\braket{\aorb{2}|h|\aorb{2}}
		+
		\braket{\borb{2}|h|\borb{2}}
		+
		\antitwo{\aorb{2}\borb{2}}{\aorb{2}\borb{2}}
	\end{array}
	\right]
\end{align}

1電子積分から見ていく。
\begin{align}
	\braket{\aorb{1}|h|\aorb{1}}
&=
	\int\d\r_1\d\omega_1\
		\conju{\psi_1}(\r_1) \conju{\alpha}(\omega_1)
		h(\r_1)
		\psi_1(\r_1) \alpha(\omega_1) \\
%
%
&=
	\int\d\r_1\
		\conju{\psi_1}(\r_1)
		h(\r_1)
		\psi_1(\r_1) \\
%
%
&=
	\spaceone{1}{h}{1}
\end{align}
\begin{align}
	\braket{\borb{1}|h|\borb{1}}
&=
	\int\d\r_1\d\omega_1\
		\conju{\psi_1}(\r_1) \conju{\beta}(\omega_1)
		h(\r_1)
		\psi_1(\r_1) \beta(\omega_1) \\
%
%
&=
	\int\d\r_1\
		\conju{\psi_1}(\r_1)
		h(\r_1)
		\psi_1(\r_1) \\
%
%
&=
	\spaceone{1}{h}{1}
\end{align}
\begin{align}
	\braket{\aorb{2}|h|\aorb{2}}
&=
	\int\d\r_1\d\omega_1\
		\conju{\psi_2}(\r_1) \conju{\alpha}(\omega_1)
		h(\r_1)
		\psi_2(\r_1) \alpha(\omega_1) \\
%
%
&=
	\int\d\r_1\
		\conju{\psi_2}(\r_1)
		h(\r_1)
		\psi_2(\r_1) \\
%
%
&=
	\spaceone{2}{h}{2}
\end{align}
\begin{align}
	\braket{\borb{2}|h|\borb{2}}
&=
	\int\d\r_1\d\omega_1\
		\conju{\psi_2}(\r_1) \conju{\beta}(\omega_1)
		h(\r_1)
		\psi_2(\r_1) \beta(\omega_1) \\
%
%
&=
	\int\d\r_1\
		\conju{\psi_2}(\r_1)
		h(\r_1)
		\psi_2(\r_2) \\
%
%
&=
	\spaceone{2}{h}{2}
\end{align}
である。

次に、2電子積分を見ていく。
\begin{align}
	\phystwo{\aorb{1}\borb{1}}{\aorb{1}\borb{1}}
&=
	\int\d\r_1\d\omega_1 \d\r_2\d\omega_2\
		\conju{\psi_1}(\r_1) \conju{\alpha}(\omega_1)
		\conju{\psi_1}(\r_2) \conju{\beta}(\omega_2)
		r_{12}^{-1}
		\psi_1(\r_1) \alpha(\omega_1)
		\psi_1(\r_2) \beta(\omega_2) \\
%
%
&=
	\int\d\r_1 \d\r_2\
		\conju{\psi_1}(\r_1)
		\conju{\psi_1}(\r_2)
		r_{12}^{-1}
		\psi_1(\r_1)
		\psi_1(\r_2) \\
%
%
&=
	\int\d\r_1 \d\r_2\
		\conju{\psi_1}(\r_1)
		\psi_1(\r_1)
		r_{12}^{-1}
		\conju{\psi_1}(\r_2)
		\psi_1(\r_2) \\
%
%
&=
	\spacetwo{11}{11}
\end{align}
\begin{align}
	\phystwo{\aorb{1}\borb{1}}{\borb{1}\aorb{1}}
&=
	\int\d\r_1\d\omega_1 \d\r_2\d\omega_2\
		\conju{\psi_1}(\r_1) \conju{\alpha}(\omega_1)
		\conju{\psi_1}(\r_2) \conju{\beta}(\omega_2)
		r_{12}^{-1}
		\psi_1(\r_1) \beta(\omega_1)
		\psi_1(\r_2) \alpha(\omega_2) \\
%
%
&=
	0
\end{align}
\begin{align}
	\therefore
	\antitwo{\aorb{1}\borb{1}}{\aorb{1}\borb{1}}
&=
	\phystwo{\aorb{1}\borb{1}}{\aorb{1}\borb{1}}
	-
	\phystwo{\aorb{1}\borb{1}}{\borb{1}\aorb{1}} \\
%
%
&=
	\spacetwo{11}{11}
\end{align}
%------------------------------------------------------------------------------
\begin{align}
	\phystwo{\aorb{1}\borb{1}}{\aorb{2}\borb{2}}
&=
	\int\d\r_1\d\omega_1 \d\r_2\d\omega_2\
		\conju{\psi_1}(\r_1) \conju{\alpha}(\omega_1)
		\conju{\psi_1}(\r_2) \conju{\beta}(\omega_2)
		r_{12}^{-1}
		\psi_2(\r_1) \alpha(\omega_1)
		\psi_2(\r_2) \beta(\omega_2) \\
%
%
&=
	\int\d\r_1 \d\r_2\
		\conju{\psi_1}(\r_1)
		\conju{\psi_1}(\r_2)
		r_{12}^{-1}
		\psi_2(\r_1)
		\psi_2(\r_2) \\
%
%
&=
	\int\d\r_1 \d\r_2\
		\conju{\psi_1}(\r_1)
		\psi_2(\r_1)
		r_{12}^{-1}
		\conju{\psi_1}(\r_2)
		\psi_2(\r_2) \\
%
%
&=
	\spacetwo{12}{12}
\end{align}
\begin{align}
	\phystwo{\aorb{1}\borb{1}}{\borb{2}\aorb{2}}
&=
	\int\d\r_1\d\omega_1 \d\r_2\d\omega_2\
		\conju{\psi_1}(\r_1) \conju{\alpha}(\omega_1)
		\conju{\psi_1}(\r_2) \conju{\beta}(\omega_2)
		r_{12}^{-1}
		\psi_2(\r_1) \beta(\omega_1)
		\psi_2(\r_2) \alpha(\omega_2) \\
%
%
&=
	0
\end{align}
\begin{align}
	\therefore
	\antitwo{\aorb{1}\borb{1}}{\aorb{2}\borb{2}}
&=
	\phystwo{\aorb{1}\borb{1}}{\aorb{2}\borb{2}}
	-
	\phystwo{\aorb{1}\borb{1}}{\borb{2}\aorb{2}} \\
%
%
&=
	\spacetwo{12}{12}
\end{align}
%------------------------------------------------------------------
\begin{align}
	\antitwo{\aorb{2}\borb{2}}{\aorb{1}\borb{1}}
&=
	\phystwo{\aorb{2}\borb{2}}{\aorb{1}\borb{1}}
	-
	\phystwo{\aorb{2}\borb{2}}{\borb{1}\aorb{1}} \\
%
%
&=
	\conju{
		\phystwo{\aorb{1}\borb{1}}{\aorb{2}\borb{2}}
	}
	-
	\conju{
		\phystwo{\aorb{1}\borb{1}}{\borb{2}\aorb{2}}
	} \\
%
%
&=
	\conju{
		\spacetwo{12}{12}
	}
	-
	\conju{
		0
	} \\
%
%
&=
	\int\d\r_1 \d\r_2\
		\conju{\psi_2}(\r_1)
		\psi_1(\r_1)
		r_{12}^{-1}
		\conju{\psi_2}(\r_2)
		\psi_1(\r_2) \\
%
%
	\therefore
	\antitwo{\aorb{2}\borb{2}}{\aorb{1}\borb{1}}
&=
	\spacetwo{21}{21}
\end{align}
%------------------------------------------------------------------
\begin{align}
	\phystwo{\aorb{2}\borb{2}}{\aorb{2}\borb{2}}
&=
	\int\d\r_1\d\omega_1 \d\r_2\d\omega_2\
		\conju{\psi_2}(\r_1) \conju{\alpha}(\omega_1)
		\conju{\psi_2}(\r_2) \conju{\beta}(\omega_2)
		r_{12}^{-1}
		\psi_2(\r_1) \alpha(\omega_1)
		\psi_2(\r_2) \beta(\omega_2) \\
%
%
&=
	\int\d\r_1 \d\r_2\
		\conju{\psi_2}(\r_1)
		\conju{\psi_2}(\r_2)
		r_{12}^{-1}
		\psi_2(\r_1)
		\psi_2(\r_2) \\
%
%
&=
	\int\d\r_1 \d\r_2\
		\conju{\psi_2}(\r_1)
		\psi_2(\r_1)
		r_{12}^{-1}
		\conju{\psi_2}(\r_2)
		\psi_2(\r_2) \\
%
%
&=
	\spacetwo{22}{22}
\end{align}
\begin{align}
	\phystwo{\aorb{2}\borb{2}}{\borb{2}\aorb{2}}
&=
	\int\d\r_1\d\omega_1 \d\r_2\d\omega_2\
		\conju{\psi_2}(\r_1) \conju{\alpha}(\omega_1)
		\conju{\psi_2}(\r_2) \conju{\beta}(\omega_2)
		r_{12}^{-1}
		\psi_2(\r_1) \beta(\omega_1)
		\psi_2(\r_2) \alpha(\omega_2) \\
%
%
&=
	0
\end{align}
\begin{align}
	\therefore
	\antitwo{\aorb{2}\borb{2}}{\aorb{2}\borb{2}}
&=
	\phystwo{\aorb{2}\borb{2}}{\aorb{2}\borb{2}}
	-
	\phystwo{\aorb{2}\borb{2}}{\borb{2}\aorb{2}} \\
%
%
&=
	\spacetwo{22}{22}
\end{align}
である。

従って、完全CI行列は
\begin{align}
	H
&=
	\left[
	\begin{array}{cc}
		\spaceone{1}{h}{1}
		+
		\spaceone{1}{h}{1}
		+
		\spacetwo{11}{11} &
		%
		\spacetwo{12}{12} \\
	%
		\spacetwo{21}{21} &
		%
		\spaceone{2}{h}{2}
		+
		\spaceone{2}{h}{2}
		+
		\spacetwo{22}{22}
	\end{array}
	\right] \\
%
%
&=
	\left[
	\begin{array}{cc}
		2\spaceone{1}{h}{1}
		+
		\spacetwo{11}{11} &
		%
		\spacetwo{12}{12} \\
	%
		\spacetwo{21}{21} &
		%
		2\spaceone{2}{h}{2}
		+
		\spacetwo{22}{22}
	\end{array}
	\right]
\end{align}


