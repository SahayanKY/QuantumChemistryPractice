%ファイルID
%2020/06/07 15:14
%->2006071514(10進数)->(36進数)(ファイル作成者)
%->X6D2OQSahayanKY
\subsection{問}
実の空間軌道に対して
\begin{align}
	K_{ij}
&=
	\spacetwo{ij}{ij}
=
	\spacetwo{ji}{ji} \\
%
&=
	\phystwo{ii}{jj}
=
	\phystwo{jj}{ii}
\end{align}
が成立することを示せ。


\subsection{解}
交換積分$K_{ij}$は
\begin{align}
	K_{ij}
&=
	\spacetwo{ij}{ji} \\
%
%
&=
	\int\d\r_1 \d\r_2
		\conju{\psi_i}(\r_1) \psi_j(\r_1)
		r_{12}^{-1}
		\conju{\psi_j}(\r_2) \psi_i(\r_2) \\
%
%
&=
	\int\d\r_1 \d\r_2
		\psi_i(\r_1) \psi_j(\r_1)
		r_{12}^{-1}
		\psi_j(\r_2) \psi_i(\r_2)
	%
	\qquad
	(\because \forall i\quad \conju{\psi_i}(\r)=\psi_i(\r)) \\
%
%
&=
	\int\d\r_1 \d\r_2
		\psi_i(\r_1) \psi_j(\r_1)
		r_{12}^{-1}
		\psi_i(\r_2) \psi_j(\r_2)
	%
	\label{eqX6D2OQSahayanKY_Kij_realspaceorbital} \\
%
%
&=
	\spacetwo{ij}{ij}
\end{align}
となる。

また、上式\ref{eqX6D2OQSahayanKY_Kij_realspaceorbital}より、
\begin{align}
	K_{ij}
&=
	\int\d\r_1 \d\r_2
		\psi_i(\r_1) \psi_j(\r_1)
		r_{12}^{-1}
		\psi_i(\r_2) \psi_j(\r_2) \\
%
%
&=
	\int\d\r_1 \d\r_2
		\psi_j(\r_1) \psi_i(\r_1)
		r_{12}^{-1}
		\psi_j(\r_2) \psi_i(\r_2) \\
%
%
&=
	\spacetwo{ji}{ji}
\end{align}
となる。

あと2つの関係については、$\spacetwo{ij}{kl}=\phystwo{ik}{jl}$より求められる。


