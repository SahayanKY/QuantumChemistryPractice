%ファイルID
%2020/06/16 17:17
%->2006161717SahayanKY(ファイル作成者)
\subsection{問}
演算子$\mathscr{A}$をハミルトニアン$\H$と可換な演算子とする。
また、演算子$\mathscr{A}$の非縮退な固有関数$\ket{\Psi_1}, \ket{\Psi_2}$を考える。
つまり、
\begin{align}
	\mathscr{A}\ket{\Psi_1}
&=
	a_1\ket{\Psi_1} &
%
%
	\mathscr{A}\ket{\Psi_2}
&=
	a_2\ket{\Psi_2} &
%
%
	(a_1 \neq a_2)
\end{align}
である。

このとき、$\braket{\Psi_1|\H|\Psi_2}=0$であることを示せ。

\subsection{解}
まず、$\mathscr{A}$がエルミート演算子であることを仮定する。

$\mathscr{A}$と$\H$は可換であるので、
\begin{align}
	\mathscr{A} \H
	-
	\H \mathscr{A}
&=
	0 \\
%
%
	\braket{\Psi_1|\mathscr{A}\H|\Psi_2}
	-
	\braket{\Psi_1|\H\mathscr{A}|\Psi_2}
&=
	0
\end{align}
となる。左辺第1項について、$\mathscr{A}\ket{\Psi_1}=a_1\ket{\Psi_1}$であり、
$\mathscr{A}=\adj{\mathscr{A}}, \conju{a_1}=a_1$であるので、
$\bra{\Psi_1}\mathscr{A}=\bra{\Psi_1}\adj{\mathscr{A}}=\conju{a_1}\bra{\Psi_1}=a_1\bra{\Psi_1}$である。
従って、
\begin{align}
	a_1
	\braket{\Psi_1|\H|\Psi_2}
	-
	a_2
	\braket{\Psi_1|\H|\Psi_2}
&=
	0 \\
%
%
	(
		a_1 -a_2
	)
	\braket{\Psi_1|\H|\Psi_2}
&=
	0
\end{align}
となる。ここで、$\Psi_1$と$\Psi_2$が共に非縮退な固有関数であることから、
$a_1\neq a_2$であるので、
\begin{align}
	\braket{\Psi_1|\H|\Psi_2}
&=
	0
\end{align}
となる。








