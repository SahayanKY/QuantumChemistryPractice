%ファイルID
%2020/06/08 18:00
%->2006081800SahayanKY(ファイル作成者)
\subsection{問}
2つのHartree積
\begin{align}
	\Psi^{\rm HP}_{\aspin\bspin}
&=
	\psi_1(\r_1) \alpha(\omega_1)
	\psi_2(\r_2) \beta(\omega_2) \\
%
%
	\Psi^{\rm HP}_{\bspin\bspin}
&=
	\psi_1(\r_1) \beta(\omega_1)
	\psi_2(\r_2) \beta(\omega_2)
\end{align}
を考える。これらのエネルギーが等しいことを示せ。

また、Hartree積は平行スピンをもった電子の相関がないことから予想される通り、
そのエネルギーがSlater行列式$\ket{\aorb{\psi_1}\borb{\psi_2}}$のエネルギー
$E(\aspin\bspin)$に等しくなることを示せ。


\subsection{解}
まず、前者のHartree積$\Psi^{\rm HP}_{\aspin\bspin}$について考える。
その波動関数のエネルギーは
\begin{align}
	\braket{\Psi^{\rm HP}_{\aspin\bspin}|\H|\Psi^{\rm HP}_{\aspin\bspin}}
&=
	\int\d\r_1\d\omega_1 \d\r_2\d\omega_2
		\conju{\psi_1}(\r_1) \conju{\alpha}(\omega_1)
		\conju{\psi_2}(\r_2) \conju{\beta}(\omega_2)
		\H
		\psi_1(\r_1) \alpha(\omega_1)
		\psi_2(\r_2) \beta(\omega_2) \\
%
%
&=
	\int\d\r_1 \d\r_2
		\conju{\psi_1}(\r_1) \conju{\psi_2}(\r_2)
		\Bigl(
			h(\r_1) +h(\r_2) +r_{12}^{-1}
		\Bigr)
		\psi_1(\r_1) \psi_2(\r_2) \\
%
%
&=
	\int\d\r_1
		\conju{\psi_1}(\r_1)
		h(\r_1)
		\psi_1(\r_1) \nonumber\\&\quad
	+
	\int\d\r_2
		\conju{\psi_2}(\r_2)
		h(\r_2)
		\psi_2(\r_2) \nonumber\\&\quad
	+
	\int\d\r_1\d\r_2
		\conju{\psi_1}(\r_1) \psi_1(\r_1)
		r_{12}^{-1}
		\conju{\psi_2}(\r_2) \psi_2(\r_2) \\
%
%
&=
	\spaceone{1}{h}{1}
	+
	\spaceone{2}{h}{2}
	+
	\spacetwo{11}{22} \\
%
%
&=
	E(\aspin\bspin)
\end{align}
である。

次に、後者のHartree積$\Psi^{\rm HP}_{\bspin\bspin}$について考える。
その波動関数のエネルギーは
\begin{align}
	\braket{\Psi^{\rm HP}_{\bspin\bspin}|\H|\Psi^{\rm HP}_{\bspin\bspin}}
&=
	\int\d\r_1\d\omega_1 \d\r_2\d\omega_2
		\conju{\psi_1}(\r_1) \conju{\beta}(\omega_1)
		\conju{\psi_2}(\r_2) \conju{\beta}(\omega_2)
		\H
		\psi_1(\r_1) \beta(\omega_1)
		\psi_2(\r_2) \beta(\omega_2) \\
%
%
&=
	\int\d\r_1 \d\r_2
		\conju{\psi_1}(\r_1) \conju{\psi_2}(\r_2)
		\Bigl(
			h(\r_1) +h(\r_2) +r_{12}^{-1}
		\Bigr)
		\psi_1(\r_1) \psi_2(\r_2) \\
%
%
&=
	\int\d\r_1
		\conju{\psi_1}(\r_1)
		h(\r_1)
		\psi_1(\r_1) \nonumber\\&\quad
	+
	\int\d\r_2
		\conju{\psi_2}(\r_2)
		h(\r_2)
		\psi_2(\r_2) \nonumber\\&\quad
	+
	\int\d\r_1 \d\r_2
		\conju{\psi_1}(\r_1) \psi_1(\r_1)
		r_{12}^{-1}
		\conju{\psi_2}(\r_2) \psi_2(\r_2) \\
%
%
&=
	\spaceone{1}{h}{1}
	+
	\spaceone{2}{h}{2}
	+
	\spacetwo{11}{22} \\
%
%
&=
	E(\aspin\bspin)
\end{align}
である。

従って、$\ket{\Psi^{\rm HP}_{\aspin\bspin}},\ket{\Psi^{\rm HP}_{\bspin\bspin}},\ket{\aorb{\psi_1}\borb{\psi_2}}$の
エネルギーは等しいことが言える。

