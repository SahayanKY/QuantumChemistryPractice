%ファイルID
%2020/06/23 17:11
%->2006231711SahayanKY(ファイル作成者)
\subsection{問}
次式を示せ。
\begin{align}
	\braket{^1\Psi_1^2|\H|^1\Psi_1^2}
&=
	h_{11}
	+
	h_{22}
	+
	J_{12}
	+
	K_{12} \\
%
%
	\braket{^3\Psi_1^2|\H|^3\Psi_1^2}
&=
	h_{11}
	+
	h_{22}
	+
	J_{12}
	-
	K_{12}
\end{align}


\subsection{解}
まず、Slater行列式$\ket{ij}$に対応するブラベクトルを
$\bra{ij}$と定義する。

$\ket{^1\Psi_1^2},\ket{^3\Psi_1^2}$のエネルギーを計算する前に、
$\ket{\aorb{1}\borb{2}},\ket{\aorb{2}\borb{1}}$のエネルギーを計算する。
これは問題2.23を参考にすると、
\begin{align}
	\braket{\aorb{1}\borb{2}|\H|\aorb{1}\borb{2}}
&=
	h_{11}
	+
	h_{22}
	+
	J_{12} \\
%
%
	\braket{\aorb{2}\borb{1}|\H|\aorb{2}\borb{1}}
&=
	h_{11}
	+
	h_{22}
	+
	J_{12}
\end{align}
となる。また、$\braket{\aorb{1}\borb{2}|\H|\aorb{2}\borb{1}}$も計算する。
これは表2.3と表2.4を参考にすると
\begin{align}
	\braket{\aorb{1}\borb{2}|\H|\aorb{2}\borb{1}}
&=
	\braket{\aorb{1}\borb{2}|\Oone|\aorb{2}\borb{1}}
	+
	\braket{\aorb{1}\borb{2}|\Otwo|\aorb{2}\borb{1}} \\
%
%
&=
	0
	+
	\antitwo{\aorb{1}\borb{2}}{\aorb{2}\borb{1}} \\
%
%
&=
	\phystwo{\aorb{1}\borb{2}}{\aorb{2}\borb{1}}
	-
	\phystwo{\aorb{1}\borb{2}}{\borb{1}\aorb{2}} \\
%
%
&=
	\int \d\x_1\d\x_2
		\conju{\aorb{\psi}_1}(\x_1) \conju{\borb{\psi}_2}(\x_2)
		r_{12}^{-1}
		\aorb{\psi}_2(\x_1) \borb{\psi}_1(\x_2) \nonumber \\ &\quad
	-
	\int \d\x_1\d\x_2
		\conju{\aorb{\psi}_1}(\x_1) \conju{\borb{\psi}_2}(\x_2)
		r_{12}^{-1}
		\borb{\psi}_1(\x_1) \aorb{\psi}_2(\x_2) \\
%
%
&=
	\int \d\r_1\d\r_2
		\conju{\psi_1}(\r_1) \psi_2(\r_1)
		r_{12}^{-1}
		\conju{\psi_2}(\r_2) \psi_1(\r_2)
	\int \d\omega_1\d\omega_2
		\conju{\alpha}(\omega_1) \conju{\beta}(\omega_2)
		\alpha(\omega_1) \beta(\omega_2) \nonumber \\ &\quad
	-
	\int \d\r_1\d\r_2
		\conju{\psi_1}(\r_1) \psi_1(\r_1)
		r_{12}^{-1}
		\conju{\psi_2}(\r_2) \psi_2(\r_2)
	\int \d\omega_1\d\omega_2
		\conju{\alpha}(\omega_1) \conju{\beta}(\omega_2)
		\beta(\omega_1) \alpha(\omega_2) \\
%
%
&=
	\int \d\r_1\d\r_2
		\conju{\psi_1}(\r_1) \psi_2(\r_1)
		r_{12}^{-1}
		\conju{\psi_2}(\r_2) \psi_1(\r_2)
	-
	0 \\
%
%
&=
	\spacetwo{12}{21} \\
%
%
&=
	K_{12}
\end{align}
となる。加えて、$\braket{\aorb{2}\borb{1}|\H|\aorb{1}\borb{2}}$は
\begin{align}
	\braket{\aorb{2}\borb{1}|\H|\aorb{1}\borb{2}}
&=
	\conju{
		\braket{\aorb{1}\borb{2}|\adj{\H}|\aorb{2}\borb{1}}
	} \\
%
%
&=
	\conju{
		\braket{\aorb{1}\borb{2}|\H|\aorb{2}\borb{1}}
	} \\
%
%
&=
	\conju{K_{12}} \\
%
%
&=
	K_{12}
	%
	\qquad
	(\because \text{問題2.19})
\end{align}

従って、$\ket{^1\Psi_1^2}$のエネルギーは
\begin{align}
	\braket{^1\Psi_1^2|\H|^1\Psi_1^2}
&=
	\left(
		\frac{1}{\sqrt{2}}
		(
			\bra{\aorb{1}\borb{2}}
			+
			\bra{\aorb{2}\borb{1}}
		)
	\right)
	\H
	\left(
		\frac{1}{\sqrt{2}}
		(
			\ket{\aorb{1}\borb{2}}
			+
			\ket{\aorb{2}\borb{1}}
		)
	\right) \\
%
%
&=
	\frac{1}{2}
	\biggl(
		\braket{\aorb{1}\borb{2}|\H|\aorb{1}\borb{2}}
		+
		\braket{\aorb{1}\borb{2}|\H|\aorb{2}\borb{1}}
		+
		\braket{\aorb{2}\borb{1}|\H|\aorb{1}\borb{2}}
		+
		\braket{\aorb{2}\borb{1}|\H|\aorb{2}\borb{1}}
	\biggr) \\
%
%
&=
	\frac{1}{2}
	\biggl(
		(h_{11} +h_{22} +J_{12})
		+
		(K_{12})
		+
		(K_{12})
		+
		(h_{11} +h_{22} +J_{12})
	\biggr) \\
%
%
&=
	h_{11}
	+
	h_{22}
	+
	J_{12}
	+
	K_{12}
\end{align}
である。また、$\ket{^3\Psi_1^2}$のエネルギーは
\begin{align}
	\braket{^3\Psi_1^2|\H|^3\Psi_1^2}
&=
	\left(
		\frac{1}{\sqrt{2}}
		(
			\bra{\aorb{1}\borb{2}}
			-
			\bra{\aorb{2}\borb{1}}
		)
	\right)
	\H
	\left(
		\frac{1}{\sqrt{2}}
		(
			\ket{\aorb{1}\borb{2}}
			-
			\ket{\aorb{2}\borb{1}}
		)
	\right) \\
%
%
&=
	\frac{1}{2}
	\biggl(
		\braket{\aorb{1}\borb{2}|\H|\aorb{1}\borb{2}}
		-
		\braket{\aorb{1}\borb{2}|\H|\aorb{2}\borb{1}}
		-
		\braket{\aorb{2}\borb{1}|\H|\aorb{1}\borb{2}}
		+
		\braket{\aorb{2}\borb{1}|\H|\aorb{2}\borb{1}}
	\biggr) \\
%
%
&=
	\frac{1}{2}
	\biggl(
		(h_{11} +h_{22} +J_{12})
		-
		(K_{12})
		-
		(K_{12})
		+
		(h_{11} +h_{22} +J_{12})
	\biggr) \\
%
%
&=
	h_{11}
	+
	h_{22}
	+
	J_{12}
	-
	K_{12}
\end{align}
である。


