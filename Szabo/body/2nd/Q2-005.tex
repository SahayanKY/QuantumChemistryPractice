%ファイルID
%2020/04/30 12:12
%->2004301212(10進数)->X5B4PO(ファイル作成者)
%->X5B4POSahayanKY
\subsection{問}
次のSlater行列式を考える。
\begin{align}
	\ket{K}
&=
	\ket{\chi_i \chi_j} &
%
%
	\ket{L}
&=
	\ket{\chi_k \chi_l}
\end{align}
このとき、
\begin{align}
	\braket{K|L}
&=
	\delta_{ik} \delta_{jl}
	-
	\delta_{il} \delta_{jk}
\end{align}
であることを示せ。


\subsection{解}
$\ket{K}$と$\ket{L}$は次の通りである。
\begin{align}
	\ket{K}
&=
	\frac{1}{\sqrt{2}}
		\left(
			\chi_i(\x_1) \chi_j(\x_2)
			-
			\chi_j(\x_1) \chi_i(\x_2)
		\right) &
%
%
	\ket{L}
&=
	\frac{1}{\sqrt{2}}
		\left(
			\chi_k(\x_1) \chi_l(\x_2)
			-
			\chi_l(\x_1) \chi_k(\x_2)
		\right)
\end{align}
従って、
\begin{align}
	\braket{K|L}
&=
	\int \d\x_1 \int \d\x_2\
		\frac{1}{2}
			\left(
				\conju{\chi_i}(\x_1) \conju{\chi_j}(\x_2)
				-
				\conju{\chi_j}(\x_1) \conju{\chi_i}(\x_2)
			\right)
			\left(
				\chi_k(\x_1) \chi_l(\x_2)
				-
				\chi_l(\x_1) \chi_k(\x_2)
			\right) \\
%
%
&=
	\frac{1}{2}
		\left\{
		\begin{array}{>{\displaystyle}l}
			\int \d\x_1 \int \d\x_2\
				\conju{\chi_i}(\x_1) \conju{\chi_j}(\x_2) \chi_k(\x_1) \chi_l(\x_2) \\[3mm]
			-
			\int \d\x_1 \int \d\x_2\
				\conju{\chi_i}(\x_1) \conju{\chi_j}(\x_2) \chi_l(\x_1) \chi_k(\x_2) \\[3mm]
			-
			\int \d\x_1 \int \d\x_2\
				\conju{\chi_j}(\x_1) \conju{\chi_i}(\x_2) \chi_k(\x_1) \chi_l(\x_2) \\[3mm]
			+
			\int \d\x_1 \int \d\x_2\
				\conju{\chi_j}(\x_1) \conju{\chi_i}(\x_2) \chi_l(\x_1) \chi_k(\x_2)
		\end{array}
		\right\} \\
%
%
&=
	\frac{1}{2}
		\left(
			\delta_{ik} \delta_{jl}
			-
			\delta_{il} \delta_{jk}
			-
			\delta_{jk} \delta_{il}
			+
			\delta_{jl} \delta_{ik}
		\right) \\
%
%
&=
	\delta_{ik} \delta_{jl}
	-
	\delta_{il} \delta_{jk}
\end{align}
である。




