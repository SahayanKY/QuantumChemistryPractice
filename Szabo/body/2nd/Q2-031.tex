%ファイルID
%2020/06/11 21:43
%->2006112143SahayanKY(ファイル作成者)
\subsection{問}
次の等式を示せ。
\begin{align}
	\braket{\Psi_a^r|\Otwo|\Psi_0}
&=
	\sum_b^N \antitwo{rb}{ab}
\end{align}

ただし、次の式を用いること。
\begin{align}
	\Otwo
&=
	\frac{1}{2}
	\sum_i^{2K} \sum_j^{2K} \sum_k^{2K} \sum_l^{2K}
		\phystwo{ij}{kl}
		\adj{a_i} \adj{a_j} a_l a_k
\end{align}

\subsection{解}
\begin{align}
	\braket{\Psi_a^r|\Otwo|\Psi_0}
&=
	\frac{1}{2}
	\sum_i^{2K} \sum_j^{2K} \sum_k^{2K} \sum_l^{2K}
		\phystwo{ij}{kl}
		\braket{\Psi_a^r|\adj{a_i} \adj{a_j} a_l a_k|\Psi_0} \\
%
%
&=
	\frac{1}{2}
	\sum_i^{2K} \sum_j^{2K} \sum_k^{2K} \sum_l^{2K}
		\phystwo{ij}{kl}
		\braket{\Psi_0|\adj{a_a} a_r \adj{a_i} \adj{a_j} a_l a_k|\Psi_0} \\
\end{align}
である。ここで、
\begin{align}
&\quad
	\adj{a_a} a_r \adj{a_i} \adj{a_j} a_l a_k \\
&=
	\adj{a_a} (\delta_{ri} -\adj{a_i} a_r) \adj{a_j} a_l a_k \\
%
%
&=
	\delta_{ri} \adj{a_a} \adj{a_j} a_l a_k
	-
	\adj{a_a} \adj{a_i} a_r \adj{a_j} a_l a_k \\
%
%
&=
	\delta_{ri} (-\adj{a_j} \adj{a_a}) a_l a_k
	-
	(-\adj{a_i} \adj{a_a}) (\delta_{rj} -\adj{a_j} a_r) a_l a_k \\
%
%
&=
	-
	\delta_{ri} \adj{a_j} \adj{a_a} a_l a_k
	+
	\delta_{rj} \adj{a_i} \adj{a_a} a_l a_k
	-
	\adj{a_i} \adj{a_a} \adj{a_j} a_r a_l a_k \\
%
%
&=
	-
	\delta_{ri} \adj{a_j} (\delta_{al} -a_l \adj{a_a}) a_k
	+
	\delta_{rj} \adj{a_i} (\delta_{al} -a_l \adj{a_a}) a_k
	-
	\adj{a_i} \adj{a_a} \adj{a_j} (-a_l a_r) a_k \\
%
%
&=
	-
	\delta_{ri} \delta_{al} \adj{a_j} a_k
	+
	\delta_{ri} \adj{a_j} a_l \adj{a_a} a_k
	+
	\delta_{rj} \delta_{al} \adj{a_i} a_k
	-
	\delta_{rj} \adj{a_i} a_l \adj{a_a} a_k
	+
	\adj{a_i} \adj{a_a} \adj{a_j} a_l a_r a_k \\
%
%
&=
	-
	\delta_{ri} \delta_{al} \adj{a_j} a_k
	+
	\delta_{ri} \adj{a_j} a_l (\delta_{ak} -a_k \adj{a_a})
	+
	\delta_{rj} \delta_{al} \adj{a_i} a_k \nonumber \\ &\qquad
	-
	\delta_{rj} \adj{a_i} a_l (\delta_{ak} -a_k \adj{a_a})
	+
	\adj{a_i} \adj{a_a} \adj{a_j} a_l (-a_k a_r) \\
%
%
&=
	-
	\delta_{ri} \delta_{al} \adj{a_j} a_k
	+
	\delta_{ri} \delta_{ak} \adj{a_j} a_l
	-
	\delta_{ri} \adj{a_j} a_l a_k \adj{a_a}
	+
	\delta_{rj} \delta_{al} \adj{a_i} a_k \nonumber \\ &\qquad
	-
	\delta_{rj} \delta_{ak} \adj{a_i} a_l
	+
	\delta_{rj} \adj{a_i} a_l a_k \adj{a_a}
	-
	\adj{a_i} \adj{a_a} \adj{a_j} a_l a_k a_r
\end{align}
である。$\adj{a_a}\ket{\Psi_0}=a_r\ket{\Psi_0}=0$であるので、
\begin{align}
	\braket{\Psi_a^r|\Otwo|\Psi_0}
&=
	\frac{1}{2}
	\sum_i^{2K} \sum_j^{2K} \sum_k^{2K} \sum_l^{2K}
		\phystwo{ij}{kl}
		\left(
		\begin{array}{l}
			-
			\delta_{ri} \delta_{al} \braket{\Psi_0|\adj{a_j} a_k|\Psi_0}
			+
			\delta_{ri} \delta_{ak} \braket{\Psi_0|\adj{a_j} a_l|\Psi_0} \\\quad
			+
			\delta_{rj} \delta_{al} \braket{\Psi_0|\adj{a_i} a_k|\Psi_0}
			-
			\delta_{rj} \delta_{ak} \braket{\Psi_0|\adj{a_i} a_l|\Psi_0}
		\end{array}
		\right) \\
%
%
&=
	-
	\frac{1}{2}
	\sum_j^{2K} \sum_k^{2K}
		\phystwo{rj}{ka}
		\braket{\Psi_0|\adj{a_j}a_k|\Psi_0}
	+
	\frac{1}{2}
	\sum_j^{2K} \sum_l^{2K}
		\phystwo{rj}{al}
		\braket{\Psi_0|\adj{a_j}a_l|\Psi_0} \nonumber\\ &\qquad
	+
	\frac{1}{2}
	\sum_i^{2K} \sum_k^{2K}
		\phystwo{ir}{ka}
		\braket{\Psi_0|\adj{a_i}a_k|\Psi_0}
	-
	\frac{1}{2}
	\sum_i^{2K} \sum_l^{2K}
		\phystwo{ir}{al}
		\braket{\Psi_0|\adj{a_i}a_l|\Psi_0}
\end{align}
問題2.27より、$\braket{\Psi_0|\adj{a_i}a_j|\Psi_0}$は
$i=j=1,2,\dots,N$のとき1であり、それ以外の時はゼロである。
従って、
\begin{align}
	\braket{\Psi_a^r|\Otwo|\Psi_0}
&=
	-
	\frac{1}{2}
	\sum_j^N
		\phystwo{rj}{ja}
	+
	\frac{1}{2}
	\sum_j^N
		\phystwo{rj}{aj}
	+
	\frac{1}{2}
	\sum_i^N
		\phystwo{ir}{ia}
	-
	\frac{1}{2}
	\sum_i^N
		\phystwo{ir}{ai} \\
%
%
&=
	-
	\frac{1}{2}
	\sum_j^N
		\phystwo{rj}{ja}
	+
	\frac{1}{2}
	\sum_j^N
		\phystwo{rj}{aj}
	+
	\frac{1}{2}
	\sum_i^N
		\phystwo{ri}{ai}
	-
	\frac{1}{2}
	\sum_i^N
		\phystwo{ri}{ia}
	%
	\qquad
	(\because \phystwo{ij}{kl} = \phystwo{ji}{lk}) \\
%
%
&=
	\sum_i^N
		\phystwo{ri}{ai}
	-
	\sum_i^N
		\phystwo{ri}{ia} \\
%
%
&=
	\sum_i^N
		\antitwo{ri}{ai}
\end{align}
となる。


