%ファイルID
%2020/05/11 09:14
%->2005110914(10進数)->X5SHHE(ファイル作成者)
%->X5SHHESahayanKY
\subsection{問}
$\braket{K|\Oone|L}$と$\braket{K|\Otwo|L}$に関する規則は
次の通りである。
\begin{align}
	\braket{K|\Oone|L}
&=
	\left\{
	\begin{array}{>{\displaystyle}ll}
		\sum_m \braket{m|h|m} &
		%
		(\ket{K}=\ket{L}=\ket{\cdots m\cdots}) \\
	%
		\braket{m|h|p} &
		%
		(\ket{K}=\ket{\cdots m\cdots}, \ket{L}=\ket{\cdots p\cdots}) \\
	%
		0 &
		%
		(\ket{K}=\ket{\cdots mn\cdots}, \ket{L}=\ket{\cdots pq\cdots})
	\end{array}
	\right. \\
%
%
	\braket{K|\Otwo|L}
&=
	\left\{
	\begin{array}{>{\displaystyle}ll}
		\frac{1}{2} \sum_m \sum_n \antitwo{mn}{mn} &
		%
		(\ket{K}=\ket{L}=\ket{\cdots m\cdots}) \\
	%
		\sum_n \antitwo{mn}{pn} &
		%
		(\ket{K}=\ket{\cdots m\cdots}, \ket{L}=\ket{\cdots p\cdots}) \\
	%
		\antitwo{mn}{pq} &
		%
		(\ket{K}=\ket{\cdots mn\cdots}, \ket{L}=\ket{\cdots pq\cdots})
	\end{array}
	\right.
\end{align}

これを利用して\ce{H2}の完全CI行列の行列要素を計算し、
問題2.9で得た結果に等しくなることを示せ。


\subsection{解}
\ce{H2}の完全CI行列は次の通りである。
\begin{align}
	H
&=
	\left[
	\begin{array}{cc}
		\braket{\Psi_0|\H|\Psi_0} &
		%
		\braket{\Psi_0|\H|\Psi_{12}^{34}} \\
	%
		\braket{\Psi_{12}^{34}|\H|\Psi_0} &
		%
		\braket{\Psi_{12}^{34}|\H|\Psi_{12}^{34}}
	\end{array}
	\right]
\end{align}
ここで、$\ket{\Psi_0}=\ket{12}, \ket{\Psi_{12}^{34}}=\ket{34}$である。
よって、$\braket{\Psi_0|\H|\Psi_0}$は
\begin{align}
	\braket{\Psi_0|\H|\Psi_0}
&=
	\sum_{m=1,2} \braket{m|h|m}
	+
	\frac{1}{2}
	\sum_{m=1,2} \sum_{n=1,2} \antitwo{mn}{mn} \\
%
%
&=
	\braket{1|h|1}
	+
	\braket{2|h|2}
	+
	\frac{1}{2}
	\biggl(
		\antitwo{11}{11}
		+
		\antitwo{12}{12}
		+
		\antitwo{21}{21}
		+
		\antitwo{22}{22}
	\biggr) \\
%
%
&=
	\braket{1|h|1}
	+
	\braket{2|h|2}
	+
	\frac{1}{2}
	\biggl(
		\antitwo{12}{12}
		+
		\antitwo{21}{21}
	\biggr)
	%
	\quad
	(\because \antitwo{ij}{kk}=0) \\
%
%
&=
	\braket{1|h|1}
	+
	\braket{2|h|2}
	+
	\antitwo{12}{12}
	%
	\quad
	(\because \phystwo{ij}{kl}=\phystwo{ji}{lk})
\end{align}
である。また、$\braket{\Psi_0|\H|\Psi_{12}^{34}}$と
$\braket{\Psi_{12}^{34}|\H|\Psi_0}$は
\begin{align}
	\braket{\Psi_0|\H|\Psi_{12}^{34}}
&=
	\antitwo{12}{34} \\
%
%
	\braket{\Psi_{12}^{34}|\H|\Psi_0}
&=
	\antitwo{34}{12}
\end{align}
である。最後に$\braket{\Psi_{12}^{34}|\H|\Psi_{12}^{34}}$は、
$\braket{\Psi_0|\H|\Psi_0}$と同様で
\begin{align}
	\braket{\Psi_{12}^{34}|\H|\Psi_{12}^{34}}
&=
	\braket{3|h|3}
	+
	\braket{4|h|4}
	+
	\antitwo{34}{34}
\end{align}
となる。

従って、確かに問題2.9で得た結果に一致することが言える。
