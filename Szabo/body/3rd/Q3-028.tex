%ファイルID
%2020/09/18 1430
%->2009181430SahayanKY(ファイル作成者)
\subsection{問}
基底関数$\phi_1,\phi_2$をSchmidtの方法で規格直交化すると
変換行列$X$は
\begin{align}
	X_{\rm Schmidt}
&=
	\left[
	\begin{array}{cc}
		1 & -\dfrac{S_{12}}{\sqrt{1-S_{12}^2}} \\[3mm]
		0 & \dfrac{1}{\sqrt{1-S_{12}^2}}
	\end{array}
	\right]
\end{align}
となる。このとき、新たに得られる基底$\phi'_\mu=\sum_\nu X_{\nu\mu} \phi_\nu$が
確かに規格直交化されていることを示せ。


\subsection{解}
新しい基底関数は
\begin{align}
	\phi'_1
&=
	X_{11} \phi_1
	+
	X_{21} \phi_2 &
%
	\phi'_2
&=
	X_{12} \phi_1
	+
	X_{22} \phi_2 \\
%
%
&=
	\phi_1 &
%
&=
	-
	\frac{S_{12}}{\sqrt{1-S_{12}^2}}
	\phi_1
	+
	\frac{1}{\sqrt{1-S_{12}^2}}
	\phi_2
\end{align}
である。

まず、規格化されているかを調べる。
\begin{align}
	\braket{\phi'_1|\phi'_1}
&=
	\braket{\phi_1|\phi_1} \\
%
%
&=
	1
\end{align}
\begin{align}
	\braket{\phi'_2|\phi'_2}
&=
	\braket{
		(X_{12}\phi_1 +X_{22}\phi_2)|
		(X_{12}\phi_1 +X_{22}\phi_2)
	} \\
%
%
&=
	\conju{X_{12}}X_{12}
	\braket{\phi_1|\phi_1}
	+
	\conju{X_{12}}X_{22}
	\braket{\phi_1|\phi_2}
	+
	\conju{X_{22}}X_{12}
	\braket{\phi_2|\phi_1}
	+
	\conju{X_{22}}X_{22}
	\braket{\phi_2|\phi_2} \\
%
%
&=
	\frac{
		S_{12}^2
	}{
		1-S_{12}^2
	} \cdot
	1
	-
	\frac{
		S_{12}
	}{
		1-S_{12}^2
	} \cdot
	S_{12}
	-
	\frac{
		S_{12}
	}{
		1-S_{12}^2
	} \cdot
	S_{12}
	+
	\frac{
		1
	}{
		1-S_{12}^2
	} \cdot
	1 \\
%
%
&=
	\frac{
		S_{12}^2
		-
		S_{12}^2
		-
		S_{12}^2
		+
		1
	}{
		1-S_{12}^2
	} \\
%
%
&=
	1
\end{align}
である。従って、確かに規格化されている。

次に直交化されているかを調べる。
\begin{align}
	\braket{\phi'_1|\phi'_2}
&=
	X_{12}
	\braket{\phi_1|\phi_1}
	+
	X_{22}
	\braket{\phi_1|\phi_2} \\
%
%
&=
	-
	\frac{S_{12}}{\sqrt{1-S_{12}^2}} \cdot
	1
	+
	\frac{1}{\sqrt{1-S_{12}^2}} \cdot
	S_{12} \\
%
%
&=
	0
\end{align}
である。従って、確かに直交化されている。


