%ファイルID
%2020/09/14 1332
%->2009141332SahayanKY(ファイル作成者)
\subsection{問}
\ce{H2}の最小基底関数モデルにおいては
2つの分子軌道$\psi$の係数が次の2つになることを、
規格化条件から示せ。
\begin{align}
	\psi_1
&=
	\frac{1}{
		\sqrt{
			2(1+S_{12})
		}
	}
	(\phi_1 +\phi_2) \\
%
%
	\psi_2
&=
	\frac{1}{
		\sqrt{
			2(1-S_{12})
		}
	}
	(\phi_1 -\phi_2)
\end{align}
ここで$\phi_1,\phi_2$は基底関数であり、
それぞれ水素1と水素2を中心とする1s軌道である。

\subsection{解}
結合性軌道の係数を$c_1$とする。
\begin{align}
	\psi_1
&=
	c_1(\phi_1 +\phi_2)
\end{align}
規格化条件より、(分子軌道、基底関数は実関数であるので内積の複素共役は考慮しない)
\begin{align}
	\int \d\r
		\psi_1 \psi_1
&=
	c_1^2
	\int \d\r
		(\phi_1 +\phi_2)^2 \\
%
%
	1
&=
	c_1^2
	\left(
		\int\d\r \phi_1^2
		+
		2
		\int\d\r \phi_1\phi_2
		+
		\int\d\r \phi_2^2
	\right) \\
%
%
&=
	c_1^2
	\left(
		1
		+
		2S_{12}
		+
		1
	\right) \\
%
%
	c_1^2
&=
	\frac{1}{
		2
		+
		2S_{12}
	} \\
%
%
	|c_1|
&=
	\frac{1}{
		\sqrt{
			2(1+S_{12})
		}
	}
\end{align}
波動関数は正負の区別がない。従って、結合性軌道は
\begin{align}
	\psi_1
&=
	\frac{1}{
		\sqrt{
			2(1+S_{12})
		}
	}
	(\phi_1 +\phi_2)
\end{align}
である。

続いて反結合性軌道について考える。係数を$c_2$とする。
\begin{align}
	\psi_2
&=
	c_2
	(\phi_1 -\phi_2)
\end{align}
同様に、規格化条件より
\begin{align}
	\int\d\r
		\psi_2\psi_2
&=
	c_2^2
	\int\d\r
		(\phi_1 -\phi_2)^2 \\
%
%
	1
&=
	c_2^2
	\left(
		\int\d\r \phi_1^2
		-
		2\int\d\r \phi_1\phi_2
		+
		\int\d\r \phi_2^2
	\right) \\
%
%
&=
	c_2^2
	\left(
		1
		-
		2S_{12}
		+
		1
	\right) \\
%
%
	c_2^2
&=
	\frac{1}{
		2
		-
		2S_{12}
	} \\
%
%
	|c_2|
&=
	\frac{1}{
		\sqrt{
			2(1-S_{12})
		}
	}
\end{align}
従って、反結合性軌道は
\begin{align}
	\psi_2
&=
	\frac{1}{
		\sqrt{
			2(1-S_{12})
		}
	}
	(\phi_1 -\phi_2)
\end{align}
である。

