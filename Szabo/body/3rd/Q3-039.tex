%ファイルID
%2020/09/27 23:14
%->2009272314SahayanKY(ファイル作成者)
\subsection{問}
次のような1電子演算子の和であるスピン依存の演算子
\begin{align}
	\hat{\rho}^S
&=
	2
	\sum_i^N
		\delta(\r_i-\bm{R})
		s_z(i)
\end{align}
を考える。任意の1個の非制限行列式による$\hat{\rho}^S$の期待値が
\begin{align}
	\average{\hat{\rho}^S}
&=
	\rho^S(\bm{R})
=
	\tr[1]{\bm{P}^S\bm{A}}
\end{align}
となることを示せ。

ここで$\rho^S(\r)$はスピン密度$\rho^\alpha(\r)-\rho^\beta(\r)$であり、
$A_{\mu\nu}=\conju{\phi_\mu}(\bm{R})\phi_\nu(\bm{R})$である。


\subsection{解}
前問で利用した1電子演算子の行列要素の計算規則
\begin{align}
	\average{\Oone}
&=
	\braket{\Psi|\Oone|\Psi} \\
%
%
&=
	\sum_a^N
		\braket{\chi_a|h|\chi_a}
\end{align}
はスピン非依存性の1電子演算子を想定して導出されたものであるが、
スピン依存性であっても同様の議論が成立する。つまり、
\begin{align}
	\average{\hat{\rho}^S}
&=
	\Braket{\Psi|2\sum_i^N \delta(\r_i-\bm{R})s_z(\omega_i)|\Psi} \\
%
%
&=
	\sum_a^N
		\braket{\chi_a(\x_1)|2\delta(\r_1-\bm{R})s_z(\omega_1)|\chi_a(\x_1)}
\end{align}
となる。これを式変形すると
\begin{align}
	\average{\hat{\rho}^S}
&=
	2
	\sum_a^{N^\alpha}
		\braket{\chi_a(\x_1)|\delta(\r_1-\bm{R})s_z(\omega_1)|\chi_a(\x_1)}
	+
	2
	\sum_a^{N^\beta}
		\braket{\chi_a(\x_1)|\delta(\r_1-\bm{R})s_z(\omega_1)|\chi_a(\x_1)} \\
%
%
&=
	2
	\sum_a^{N^\alpha}
		\braket{\psi^\alpha_a(\r_1)|\delta(\r_1-\bm{R})|\psi^\alpha_a(\r_1)}
		\braket{\alpha(\omega_1)|s_z(\omega_1)|\alpha(\omega_1)} \nonumber \\ &\qquad
	+
	2
	\sum_a^{N^\beta}
		\braket{\psi^\beta_a(\x_1)|\delta(\r_1-\bm{R})|\psi^\beta_a(\r_1)}
		\braket{\beta(\omega_1)|s_z(\omega_1)|\beta(\omega_1)} \\
%
%
&=
	2
	\sum_a^{N^\alpha}
		\conju{\psi^\alpha_a}(\bm{R})
		\psi^\alpha_a(\bm{R}) \cdot
		\frac{1}{2}
		\braket{\alpha|\alpha}
	+
	2
	\sum_a^{N^\beta}
		\conju{\psi^\beta_a}(\bm{R})
		\psi^\beta_a(\bm{R}) \cdot
		\left(
			-\frac{1}{2}
		\right)
		\braket{\beta|\beta} \\
%
%
&=
	\sum_a^{N^\alpha}
		|\psi^\alpha_a(\bm{R})|^2
	-
	\sum_a^{N^\beta}
		|\psi^\beta_a(\bm{R})|^2 \\
%
%
&=
	\rho^\alpha(\bm{R})
	-
	\rho^\beta(\bm{R}) \\
%
%
&=
	\rho^S(\bm{R})
\end{align}
となる。

さらに空間軌道を基底関数で展開すると
\begin{align}
	\average{\hat{\rho}^S}
&=
	\rho^\alpha(\bm{R})
	-
	\rho^\beta(\bm{R}) \\
%
%
&=
	\sum_\mu
	\sum_\nu
		P^\alpha_{\mu\nu}
		\phi_\mu(\bm{R})
		\conju{\phi_\nu}(\bm{R})
	-
	\sum_\mu
	\sum_\nu
		P^\beta_{\mu\nu}
		\phi_\mu(\bm{R})
		\conju{\phi_\nu}(\bm{R}) \\
%
%
&=
	\sum_\mu
	\sum_\nu
		P^S_{\mu\nu}
		\phi_\mu(\bm{R})
		\conju{\phi_\nu}(\bm{R}) \\
%
%
&=
	\sum_\mu
	\sum_\nu
		P^S_{\mu\nu}
		A_{\nu\mu} \\
%
%
&=
	\sum_\mu
		(\bm{P}^S \bm{A})_{\mu\mu} \\
%
%
&=
	\tr[1]{\bm{P}^S \bm{A}}
\end{align}
となる。
