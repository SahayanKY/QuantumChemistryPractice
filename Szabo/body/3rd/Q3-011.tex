%ファイルID
%2020/07/30 11:16
%->2007301116SahayanKY(ファイル作成者)
\subsection{問}
密度演算子$\hat{\rho}(\r)$は次の通りである。
\begin{align}
	\hat{\rho}(\r)
&=
	\sum_{i=1}^{N}
		\delta(\r_i-\r)
\end{align}
このとき、制限付きの閉殻系において
\begin{align}
	\rho(\r)
=
	\braket{\Psi_0|\hat{\rho}(\r)|\Psi_0}
%
%
=
	2
	\sum_a^{N/2}
		|\psi_a(\r)|^2
\end{align}
を導出せよ。


\subsection{解}
まず、
\begin{align}
	\rho(\r)
&=
	\braket{\Psi_0|\hat{\rho}(\r)|\Psi_0} \\
%
%
&=
	\sum_{i=1}^{N}
		\braket{\Psi_0|\delta(\r_i-\r)|\Psi_0}
\end{align}
である。したがって、$\braket{\Psi_0|\delta(\r_i-\r)|\Psi_0}$について考える。
\begin{align}
&\quad
	\braket{\Psi_0|\delta(\r_i-\r)|\Psi_0} \\
&=
	\frac{1}{N!}
	\int\d\x_1\dots\d\x_N
		\sum_j^{N!}
		\sum_k^{N!}
			(-1)^{p_j+p_k}
			\mathscr{P}_j\{
				\conju{\chi_1}(1) \conju{\chi_2}(2)\dots
			\}
			\delta(\r_i-\r)
			\mathscr{P}_k\{
				\chi_1(1) \chi_2(2)\dots
			\}
\end{align}
ここで$\x_i$以外の積分において、
スピン軌道の直交性により同じスピン軌道でなければゼロになる。
即ち、$\mathscr{P}_j$と$\mathscr{P}_k$で$\x_{l\neq i}$が占めるスピン軌道が
同じときだけ非ゼロになる。
これは$j=k$を意味するので、
\begin{align}
&\quad
	\braket{\Psi_0|\delta(\r_i-\r)|\Psi_0} \\
%
%
&=
	\frac{1}{N!}
	\int\d\x_1\dots\d\x_N
		\sum_j^{N!}
			(-1)^{2p_j}
			\mathscr{P}_j\{
				\conju{\chi_1}(1) \conju{\chi_2}(2)\dots
			\}
			\delta(\r_i-\r)
			\mathscr{P}_j\{
				\chi_1(1) \chi_2(2)\dots
			\} \\
%
%
&=
	\frac{1}{N!}
	\sum_j^{N!}
	\int\d\x_1\dots\d\x_N
		\delta(\r_i-\r)
		\mathscr{P}_j\{
			\conju{\chi_1}(1) \chi_1(1)
			\conju{\chi_2}(2) \chi_2(2) \dots
		\}
\end{align}
ここで$N!$個の和の中で$\x_i$が$\chi_1$を占める回数は$(N-1)!$回である。
他のスピン軌道についても同様である。
また、$\x_i$が$\chi_1$を占めるとき、
$\x_i$以外の積分はスピン軌道の規格性により1になる。
以上のことから、
\begin{align}
&\quad
	\braket{\Psi_0|\delta(\r_i-\r)|\Psi_0} \\
&=
	\frac{1}{N!}
	\left(
	\begin{array}{>{\displaystyle}l}
		(N-1)!
		\int\d\x_i
			\delta(\r_i-\r)
			\conju{\chi_1}(\x_i)
			\chi_1(\x_i) \\
		+
		(N-1)!
		\int\d\x_i
			\delta(\r_i-\r)
			\conju{\chi_2}(\x_i)
			\chi_2(\x_i) \\
		+
		\dots
	\end{array}
	\right) \\
%
%
&=
	\frac{1}{N}
	\left(
	\begin{array}{>{\displaystyle}l}
		\int\d\r_i\d\omega_i
			\delta(\r_i-\r)
			\conju{\psi_1}(\r_i)\conju{\alpha}(\omega_i)
			\psi_1(\r_i)\alpha(\omega_i) \\
		+
		\int\d\r_i\d\omega_i
			\delta(\r_i-\r)
			\conju{\psi_1}(\r_i)\conju{\beta}(\omega_i)
			\psi_1(\r_i)\beta(\omega_i) \\
		+
		\dots
	\end{array}
	\right) \\
%
%
&=
	\frac{1}{N}
	\left(
		\conju{\psi_1}(\r) \psi_1(\r)
		+
		\conju{\psi_1}(\r) \psi_1(\r)
		+
		\conju{\psi_2}(\r) \psi_2(\r)
		+
		\conju{\psi_2}(\r) \psi_2(\r)
		+
		\dots
	\right) \\
%
%
&=
	\frac{2}{N}
	\sum_a^{N/2}
		|\psi_a(\r)|^2
\end{align}
以上の議論は任意の$i$について成立するため、
\begin{align}
	\rho(\r)
&=
	\sum_{i=1}^{N}
		\braket{\Psi_0|\delta(\r_i-\r)|\Psi_0} \\
%
%
&=
	\sum_{i=1}^{N}
		\frac{2}{N}
		\sum_a^{N/2}
			|\psi_a(\r)|^2 \\
%
%
&=
	N\cdot
	\frac{2}{N}
	\sum_a^{N/2}
		|\psi_a(\r)|^2 \\
%
%
&=
	2
	\sum_a^{N/2}
		|\psi_a(\r)|^2
\end{align}
である。







