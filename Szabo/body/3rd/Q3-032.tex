%ファイルID
%2020/09/23 16:22
%->2009231622SahayanKY(ファイル作成者)
\subsection{問}
各基底関数系(STO-3G, 4-31G, 6-31G$^{\ast}$, 6-31G$^{\ast\ast}$)と
Hartree-Fock極限における分子のSCF全エネルギーを用いて、
以下の反応
\begin{align}
	\ce{
		N2 + 3H2 ->
	}&
	\ce{
		2NH3
	} \\
%
%
	\ce{
		CO + 3H2 ->
	}&
	\ce{
		CH4 + H2O
	}
\end{align}
の反応前後のエネルギー差$\Delta E$を計算せよ。

また、ゼロ点振動エネルギーの差についても計算し、
それの$\Delta E$への寄与を考えよ。

\subsection{解}
表\ref{tab2009231622SahayanKY_SCFtotalenergy}に本書に示されている各分子の全エネルギーを示した。
この表を元に、$\Delta E$を計算した結果を
\ce{N2}の還元反応については表\ref{tab2009231622SahayanKY_DeltaEofReductionofNitrogen}に、
\ce{CO}の還元反応については表\ref{tab2009231622SahayanKY_DeltaEofReductionofCarbonMonoxide}に示す。

%
%表を挿入
\begin{table}[htpt]
\caption{各基底関数系による\ce{H2}, \ce{N2}, \ce{NH3}, \ce{CO}, \ce{CH4}, \ce{H2O}のSCF全エネルギー(\si{\atomic})}
\label{tab2009231622SahayanKY_SCFtotalenergy}
\centering
\begin{tabular}{
		l
		S[table-format=+1.3]
		S[table-format=+3.3]
		S[table-format=+2.3]
		S[table-format=+3.3]
		S[table-format=+2.3]
		S[table-format=+2.3]
	}
	\hline
	基底関数系 & {\ce{H2}} & {\ce{N2}} & {\ce{NH3}} & {\ce{CO}} & {\ce{CH4}} & {\ce{H2O}} \\
	\hline
	STO-3G & 			-1.117 & -107.496 & -55.454 & -111.225 & -39.727 & -74.963 \\
	4-31G & 			-1.127 & -108.754 & -56.102 & -112.552 & -40.140 & -75.907 \\
	6-31G$^{\ast}$ & 	-1.127 & -108.942 & -56.184 & -112.737 & -40.195 & -76.011 \\
	6-31G$^{\ast\ast}$ &-1.131 & -108.942 & -56.195 & -112.737 & -40.202 & -76.023 \\
	HF極限 & 			-1.134 & -108.997 & -56.225 & -112.791 & -40.225 & -76.065 \\
	\hline
\end{tabular}
\end{table}

%
%表を挿入
\begin{table}[htpt]
\caption{各基底関数系での\ce{N2 + 3H2 -> 2NH3}の$\Delta E$}
\label{tab2009231622SahayanKY_DeltaEofReductionofNitrogen}
\centering
\begin{tabular}{
		l
		>{$}r<{$}@{\ }
		S[table-format=+1.3]
		S[table-format=+2.0]
	}
	\hline
	基底関数系 & \multicolumn{2}{c}{$\Delta E/\si{\atomic}$} & {$\Delta E/\si{kcal.mol^{-1}}$} \\
	\hline
	STO-3G & 			2\times (-55.454) -\{3\times (-1.117)  -107.496\}= & -0.061 & -38 \\
	4-31G & 			2\times (-56.102) -\{3\times (-1.127)  -108.754\}= & -0.069 & -43 \\
	6-31G$^{\ast}$ & 	2\times (-56.184) -\{3\times (-1.127)  -108.942\}= & -0.045 & -28 \\
	6-31G$^{\ast\ast}$ &2\times (-56.195) -\{3\times (-1.131)  -108.942\}= & -0.055 & -34 \\
	HF極限 & 			2\times (-56.225) -\{3\times (-1.134)  -108.997\}= & -0.051 & -32 \\
	\hline
\end{tabular}
\end{table}

%
%表を挿入
\begin{table}[htpt]
\caption{各基底関数系での\ce{CO + 3H2 -> CH4 + H2O}の$\Delta E$}
\label{tab2009231622SahayanKY_DeltaEofReductionofCarbonMonoxide}
\centering
\begin{tabular}{
		l
		>{$}r<{$}@{\ }
		S[table-format=+1.3]
		S[table-format=+2.1]
	}
	\hline
	基底関数系 & \multicolumn{2}{c}{$\Delta E/\si{\atomic}$} & {$\Delta E/\si{kcal.mol^{-1}}$} \\
	\hline
	STO-3G & 			 -39.727 -74.963 -\{3\times (-1.117)  -111.225\}= & -0.114 & -71.5 \\
	4-31G & 			 -40.140 -75.907 -\{3\times (-1.127)  -112.552\}= & -0.114 & -71.5 \\
	6-31G$^{\ast}$ & 	 -40.195 -76.011 -\{3\times (-1.127)  -112.737\}= & -0.088 & -55 \\
	6-31G$^{\ast\ast}$ & -40.202 -76.023 -\{3\times (-1.131)  -112.737\}= & -0.095 & -60 \\
	HF極限 & 			 -40.225 -76.065 -\{3\times (-1.134)  -112.791\}= & -0.097 & -61 \\
	\hline
\end{tabular}
\end{table}

この計算結果から次のことが言える。
どちらの反応においても、
どの基底関数をつかっても$\Delta E$の符号は変わらない。
Hartree-Fock理論ではどちらの反応も発熱反応であると言える。
また、どちらの反応においても\SI{10}{kcal.mol^{-1}}程度の誤差が基底関数によって生じていることが言える。

実験値と比較する。
実験では、
\SI{0}{K}における$\Delta H$は
\ce{N2}で\SI{-18.604}{kcal.mol^{-1}}、
\ce{CO}で\SI{-45.894}{kcal.mol^{-1}}である。
$\Delta H$と$\Delta E$の間には$\Delta (PV)$の差があるものの、
大小の順序や符号などはよく再現できていると考えられる。
ただし、どちらの反応についても
実験値の$\Delta H$に対して計算値の$\Delta E$は\SI{14}{kcal.mol^{-1}}ほど小さい。

次にゼロ点振動エネルギーについて考える。
各分子のゼロ点振動エネルギー$\frac{1}{2}h\nu_0$を
表\ref{tab2009231622SahayanKY_ZeroPointEnergy}に示した。
\ce{N2}の還元反応におけるゼロ点振動エネルギーの差$\Delta E_{\rm v,\ce{N2}}$は
\begin{align}
	\Delta E_{\rm v,\ce{N2}}
&=
	\biggl(
		2 \cdot
		(
			1.35
			+
			2.32
			+
			2.32
			+
			4.77
			+
			4.85
			+
			4.85
		)
		-
		3.35
		-
		3 \cdot
		6.18
	\biggr) \si{kcal.mol^{-1}} \\
%
%
&=
	\SI{19.03}{kcal.mol^{-1}}
\end{align}
である。
また、\ce{CO}の還元反応については
\begin{align}
	\Delta E_{\rm v,\ce{CO}}
&=
	\biggl(
	\begin{array}{l}
		(
			1.86
			+
			1.86
			+
			1.86
			+
			2.17
			+
			2.17
			+
			4.14
			+
			4.2
			+
			4.2
			+
			4.2
		)
		+
		(
			2.28
			+
			5.13
			+
			5.33
		) \\ \quad
		-
		3.08
		-
		3 \cdot
		6.18
	\end{array}
	\biggr) \si{kcal.mol^{-1}} \\
%
%
&=
	\SI{17.8}{kcal.mol^{-1}}
\end{align}
この値は、SCF全エネルギーの差$\Delta E$と実験によるエンタルピー変化$\Delta H$の誤差を
よく説明している。また、$\Delta E$と$\Delta E_{\rm v}$のオーダーは同じである。
従って、反応のエネルギー差を考えるうえでゼロ点振動エネルギーは考慮されるべきと考えられる。


%
%表を挿入
\begin{table}[htpt]
\caption{各分子のゼロ点振動エネルギーの実験値}
\label{tab2009231622SahayanKY_ZeroPointEnergy}
\centering
\begin{tabular}{cS[table-format=1.2(1)]}
	\hline
	分子 & {$\frac{1}{2}h\nu_0/\si{kcal.mol^{-1}}$} \\
	\hline
	\ce{H2} & 6.18 \\
	\ce{N2} & 3.35 \\
	\ce{CO} & 3.08 \\
	\ce{H2O} & 2.28 \\
	         & 5.13 \\
	         & 5.33 \\
	\ce{NH3} & 1.35 \\
	         & 2.32(2) \\
	         & 4.77 \\
	         & 4.85(2) \\
	\ce{CH4} & 1.86(3) \\
	         & 2.17(2) \\
	         & 4.14 \\
	         & 4.2(3) \\
	\hline
\end{tabular}
\end{table}


