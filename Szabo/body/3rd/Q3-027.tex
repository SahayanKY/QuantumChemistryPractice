%ファイルID
%2020/09/17 1843
%->2009171843SahayanKY(ファイル作成者)
\subsection{問}
最小基底関数における\ce{H2}の全電子エネルギー$E_0$が
\begin{align}
	E_0
=
	\frac{
		F_{11}
		+
		H^{\rm core}_{11}
		+
		F_{12}
		+
		H^{\rm core}_{12}
	}{
		1
		+
		S_{12}
	}
=
	\SI{-1.8310}{\atomic}
\end{align}
であることを示せ。
更に核の反発も含めた全エネルギー$E_{\rm tot}$が
\begin{align}
	E_{\rm tot}
=
	\SI{-1.1167}{\atomic}
\end{align}
であることを示せ。


\subsection{解}
全電子エネルギー$E_0$は次式より求められる。
\begin{align}
	E_0
&=
	\frac{1}{2}
	\sum_\mu
	\sum_\nu
		P_{\nu\mu}
		(H^{\rm core}_{\mu\nu} +F_{\mu\nu})
\end{align}
既に見ている通り
\begin{align}
	P_{\mu\nu}
&=
	(1+S_{12})^{-1}
	%
	\qquad
	(\mu=1,2,\ \nu=1,2)
\end{align}
である。従って、
\begin{align}
	E_0
&=
	\frac{1}{2(1+S_{12})}
	\sum_\mu
	\sum_\nu
		(H^{\rm core}_{\mu\nu} +F_{\mu\nu}) \\
%
%
&=
	\frac{1}{2(1+S_{12})}
	\left(
		H^{\rm core}_{11}
		+
		H^{\rm core}_{12}
		+
		H^{\rm core}_{21}
		+
		H^{\rm core}_{22}
		+
		F_{11}
		+
		F_{12}
		+
		F_{21}
		+
		F_{22}
	\right) \\
%
%
&=
	\frac{1}{2(1+S_{12})}
	\left(
		2H^{\rm core}_{11}
		+
		2H^{\rm core}_{12}
		+
		2F_{11}
		+
		2F_{12}
	\right) \\
%
%
&=
	\frac{
		H^{\rm core}_{11}
		+
		H^{\rm core}_{12}
		+
		F_{11}
		+
		F_{12}
	}{
		1
		+
		S_{12}
	}
\end{align}
となる。最小基底関数で核間距離が$R=\SI{1.4}{\atomic}$の\ce{H2}では
\begin{align}
	H^{\rm core}_{11}
&=
	-1.1204 &
%
	H^{\rm core}_{12}
&=
	-0.9584 &
%
	F_{11}
&=
	-0.3655 &
%
	F_{12}
&=
	-0.5939 &
%
	S_{12}
&=
	0.6593
\end{align}
であるので
\begin{align}
	E_0
&=
	\frac{
		-1.1204
		-0.9584
		-0.3655
		-0.5939
	}{
		1+0.6593
	} \\
%
%
&=
	-1.8310
\end{align}
となる。

全エネルギー$E_{\rm tot}$については
\begin{align}
	E_{\rm tot}
&=
	E_0
	+
	\sum_{A}
	\sum_{B>A}
		\frac{Z_A Z_B}{R_{AB}} \\
%
%
&=
	E_0
	+
	\frac{1}{R} \\
%
%
&=
	-
	1.8310
	+
	\frac{1}{1.4} \\
%
%
&=
	-1.1167
\end{align}
である。
