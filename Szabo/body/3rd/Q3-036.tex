%ファイルID
%2020/09/26 14:07
%->2009261407SahayanKY(ファイル作成者)
\subsection{問}
スピン密度$\rho^S$は
\begin{align}
	\rho^S(\r)
&=
	\rho^\alpha(\r)
	-
	\rho^\beta(\r) \\
%
%
	\rho^\alpha(\r)
&=
	\sum_a^{N^\alpha}
		|\psi^\alpha_a(\r)|^2 \\
%
%
	\rho^\beta(\r)
&=
	\sum_a^{N^\beta}
		|\psi^\beta_a(\r)|^2
\end{align}
と定義されている。
これを全空間で積分すると
\begin{align}
	\int\d\r
		\rho^S(\r)
&=
	2\average{\mathscr{S}_z}
\end{align}
となることを示せ。


\subsection{解}
Hartree-Fock基底状態$\ket{\Psi_0}$における
$\mathscr{S}_z$の期待値$\average{\mathscr{S}_z}$は
\begin{align}
	\average{\mathscr{S}_z}
&=
	\frac{1}{2}
	\left(
		N^\alpha
		-
		N^\beta
	\right)
\end{align}
である。

一方でスピン密度の積分については
\begin{align}
	\int\d\r
		\rho^S(\r)
&=
	\int\d\r
		\rho^\alpha(\r)
	-
	\int\d\r
		\rho^\beta(\r) \\
%
%
&=
	\sum_a^{N^\alpha}
		\int\d\r |\psi^\alpha_a(\r)|^2
	-
	\sum_a^{N^\beta}
		\int\d\r |\psi^\beta_a(\r)|^2 \\
%
%
&=
	\sum_a^{N^\alpha}
		1
	-
	\sum_a^{N^\beta}
		1 \\
%
%
&=
	N^\alpha
	-
	N^\beta \\
%
%
&=
	2\average{\mathscr{S}_z}
\end{align}
である。
