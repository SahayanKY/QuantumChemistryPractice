%ファイルID
%2020/09/24 17:16
%->2009241716SahayanKY(ファイル作成者)
\subsection{問}
非制限軌道における軌道エネルギーは
\begin{align}
	\epsilon^\alpha_i
&=
	\spaceone{\psi^\alpha_i}{f^\alpha}{\psi^\alpha_i} \\
%
%
	\epsilon^\beta_i
&=
	\spaceone{\psi^\beta_i}{f^\beta}{\psi^\beta_i}
\end{align}
である。これを式変形して
\begin{align}
	\epsilon^\alpha_i
&=
	h^\alpha_{ii}
	+
	\sum_a^{N^\alpha}\left(
		J^{\alpha\alpha}_{ia}
		-
		K^{\alpha\alpha}_{ia}
	\right)
	+
	\sum_a^{N^\beta}
		J^{\alpha\beta}_{ia} \\
%
%
	\epsilon^\beta_i
&=
	h^\beta_{ii}
	+
	\sum_a^{N^\beta}\left(
		J^{\beta\beta}_{ia}
		-
		K^{\beta\beta}_{ia}
	\right)
	+
	\sum_a^{N^\beta}
		J^{\beta\alpha}_{ia}
\end{align}
となることを示せ。

また、全電子エネルギー$E_0$を軌道エネルギーとクーロンエネルギー、交換エネルギーによって表せ。

\subsection{解}
まず、Fock演算子は
\begin{align}
	f^\alpha(\r_1)
&=
	h(\r_1)
	+
	\sum_a^{N^\alpha}\left(
		J^\alpha_a(\r_1)
		-
		K^\alpha_a(\r_1)
	\right)
	+
	\sum_a^{N^\beta}
		J^\beta_a(\r_1) \\
%
%
	f^\beta(\r_1)
&=
	h(\r_1)
	+
	\sum_a^{N^\beta}\left(
		J^\beta_a(\r_1)
		-
		K^\beta_a(\r_1)
	\right)
	+
	\sum_a^{N^\alpha}
		J^\alpha_a(\r_1)
\end{align}
である。

従って、軌道エネルギー$\epsilon^\alpha_i$は
\begin{align}
	\epsilon^\alpha_i
&=
	\spaceone{\psi^\alpha_i}{f^\alpha}{\psi^\alpha_i} \\
%
%
&=
	\spaceone{\psi^\alpha_i}{h}{\psi^\alpha_i}
	+
	\sum_a^{N^\alpha}\left\{
		\spaceone{\psi^\alpha_i}{J^\alpha_a}{\psi^\alpha_i}
		-
		\spaceone{\psi^\alpha_i}{K^\alpha_a}{\psi^\alpha_i}
	\right\}
	+
	\sum_a^{N^\beta}
		\spaceone{\psi^\alpha_i}{J^\beta_a}{\psi^\alpha_i} \\
%
%
&=
	h^\alpha_{ii}
	+
	\sum_a^{N^\alpha}\left(
		J^{\alpha\alpha}_{ia}
		-
		K^{\alpha\alpha}_{ia}
	\right)
	+
	\sum_a^{N^\beta}
		J^{\alpha\beta}_{ia}
\end{align}


