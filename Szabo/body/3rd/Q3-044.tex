%ファイルID
%2020/10/15 17:16
%->2010151716SahayanKY(ファイル作成者)
\subsection{問}
次式
\begin{align}
	\lim_{R\rightarrow\infty}
		\ket{\Psi_0}
&=
	\frac{1}{2}
	\left[
		\ket{\aorb{\psi_1}\borb{\psi_1}}
		-
		\ket{\aorb{\psi_2}\borb{\psi_2}}
		-
		\sqrt{2}
		\ket{^3\Psi^2_1}
	\right]
\end{align}
から、
\begin{align}
	\lim_{R\rightarrow\infty}
		\ket{\Psi_0}
&=
	\ket{\aorb{\phi_1}\borb{\phi_2}}
\end{align}
を導出せよ。

ここで$\phi_1,\phi_2,\psi_1,\psi_2$は問題3.42の通りであり、
$\ket{^3\Psi_1^2}$は制限軌道における1電子励起3重項状態
\begin{align}
	\ket{^3\Psi_1^2}
&=
	\frac{1}{\sqrt{2}}
	\left(
		\ket{\Psi_{\borb{1}}^{\aorb{2}}}
		-
		\ket{\Psi_{\aorb{1}}^{\aorb{2}}}
	\right) \\
%
%
&=
	\frac{1}{\sqrt{2}}
	\left(
		\ket{\aorb{\psi_1}\borb{\psi_2}}
		-
		\ket{\aorb{\psi_2}\borb{\psi_1}}
	\right)
\end{align}
である。

\subsection{解}
まず、制限分子軌道$\psi_1,\psi_2$と原子軌道(基底関数)$\phi_1,\phi_2$の間には
\begin{align}
	\psi_1
&=
	\frac{1}{\sqrt{2(1+S_{12})}}
	(\phi_1 +\phi_2) \\
%
%
	\psi_2
&=
	\frac{1}{\sqrt{2(1-S_{12})}}
	(\phi_1 -\phi_2)
\end{align}
の関係がある。$R\rightarrow\infty$では重なりがゼロに漸近し$S_{12}=0$となるため、
\begin{align}
	\psi_1
&=
	\frac{1}{\sqrt{2}}
	(\phi_1 +\phi_2) \\
%
%
	\psi_2
&=
	\frac{1}{\sqrt{2}}
	(\phi_1 -\phi_2)
\end{align}
となる。

行列式の性質を利用すると
\begin{align}
	\lim_{R\rightarrow\infty}
		\ket{\Psi_0}
&=
	\frac{1}{2}
	\left[
		\ket{\aorb{\psi_1}\borb{\psi_1}}
		-
		\ket{\aorb{\psi_2}\borb{\psi_2}}
		-
		\ket{\aorb{\psi_1}\borb{\psi_2}}
		+
		\ket{\aorb{\psi_2}\borb{\psi_1}}
	\right] \\
%
%
&=
	\frac{1}{2}
	\left[
		\ket{(\aorb{\psi_1}+\aorb{\psi_2})\borb{\psi_1}}
		-
		\ket{(\aorb{\psi_2}+\aorb{\psi_1})\borb{\psi_2}}
	\right] \\
%
%
&=
	\frac{1}{2}
	\left[
		\sqrt{2} \ket{\aorb{\phi_1}\borb{\psi_1}}
		-
		\sqrt{2} \ket{\aorb{\phi_1}\borb{\psi_2}}
	\right] \\
%
%
&=
	\frac{1}{\sqrt{2}}
	\ket{\aorb{\phi_1}(\borb{\psi_1}-\borb{\psi_2})} \\
%
%
&=
	\frac{1}{\sqrt{2}} \cdot
	\sqrt{2} \ket{\aorb{\phi_1}\borb{\phi_2}} \\
%
%
&=
	\ket{\aorb{\phi_1}\borb{\phi_2}}
\end{align}
となる。
