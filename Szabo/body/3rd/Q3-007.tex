%ファイルID
%2020/07/27 15:52
%->2007271552SahayanKY(ファイル作成者)
\subsection{問}
Hartree-Fockハミルトニアン$\H_0$を次の通りに定義する。
\begin{align}
	\H_0
&=
	\sum_{i=1}^{N}
		f(i)
\end{align}
ここで$f(i)$は$i$番目の電子に関するFock演算子である。

まず、$\H_0$と置換演算子が可換であることを示せ。
つぎに、Hartree-Fock基底状態$\ket{\Psi_0}$が
$\H_0$の固有関数であること、
またその固有値が$\sum_a \epsilon_a$であることを示せ。


\subsection{解}
任意の$N$電子の波動関数$\psi(\x_1,\x_2,\dots,\x_N)$を考える。
また、置換演算子として$\x_1$と$\x_2$を入れ替える$\mathscr{P}_{12}$を考える。
このとき、
\begin{align}
	\H_0
	\mathscr{P}_{12}
		\psi
&=
	\H_0
		\psi(\x_2,\x_1,\dots,\x_N) \\
%
%
&=
	f(\x_1)
		\psi(\x_2,\x_1,\dots,\x_N)
	+
	f(\x_2)
		\psi(\x_2,\x_1,\dots,\x_N)
	+
	\dots
	+
	f(\x_N)
		\psi(\x_2,\x_1,\dots,\x_N) \\
%
%
&=
	f(\x_2)
		\psi(\x_2,\x_1,\dots,\x_N)
	+
	f(\x_1)
		\psi(\x_2,\x_1,\dots,\x_N)
	+
	\dots
	+
	f(\x_N)
		\psi(\x_2,\x_1,\dots,\x_N) \\
%
%
&=
	\mathscr{P}_{12}\{
		f(\x_1)
			\psi(\x_1,\x_2,\dots,\x_N)
		+
		f(\x_2)
			\psi(\x_1,\x_2,\dots,\x_N)
		+
		\dots
		+
		f(\x_N)
			\psi(\x_1,\x_2,\dots,\x_N)
	\} \\
%
%
&=
	\mathscr{P}_{12}
	\left(
		\sum_{i} f(\x_i)
	\right)
		\psi(\x_1,\x_2,\dots,\x_N) \\
%
%
&=
	\mathscr{P}_{12}
	\H_0
		\psi
\end{align}
である。従って、$\mathscr{P}_{12}$と$\H_0$は可換である。
同様のことが任意の置換演算子についても成立するので、
一般に置換演算子と$\H_0$は可換である。

Hartree-Fock基底状態$\ket{\Psi_0}$をSlater行列式であらわに書くと
\begin{align}
	\ket{\Psi_0}
&=
	\frac{1}{\sqrt{N!}}
	\sum_{n}^{N!}
		(-1)^{p_n}
		\mathscr{P}_n\{
			\chi_1(\x_1)\chi_2(\x_2)\dots\chi_N(\x_N)
		\}
\end{align}
である。これに$\H_0$を作用させると、$\H_0$と$\mathscr{P}_n$は可換であるので
\begin{align}
	\H_0 \ket{\Psi_0}
&=
	\frac{1}{\sqrt{N!}}
	\sum_{n}^{N!}
		(-1)^{p_n}
		\H_0
		\mathscr{P}_n\{
			\chi_1(\x_1)\chi_2(\x_2)\dots\chi_N(\x_N)
		\} \\
%
%
&=
	\frac{1}{\sqrt{N!}}
	\sum_{n}^{N!}
		(-1)^{p_n}
		\mathscr{P}_n
		\H_0\{
			\chi_1(\x_1)\chi_2(\x_2)\dots\chi_N(\x_N)
		\}
\end{align}
となる。$\H_0$以降について見ると、
\begin{align}
	\H_0\{
		\chi_1(\x_1)\chi_2(\x_2)\dots\chi_N(\x_N)
	\}
&=
	\left(
		\sum_{i}^{N} f(\x_i)
	\right)\{
		\chi_1(\x_1)\chi_2(\x_2)\dots\chi_N(\x_N)
	\} \\
%
%
&=
	f(\x_1) \chi_1(\x_1)\chi_2(\x_2)\dots\chi_N(\x_N) \nonumber \\&\quad
	+
	f(\x_2) \chi_1(\x_1)\chi_2(\x_2)\dots\chi_N(\x_N) \nonumber \\&\quad
	+
	\dots
	+
	f(\x_N) \chi_1(\x_1)\chi_2(\x_2)\dots\chi_N(\x_N) \\
%
%
&=
	\epsilon_1 \chi_1(\x_1)\chi_2(\x_2)\dots\chi_N(\x_N) \nonumber \\&\quad
	+
	\epsilon_2 \chi_1(\x_1)\chi_2(\x_2)\dots\chi_N(\x_N) \nonumber \\&\quad
	+
	\dots
	+
	\epsilon_N \chi_1(\x_1)\chi_2(\x_2)\dots\chi_N(\x_N) \\
%
%
&=
	\left(
		\sum_{a}^{N} \epsilon_a
	\right)
	\chi_1(\x_1)\chi_2\dots\chi_N(\x_N)
\end{align}
となるので、
\begin{align}
	\H_0\ket{\Psi_0}
&=
	\frac{1}{\sqrt{N!}}
	\sum_{n}^{N!}
		(-1)^{p_n}
		\mathscr{P}_n\left\{
			\left(
				\sum_{a}^{N} \epsilon_a
			\right)
			\chi_1(\x_1)\chi_2(\x_2)\dots\chi_N(\x_N)
		\right\} \\
%
%
&=
	\left(
		\sum_{a}^{N} \epsilon_a
	\right)
	\frac{1}{\sqrt{N!}}
	\sum_{n}^{N!}
		(-1)^{p_n}
		\mathscr{P}_n\{
			\chi_1(\x_1)\chi_2(\x_2)\dots\chi_N(\x_N)
		\} \\
%
%
&=
	\left(
		\sum_{a}^{N} \epsilon_a
	\right)
	\ket{\Psi_0}
\end{align}
となる。従って、$\ket{\Psi_0}$は$\H_0$の固有関数であり、
その固有値は$\sum_a \epsilon_a$である。

