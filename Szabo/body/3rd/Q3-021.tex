%ファイルID
%2020/09/13 1840
%->2009131840SahayanKY(ファイル作成者)
\subsection{問}
STO-1G($\zeta=1.24$)をもった2つの軌道を考える。
その軌道の中心間の距離が$R=\SI{1.4}{\atomic}$であるとき
重なりが$S_{12}=0.6648$であることを示せ。


\subsection{解}
STO-1G($\zeta=1.24$)の各原始Gauss型関数の軌道指数$\alpha$は
\begin{align}
	\alpha
&=
	\alpha(\zeta=1.0) \times \zeta^2
\end{align}
より求められるので、STO-1G($\zeta=1.24$)は
\begin{align}
	\phi^{\rm CGF}_{\rm 1s}(\zeta=1.24,\text{STO-1G})
&=
	\phi^{\rm GF}_{\rm 1s}(0.270950\times 1.24^2) \\
%
%
&=
	\phi^{\rm GF}_{\rm 1s}(0.416613) \\
%
%
&=
	\left(
		\frac{2\cdot 0.416613}{\pi}
	\right)^{\frac{3}{4}}
	\exp(-0.416613|\r|^2) \\
%
%
&=
	0.369581
	\exp(-0.416613|\r|^2)
\end{align}
である。

重なり積分を計算するにあたって
一方の中心を$\bm{R}_A=\bm{0}$、
もう一方の中心を$\bm{R}_B=\SI{1.4}{\atomic}\e{x}$と置く。
重なり積分$S_{12}$は
\begin{align}
	S_{12}
&=
	\int\d\r
		\phi^{\rm CGF}_{\rm 1s}(\zeta=1.24,\text{STO-1G},\r-\bm{R}_A)
		\phi^{\rm CGF}_{\rm 1s}(\zeta=1.24,\text{STO-1G},\r-\bm{R}_B) \\
%
%
&=
	0.369581^2
	\int\d\r
		\exp(-0.416613|\r-\bm{R}_A|^2)
		\exp(-0.416613|\r-\bm{R}_B|^2)
\end{align}
積分部分は付録の積分公式を用いることで
\begin{align}
	S_{12}
&=
	0.369581^2\cdot
	\left(
		\frac{\pi}{2\cdot 0.416613}
	\right)^{\frac{3}{2}}
	\exp\left(
		-\frac{0.416613^2}{2\cdot 0.416613} \cdot
		|1.4\e{x}|^2
	\right) \\
%
%
&=
	0.664792 \\
%
%
&=
	0.6648
\end{align}
である。
