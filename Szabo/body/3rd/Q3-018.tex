%ファイルID
%2020/08/08 1842
%->2008081842SahayanKY(ファイル作成者)
\subsection{問}
電子数$N$は密度行列$\bm{P}$と重なり行列$\bm{S}$を用いて
\begin{align}
	N
=
	\sum_\mu
		(\bm{P}\bm{S})_{\mu\mu}
=
	\sum_\mu
		(\bm{S}^{\alpha}\bm{P}\bm{S}^{1-\alpha})_{\mu\mu}
\end{align}
と表すことが出来る。ここで$\tr{\bm{A}\bm{B}}=\tr{\bm{B}\bm{A}}$を用いている。

$\alpha=\frac{1}{2}$とすると
\begin{align}
	\sum_\mu
		(\bm{S}^{\frac{1}{2}}\bm{P}\bm{S}^{\frac{1}{2}})_{\mu\mu}
=
	\sum_\mu
		(\bm{P}')_{\mu\mu}
\end{align}
となることを示せ。ここで$\bm{P}'$は対称直交化された基底における密度行列
\begin{align}
	\rho(\r)
&=
	\sum_\mu
	\sum_\nu
		P'_{\mu\nu}
		\phi'_{\mu}(\r)
		\conju{\phi'_{\nu}}(\r) \\
%
%
	\phi'_{\mu}(\r)
&=
	\sum_\nu
		(\bm{S}^{-\frac{1}{2}})_{\nu\mu}
		\phi_\nu(\r)
\end{align}
である。

\subsection{解}
$\bm{P}$と$\bm{P}'$の関係を考える。
\begin{align}
	\rho(\r)
&=
	\sum_\mu
	\sum_\nu
		P'_{\mu\nu}
		\phi'_{\mu}(\r)
		\conju{\phi'_{\nu}}(\r) \\
%
%
&=
	\sum_\mu
	\sum_\nu
		P'_{\mu\nu}
		\left(
			\sum_\sigma
				(\bm{S}^{-\frac{1}{2}})_{\sigma\mu}
				\phi_{\sigma}(\r)
		\right)
		\conju[1]{
			\sum_\lambda
				(\bm{S}^{-\frac{1}{2}})_{\lambda\nu}
				\phi_{\lambda}(\r)
		} \\
%
%
&=
	\sum_\mu
	\sum_\nu
	\sum_\sigma
	\sum_\lambda
		P'_{\mu\nu}
		(\bm{S}^{-\frac{1}{2}})_{\sigma\mu}
		\conju{(\bm{S}^{-\frac{1}{2}})_{\lambda\nu}}
		\phi_\sigma(\r)
		\conju{\phi_\lambda}(\r) \\
%
%
&=
	\sum_\sigma
	\sum_\lambda
		P_{\sigma\lambda}
		\phi_\sigma(\r)
		\conju{\phi_\lambda}(\r)
\end{align}
従って、
\begin{align}
	P_{\sigma\lambda}
&=
	\sum_\mu
	\sum_\nu
		P'_{\mu\nu}
		(\bm{S}^{-\frac{1}{2}})_{\sigma\mu}
		\conju{(\bm{S}^{-\frac{1}{2}})_{\lambda\nu}} \\
%
%
&=
	\sum_\mu
	\sum_\nu
		(\bm{S}^{-\frac{1}{2}})_{\sigma\mu}
		P'_{\mu\nu}
		(\adj{\bm{S}^{-\frac{1}{2}}})_{\nu\lambda} \\
%
%
	\bm{P}
&=
	\bm{S}^{-\frac{1}{2}}
	\bm{P}'
	\adj{\bm{S}^{-\frac{1}{2}}}
\end{align}
$\adj{\bm{S}^{-\frac{1}{2}}}$は
\begin{align}
	\bm{S}^{-\frac{1}{2}}
&=
	\bm{U}
	\bm{s}^{-\frac{1}{2}}
	\adj{\bm{U}} \\
%
%
	\adj{\bm{S}^{-\frac{1}{2}}}
&=
	\bm{\bm{U}}
	\adj{\bm{s}^{-\frac{1}{2}}}
	\adj{\bm{U}}
\end{align}
である。$\bm{s}$は$\bm{S}$の固有値を並べた対角行列である。
$\bm{S}$の固有値は前問より正数であるので、
$\bm{s}^{-\frac{1}{2}}$は正数の対角行列である。
よって、エルミート行列($\adj{\bm{s}^{-\frac{1}{2}}}=\bm{s}^{-\frac{1}{2}}$)
であるので、
\begin{align}
	\adj{\bm{S}^{-\frac{1}{2}}}
&=
	\bm{S}^{-\frac{1}{2}}
\end{align}
である。よって、
\begin{align}
	\bm{P}
&=
	\bm{S}^{-\frac{1}{2}}
	\bm{P}'
	\bm{S}^{-\frac{1}{2}}
\end{align}
である。

以上から、
\begin{align}
	\sum_\mu
		\left(
			\bm{S}^{\frac{1}{2}}
			\bm{P}
			\bm{S}^{\frac{1}{2}}
		\right)_{\mu\mu}
&=
	\sum_\mu
		\left(
			\bm{S}^{\frac{1}{2}}
			\bm{S}^{-\frac{1}{2}}
			\bm{P}'
			\bm{S}^{-\frac{1}{2}}
			\bm{S}^{\frac{1}{2}}
		\right)_{\mu\mu} \\
%
%
&=
	\sum_\mu
		\left(
			\identity
			\bm{P}'
			\identity
		\right)_{\mu\mu} \\
%
%
&=
	\sum_\mu
		(\bm{P}')_{\mu\mu}
\end{align}
である。

