%ファイルID
%2020/08/05 2039
%->2008052039SahayanKY(ファイル作成者)
\subsection{問}
Roothaanの方程式の計算を簡単にするために、
基底関数$\phi_\nu$を直交化させた$\phi'_\mu$を考える。
つまり、
\begin{align}
	\phi'_\mu
&=
	\sum_\nu
		X_{\nu\mu} \phi_\nu
\end{align}
\begin{align}
	\psi_i
=
	\sum_\mu
		C'_{\mu i}\phi'_\mu
=
	\sum_\nu
		C_{\nu i}\phi_\nu
\end{align}
また、直交化基底関数におけるFock行列は次の通りである。
\begin{align}
	F'_{\mu\nu}
&=
	\int\d\r_1
		\conju{\phi'_\mu}(1)
		f(1)
		\phi'_\nu(1)
\end{align}

このとき、以下の式が成立することを示せ。
\begin{align}
	\bm{C}'
&=
	\bm{X}^{-1}
	\bm{C} \\
%
%
	\bm{F}'
	\bm{C}'
&=
	\bm{C}'
	\bm{\epsilon}
\end{align}


\subsection{解}



