%ファイルID
%2020/08/05 2039
%->2008052039SahayanKY(ファイル作成者)
\subsection{問}
Roothaanの方程式の計算を簡単にするために、
基底関数$\phi_\nu$を直交化させた$\phi'_\mu$を考える。
つまり、
\begin{align}
	\phi'_\mu
&=
	\sum_\nu
		X_{\nu\mu} \phi_\nu
\end{align}
\begin{align}
	\psi_i
=
	\sum_\mu
		C'_{\mu i}\phi'_\mu
=
	\sum_\nu
		C_{\nu i}\phi_\nu
\end{align}
また、直交化基底関数におけるFock行列は次の通りである。
\begin{align}
	F'_{\mu\nu}
&=
	\int\d\r_1
		\conju{\phi'_\mu}(1)
		f(1)
		\phi'_\nu(1)
\end{align}

このとき、以下の式が成立することを示せ。
\begin{align}
	\bm{C}'
&=
	\bm{X}^{-1}
	\bm{C} \\
%
%
	\bm{F}'
&=
	\adj{\bm{X}}
	\bm{F}
	\bm{X}
\end{align}


\subsection{解}
前者の式についてまず見ていく。
空間軌道の展開の式から、
\begin{align}
	\sum_\mu
		C'_{\mu i}
		\phi'_\mu
&=
	\sum_\nu
		C_{\nu i}
		\phi_\nu \\
%
%
	\sum_\mu
		C'_{\mu i}
		\left(
			\sum_\nu
				X_{\nu\mu}
				\phi_\nu
		\right)
&=
	\sum_\nu
		C_{\nu i}
		\phi_\nu \\
%
%
	\sum_\nu
		\left(
			\sum_\mu
				C'_{\mu i}
				X_{\nu\mu}
		\right)
		\phi_\nu
&=
	\sum_\nu
		C_{\nu i}
		\phi_\nu \\
%
%
	\sum_\nu
		\left(
			C_{\nu i}
			-
			\sum_{\mu}
				C'_{\mu i}
				X_{\nu\mu}
		\right)
		\phi_\nu
&=
	0
\end{align}
$\phi_\nu$は互いに線形独立であることから、係数はゼロになる必要がある。
従って、
\begin{align}
	C_{\nu i}
&=
	\sum_\mu
		C'_{\mu i}
		X_{\nu\mu} \\
%
%
	\bm{C}
&=
	\bm{X}
	\bm{C}' \\
%
%
	\bm{C}'
&=
	\bm{X}^{-1}
	\bm{C}
\end{align}
である。


次に後者の式について見ていく。
\begin{align}
	F'_{\mu\nu}
&=
	\int\d\r_1
		\conju{\phi'_\mu}(1)
		f(1)
		\phi'_\nu(1) \\
%
%
&=
	\int\d\r_1
		\conju[1]{
			\sum_\sigma
				X_{\sigma\mu}
				\phi_\sigma(1)
		}
		f(1)
		\left(
			\sum_\lambda
				X_{\lambda\nu}
				\phi_\lambda(1)
		\right) \\
%
%
&=
	\sum_\sigma
	\sum_\lambda
		\conju{X_{\sigma\mu}}
		X_{\lambda\nu}
		\int\d\r_1
			\conju{\phi_\sigma}(1)
			f(1)
			\phi_\lambda(1) \\
%
%
&=
	\sum_\sigma
	\sum_\lambda
		\conju{X_{\sigma\mu}}
		X_{\lambda\nu}
		F_{\sigma\lambda} \\
%
%
&=
	\sum_\sigma
	\sum_\lambda
		\adj{X}_{\mu\sigma}
		F_{\sigma\lambda}
		X_{\lambda\nu} \\
%
%
&=
	\adj{\bm{X}}
	\bm{F}
	\bm{X}
\end{align}




