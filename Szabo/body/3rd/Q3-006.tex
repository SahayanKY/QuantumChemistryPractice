%ファイルID
%2020/07/26 18:56
%->2007261856SahayanKY(ファイル作成者)
\subsection{問}
次の式
\begin{align}
	E
&=
	\sum_i^{\rm occ}
		\braket{i|h|i}
	+
	\frac{1}{2}
	\sum_i^{\rm occ}
	\sum_j^{\rm occ}
		\antitwo{ij}{ij}
\end{align}
を用いて、
\begin{align}
	^{N}E_{0}
	-
	^{N+1}E^r
&=
	-
	\braket{r|h|r}
	-
	\sum_b
		\antitwo{rb}{rb}
\end{align}
を示せ。ここで$^{N}E_0$は
$N$電子系のHartree-Fock方程式のスピン軌道で得られる
$N$電子系基底状態のSlater行列式$\ket{^{N}\Psi_0}$のエネルギーの期待値、
$^{N+1}E^r$は
そのスピン軌道の内の$\chi_r$を$\ket{^{N}\Psi_0}$に追加して得られる
$\ket{^{N+1}\Psi^r}$のエネルギーの期待値である。
\begin{align}
	^{N}E_0
&=
	\braket{^{N}\Psi_0|\H|^{N}\Psi_0} &
%
	^{N+1}E^r
&=
	\braket{^{N+1}\Psi^r|\H|^{N+1}\Psi^r}
\end{align}



\subsection{解}
$^{N+1}E^r$は
\begin{align}
	^{N+1}E^r
&=
	\sum_{i=1,\dots,N,r}
		\braket{i|h|i}
	+
	\frac{1}{2}
	\sum_{i=1,\dots,N,r}
	\sum_{j=1,\dots,N,r}
		\antitwo{ij}{ij} \\
%
%
&=
	\sum_{i=1,\dots,N}
		\braket{i|h|i}
	+
	\braket{r|h|r} \nonumber \\&\qquad
	+
	\frac{1}{2}
	\sum_{i=1,\dots,N,r}\left(
		\sum_{j=1,\dots,N}
			\antitwo{ij}{ij}
		+
		\antitwo{ir}{ir}
	\right) \\
%
%
&=
	\sum_{i=1,\dots,N}
		\braket{i|h|i}
	+
	\braket{r|h|r} \nonumber \\&\qquad
	+
	\frac{1}{2}
	\left\{
		\sum_{i=1,\dots,N}\left(
			\sum_{j=1,\dots,N}
				\antitwo{ij}{ij}
			+
			\antitwo{ir}{ir}
		\right)
		+
		\left(
			\sum_{j=1,\dots,N}
				\antitwo{rj}{rj}
			+
			\antitwo{rr}{rr}
		\right)
	\right\} \\
%
%
&=
	\sum_{i=1,\dots,N}
		\braket{i|h|i}
	+
	\frac{1}{2}
	\sum_{i=1,\dots,N}
	\sum_{j=1,\dots,N}
		\antitwo{ij}{ij} \nonumber \\&\qquad
	+
	\braket{r|h|r}
	+
	\frac{1}{2}\left\{
		\sum_{i=1,\dots,N}
			\antitwo{ir}{ir}
		+
		\sum_{j=1,\dots,N}
			\antitwo{rj}{rj}
	\right\} \nonumber \\&\qquad
	%
	(\because \antitwo{ij}{kk}=0) \\
%
%
&=
	^{N}E_0
	+
	\braket{r|h|r}
	+
	\sum_{j=1,\dots,N}
		\antitwo{rj}{rj}
	%
	\qquad
	(\because \antitwo{ij}{kl}=\antitwo{ji}{lk})
\end{align}
となる。従って、
\begin{align}
	^{N}E_{0}
	-
	^{N+1}E^r
&=
	-
	\braket{r|h|r}
	-
	\sum_b
		\antitwo{rb}{rb}
\end{align}
となる。

