%ファイルID
%2020/09/16 1908
%->2009161908SahayanKY(ファイル作成者)
\subsection{問}
最小基底関数における\ce{H2}の軌道エネルギーが、
Roothaan方程式より
\begin{align}
	\epsilon_1
&=
	\frac{F_{11}+F_{12}}{1+S_{12}}
=
	\SI{-0.5782}{\atomic} \\
%
%
	\epsilon_2
&=
	\frac{F_{11}-F_{12}}{1-S_{12}}
=
	\SI{+0.6703}{\atomic}
\end{align}
となることを示せ。


\subsection{解}
前問3.25より$F_{11}=F_{22}, F_{12}=F_{21}$である。
従って、Roothaan方程式は
\begin{align}
&
	\left[
	\begin{array}{cc}
		F_{11} & F_{12} \\
		F_{12} & F_{11}
	\end{array}
	\right]
	\left[
	\begin{array}{cc}
		\left[2(1+S_{12})\right]^{-1/2} & \left[2(1-S_{12})\right]^{-1/2} \\
		%
		\left[2(1+S_{12})\right]^{-1/2} & \left[2(1-S_{12})\right]^{-1/2}
	\end{array}
	\right] \nonumber\\
&\qquad\qquad
=
	\left[
	\begin{array}{cc}
		1 & S_{12} \\
		S_{12} & 1
	\end{array}
	\right]
	\left[
	\begin{array}{cc}
		\left[2(1+S_{12})\right]^{-1/2} & \left[2(1-S_{12})\right]^{-1/2} \\
		%
		\left[2(1+S_{12})\right]^{-1/2} & -\left[2(1-S_{12})\right]^{-1/2}
	\end{array}
	\right]
	\left[
	\begin{array}{cc}
		\epsilon_1 & 0 \\
		0 & \epsilon_2
	\end{array}
	\right]
\end{align}
\begin{align}
	\left\{
	\begin{array}{>{\displaystyle}l}
			\epsilon_1
		=
			\frac{F_{11}+F_{12}}{1+S_{12}} \\[3mm]
		%
			\epsilon_2
		=
			\frac{F_{11}-F_{12}}{1-S_{12}}
	\end{array}
	\right.
\end{align}
となる。解く過程は問題3.23と同一であるので省略する。

前問より$F_{11}=-0.3655, F_{12}=-0.5939$、また、$S_{12}=0.6593$であるので
\begin{align}
	\left\{
	\begin{array}{>{\displaystyle}l}
			\epsilon_1
		=
			\frac{-0.3655-0.5939}{1+0.6593}
		=
			-0.5782 \\[3mm]
		%
			\epsilon_2
		=
			\frac{-0.3655+0.5939}{1-0.6593}
		=
			0.6704
	\end{array}
	\right.
\end{align}
である。