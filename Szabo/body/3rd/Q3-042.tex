%ファイルID
%2020/10/01 13:53
%->2010011353SahayanKY(ファイル作成者)
\subsection{問}
水素分子\ce{H2}の最小基底関数系における制限分子軌道は次の二つである。
\begin{align}
	\psi_1
&=
	\left[
		2(1+S_{12})
	\right]^{-1/2}
	(\phi_1 +\phi_2) \\
%
%
	\psi_2
&=
	\left[
		2(1-S_{12})
	\right]^{-1/2}
	(\phi_1 -\phi_2)
\end{align}
この2つから非制限分子軌道を書き表す。
占有軌道については
\begin{align}
	\psi^\alpha_1
&=
	\cos\theta
	\psi_1
	+
	\sin\theta
	\psi_2 \\
%
%
	\psi^\beta_1
&=
	\cos\theta
	\psi_1
	-
	\sin\theta
	\psi_2
\end{align}
であり、仮想軌道については
\begin{align}
	\psi^\alpha_2
&=
	-\sin\theta
	\psi_1
	+
	\cos\theta
	\psi_2 \\
%
%
	\psi^\beta_2
&=
	\sin\theta
	\psi_1
	+
	\cos\theta
	\psi_2
\end{align}
とする。

このとき、$\alpha$軌道の空間軌道$\psi^\alpha_1,\psi^\alpha_2$同士、
$\beta$軌道の空間軌道$\psi^\beta_1,\psi^\beta_2$同士で規格直交系をなしていることを確かめよ。


\subsection{解}
$\alpha$軌道から見ていく。
\begin{align}
	\braket{\psi^\alpha_1|\psi^\alpha_1}
&=
	\braket{
		(\cos\theta\psi_1 +\sin\theta\psi_2)|
		(\cos\theta\psi_1 +\sin\theta\psi_2)
	} \\
%
%
&=
	\conju[1]{\cos\theta}
	\cos\theta
	\braket{\psi_1|\psi_1}
	+
	\conju[1]{\cos\theta}
	\sin\theta
	\braket{\psi_1|\psi_2} \nonumber \\ &\qquad
	+
	\conju[1]{\sin\theta}
	\cos\theta
	\braket{\psi_2|\psi_1}
	+
	\conju[1]{\sin\theta}
	\sin\theta
	\braket{\psi_2|\psi_2} \\
%
%
&=
	\cos^2\theta
	+
	\sin^2\theta \\
%
%
&=
	1
\end{align}
\begin{align}
	\braket{\psi^\alpha_2|\psi^\alpha_2}
&=
	\braket{
		(-\sin\theta\psi_1 +\cos\theta\psi_2)|
		(-\sin\theta\psi_1 +\cos\theta\psi_2)
	} \\
%
%
&=
	\conju[1]{-\sin\theta}
	(-\sin\theta)
	+
	\conju[1]{\cos\theta}
	\cos\theta \\
%
%
&=
	1
\end{align}
\begin{align}
	\braket{\psi^\alpha_1|\psi^\alpha_2}
&=
	\braket{
		(\cos\theta\psi_1 +\sin\theta\psi_2)|
		(-\sin\theta\psi_1 +\cos\theta\psi_2)
	} \\
%
%
&=
	\conju[1]{\cos\theta}
	(-\sin\theta)
	+
	\conju[1]{\sin\theta}
	\cos\theta \\
%
%
&=
	0
\end{align}
従って、$\alpha$軌道$\psi^\alpha_1,\psi^\alpha_2$は規格直交系をなしている。

次に$\beta$軌道について見ていく。
\begin{align}
	\braket{\psi^\beta_1|\psi^\beta_1}
&=
	\braket{
		(\cos\theta\psi_1 -\sin\theta\psi_2)|
		(\cos\theta\psi_1 -\sin\theta\psi_2)
	} \\
%
%
&=
	\conju[1]{\cos\theta}
	\cos\theta
	+
	\conju[1]{-\sin\theta}
	(-\sin\theta) \\
%
%
&=
	1
\end{align}
\begin{align}
	\braket{\psi^\beta_2|\psi^\beta_2}
&=
	\braket{
		(\sin\theta\psi_1 +\cos\theta\psi_2)|
		(\sin\theta\psi_1 +\cos\theta\psi_2)
	} \\
%
%
&=
	\conju[1]{\sin\theta}
	\sin\theta
	+
	\conju[1]{\cos\theta}
	\cos\theta \\
%
%
&=
	1
\end{align}
\begin{align}
	\braket{\psi^\beta_1|\psi^\beta_2}
&=
	\braket{
		(\cos\theta\psi_1 -\sin\theta\psi_2)|
		(\sin\theta\psi_1 +\cos\theta\psi_2)
	} \\
%
%
&=
	\conju[1]{\cos\theta}
	\sin\theta
	+
	\conju[1]{-\sin\theta}
	\cos\theta \\
%
%
&=
	0
\end{align}
従って、$\beta$軌道$\psi^\beta_1,\psi^\beta_2$は規格直交系をなしている。

