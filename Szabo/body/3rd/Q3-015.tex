%ファイルID
%2020/08/04 1655
%->2008041655SahayanKY(ファイル作成者)
\subsection{問}
Roothaan方程式における重なり行列$\bm{S}$は次の通りである。
\begin{align}
	S_{\mu\nu}
&=
	\int\d\r
		\conju{\phi_\mu}
		\phi_\nu
\end{align}
このとき、$\bm{S}$の固有値が全て正になることを示せ。


\subsection{解}
$c_\mu^i$を$\bm{S}$の$i$番目の固有ベクトルの$\mu$行目の要素とする。
このとき、以下の式が成立する。
\begin{align}
	\sum_\nu
		S_{\mu\nu} c_\nu^i
&=
	s_i
	c_\mu^i
\end{align}
両辺に$\conju{c_\mu^i}$をかけ、$\mu$で和をとると
\begin{align}
	\sum_\mu
	\sum_\nu
		\conju{c_\mu^i}
		S_{\mu\nu}
		c_\nu^i
&=
	\sum_\mu
		\conju{c_\mu^i}
		s_i
		c_\mu^i \\
%
%
&=
	s_i
	\sum_\mu
		|c_\mu^i|^2 \\
%
%
	s_i
&=
	\frac{
		\sum_\mu
		\sum_\nu
			\conju{c_\mu^i}
			S_{\mu\nu}
			c_\nu^i
	}{
		\sum_\mu
			|c_\mu^i|^2
	}
\end{align}
ここで分母は正数$|c_\mu^i|^2$の総和であるので常に正数である。
従って、固有値$s_i$が正数かどうかは
分子によって決まる。
分子については、$S_{\mu\nu}$の定義を利用すると
\begin{align}
	\sum_\mu
	\sum_\nu
		\conju{c_\mu^i}
		S_{\mu\nu}
		c_\nu^i
&=
	\sum_\mu
	\sum_\nu
		\int\d\r
			\conju{c_\mu^i}
			\conju{\phi_\mu}
			\phi_\nu
			c_\nu^i \\
%
%
&=
	\int\d\r
		\conju[1]{
			\sum_\mu
				c_\mu^i
				\phi_\mu
		}
		\left(
			\sum_\nu
				c_\nu^i
				\phi_\nu
		\right) \\
%
%
&=
	\int\d\r
		\conju{\varphi^i}
		\varphi^i \\
%
%
&=
	\int\d\r
		|\varphi^i|^2
\end{align}
となる。(ここで$c_\mu^i$は重なり行列$\bm{S}$を対角化する行列中のベクトルなので
$\varphi^i$は空間軌道$\psi_i$とは関係がないことに注意)従って、分子も常に正数になることが言える。

よって、全ての$i$について常に$s_i$は正数になることから、
重なり行列$\bm{S}$の固有値は常に正数である。




