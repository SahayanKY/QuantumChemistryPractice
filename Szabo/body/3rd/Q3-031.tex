%ファイルID
%2020/09/23 11:45
%->2009231145SahayanKY(ファイル作成者)
\subsection{問}
ベンゼンについて、STO-3G、4-31G、6-31G$^{\ast}$、6-31G$^{\ast\ast}$の
基底関数の総数を求めよ。

\subsection{解}
ベンゼンの分子式は\ce{C6H6}である。従って、元素\ce{C}、\ce{H}それぞれについて
基底関数を求めればよい。

元素\ce{C}について考える。
STO-3Gでは\orbital{1}{s}{}, \orbital{2}{s}{}, \orbital{2}{p}{x}, \orbital{2}{p}{y}, \orbital{2}{p}{z}の5個の基底関数を用いる。
4-31Gでは\orbital{1}{s}{}, \orbital{2}{s'}{}, \orbital{2}{p'}{x}, \orbital{2}{p'}{y}, \orbital{2}{p'}{z},
\orbital{2}{s''}{}, \orbital{2}{p''}{x}, \orbital{2}{p''}{y}, \orbital{2}{p''}{z}
の9個である。
6-31G$^{\ast}$と6-31G$^{\ast\ast}$では
\orbital{1}{s}{}, \orbital{2}{s'}{}, \orbital{2}{p'}{x}, \orbital{2}{p'}{y}, \orbital{2}{p'}{z},
\orbital{2}{s''}{}, \orbital{2}{p''}{x}, \orbital{2}{p''}{y}, \orbital{2}{p''}{z},
\orbital{3}{d}{xx}, \orbital{3}{d}{yy}, \orbital{3}{d}{zz}, \orbital{3}{d}{xy}, \orbital{3}{d}{yz}, \orbital{3}{d}{zx}
の15個である。

次に元素\ce{H}について考える。
STO-3Gでは\orbital{1}{s}{}の1個である。
4-31Gと6-31G$^{\ast}$では\orbital{1}{s'}{}, \orbital{1}{s''}{}の2個である。
6-31G$^{\ast\ast}$では
\orbital{1}{s'}{}, \orbital{1}{s''}{}, \orbital{2}{p}{x}, \orbital{2}{p}{y}, \orbital{2}{p}{z}の5個である。

従って、
STO-3Gでは$6\times 5 + 6\times 1 =36$、
4-31Gでは$6\times 9 + 6\times 2 =66$、
6-31G$^{\ast}$では$6\times 15 + 6\times 2 =102$、
6-31G$^{\ast\ast}$では$6\times 15 + 6\times 5 =120$
個の基底関数を用いることになる。

