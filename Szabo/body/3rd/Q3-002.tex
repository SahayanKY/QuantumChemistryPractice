%ファイルID
%2020/07/09 16:00
%->2007091600SahayanKY(ファイル作成者)
\subsection{問}
Hartree-Fock方程式を導くにあたって、以下の汎関数を極小化する。
\begin{align}
	\mathscr{L}[\{\chi_a\}]
&=
	E_0[\{\chi_a\}]
	-
	\sum_{a=1}^{N}
	\sum_{b=1}^{N}
		\epsilon_{ba}
		(\braket{a|b}-\delta_{ab})
\end{align}
ここで$E_0[\{\chi_a\}]=\braket{\Psi_0|\H|\Psi_0}$であり、
$\{\chi_a\}$は$\ket{\Psi_0}$のスピン軌道の組である。

Lagrangeの乗数$\epsilon_{ba}$がエルミート行列の要素であること、
すなわち
\begin{align}
	\epsilon_{ba}
&=
	\conju{\epsilon_{ab}}
\end{align}
であることを示せ。


\subsection{解}
$\mathscr{L}$を極小化するためには、
$\mathscr{L}$の値域は実数でなければならない。
また、$\H$がエルミート演算子であることから、
$E_0=\braket{\Psi_0|\H|\Psi_0}$は実数である。
従って、
\begin{align}
	\conju{\mathscr{L}}
&=
	\conju{E_0}
	-
	\conju[1]{
		\sum_{a=1}^{N}
		\sum_{b=1}^{N}
			\epsilon_{ba}
			(\braket{a|b}-\delta_{ab})
	} \\
%
%
	\mathscr{L}
&=
	E_0
	-
	\sum_{a=1}^{N}
	\sum_{b=1}^{N}
		\conju{\epsilon_{ba}}
		(\conju{\braket{a|b}}-\conju{\delta_{ab}}) \\
%
%
&=
	E_0
	-
	\sum_{a=1}^{N}
	\sum_{b=1}^{N}
		\conju{\epsilon_{ba}}
		(\braket{b|a}-\delta_{ab}) \\
%
%
&=
	E_0
	-
	\sum_{b=1}^{N}
	\sum_{a=1}^{N}
		\conju{\epsilon_{ab}}
		(\braket{a|b}-\delta_{ba}) \\
%
%
&=
	E_0
	-
	\sum_{a=1}^{N}
	\sum_{b=1}^{N}
		\conju{\epsilon_{ab}}
		(\braket{a|b}-\delta_{ab})
\end{align}
である。したがって、元の式と辺々引くと
\begin{align}
	0
&=
	\sum_{a=1}^{N}
	\sum_{b=1}^{N}
		(
			\epsilon_{ba}-\conju{\epsilon_{ab}}
		)
		(
			\braket{a|b}-\delta_{ab}
		)
\end{align}
($\mathscr{L}$を極小化する範囲として)$\ket{a}$や$\ket{b}$は独立に選ぶことができるので、
各項の$\braket{a|b}-\delta_{ab}$は独立である。従って、
各項の係数がゼロでなくてはならず、
\begin{align}
	\forall a,b
	\quad
	%
	\epsilon_{ba}
&=
	\conju{\epsilon_{ab}}
\end{align}
であることが、$\mathscr{L}$を極小化するために必要である。
