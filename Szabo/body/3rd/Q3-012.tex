%ファイルID
%2020/07/30 1240
%->2007301240SahayanKY(ファイル作成者)
\subsection{問}
問題3.10の結果を用いて以下の式を示せ。
\begin{align}
	\bm{P}
	\bm{S}
	\bm{P}
&=
	2\bm{P}
\end{align}
ここで$\bm{P}$は密度行列であり、
\begin{align}
	P_{\mu\nu}
&=
	2
	\sum_a^{N/2}
		C_{\mu a} \conju{C_{\nu a}}
	%
	\qquad
	(\mu,\nu=1,\dots,K)
\end{align}
である。

また、基底関数が規格直交であるとき$\frac{1}{2}\bm{P}$が
べき等元($\bm{A}^2=\bm{A}$)であることを示せ。

\subsection{解}
問題3.10より
\begin{align}
	\adj{\bm{C}}\bm{S}\bm{C}
&=
	\identity \\
%
%
	\sum_k^K
	\sum_l^K
		\conju{C_{ki}}S_{kl}C_{lj}
&=
	\delta_{ij} \\
%
%
	C_{hi}
	\left(
		\sum_k^K
		\sum_l^K
			\conju{C_{ki}}S_{kl}C_{lj}
	\right)
	\conju{C_{mj}}
&=
	C_{hi}
	\delta_{ij}
	\conju{C_{mj}} \\
%
%
	\sum_a^{N/2}
	\sum_b^{N/2}
		C_{ha}
		\left(
			\sum_k^K
			\sum_l^K
				\conju{C_{ka}}S_{kl}C_{lb}
		\right)
		\conju{C_{mb}}
&=
	\sum_a^{N/2}
	\sum_b^{N/2}
		C_{ha}
		\delta_{ab}
		\conju{C_{mb}} \\
%
%
	\sum_k^K
	\sum_l^K
		\left(
			\sum_a^{N/2}
				C_{ha} \conju{C_{ka}}
		\right)
		S_{kl}
		\left(
			\sum_b^{N/2}
				C_{lb} \conju{C_{mb}}
		\right)
&=
	\sum_a^{N/2}
		C_{ha} \conju{C_{ma}} \\
%
%
	\sum_k^K
	\sum_l^K
		\left(
			\frac{1}{2} P_{hk}
		\right)
		S_{kl}
		\left(
			\frac{1}{2} P_{lm}
		\right)
&=
	\frac{1}{2} P_{hm} \\
%
%
	\sum_k^K
	\sum_l^K
		P_{hk}
		S_{kl}
		P_{lm}
&=
	2 P_{hm} \\
%
%
	\bm{P}\bm{S}\bm{P}
&=
	2\bm{P}
\end{align}
となる。

また、基底関数が規格直交であるとき、$\bm{S}=\identity$であるので、
\begin{align}
	\bm{P}
	\identity
	\bm{P}
&=
	2\bm{P} \\
%
%
	\bm{P}^2
&=
	2\bm{P} \\
%
%
	\left(
		\frac{1}{2} \bm{P}
	\right)^2
&=
	\frac{1}{2} \bm{P}
\end{align}
となるため、$\frac{1}{2}\bm{P}$がべき等元になる。

