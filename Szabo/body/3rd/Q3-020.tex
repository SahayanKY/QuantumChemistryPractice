%ファイルID
%2020/09/03 1647
%->2009031647SahayanKY(ファイル作成者)
\subsection{問}
3つのSTO-$L$G短縮関数について
原点における$\phi(\r)$の値を計算し、
Slater型関数($\zeta=1.0$)に対する値($\pi^{-\frac{1}{2}}$)と比較せよ。


\subsection{解}
p.172より、STO-1G、STO-2G、STO-3Gは次のとおりである。
\begin{align}
	\phi^{\rm CGF}_{\rm 1s}(\zeta=1.0,\text{STO-1G})
&=
	\phi^{\rm GF}_{\rm 1s}(0.270950) \\
%
%
	\phi^{\rm CGF}_{\rm 1s}(\zeta=1.0,\text{STO-2G})
&=
	0.678914
	\phi^{\rm GF}_{\rm 1s}(0.151623)
	+
	0.430129
	\phi^{\rm GF}_{\rm 1s}(0.851819) \\
%
%
	\phi^{\rm CGF}_{\rm 1s}(\zeta=1.0,\text{STO-3G})
&=
	0.444635
	\phi^{\rm GF}_{\rm 1s}(0.109818)
	+
	0.535328
	\phi^{\rm GF}_{\rm 1s}(0.405771)
	+
	0.154329
	\phi^{\rm GF}_{\rm 1s}(2.22766)
\end{align}
ここで$\phi^{\rm GF}_{\rm 1s}$は原始Gauss型関数であり、
\begin{align}
	\phi^{\rm GF}_{\rm 1s}(\alpha)
&=
	\left(
		\frac{2\alpha}{\pi}
	\right)^{\frac{3}{4}}
	\exp(-\alpha|\r|^2)
\end{align}
である。
