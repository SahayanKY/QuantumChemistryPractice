%ファイルID
%2020/09/24 16:24
%->2009241624SahayanKY(ファイル作成者)
\subsection{問}
\ce{Li}原子の2重項基底状態の非制限波動関数は
$\ket{\Psi_0}=\Ket{\aorb{\psi^\alpha_1}\borb{\psi^\beta_1}\aorb{\psi^\alpha_2}}$である。
この状態の全電子エネルギーが
\begin{align}
	E_0
&=
	h^\alpha_{11}
	+
	h^\beta_{11}
	+
	h^\alpha_{22}
	+
	J^{\alpha\alpha}_{12}
	-
	K^{\alpha\alpha}_{12}
	+
	J^{\alpha\beta}_{11}
	+
	J^{\alpha\beta}_{21}
\end{align}
であることを示せ。

\subsection{解}
本書より、一般に全電子エネルギーは
\begin{align}
	E_0
&=
	\sum_a^{N^\alpha}
		h^\alpha_{aa}
	+
	\sum_a^{N^\beta}
		h^\beta_{aa}
	+
	\frac{1}{2}
	\sum_a^{N^\alpha}
	\sum_b^{N^\alpha}\left(
		J^{\alpha\alpha}_{ab}
		-
		K^{\alpha\alpha}_{ab}
	\right)
	+
	\frac{1}{2}
	\sum_a^{N^\beta}
	\sum_b^{N^\beta}\left(
		J^{\beta\beta}_{ab}
		-
		K^{\beta\beta}_{ab}
	\right)
	+
	\sum_a^{N^\alpha}
	\sum_b^{N^\beta}
		J^{\alpha\beta}_{ab}
\end{align}
で与えられる。

従って、\ce{Li}の全電子エネルギーは
\begin{align}
	E_0
&=
	\sum_a^{2}
		h^\alpha_{aa}
	+
	\sum_a^{1}
		h^\beta_{aa}
	+
	\frac{1}{2}
	\sum_a^{2}
	\sum_b^{2}\left(
		J^{\alpha\alpha}_{ab}
		-
		K^{\alpha\alpha}_{ab}
	\right)
	+
	\frac{1}{2}
	\sum_a^{1}
	\sum_b^{1}\left(
		J^{\beta\beta}_{ab}
		-
		K^{\beta\beta}_{ab}
	\right)
	+
	\sum_a^{2}
	\sum_b^{1}
		J^{\alpha\beta}_{ab} \\
%
%
&=
	\left(
		h^\alpha_{11}
		+
		h^\alpha_{22}
	\right)
	+
	\left(
		h^\beta_{11}
	\right)
	+
	\frac{1}{2}
	\left(
		J^{\alpha\alpha}_{11}
		+
		J^{\alpha\alpha}_{12}
		+
		J^{\alpha\alpha}_{21}
		+
		J^{\alpha\alpha}_{22}
		-
		K^{\alpha\alpha}_{11}
		-
		K^{\alpha\alpha}_{12}
		-
		K^{\alpha\alpha}_{21}
		-
		K^{\alpha\alpha}_{22}
	\right) \nonumber \\ &\qquad
	+
	\frac{1}{2}
	\left(
		J^{\beta\beta}_{11}
		-
		K^{\beta\beta}_{11}
	\right)
	+
	\left(
		J^{\alpha\beta}_{11}
		+
		J^{\alpha\beta}_{21}
	\right) \\
%
%
&=
	h^\alpha_{11}
	+
	h^\alpha_{22}
	+
	h^\beta_{11}
	+
	\frac{1}{2}
	\left(
		J^{\alpha\alpha}_{12}
		+
		J^{\alpha\alpha}_{21}
		-
		K^{\alpha\alpha}_{12}
		-
		K^{\alpha\alpha}_{21}
	\right)
	+
	J^{\alpha\beta}_{11}
	+
	J^{\alpha\beta}_{21} \nonumber \\ &\qquad
	%
	(\because
		J^{\alpha\alpha}_{aa}-K^{\alpha\alpha}_{aa}
		=
		J^{\beta\beta}_{aa}-K^{\beta\beta}_{aa}
		=0
	) \\
%
%
&=
	h^\alpha_{11}
	+
	h^\alpha_{22}
	+
	h^\beta_{11}
	+
	\frac{1}{2}
	\left(
		J^{\alpha\alpha}_{12}
		+
		J^{\alpha\alpha}_{12}
		-
		K^{\alpha\alpha}_{12}
		-
		K^{\alpha\alpha}_{12}
	\right)
	+
	J^{\alpha\beta}_{11}
	+
	J^{\alpha\beta}_{21} \\
%
%
&=
	h^\alpha_{11}
	+
	h^\alpha_{22}
	+
	h^\beta_{11}
	+
	J^{\alpha\alpha}_{12}
	-
	K^{\alpha\alpha}_{12}
	+
	J^{\alpha\beta}_{11}
	+
	J^{\alpha\beta}_{21}
\end{align}
である。


