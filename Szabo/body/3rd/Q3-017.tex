%ファイルID
%2020/08/06 1337
%->2008061337SahayanKY(ファイル作成者)
\subsection{問}
Hartree-Fock基底状態におけるエネルギー$E_0$は次のとおりに書き表すことができる。
\begin{align}
	E_0
&=
	\braket{\Psi_0|\mathscr{H}|\Psi_0} \\
&=
	2
	\sum_a^{N/2}
		h_{aa}
	+
	\sum_a^{N/2}
	\sum_b^{N/2}\left(
		2J_{ab}
		-
		K_{ab}
	\right) \\
&=
	\sum_a^{N/2}\left(
		h_{aa}
		+
		\epsilon_a
	\right)
\end{align}
これを変形すると
\begin{align}
	E_0
&=
	\frac{1}{2}
	\sum_\mu
	\sum_\nu
		P_{\nu\mu}
		\left(
			H^{\rm core}_{\mu\nu}
			+
			F_{\mu\nu}
		\right)
\end{align}
となることを示せ。
ここで$H^{\rm core}_{\mu\nu}$と$F_{\mu\nu}$は
核-1電子ハミルトニアン行列とFock行列であり、
\begin{align}
	H^{\rm core}_{\mu\nu}
&=
	\int\d\r_1
		\conju{\phi_\mu}(1)
		h(1)
		\phi_\nu(1) \\
%
%
	F_{\mu\nu}
&=
	\int\d\r_1
		\conju{\phi_\mu}(1)
		f(1)
		\phi_\nu(1)
\end{align}
である。

\subsection{解}
まず、1電子積分$h_{aa}$は
\begin{align}
	h_{aa}
&=
	\spaceone{\psi_a}{h}{\psi_a} \\
%
%
&=
	\int\d\r_1
		\conju{\psi_a}(1)
		h(1)
		\psi_a(1) \\
%
%
&=
	\int\d\r_1
		\conju[1]{
			\sum_\mu C_{\mu a}\phi_\mu(1)
		}
		h(1)
		\left(
			\sum_\nu C_{\nu a}\phi_\nu(1)
		\right) \\
%
%
&=
	\sum_\mu
	\sum_\nu
		\conju{C_{\mu a}}
		C_{\nu a}
		\int\d\r_1
			\conju{\phi_\mu}(1)
			h(1)
			\phi_\nu(1) \\
%
%
&=
	\sum_\mu
	\sum_\nu
		\conju{C_{\mu a}}
		C_{\nu a}
		H^{\rm core}_{\mu\nu}
\end{align}
と書き換えることができる。
次に、軌道エネルギー$\epsilon_a$は
閉殻系の制限付きHartree-Fock方程式より
\begin{align}
	f(1)
	\psi_a(1)
&=
	\epsilon_a
	\psi_a(1) \\
%
%
	\int\d\r_1
		\conju{\psi_a}(1)
		f(1)
		\psi_a(1)
&=
	\epsilon_a
	\int\d\r_1
		\conju{\psi_a}(1)
		\psi_a(1) \\
%
%
	\int\d\r_1
		\conju{\psi_a}(1)
		f(1)
		\psi_a(1)
&=
	\epsilon_a
\end{align}
\begin{align}
	\epsilon_a
&=
	\int\d\r_1
		\conju[1]{
			\sum_\mu C_{\mu a} \phi_\mu(1)
		}
		f(1)
		\left(
			\sum_\nu C_{\nu a} \phi_\nu(1)
		\right) \\
%
%
&=
	\sum_\mu
	\sum_\nu
		\conju{C_{\mu a}}
		C_{\nu a}
		\int\d\r_1
			\conju{\phi_\mu}(1)
			f(1)
			\phi_\nu(1) \\
%
%
&=
	\sum_\mu
	\sum_\nu
		\conju{C_{\mu a}}
		C_{\nu a}
		F_{\mu\nu}
\end{align}
と書き換えることができる。

従って、
\begin{align}
	E_0
&=
	\sum_a^{N/2}\left(
		h_{aa}
		+
		\epsilon_a
	\right) \\
%
%
&=
	\sum_a^{N/2}\left(
		\left(
			\sum_\mu
			\sum_\nu
				\conju{C_{\mu a}}
				C_{\nu a}
				H^{\rm core}_{\mu\nu}
		\right)
		+
		\left(
			\sum_\mu
			\sum_\nu
				\conju{C_{\mu a}}
				C_{\nu a}
				F_{\mu\nu}
		\right)
	\right) \\
%
%
&=
	\sum_a^{N/2}
	\sum_\mu
	\sum_\nu
		\conju{C_{\mu a}}
		C_{\nu a}
		\left(
			H^{\rm core}_{\mu\nu}
			+
			F_{\mu\nu}
		\right) \\
%
%
&=
	\frac{1}{2}
	\sum_\mu
	\sum_\nu
		\left(
			2
			\sum_a^{N/2}
				C_{\nu a}
				\conju{C_{\mu a}}
		\right)
		\left(
			H^{\rm core}_{\mu\nu}
			+
			F_{\mu\nu}
		\right) \\
%
%
&=
	\frac{1}{2}
	\sum_\mu
	\sum_\nu
		P_{\nu\mu}
		\left(
			H^{\rm core}_{\mu\nu}
			+
			F_{\mu\nu}
		\right)
\end{align}
となる。



