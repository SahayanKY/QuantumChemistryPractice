%ファイルID
%2020/07/30 1607
%->2007301607SahayanKY(ファイル作成者)
\subsection{問}
制限付きの閉殻Fock演算子は次の通りである。
\begin{align}
	f(\r_1)
&=
	h(\r_1)
	+
	\sum_{a}^{N/2}
		\int\d\r_2
			\conju{\psi_a}(\r_2)
			(2-\mathscr{P}_{12})
			r_{12}^{-1}
			\psi_a(\r_2)
\end{align}

基底関数$\phi_\mu$により空間軌道$\psi_a$を展開すると、
次の通りに書き表せることを示せ。
\begin{align}
	f(\r_1)
&=
	h(\r_1)
	+
	\frac{1}{2}
	\sum_{\lambda,\sigma}
		P_{\lambda\sigma}
		\left[
			\int\d\r_2
				\conju{\phi_\sigma}(\r_2)
				(2-\mathscr{P}_{12})
				r_{12}^{-1}
				\phi_{\lambda}(\r_2)
		\right]
\end{align}


\subsection{解}
基底関数により展開すると
\begin{align}
	\psi_a
&=
	\sum_{\mu}
		C_{\mu i} \phi_{\mu}
\end{align}
となるので、
\begin{align}
	f(\r_1)
&=
	h(\r_1)
	+
	\sum_a^{N/2}
		\int\d\r_2
			\conju[1]{
				\sum_\sigma C_{\sigma a} \phi_{\sigma}(\r_2)
			}
			(2-\mathscr{P}_{12})
			r_{12}^{-1}
			\left(
				\sum_\lambda C_{\lambda a} \phi_{\lambda}(\r_2)
			\right) \\
%
%
&=
	h(\r_1)
	+
	\sum_a^{N/2}
	\sum_\sigma
	\sum_\lambda
		\int\d\r_2
			\conju{C_{\sigma a}}
			\conju{\phi_{\sigma}}(\r_2)
			(2-\mathscr{P}_{12})
			r_{12}^{-1}
			C_{\lambda a}
			\phi_{\lambda}(\r_2) \\
%
%
&=
	h(\r_1)
	+
	\sum_\sigma
	\sum_\lambda
		\sum_a^{N/2}
			\conju{C_{\sigma a}}
			C_{\lambda a}
		\int\d\r_2
			\conju{\phi_\sigma}(\r_2)
			(2-\mathscr{P}_{12})
			r_{12}^{-1}
			\phi_\lambda(\r_2) \\
%
%
&=
	h(\r_1)
	+
	\sum_\sigma
	\sum_\lambda
		\left(
			\frac{1}{2} P_{\lambda \sigma}
		\right)
		\int\d\r_2
			\conju{\phi_\sigma}(\r_2)
			(2-\mathscr{P}_{12})
			r_{12}^{-1}
			\phi_\lambda(\r_2) \\
%
%
&=
	h(\r_1)
	+
	\frac{1}{2}
	\sum_{\lambda,\sigma}
		P_{\lambda\sigma}
		\int\d\r_2
			\conju{\phi_\sigma}(\r_2)
			(2-\mathscr{P}_{12})
			r_{12}^{-1}
			\phi_\lambda(\r_2)
\end{align}
である。
