%ファイルID
%2020/09/21 13:50
%->2009211350SahayanKY(ファイル作成者)
\subsection{問}
4個の短縮していない1s Gauss型関数を用いた\ce{He}原子のSCF計算において、
最適化された1s軌道が
\begin{align}
	\psi_{\rm 1s}
&=
	0.51380 g_{\rm 1s}(0.298073)
	+
	0.46954 g_{\rm 1s}(1.242567)
	+
	0.15457 g_{\rm 1s}(5.782948)
	+
	0.02373 g_{\rm 1s}(38.47497)
	%
	\label{eq2009211350SahayanKY_optimized1sOrbitalBy4PrimitiveGaussFunctions}
\end{align}
であることが知られている。これを用いて\ce{He}の4-31G基底関数に類するものを求めよ。

\subsection{解}
\ce{He}の4-31G基底関数は次の通りである。
\begin{align}
	\phi'_{\rm 1s}(\r)
&=
	\sum_{i=1}^{3}
		d'_{i,\rm 1s}
		g_{\rm 1s}(\alpha'_{i,\rm 1s},\r) \\
%
%
	\phi''_{\rm 1s}(\r)
&=
	g_{\rm 1s}(\alpha''_{\rm 1s},\r)
\end{align}
このうち2つ目の$\phi''_{\rm 1s}$は外側の原子価殻関数と呼ばれるものであり、
より広がった基底関数である。
従って、これの軌道指数として
式\ref{eq2009211350SahayanKY_optimized1sOrbitalBy4PrimitiveGaussFunctions}中で
最も小さい0.298073を採用することが自然である。
よって、求める基底関数は
\begin{align}
	\phi'_{\rm 1s}(\r)
&=
	N
	\left(
		0.46954 g_{\rm 1s}(1.242567)
		+
		0.15457 g_{\rm 1s}(5.782948)
		+
		0.02373 g_{\rm 1s}(38.47497)
	\right) \\
%
%
	\phi''_{\rm 1s}(\r)
&=
	g_{\rm 1s}(0.298073)
\end{align}
となる。ここで$N$は規格化定数であり、以降の議論で求める。

本書の付録中にある、規格化されていない1s原始Gauss関数の積の積分公式を用いると
\begin{align}
	\int\d\r
		g_{\rm 1s}(\alpha,\r)
		g_{\rm 1s}(\beta,\r)
&=
	\int\d\r
		\left(
			\frac{2\alpha}{\pi}
		\right)^{\frac{3}{4}}
		\tilde{g}_{\rm 1s}(\alpha,\r)
		\left(
			\frac{2\beta}{\pi}
		\right)^{\frac{3}{4}}
		\tilde{g}_{\rm 1s}(\beta,\r) \\
%
%
&=
	\left(
		\frac{4\alpha\beta}{\pi^2}
	\right)^{\frac{3}{4}}
	\int\d\r
		\tilde{g}_{\rm 1s}(\alpha,\r)
		\tilde{g}_{\rm 1s}(\beta,\r) \\
%
%
&=
	\left(
		\frac{4\alpha\beta}{\pi^2}
	\right)^{\frac{3}{4}} \cdot
	\left(
		\frac{\pi}{\alpha+\beta}
	\right)^{\frac{3}{2}}
	\exp\left(
		-\frac{\alpha\beta}{\alpha+\beta}|\bm{0}|^2
	\right) \\
%
%
&=
	2^{\frac{3}{2}}
	\alpha^{\frac{3}{4}}
	\beta^{\frac{3}{4}}
	(\alpha+\beta)^{-\frac{3}{2}} \\
%
%
&=
	2^{\frac{3}{2}}
	\left(
		\frac{
			\alpha\beta
		}{
			(\alpha+\beta)^2
		}
	\right)^{\frac{3}{4}}
\end{align}
となる。従って、
\begin{align}
	\braket{\phi'_{\rm 1s}|\phi'_{\rm 1s}}
&=
	1 \\
%
%
	N^2
	2^{\frac{3}{2}}
	\left(
	\begin{array}{c}
		0.46954^2
		\left(
			\frac{
				1.242567^2
			}{
				(
					1.242567
					+
					1.242567
				)^2
			}
		\right)^{\frac{3}{4}}
		+
		2 \cdot 0.46954 \cdot 0.15457
		\left(
			\frac{
				1.242567 \cdot 5.782948
			}{
				(
					1.242567
					+
					5.782948
				)^2
			}
		\right)^{\frac{3}{4}} \\
		+
		0.15457^2
		\left(
			\frac{
				5.782948^2
			}{
				(
					5.782948
					+
					5.782948
				)^2
			}
		\right)^{\frac{3}{4}}
		+
		2 \cdot 0.15457 \cdot 0.02373
		\left(
			\frac{
				5.782948 \cdot 38.47497
			}{
				(
					5.782948
					+
					38.47497
				)^2
			}
		\right)^{\frac{3}{4}} \\
		+
		0.02373^2
		\left(
			\frac{
				38.47497^2
			}{
				(
					38.47497
					+
					38.47497
				)^2
			}
		\right)^{\frac{3}{4}}
		+
		2 \cdot 0.02373 \cdot 0.46954
		\left(
			\frac{
				38.47497 \cdot 1.242567
			}{
				(
					38.47497
					+
					1.242567
				)^2
			}
		\right)^{\frac{3}{4}}
	\end{array}
	\right)
&=
	1 \\
%
%
	N^2 \cdot
	2^{\frac{3}{2}} \cdot
	0.12386
&=
	1 \\
%
%
	N
&=
	1.68953
\end{align}
となるので、求める基底関数は
\begin{align}
	\phi'_{\rm 1s}(\r)
&=
	0.79330 g_{\rm 1s}(1.242567)
	+
	0.26115 g_{\rm 1s}(5.782948)
	+
	0.04009 g_{\rm 1s}(38.47497) \\
%
%
	\phi''_{\rm 1s}(\r)
&=
	g_{\rm 1s}(0.298073)
\end{align}
である。




