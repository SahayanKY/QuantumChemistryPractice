%ファイルID
%2020/09/18 2045
%->2009182045SahayanKY(ファイル作成者)
\subsection{問}
STO-3Gにおける\ce{HeH+}の全電子エネルギー$E_0$は
結合距離$R\rightarrow\infty$のとき
\begin{align}
	E_0(R\rightarrow\infty)
&=
	2T_{11}
	+
	2V^1_{11}
	+
	\spacetwo{\phi_1\phi_1}{\phi_2\phi_2}
\end{align}
となることを示せ。
ここで\ce{He}を原子1とし、
\ce{H}を原子2としている。


\subsection{解}
極限における密度行列は
\begin{align}
	\bm{P}_{R\rightarrow\infty}
&=
	\left[
	\begin{array}{cc}
		2 & 0 \\
		0 & 0
	\end{array}
	\right]
\end{align}
である。従って、全電子エネルギー$E_0$は
\begin{align}
	E_0(R\rightarrow\infty)
&=
	\frac{1}{2}
	\sum_\mu
	\sum_\nu
		P_{\nu\mu}
		\left(
			H^{\rm core}_{\mu\nu}
			+
			F_{\mu\nu}
		\right) \\
%
%
&=
	\frac{1}{2}
	P_{11}
	\left(
		H^{\rm core}_{11}
		+
		F_{11}
	\right) \\
%
%
&=
	H^{\rm core}_{11}
	+
	F_{11} \\
%
%
&=
	2H^{\rm core}_{11}
	+
	\sum_{\lambda\sigma}
		P_{\lambda\sigma}
		\left[
			\spacetwo{11}{\sigma\lambda}
			-
			\frac{1}{2}
			\spacetwo{1\lambda}{\sigma 1}
		\right] \\
%
%
&=
	2H^{\rm core}_{11}
	+
	2
	\left[
		\spacetwo{11}{11}
		-
		\frac{1}{2}
		\spacetwo{11}{11}
	\right] \\
%
%
&=
	2T_{11}
	+
	2V^{\rm nucl}_{11}
	+
	\spacetwo{11}{11}
\end{align}
となる。$V^{\rm nucl}$は全ての原子核と1つの電子の間の引力のエネルギーを表しており、
\begin{align}
	V^{\rm nucl}_{11}
&=
	\int\d\r_1
		\conju{\phi_1}(1)
		\left[
			-
			\sum_A \frac{Z_A}{|\r_1-\bm{R}_A|}
		\right]
		\phi_1(1) \\
%
%
&=
	\sum_A
		\int\d\r_1
			\conju{\phi_1}(1)
			\left[
				- \frac{Z_A}{|\r_1-\bm{R}_A|}
			\right]
			\phi_1(1) \\
%
%
&=
	\sum_A V^{A}_{11} \\
%
%
&=
	V^1_{11}
	+
	V^2_{11}
\end{align}
となっている。このうち$V^2_{11}$については
原子2(水素)の原子核と電子の間のクーロン引力を表す。
結合距離を極限まで大きくとると電子は原子1(ヘリウム)の周りに局在化するため、
電子と水素原子核の間の相互作用はゼロに漸近する。
従って、
\begin{align}
	V^{2}_{11}(R\rightarrow\infty)
&=
	0
\end{align}
となる。

よって、全電子エネルギー$E_0$は
\begin{align}
	E_0(R\rightarrow\infty)
&=
	2T_{11}
	+
	2V^{1}_{11}
	+
	\spacetwo{11}{11}
\end{align}
になる。










