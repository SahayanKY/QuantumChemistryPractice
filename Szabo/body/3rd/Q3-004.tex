%ファイルID
%2020/07/18 21:41
%->2007182141SahayanKY(ファイル作成者)
\subsection{問}
問題3.1の結果を使い、
$f_{ij}=\braket{\chi_i|f|\chi_j}$がエルミート行列の要素となることを示すことで、
Fock演算子$f$がエルミート演算子であることを示せ。

\subsection{解}
問題3.1より
\begin{align}
	f_{ij}
=
	\braket{\chi_i|f|\chi_j}
=
	\braket{i|h|j}
	+
	\sum_{b=1}^{N}\biggl(
		\chemtwo{ij}{bb}
		-
		\chemtwo{ib}{bj}
	\biggr)
\end{align}
である。従って、$\conju{f_{ji}}$の値は
\begin{align}
	\conju{f_{ji}}
&=
	\conju[1]{
		\braket{j|h|i}
		+
		\sum_{b=1}^{N}\biggl(
			\chemtwo{ji}{bb}
			-
			\chemtwo{jb}{bi}
		\biggr)
	} \\
%
%
&=
	\conju{\braket{j|h|i}}
	+
	\sum_{b=1}^{N}\biggl(
		\conju{\chemtwo{ji}{bb}}
		-
		\conju{\chemtwo{jb}{bi}}
	\biggr) \\
%
%
&=
	\braket{i|\adj{h}|j}
	+
	\sum_{b=1}^{N}\biggl(
		\chemtwo{ij}{bb}
		-
		\chemtwo{bj}{ib}
	\biggr) \\
%
%
&=
	\braket{i|h|j}
	+
	\sum_{b=1}^{N}\biggl(
		\chemtwo{ij}{bb}
		-
		\chemtwo{ib}{bj}
	\biggr)
	%
	\qquad
	(
		\because
		\adj{h}=h,
		\chemtwo{ij}{kl}=\chemtwo{kl}{ij}
	) \\
%
%
&=
	f_{ij}
\end{align}
である。よって、$f_{ij}$を要素としてもつ行列はエルミート行列であり、
また、このことから、
\begin{align}
	\conju{f_{ji}}
&=
	f_{ij} \\
%
%
	\conju{\braket{\chi_j|f|\chi_i}}
&=
	\braket{\chi_i|f|\chi_j} \\
%
%
	\braket{\chi_i|\adj{f}|\chi_j}
&=
	\braket{\chi_i|f|\chi_j}
\end{align}
となるので、$\adj{f}=f$となる。(エルミート演算子である。)



