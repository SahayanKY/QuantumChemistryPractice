%ファイルID
%2020/07/31 2053
%->2007312053SahayanKY(ファイル作成者)
\subsection{問}
基底関数が実関数であるとして
2電子積分の対称性を考慮すると、
基底関数の大きさが$K=100$のときに
$\num{12753775}=O(K^4/8)$個の相異なる2電子積分が存在することを示せ。


\subsection{解}
2電子積分は次の通りである。
\begin{align}
	\spacetwo{\mu\nu}{\lambda\sigma}
&=
	\int\d\r_1\d\r_2
		\conju{\phi_\mu}(1)
		\phi_\nu(1)
		r_{12}^{-1}
		\conju{\phi_\lambda}(2)
		\phi_\sigma(2)
\end{align}

基底関数$\phi$が実関数であるとき、
\begin{align}
	\spacetwo{\mu\nu}{\lambda\sigma}
=
	\spacetwo{\nu\mu}{\lambda\sigma}
=
	\spacetwo{\mu\nu}{\sigma\lambda}
=
	\spacetwo{\nu\mu}{\sigma\lambda}
\end{align}
が成立する。また、積分変数の書き換えにより
\begin{align}
	\spacetwo{\mu\nu}{\lambda\sigma}
=
	\spacetwo{\lambda\sigma}{\mu\nu}
\end{align}
が成立する。
従って、
\begin{align}
	\spacetwo{\mu\nu}{\lambda\sigma}
&=
	\spacetwo{\nu\mu}{\lambda\sigma}
=
	\spacetwo{\mu\nu}{\sigma\lambda}
=
	\spacetwo{\nu\mu}{\sigma\lambda} \\
&=
	\spacetwo{\lambda\sigma}{\mu\nu}
=
	\spacetwo{\lambda\sigma}{\nu\mu}
=
	\spacetwo{\sigma\lambda}{\mu\nu}
=
	\spacetwo{\sigma\lambda}{\nu\mu}
\end{align}
である。

このあとの方針としては、
4つの数の組み合わせ方を考えた後
それぞれの組での順列を考え、
二電子積分が相異なる順列を数え上げていく。
まず、4つの数の組み合わせ方は次の5つのパターンがある。
ここで$i,j,k,l$は互いに異なる数である。
\begin{align}
	(i,i,i,i),\
	(i,i,i,j),\
	(i,i,j,j),\
	(i,i,j,k),\
	(i,j,k,l)
\end{align}
1つ目のパターンに一致する数の組み合わせ方は、
$K$個ある数のうちから1つを選ぶ組合せの数に等しいので
$\bicoeff{K}{1}=K$通りある。
2つ目のパターンに一致する数の組み合わせ方は、
まず$K$個ある数のうちから2つを選び、
そのうち一方を$i$とする組合せの数に等しいので
$\bicoeff{K}{2}\cdot\bicoeff{2}{1}=K(K-1)$通りある。
3つ目のパターンに一致する数の組み合わせ方は、
$K$個ある数のうちから2つを選ぶ組合せの数に等しいので
$\bicoeff{K}{2}=\frac{1}{2}K(K-1)$通りある。
4つ目のパターンに一致する数の組み合わせ方は、
$K$個ある数のうちから3つの数を選び、
更にそこから1つの数を選ぶ組合せの数に等しいので
$\bicoeff{K}{3}\cdot\bicoeff{3}{1}=\frac{1}{2}K(K-1)(K-2)$通りある。
最後に5つ目のパターンに一致する数の組み合わせ方は
$K$個ある数のうちから4つの数を選ぶ組合せの数に等しいので
$\bicoeff{K}{4}=\frac{1}{24}K(K-1)(K-2)(K-3)$通りある。

ここから、それぞれのパターンにおいて順列を考え、
前述した二電子積分が等しくなる順列ごとにグループ分けをする。
最終的に求める数は
\begin{align}
	\sum_i
		(\text{パターン$i$の組合せの数})\times
		(\text{パターン$i$の二電子積分が等しいグループの数})
\end{align}
によって求まるからである。








