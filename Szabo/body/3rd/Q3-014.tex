%ファイルID
%2020/07/31 2053
%->2007312053SahayanKY(ファイル作成者)
\subsection{問}
基底関数が実関数であるとして
2電子積分の対称性を考慮すると、
基底関数の大きさが$K=100$のときに
$\num{12753775}=O(K^4/8)$個の相異なる2電子積分が存在することを示せ。


\subsection{解}
2電子積分は次の通りである。
\begin{align}
	\spacetwo{\mu\nu}{\lambda\sigma}
&=
	\int\d\r_1\d\r_2
		\conju{\phi_\mu}(1)
		\phi_\nu(1)
		r_{12}^{-1}
		\conju{\phi_\lambda}(2)
		\phi_\sigma(2)
\end{align}

基底関数$\phi$が実関数であるとき、
\begin{align}
	\spacetwo{\mu\nu}{\lambda\sigma}
=
	\spacetwo{\nu\mu}{\lambda\sigma}
=
	\spacetwo{\mu\nu}{\sigma\lambda}
=
	\spacetwo{\nu\mu}{\sigma\lambda}
\end{align}
が成立する。また、積分変数の書き換えにより
\begin{align}
	\spacetwo{\mu\nu}{\lambda\sigma}
=
	\spacetwo{\lambda\sigma}{\mu\nu}
\end{align}
が成立する。
従って、
\begin{align}
	\spacetwo{\mu\nu}{\lambda\sigma}
&=
	\spacetwo{\nu\mu}{\lambda\sigma}
=
	\spacetwo{\mu\nu}{\sigma\lambda}
=
	\spacetwo{\nu\mu}{\sigma\lambda} \\
&=
	\spacetwo{\lambda\sigma}{\mu\nu}
=
	\spacetwo{\lambda\sigma}{\nu\mu}
=
	\spacetwo{\sigma\lambda}{\mu\nu}
=
	\spacetwo{\sigma\lambda}{\nu\mu}
\end{align}


