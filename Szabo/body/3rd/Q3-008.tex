%ファイルID
%2020/07/28 18:00
%->2007281800SahayanKY(ファイル作成者)
\subsection{問}
Hartree-Fockハミルトニアン$\H_0$の摂動$\mathscr{V}$は
\begin{align}
	\mathscr{V}
&=
	\H
	-
	\H_0 \\
%
%
&=
	\left(
		\sum_{i=1}^{N}
			h(i)
		+
		\sum_{i=1}^{N}
		\sum_{j>i}^{N}
			r_{ij}^{-1}
	\right)
	-
	\sum_{i=1}^{N}
		f(i) \\
%
%
&=
	\left(
		\sum_{i=1}^{N}
			h(i)
		+
		\sum_{i=1}^{N}
		\sum_{j>i}^{N}
			r_{ij}^{-1}
	\right)
	-
	\left(
		\sum_{i=1}^{N}
			h(i)
		+
		\sum_{i=1}^{N}
			v^{\rm HF}(i)
	\right) \\
%
%
&=
	\sum_{i=1}^{N}
	\sum_{j>i}^{N}
		r_{ij}^{-1}
	-
	\sum_{i=1}^{N}
		v^{\rm HF}(i)
\end{align}
である。ここで$v^{\rm HF}$はHartree-Fockポテンシャルであり、
\begin{align}
	v^{\rm HF}(i)
&=
	\sum_b\left(
		\mathscr{J}_b(i)
		-
		\mathscr{K}_b(i)
	\right) \\
%
%
&=
	\sum_b
		\int\d\x_j
			\conju{\chi_b}(j)
			r_{ij}^{-1}
			(1-\mathscr{P}_{ij})
			\chi_b(j)
	%
	\qquad
	(j\neq i)
\end{align}
である。

次式
\begin{align}
	\braket{\Psi_0|\mathscr{V}|\Psi_0}
&=
	-
	\frac{1}{2}
	\sum_{a}
	\sum_{b}
		\antitwo{ab}{ab}
\end{align}
を左辺から直接示せ。

\subsection{解}
摂動の第1項をまず考える。
\begin{align}
	\Braket{
		\Psi_0
		|
		\sum_{i=1}^{N}
		\sum_{j>i}^{N}
			r_{ij}^{-1}
		|
		\Psi_0
	}
&=
	\braket{\Psi_0|\Otwo|\Psi_0} \\
%
%
&=
	\frac{1}{2}
	\sum_{a}^{N}
	\sum_{b}^{N}
		\antitwo{ab}{ab}
\end{align}
である。ここで表2.4(または表2.6)に示されている規則を利用した。

摂動の第2項について考える。
まず、$\ket{\Psi_0}$を生成演算子$\adj{a_i}$と消滅演算子$a_i$で表現することを考える。
\begin{align}
	\ket{\Psi_0}
&=
	\ket{\chi_1\chi_2\dots\chi_i\dots\chi_N} \\
%
%
&=
	\left\{
	\begin{array}{ll}
		-
		\ket{\chi_1\chi_2\dots\chi_N\dots\chi_i}
		&
		(i\neq N) \\
	%
		\ket{\chi_1\chi_2\dots\chi_N}
		&
		(i=N)
	\end{array}
	\right. \\
%
%
&=
	\left\{
	\begin{array}{ll}
		-
		\adj{a_1}\adj{a_2}\dots\adj{a_N}\dots\ket{\chi_i}
		&
		(i\neq N) \\
	%
		\adj{a_1}\adj{a_2}\dots\ket{\chi_N}
		&
		(i=N)
	\end{array}
	\right.
\end{align}
従って、
\begin{align}
	\Braket{
		\Psi_0
		|
		\sum_{i=1}^{N}
			v^{\rm HF}(i)
		|
		\Psi_0
	}
&=
	\sum_{i=1}^{N}
		\braket{\Psi_0|v^{\rm HF}(i)|\Psi_0} \\
%
%
&=
	\sum_{i=1}^{N-1}
		\bra{\chi_i}\dots a_N \dots a_2 a_1
		v^{\rm HF}(i)
		\adj{a_1}\adj{a_2}\dots\adj{a_N}\dots\ket{\chi_i} \nonumber \\&\qquad
	+
	\bra{\chi_N}\dots a_2 a_1
	v^{\rm HF}(N)
	\adj{a_1}\adj{a_2}\dots\ket{\chi_N} \nonumber \\&\qquad
	%
	(\because a_i\adj{a_j}=-\adj{a_j}a_i+\delta_{ij}) \\
%
%
&=
	\sum_{i=1}^{N-1}
		\braket{\chi_i|v^{\rm HF}(i)|\chi_i}
	+
	\braket{\chi_N|v^{\rm HF}(N)|\chi_N} \\
%
%
&=
	\sum_{i=1}^{N}
		\braket{\chi_i|v^{\rm HF}(i)|\chi_i} \\
%
%
&=
	\sum_{i=1}^{N}
	\sum_b^{N}
		\braket{\chi_i|(\mathscr{J}_b-\mathscr{K}_b)|\chi_i} \\
%
%
&=
	\sum_{i=1}^{N}
	\sum_b^{N}
		\antitwo{ib}{ib} \\
%
%
&=
	\sum_a^{N}
	\sum_b^{N}
		\antitwo{ab}{ab}
\end{align}
である。\footnote{本当に$v^{\rm HF}(i)\adj{a_j}(i\neq j)$は可換なのか?}
従って、
\begin{align}
	\braket{\Psi_0|\mathscr{V}|\Psi_0}
&=
	\frac{1}{2}
	\sum_a^{N}
	\sum_b^{N}
		\antitwo{ab}{ab}
	-
	\sum_a^{N}
	\sum_b^{N}
		\antitwo{ab}{ab} \\
%
%
&=
	-
	\frac{1}{2}
	\sum_a^{N}
	\sum_b^{N}
		\antitwo{ab}{ab}
\end{align}
となる。この値の負符号を撮った量は
電子間相互作用の量に等しい。
従って、この分を$E_0^{(0)}=\sum_a \epsilon_a$から引くことで、
二倍に見積もっていた電子間相互作用を打ち消していることが分かる。




