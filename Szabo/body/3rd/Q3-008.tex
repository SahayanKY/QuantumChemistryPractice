%ファイルID
%2020/07/28 18:00
%->2007281800SahayanKY(ファイル作成者)
\subsection{問}
Hartree-Fockハミルトニアン$\H_0$の摂動$\mathscr{V}$は
\begin{align}
	\mathscr{V}
&=
	\H
	-
	\H_0 \\
%
%
&=
	\left(
		\sum_{i=1}^{N}
			h(i)
		+
		\sum_{i=1}^{N}
		\sum_{j>i}^{N}
			r_{ij}^{-1}
	\right)
	-
	\sum_{i=1}^{N}
		f(i) \\
%
%
&=
	\left(
		\sum_{i=1}^{N}
			h(i)
		+
		\sum_{i=1}^{N}
		\sum_{j>i}^{N}
			r_{ij}^{-1}
	\right)
	-
	\left(
		\sum_{i=1}^{N}
			h(i)
		+
		\sum_{i=1}^{N}
			v^{\rm HF}(i)
	\right) \\
%
%
&=
	\sum_{i=1}^{N}
	\sum_{j>i}^{N}
		r_{ij}^{-1}
	-
	\sum_{i=1}^{N}
		v^{\rm HF}(i)
\end{align}
である。ここで$v^{\rm HF}$はHartree-Fockポテンシャルであり、
\begin{align}
	v^{\rm HF}(i)
&=
	\sum_b\left(
		\mathscr{J}_b(i)
		-
		\mathscr{K}_b(i)
	\right) \\
%
%
&=
	\sum_b
		\int\d\x_j
			\conju{\chi_b}(j)
			r_{ij}^{-1}
			(1-\mathscr{P}_{ij})
			\chi_b(j)
	%
	\qquad
	(j\neq i)
\end{align}
である。

次式
\begin{align}
	\braket{\Psi_0|\mathscr{V}|\Psi_0}
&=
	-
	\frac{1}{2}
	\sum_{a}
	\sum_{b}
		\antitwo{ab}{ab}
\end{align}
を示せ。

\subsection{解}



