%ファイルID
%2020/09/24 14:53
%->2009241453SahayanKY(ファイル作成者)
\subsection{問}
非制限スピン軌道のもとでの
空間軌道$\psi^\alpha_j$に対するFock演算子は
\begin{align}
	f^\alpha(\r_1)
&=
	\int\d\omega_1
		\conju{\alpha}(\omega_1)
		f(\x_1)
		\alpha(\omega_1)
\end{align}
である。これを式変形して
\begin{align}
	f^\alpha(1)
&=
	h(1)
	+
	\sum_a^{N^\alpha}\left(
		J^\alpha_a(1)
		-
		K^\alpha_a(1)
	\right)
	+
	\sum_a^{N^\beta}
		J^\beta_a(1)
\end{align}
となることを示せ。
ここで
\begin{align}
	J^\alpha_a(1)
&=
	\int\d\r_2
		\conju{\psi^\alpha_a}(2)
		r_{12}^{-1}
		\psi^\alpha_a(2) \\
%
%
	J^\beta_a(1)
&=
	\int\d\r_2
		\conju{\psi^\beta_a}(2)
		r_{12}^{-1}
		\psi^\beta_a(2) \\
%
%
	K^\alpha_a(1)
&=
	\int\d\r_2
		\conju{\psi^\alpha_a}(2)
		r_{12}^{-1}
		\mathscr{P}_{12}
		\psi^\alpha_a(2)
\end{align}
である。


\subsection{解}
スピン軌道に対するFock演算子は
\begin{align}
	f(\x_1)
&=
	h(\r_1)
	+
	\sum_a^N
		\int\d\x_2
			\conju{\chi_a}(\x_2)
			r_{12}^{-1}
			(1-\mathscr{P}_{12})
			\chi_a(\x_2)
\end{align}
である。
第2項の全占有スピン軌道についての和を、
全占有$\alpha$スピン軌道についての和と
全占有$\beta$スピン軌道についての和に分解すると、
\begin{align}
	f(\x_1)
&=
	h(\r_1)
	+
	\sum_a^{N^\alpha}
		\int\d\x_2
			\conju{\psi^\alpha_a}(\r_2)
			\conju{\alpha}(\omega_2)
			r_{12}^{-1}
			(1-\mathscr{P}_{12})
			\psi^\alpha_a(\r_2)
			\alpha(\omega_2) \nonumber \\ &\qquad
	+
	\sum_a^{N^\beta}
		\int\d\x_2
			\conju{\psi^\beta_a}(\r_2)
			\conju{\beta}(\omega_2)
			r_{12}^{-1}
			(1-\mathscr{P}_{12})
			\psi^\beta_a(\r_2)
			\beta(\omega_2) \\
%
%
&=
	h(\r_1)
	+
	\sum_a^{N^\alpha}
		\int\d\r_2
			\conju{\psi^\alpha_a}(\r_2)
			r_{12}^{-1}
			\psi^\alpha_a(\r_2)
	-
	\sum_a^{N^\alpha}
		\int\d\x_2
			\conju{\psi^\alpha_a}(\r_2)
			\conju{\alpha}(\omega_2)
			r_{12}^{-1}
			\psi^\alpha_a(\r_1)
			\alpha(\omega_1)
			\mathscr{P}_{1\rightarrow 2} \nonumber \\ &\qquad
	+
	\sum_a^{N^\beta}
		\int\d\r_2
			\conju{\psi^\beta_a}(\r_2)
			r_{12}^{-1}
			\psi^\beta_a(\r_2)
	-
	\sum_a^{N^\beta}
		\int\d\x_2
			\conju{\psi^\beta_a}(\r_2)
			\conju{\beta}(\omega_2)
			r_{12}^{-1}
			\psi^\beta_a(\r_1)
			\beta(\omega_1)
			\mathscr{P}_{1\rightarrow 2} \\
%
%
&=
	h(\r_1)
	+
	\sum_a^{N^\alpha}
		J^\alpha_a(\r_1)
	-
	\sum_a^{N^\alpha}
		\int\d\x_2
			\conju{\psi^\alpha_a}(\r_2)
			\conju{\alpha}(\omega_2)
			r_{12}^{-1}
			\psi^\alpha_a(\r_1)
			\alpha(\omega_1)
			\mathscr{P}_{1\rightarrow 2} \nonumber \\ &\qquad
	+
	\sum_a^{N^\beta}
		J^\beta_a(\r_1)
	-
	\sum_a^{N^\beta}
		\int\d\x_2
			\conju{\psi^\beta_a}(\r_2)
			\conju{\beta}(\omega_2)
			r_{12}^{-1}
			\psi^\beta_a(\r_1)
			\beta(\omega_1)
			\mathscr{P}_{1\rightarrow 2}
\end{align}
となる。従って、$f^\alpha(1)$は
\begin{align}
	f^\alpha(\r_1)
&=
	\int\d\omega_1
		\conju{\alpha}(\omega_1)
		f(\x_1)
		\alpha(\omega_1) \\
%
%
&=
	\left(
		h(\r_1)
		+
		\sum_a^{N^\alpha}
			J^\alpha_a(\r_1)
		+
		\sum_a^{N^\beta}
			J^\beta_a(\r_1)
	\right)
	\int\d\omega_1
		\conju{\alpha}(\omega_1)
		\alpha(\omega_1) \nonumber \\ &\qquad
	-
	\int\d\omega_1
		\conju{\alpha}(\omega_1)
		\sum_a^{N^\alpha}
			\int\d\x_2
				\conju{\psi^\alpha_a}(\r_2)
				\conju{\alpha}(\omega_2)
				r_{12}^{-1}
				\psi^\alpha_a(\r_1)
				\alpha(\omega_1)
				\mathscr{P}_{1\rightarrow 2}
				\alpha(\omega_1) \nonumber \\ &\qquad
	-
	\int\d\omega_1
		\conju{\alpha}(\omega_1)
		\sum_a^{N^\beta}
			\int\d\x_2
				\conju{\psi^\beta_a}(\r_2)
				\conju{\beta}(\omega_2)
				r_{12}^{-1}
				\psi^\beta_a(\r_1)
				\beta(\omega_1)
				\mathscr{P}_{1\rightarrow 2}
				\alpha(\omega_1) \\
%
%
&=
	h(\r_1)
	+
	\sum_a^{N^\alpha}
		J^\alpha_a(\r_1)
	+
	\sum_a^{N^\beta}
		J^\beta_a(\r_1) \nonumber \\ &\qquad
	-
	\sum_a^{N^\alpha}
		\int\d\omega_1
			\conju{\alpha}(\omega_1)
			\int\d\x_2
				\conju{\psi^\alpha_a}(\r_2)
				\conju{\alpha}(\omega_2)
				r_{12}^{-1}
				\psi^\alpha_a(\r_1)
				\alpha(\omega_1)
				\alpha(\omega_2)
				\mathscr{P}_{\r_1\rightarrow\r_2} \nonumber \\ &\qquad
	-
	\sum_a^{N^\beta}
		\int\d\omega_1
			\conju{\alpha}(\omega_1)
			\int\d\x_2
				\conju{\psi^\beta_a}(\r_2)
				\conju{\beta}(\omega_2)
				r_{12}^{-1}
				\psi^\beta_a(\r_1)
				\beta(\omega_1)
				\alpha(\omega_2)
				\mathscr{P}_{\r_1\rightarrow\r_2} \\
%
%
&=
	h(\r_1)
	+
	\sum_a^{N^\alpha}
		J^\alpha_a(\r_1)
	+
	\sum_a^{N^\beta}
		J^\beta_a(\r_1) \nonumber \\ &\qquad
	-
	\sum_a^{N^\alpha}
		\int\d\omega_1
			\conju{\alpha}(\omega_1)
			\alpha(\omega_1)
			\int\d\x_2
				\conju{\psi^\alpha_a}(\r_2)
				\conju{\alpha}(\omega_2)
				r_{12}^{-1}
				\psi^\alpha_a(\r_1)
				\alpha(\omega_2)
				\mathscr{P}_{\r_1\rightarrow\r_2} \nonumber \\ &\qquad
	-
	\sum_a^{N^\beta}
		\int\d\omega_1
			\conju{\alpha}(\omega_1)
			\beta(\omega_1)
			\int\d\x_2
				\conju{\psi^\beta_a}(\r_2)
				\conju{\beta}(\omega_2)
				r_{12}^{-1}
				\psi^\beta_a(\r_1)
				\alpha(\omega_2)
				\mathscr{P}_{\r_1\rightarrow\r_2} \\
%
%
&=
	h(\r_1)
	+
	\sum_a^{N^\alpha}
		J^\alpha_a(\r_1)
	+
	\sum_a^{N^\beta}
		J^\beta_a(\r_1) \nonumber \\ &\qquad
	-
	\sum_a^{N^\alpha}
		\int\d\r_2
			\conju{\psi^\alpha_a}(\r_2)
			r_{12}^{-1}
			\psi^\alpha_a(\r_1)
			\mathscr{P}_{\r_1\rightarrow\r_2} \\
%
%
&=
	h(\r_1)
	+
	\sum_a^{N^\alpha}
		J^\alpha_a(\r_1)
	+
	\sum_a^{N^\beta}
		J^\beta_a(\r_1)
	-
	\sum_a^{N^\alpha}
		\int\d\r_2
			\conju{\psi^\alpha_a}(\r_2)
			r_{12}^{-1}
			\mathscr{P}_{\r_1\r_2}
			\psi^\alpha_a(\r_2) \\
%
%
&=
	h(\r_1)
	+
	\sum_a^{N^\alpha}
		J^\alpha_a(\r_1)
	+
	\sum_a^{N^\beta}
		J^\beta_a(\r_1)
	-
	\sum_a^{N^\alpha}
		K^\alpha_a(\r_1)
\end{align}
となる。

















