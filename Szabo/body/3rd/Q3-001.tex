%ファイルID
%2020/07/06 15:45
%->2007061545SahayanKY(ファイル作成者)
\subsection{問}
Hartree-Fock方程式は次の通りである。
\begin{align}
	f\ket{\chi_a}
&=
	\epsilon_a \ket{\chi_a}
\end{align}
このうち、$f$はFock演算子と呼ばれるものであり、
\begin{align}
	f(1)
&=
	h(1)
	+
	v^{\rm HF}(1) \\
%
%
&=
	h(1)
	+
	\sum_{b}^{N}\left(
		\mathscr{J}_b(1)
		-
		\mathscr{K}_b(1)
	\right) \\
%
%
&=
	h(1)
	+
	\sum_{b}^{N}
		\int\d\x_2
			\conju{\chi_b}(2)
			r_{12}^{-1}
			(1-\mathscr{P}_{12})
			\chi_{b}(2)
\end{align}
である。

Fock演算子の行列要素が一般に
\begin{align}
	\braket{\chi_i|f|\chi_j}
&=
	\braket{i|h|j}
	+
	\sum_{b}^{N}\left(
		\chemtwo{ij}{bb}
		-
		\chemtwo{ib}{bj}
	\right) \\
%
%
&=
	\braket{i|h|j}
	+
	\sum_{b}^{N}
		\antitwo{ib}{jb}
\end{align}
となることを示せ。


\subsection{解}
\begin{align}
	\braket{\chi_i|f|\chi_j}
&=
	\int\d\x_1
		\conju{\chi_i}(1)
		f(1)
		\chi_j(1) \\
%
%
&=
	\int\d\x_1
		\conju{\chi_i}(1)
		h(1)
		\chi_j(1)
	+
	\int\d\x_1
		\conju{\chi_i}(1)
		\left(
			\sum_{b}^{N}
				\int\d\x_2
					\conju{\chi_b}(2)
					r_{12}^{-1}
					(1-\mathscr{P}_{12})
					\chi_b(2)
		\right)
		\chi_j(1) \\
%
%
&=
	\braket{i|h|j}
	+
	\sum_{b}^{N}\left(
		\int\d\x_1
			\conju{\chi_i}(1)
			\left(
				\int\d\x_2
					\conju{\chi_b}(2)
					r_{12}^{-1}
					(1-\mathscr{P}_{12})
					\chi_b(2)
			\right)
			\chi_j(1)
	\right) \\
%
%
&=
	\braket{i|h|j}
	+
	\sum_{b}^{N}\left(
	\begin{array}{>{\displaystyle}l}
		\int\d\x_1 \d\x_2
			\conju{\chi_i}(1) \conju{\chi_b}(2)
			r_{12}^{-1}
			\chi_b(2) \chi_j(1) \\
		-
		\int\d\x_1 \d\x_2
			\conju{\chi_i}(1) \conju{\chi_b}(2)
			r_{12}^{-1}
			\chi_b(1) \chi_j(2)
	\end{array}
	\right) \\
%
%
&=
	\braket{i|h|j}
	+
	\sum_{b}^{N}\left(
		\phystwo{ib}{jb}
		-
		\phystwo{ib}{bj}
	\right) \\
%
%
&=
	\braket{i|h|j}
	+
	\sum_{b}^{N}\left(
		\antitwo{ib}{jb}
	\right) \\
%
%
&=
	\braket{i|h|j}
	+
	\sum_{b}^{N}\left(
		\chemtwo{ij}{bb}
		-
		\chemtwo{ib}{bj}
	\right)
\end{align}
となる。



