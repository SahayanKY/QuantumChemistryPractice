%ファイルID
%2020/10/15 15:53
%->2010151553SahayanKY(ファイル作成者)
\subsection{問}
最小基底関数系(STO-3G)における\ce{H2}の非制限エネルギーは、
制限付き分子軌道における各積分を用いて
\begin{align}
	E_0(\theta)
&=
	2
	\cos^2\theta
	h_{11}
	+
	2
	\sin^2\theta
	h_{22}
	+
	\cos^4\theta
	J_{11}
	+
	\sin^4\theta
	J_{22}
	+
	2
	\sin^2\theta
	\cos^2\theta
	(J_{12} -2K_{12})
\end{align}
と書き表すことができる。ここで$\theta$は前問の定義の通りである。

ここから、エネルギー$E_0$が停留値をとる条件は$\theta=0$、または
\begin{align}
	\cos^2\theta
&=
	\eta
=
	\frac{
		h_{22} -h_{11} +J_{22} -J_{12} +2K_{12}
	}{
		J_{11} +J_{22} -2J_{12} +4K_{12}
	}
\end{align}
を満たす$\theta(0<\theta\leq\frac{\pi}{4})$である。
前者のときは制限解であり、
後者のときは非制限解である。

本書の付録を用いて
$R=\SI{1.4}{\atomic}$においては非制限解が存在しないことを示せ。
また、$R=\SI{4.0}{\atomic}$においては$\theta=\SI{39.5}{\degree}$に非制限解が存在することを示せ。


\subsection{解}
付録より$R=\SI{1.4}{\atomic}$における各種積分値は
\begin{align}
	\epsilon_1
&=
	-0.5782 &
%
%
	\epsilon_2
&=
	0.6703 &
%
%
	J_{11}
&=
	0.6746 &
%
%
	J_{12}
&=
	0.6636 &
%
%
	J_{22}
&=
	0.6975 &
%
%
	K_{12}
&=
	0.1813
\end{align}
である。ここで軌道エネルギーと各積分の関係を考えると
\begin{align}
	\epsilon_1
&=
	h_{11}
	+
	J_{11} &
%
%
	\epsilon_2
&=
	h_{22}
	+
	2J_{12}
	-
	K_{12}
\end{align}
であるので、
\begin{align}
	\eta
&=
	\frac{
		h_{22} -h_{11} +J_{22} -J_{12} +2K_{12}
	}{
		J_{11} +J_{22} -2J_{12} +4K_{12}
	} \\
%
%
&=
	\frac{
		(\epsilon_2 -2J_{12} +K_{12}) -(\epsilon_1 -J_{11}) +J_{22} -J_{12} +2K_{12}
	}{
		J_{11} +J_{22} -2J_{12} +4K_{12}
	} \\
%
%
&=
	\frac{
		\epsilon_2 -\epsilon_1
		+J_{11} +J_{22} -3J_{12} +3K_{12}
	}{
		J_{11} +J_{22} -2J_{12} +4K_{12}
	} \\
%
%
&=
	\frac{
		0.6703 -(-0.5782)
		+0.6746 +0.6975 -3\cdot0.6636 +3\cdot0.1813
	}{
		0.6746 +0.6975 -2\cdot0.6636 +4\cdot0.1813
	} \\
%
%
&=
	1.524
\end{align}
となる。$\cos\theta$の値域は$\frac{1}{\sqrt{2}}\leq\cos\theta\leq 1$である。従って、
$R=\SI{1.4}{\atomic}$では$\cos^2\theta=\eta$を満たすような$\theta$は存在せず、
非制限解は存在しないことが言える。

次に$R=\SI{4.0}{\atomic}$を考える。このとき各種積分値は
\begin{align}
	\epsilon_1
&=
	-0.2542 &
%
	\epsilon_2
&=
	0.0916 &
%
	J_{11}
&=
	0.5026 &
%
	J_{12}
&=
	0.5121 &
%
	J_{22}
&=
	0.5259 &
%
	K_{12}
&=
	0.2651
\end{align}
である。従って、$\eta$は
\begin{align}
	\eta
&=
	\frac{
		\epsilon_2 -\epsilon_1
		+J_{11} +J_{22} -3J_{12} +3K_{12}
	}{
		J_{11} +J_{22} -2J_{12} +4K_{12}
	} \\
%
%
&=
	\frac{
		0.0916 -(-0.2542)
		+0.5026 +0.5259 -3\cdot0.5121 +3\cdot0.2651
	}{
		0.5026 +0.5259 -2\cdot0.5121 +4\cdot0.2651
	} \\
%
%
&=
	0.5948
\end{align}
となり、このとき$\cos\theta$の値域に含まれており、
\begin{align}
	\cos^2\theta
&=
	\eta \\
%
%
	\theta
&=
	\SI{39.53}{\degree}
\end{align}
である。よって、$\theta=\SI{39.53}{\degree}$の非制限軌道がエネルギーの極値をとり、
非制限解が存在することが言える。


