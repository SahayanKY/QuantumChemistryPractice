%ファイルID
%2020/09/15 2053
%->2009152053SahayanKY(ファイル作成者)
\subsection{問}
最小基底関数による\ce{H2}のFock行列要素が
\begin{align}
	F_{11}
=
	F_{22}
&=
	H^{\rm core}_{11}
	+
	(1+S_{12})^{-1}
	\left[
		\frac{1}{2}
		\spacetwo{\phi_1\phi_1}{\phi_1\phi_1}
		+
		\spacetwo{\phi_1\phi_1}{\phi_2\phi_2}
		+
		\spacetwo{\phi_1\phi_1}{\phi_1\phi_2}
		-
		\frac{1}{2}
		\spacetwo{\phi_1\phi_2}{\phi_1\phi_2}
	\right] \\
&=
	\SI{-0.3655}{\atomic} \\[3mm]
%
%
	F_{12}
=
	F_{21}
&=
	H^{\rm core}_{12}
	+
	(1+S_{12})^{-1}
	\left[
		-
		\frac{1}{2}
		\spacetwo{\phi_1\phi_1}{\phi_2\phi_2}
		+
		\spacetwo{\phi_1\phi_1}{\phi_1\phi_2}
		+
		\frac{3}{2}
		\spacetwo{\phi_1\phi_2}{\phi_1\phi_2}
	\right] \\
%
%
&=
	\SI{-0.5939}{\atomic}
\end{align}
となることを示せ。

\subsection{解}
Fock行列は次の通りである。
\begin{align}
	F_{\mu\nu}
&=
	H^{\rm core}_{\mu\nu}
	+
	\sum_{\lambda,\sigma}
		P_{\lambda\sigma}
		\left[
			\spacetwo{\mu\nu}{\sigma\lambda}
			-
			\frac{1}{2}
			\spacetwo{\mu\lambda}{\sigma\nu}
		\right]
\end{align}
問題3.24で見た通り、
最小基底関数における\ce{H2}の密度行列の全要素は等しく、
\begin{align}
	P_{\mu\nu}
&=
	(1+S_{12})^{-1}
	%
	\qquad
	(\mu = 1,2,\ \nu = 1,2)
\end{align}
である。

従って、\ce{H2}のFock行列は
\begin{align}
	F_{11}
&=
	H^{\rm core}_{11}
	+
	(1+S_{12})^{-1}
	\sum_{\lambda,\sigma}
		\left[
			\spacetwo{11}{\sigma\lambda}
			-
			\frac{1}{2}
			\spacetwo{1\lambda}{\sigma 1}
		\right] \\
%
%
&=
	H^{\rm core}_{11}
	+
	(1+S_{12})^{-1}
	\left[
	\begin{array}{l}
		\spacetwo{11}{11}
		+
		\spacetwo{11}{12}
		+
		\spacetwo{11}{21}
		+
		\spacetwo{11}{22} \\
		-
		\frac{1}{2}
		\spacetwo{11}{11}
		-
		\frac{1}{2}
		\spacetwo{11}{21}
		-
		\frac{1}{2}
		\spacetwo{12}{11}
		-
		\frac{1}{2}
		\spacetwo{12}{21}
	\end{array}
	\right] \\
%
%
&=
	H^{\rm core}_{11}
	+
	(1+S_{12})^{-1}
	\left[
	\begin{array}{l}
		\spacetwo{11}{11}
		+
		\spacetwo{11}{12}
		+
		\spacetwo{11}{12}
		+
		\spacetwo{11}{22} \\
		-
		\frac{1}{2}
		\spacetwo{11}{11}
		-
		\frac{1}{2}
		\spacetwo{11}{12}
		-
		\frac{1}{2}
		\spacetwo{11}{12}
		-
		\frac{1}{2}
		\spacetwo{12}{12}
	\end{array}
	\right] \\
%
%
&=
	H^{\rm core}_{11}
	+
	(1+S_{12})^{-1}
	\left[
		\frac{1}{2}
		\spacetwo{11}{11}
		+
		\spacetwo{11}{12}
		+
		\spacetwo{11}{22}
		-
		\frac{1}{2}
		\spacetwo{12}{12}
	\right]
\end{align}
\begin{align}
	F_{12}
&=
	H^{\rm core}_{12}
	+
	(1+S_{12})^{-1}
	\sum_{\lambda,\sigma}
		\left[
			\spacetwo{12}{\sigma\lambda}
			-
			\frac{1}{2}
			\spacetwo{1\lambda}{\sigma 2}
		\right] \\
%
%
&=
	H^{\rm core}_{12}
	+
	(1+S_{12})^{-1}
	\left[
	\begin{array}{l}
		\spacetwo{12}{11}
		+
		\spacetwo{12}{12}
		+
		\spacetwo{12}{21}
		+
		\spacetwo{12}{22} \\
		-
		\frac{1}{2}
		\spacetwo{11}{12}
		-
		\frac{1}{2}
		\spacetwo{11}{22}
		-
		\frac{1}{2}
		\spacetwo{12}{12}
		-
		\frac{1}{2}
		\spacetwo{12}{22}
	\end{array}
	\right] \\
%
%
&=
	H^{\rm core}_{12}
	+
	(1+S_{12})^{-1}
	\left[
	\begin{array}{l}
		\spacetwo{11}{12}
		+
		\spacetwo{12}{12}
		+
		\spacetwo{12}{12}
		+
		\spacetwo{11}{12} \\
		-
		\frac{1}{2}
		\spacetwo{11}{12}
		-
		\frac{1}{2}
		\spacetwo{11}{22}
		-
		\frac{1}{2}
		\spacetwo{12}{12}
		-
		\frac{1}{2}
		\spacetwo{11}{12}
	\end{array}
	\right] \\
%
%
&=
	H^{\rm core}_{12}
	+
	(1+S_{12})^{-1}
	\left[
		\spacetwo{11}{12}
		+
		\frac{3}{2}
		\spacetwo{12}{12}
		-
		\frac{1}{2}
		\spacetwo{11}{22}
	\right]
\end{align}
\begin{align}
	F_{21}
&=
	F_{12}
	%
	\qquad
	\left(
	\begin{array}{l}
		\because \bm{F}はエルミート行列であり、 \\
		基底関数として実関数を用いているので \\
		\bm{F}が対称行列になるため
	\end{array}
	\right)
\end{align}
\begin{align}
	F_{22}
&=
	H^{\rm core}_{22}
	+
	(1+S_{12})^{-1}
	\left[
	\begin{array}{l}
		\spacetwo{22}{11}
		+
		\spacetwo{22}{12}
		+
		\spacetwo{22}{21}
		+
		\spacetwo{22}{22} \\
		-
		\frac{1}{2}
		\spacetwo{21}{12}
		-
		\frac{1}{2}
		\spacetwo{21}{22}
		-
		\frac{1}{2}
		\spacetwo{22}{12}
		-
		\frac{1}{2}
		\spacetwo{22}{22}
	\end{array}
	\right] \\
%
%
&=
	H^{\rm core}_{22}
	+
	(1+S_{12})^{-1}
	\left[
	\begin{array}{l}
		\spacetwo{22}{11}
		+
		\spacetwo{22}{12}
		+
		\spacetwo{22}{12}
		+
		\spacetwo{22}{22} \\
		-
		\frac{1}{2}
		\spacetwo{21}{12}
		-
		\frac{1}{2}
		\spacetwo{22}{12}
		-
		\frac{1}{2}
		\spacetwo{22}{12}
		-
		\frac{1}{2}
		\spacetwo{22}{22}
	\end{array}
	\right] \\
%
%
&=
	H^{\rm core}_{22}
	+
	(1+S_{12})^{-1}
	\left[
		\spacetwo{22}{11}
		+
		\spacetwo{22}{12}
		+
		\frac{1}{2}
		\spacetwo{22}{22}
		-
		\frac{1}{2}
		\spacetwo{21}{12}
	\right] \\
%
%
&=
	H^{\rm core}_{11}
	+
	(1+S_{12})^{-1}
	\left[
		\spacetwo{11}{22}
		+
		\spacetwo{11}{12}
		+
		\frac{1}{2}
		\spacetwo{11}{11}
		-
		\frac{1}{2}
		\spacetwo{12}{12}
	\right] \\
&=
	F_{11}
\end{align}
である。ここで$H^{\rm core}_{11}=H^{\rm core}_{22}$としたが、
これは問題3.23での結果を用いている。
問題3.23では\ce{H2+}の核-1電子ハミルトニアン行列$H^{\rm core}_{\mu\nu}$について考えたが、
\ce{H2}の$H^{\rm core}_{\mu\nu}$についても同様の議論が成立するためである。
これは最小基底関数系を用いていることと等核二原子分子であることに起因する。

Fock行列の各要素の値について考える。
核間距離が$R=\SI{1.4}{\atomic}$であるとき
核-1電子ハミルトニアン行列は
\begin{align}
	H^{\rm core}_{11}
&=
	-1.1204 &
%
	H^{\rm core}_{12}
&=
	-0.9584
\end{align}
であり、2電子積分は
\begin{align}
	\spacetwo{11}{11}
&=
	0.7746 &
%
	\spacetwo{11}{22}
&=
	0.5697 &
%
	\spacetwo{11}{12}
&=
	0.4441 &
%
	\spacetwo{12}{12}
&=
	0.2970
\end{align}
である。従って、Fock行列の要素は
\begin{align}
	F_{11}
=
	F_{22}
&=
	H^{\rm core}_{11}
	+
	(1+S_{12})^{-1}
	\left[
		\frac{1}{2}
		\spacetwo{11}{11}
		+
		\spacetwo{11}{12}
		+
		\spacetwo{11}{22}
		-
		\frac{1}{2}
		\spacetwo{12}{12}
	\right] \\
%
%
&=
	-
	1.1204
	+
	(1+0.6593)^{-1}
	\left[
		\frac{1}{2} \cdot
		0.7746
		+
		0.4441
		+
		0.5697
		-
		\frac{1}{2} \cdot
		0.2970
	\right] \\
%
%
&=
	-0.3655
\end{align}
\begin{align}
	F_{12}
=
	F_{21}
&=
	H^{\rm core}_{12}
	+
	(1+S_{12})^{-1}
	\left[
		\spacetwo{11}{12}
		+
		\frac{3}{2}
		\spacetwo{12}{12}
		-
		\frac{1}{2}
		\spacetwo{11}{22}
	\right] \\
%
%
&=
	-
	0.9584
	+
	(1+0.6593)^{-1}
	\left[
		0.4441
		+
		\frac{3}{2} \cdot
		0.2970
		-
		\frac{1}{2} \cdot
		0.5697
	\right] \\
%
%
&=
	-0.5939
\end{align}
である。


