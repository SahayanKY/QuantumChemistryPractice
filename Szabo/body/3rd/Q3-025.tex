%ファイルID
%2020/09/15 2053
%->2009152053SahayanKY(ファイル作成者)
\subsection{問}
最小基底関数による\ce{H2}のFock行列要素が
\begin{align}
	F_{11}
=
	F_{22}
&=
	H^{\rm core}_{11}
	+
	(1+S_{12})^{-1}
	\left[
		\frac{1}{2}
		\spacetwo{\phi_1\phi_1}{\phi_1\phi_1}
		+
		\spacetwo{\phi_1\phi_1}{\phi_2\phi_2}
		+
		\spacetwo{\phi_1\phi_1}{\phi_1\phi_2}
		-
		\frac{1}{2}
		\spacetwo{\phi_1\phi_2}{\phi_1\phi_2}
	\right] \\
&=
	\SI{-0.3655}{\atomic} \\[3mm]
%
%
	F_{12}
=
	F_{21}
&=
	H^{\rm core}_{12}
	+
	(1+S_{12})^{-1}
	\left[
		-
		\frac{1}{2}
		\spacetwo{\phi_1\phi_1}{\phi_2\phi_2}
		+
		\spacetwo{\phi_1\phi_1}{\phi_1\phi_2}
		+
		\frac{3}{2}
		\spacetwo{\phi_1\phi_2}{\phi_1\phi_2}
	\right] \\
%
%
&=
	\SI{-0.5939}{\atomic}
\end{align}
となることを示せ。

\subsection{解}



