%ファイルID
%2020/08/10 2124
%->2008102124SahayanKY(ファイル作成者)
\subsection{問}
$\bm{R}_A$を中心とする規格化1s Gauss型関数は
\begin{align}
	\phi^{\rm GF}_{\rm 1s}(\alpha,\r-\bm{R}_{A})
&=
	\left(
		\frac{2\alpha}{\pi}
	\right)^{\frac{3}{4}}
	\exp\left(
		-\alpha|\r-\bm{R}_{A}|^2
	\right)
\end{align}
である。

2個の1s Gauss型関数の積が
1個の1s Gauss型関数にまとめることが出来ることを示せ。
つまり、
\begin{align}
	\phi^{\rm GF}_{\rm 1s}(\alpha,\r-\bm{R}_A)
	\phi^{\rm GF}_{\rm 1s}(\beta,\r-\bm{R}_B)
&=
	K_{AB}
	\phi^{\rm GF}_{\rm 1s}(p,\r-\bm{R}_p) \\
%
%
	K_{AB}
&=
	\left(
		\frac{2\alpha\beta}{(\alpha+\beta)\pi}
	\right)^{\frac{3}{4}}
	\exp\left(
		-\frac{\alpha\beta}{\alpha+\beta}
		|\bm{R}_A-\bm{R}_B|^2
	\right) \\
%
%
	p
&=
	\alpha
	+
	\beta \\
%
%
	\bm{R}_p
&=
	\frac{
		\alpha\bm{R}_A
		+
		\beta\bm{R}_B
	}{
		\alpha
		+
		\beta
	}
\end{align}
が成立することを示せ。


\subsection{解}
\begin{align}
	\phi^{\rm GF}_{\rm 1s}(\alpha,\r-\bm{R}_A)
	\phi^{\rm GF}_{\rm 1s}(\beta,\r-\bm{R}_B)
&=
	\left(
		\frac{
			4\alpha\beta
		}{
			\pi^2
		}
	\right)^{\frac{3}{4}}
	\exp\left(
		-\alpha|\r-\bm{R}_A|^2
		-\beta|\r-\bm{R}_B|^2
	\right)
\end{align}
ここで$\exp$内については
\begin{align}
	-
	\alpha|\r-\bm{R}_A|^2
	-
	\beta|\r-\bm{R}_B|^2
&=
	-
	(\alpha+\beta)|\r|^2
	+
	2(\alpha\bm{R}_A+\beta\bm{R}_B)\cdot\r
	-
	\alpha|\bm{R}_A|^2
	-
	\beta|\bm{R}_B|^2 \\
%
%
&=
	-
	(\alpha+\beta)
	\left(
		|\r|^2
		-
		2
		\frac{\alpha\bm{R}_A+\beta\bm{R}_B}{\alpha+\beta}\cdot\r
		+
		\left|
			\frac{\alpha\bm{R}_A+\beta\bm{R}_B}{\alpha+\beta}
		\right|^2
	\right) \nonumber \\&\qquad
	+
	(\alpha+\beta)
	\left|
		\frac{\alpha\bm{R}_A+\beta\bm{R}_B}{\alpha+\beta}
	\right|^2
	-
	\alpha|\bm{R}_A|^2
	-
	\beta|\bm{R}_B|^2 \\
%
%
&=
	-
	p|\r-\bm{R}_p|^2 \nonumber \\&\qquad
	+
	\frac{
		\alpha^2|\bm{R}_A|^2
		+
		2\alpha\beta\bm{R}_A\cdot\bm{R}_B
		+
		\beta^2|\bm{R}_B|^2
		-
		\alpha(\alpha+\beta)|\bm{R}_A|^2
		-
		\beta(\alpha+\beta)|\bm{R}_B|^2
	}{
		\alpha+\beta
	} \\
%
%
&=
	-
	p|\r-\bm{R}_p|^2
	+
	\frac{
		-
		\alpha\beta|\bm{R}_A|^2
		+
		2\alpha\beta\bm{R}_A\cdot\bm{R}_B
		-
		\alpha\beta|\bm{R}_B|^2
	}{
		\alpha+\beta
	} \\
%
%
&=
	-
	p|\r-\bm{R}_p|^2
	-
	\frac{\alpha\beta}{\alpha+\beta}
	|\bm{R}_A-\bm{R}_B|^2
\end{align}
と書き換えられる。また、$\exp$の前の係数の部分については
\begin{align}
	\left(
		\frac{4\alpha\beta}{\pi^2}
	\right)^{\frac{3}{4}}
&=
	\left(
		\frac{2\alpha\beta}{(\alpha+\beta)\pi}\cdot
		\frac{2(\alpha+\beta)}{\pi}
	\right)^{\frac{3}{4}} \\
%
%
&=
	\left(
		\frac{2\alpha\beta}{(\alpha+\beta)\pi}
	\right)^{\frac{3}{4}}
	\left(
		\frac{2p}{\pi}
	\right)^{\frac{3}{4}}
\end{align}
と書き換えられる。従って、
\begin{align}
	\phi^{\rm GF}_{\rm 1s}(\alpha,\r-\bm{R}_A)
	\phi^{\rm GF}_{\rm 1s}(\beta,\r-\bm{R}_B)
&=
	\left(
		\frac{2\alpha\beta}{(\alpha+\beta)\pi}
	\right)^{\frac{3}{4}}
	\left(
		\frac{2p}{\pi}
	\right)^{\frac{3}{4}}
	\exp\left(
		-p|\r-\bm{R}_p|^2
	\right)
	\exp\left(
		-\frac{\alpha\beta}{\alpha+\beta}|\bm{R}_A-\bm{R}_B|^2
	\right) \\
%
%
&=
	K_{AB}
	\phi^{\rm GF}_{\rm 1s}(p,\r-\bm{R}_p)
\end{align}
となる。
