%ファイルID
%2020/09/26 16:43
%->2009261643SahayanKY(ファイル作成者)
\subsection{問}
1電子演算子の和$\Oone=\sum_i^N h(i)$の
任意の1個の非制限行列式による期待値が
\begin{align}
	\average{\Oone}
&=
	\sum_\mu
	\sum_\nu
		P^T_{\mu\nu}
		\spaceone{\nu}{h}{\mu}
\end{align}
となることを示せ。ここで$P^T_{\mu\nu}=P^\alpha_{\mu\nu}+P^\beta_{\mu\nu}$である。

\subsection{解}
あるSlater行列式$\ket{\Psi}=\ket{\chi_a\chi_b\dots}$での
期待値$\average{\Oone}$は、
本書の表2.3にて示されている通り
\begin{align}
	\average{\Oone}
&=
	\braket{\Psi|\Oone|\Psi} \\
%
%
&=
	\sum_a^N
		\braket{\chi_a|h|\chi_a}
\end{align}
である。この総和を、$\alpha$スピン軌道についての和と
$\beta$スピン軌道についての和に分解すると
\begin{align}
	\average{\Oone}
&=
	\sum_a^{N^\alpha}
		\braket{\chi_a|h|\chi_a}
	+
	\sum_a^{N^\beta}
		\braket{\chi_a|h|\chi_a} \\
%
%
&=
	\sum_a^{N^\alpha}
		\braket{\psi^\alpha_a|h|\psi^\alpha_a}
	+
	\sum_a^{N^\beta}
		\braket{\psi^\beta_a|h|\psi^\beta_a} \\
%
%
&=
	\sum_a^{N^\alpha}
		\Braket{
			\sum_\nu C^\alpha_{\nu a} \phi_\nu |
			h |
			\sum_\mu C^\alpha_{\mu a} \phi_\mu
		}
	+
	\sum_a^{N^\beta}
		\Braket{
			\sum_\nu C^\beta_{\nu a} \phi_\nu |
			h |
			\sum_\mu C^\beta_{\mu a} \phi_\mu
		} \\
%
%
&=
	\sum_\nu
	\sum_\mu
	\sum_a^{N^\alpha}
		\conju{C^\alpha_{\nu a}}
		C^\alpha_{\mu a}
		\braket{\nu|h|\mu}
	+
	\sum_\nu
	\sum_\mu
	\sum_a^{N^\beta}
		\conju{C^\beta_{\nu a}}
		C^\beta_{\mu a}
		\braket{\nu|h|\mu} \\
%
%
&=
	\sum_\nu
	\sum_\mu
		P^\alpha_{\mu\nu}
		\spaceone{\nu}{h}{\mu}
	+
	\sum_\nu
	\sum_\mu
		P^\beta_{\mu\nu}
		\spaceone{\nu}{h}{\mu} \\
%
%
&=
	\sum_\mu
	\sum_\nu
		\left(
			P^\alpha_{\mu\nu}
			+
			P^\beta_{\mu\nu}
		\right)
		\spaceone{\nu}{h}{\mu} \\
%
%
&=
	\sum_\mu
	\sum_\nu
		P^T_{\mu\nu}
		\spaceone{\nu}{h}{\mu}
\end{align}
となる。



