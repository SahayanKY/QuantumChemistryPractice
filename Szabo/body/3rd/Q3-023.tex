%ファイルID
%2020/09/14 1637
%->2009141637SahayanKY(ファイル作成者)
\subsection{問}
\ce{H2+}のRoothaan方程式は次の通りである。
\begin{align}
	\bm{H}^{\rm core}
	\bm{C}
&=
	\bm{S}
	\bm{C}
	\bm{\epsilon}
\end{align}
ここで$\bm{H}^{\rm core}$は核-1電子ハミルトニアンである。
最小基底関数ではこれを解くと、
\begin{align}
	\epsilon_1
&=
	\frac{
		H_{11}^{\rm core}
		+
		H_{12}^{\rm core}
	}{
		1
		+
		S_{12}
	} \\
%
%
	\epsilon_2
&=
	\frac{
		H_{11}^{\rm core}
		-
		H_{12}^{\rm core}
	}{
		1
		-
		S_{12}
	}
\end{align}
であること、$R=\SI{1.4}{\atomic}$では
\begin{align}
	\epsilon_1
&=
	\SI{-1.2528}{\atomic} \\
%
%
	\epsilon_2
&=
	\SI{-0.4756}{\atomic}
\end{align}
であること示せ。


\subsection{解}
水素Aの中心を$\bm{R}_A=-\frac{1}{2}R\e{x}$、
水素Bの中心を$\bm{R}_B=\frac{1}{2}R\e{x}$とする。

核-1電子ハミルトニアン行列は
\begin{align}
	H^{\rm core}_{\mu\nu}
&=
	\int\d\r
		\conju{\phi_\mu}(\r)
		h(\r)
		\phi_\nu(\r)
\end{align}
である。ここで$H^{\rm core}_{21}$と$H^{\rm core}_{22}$について考える。
また、変数変換$\r'=-\r$を考える。
このとき、
\begin{align}
	\phi_1(\r)
&=
	\phi_2(\r')
\end{align}
が成立する。また、
\begin{align}
	\pard{x}{}{}
&=
	-
	\pard{x'}{}{} &
%
	\pard{y}{}{}
&=
	-
	\pard{y'}{}{} &
%
	\pard{z}{}{}
&=
	-
	\pard{z'}{}{}
\end{align}
より
\begin{align}
	\laplacian{}_{\r}
&=
	\laplacian{}_{\r'}
\end{align}
である。また、
\begin{align}
&\quad
	-
	\frac{Z_A}{|\r-\bm{R}_A|}
	-
	\frac{Z_B}{|\r-\bm{R}_B|} \\
%
%
&=
	-
	\frac{Z_A}{|-\r'+\bm{R}_B|}
	-
	\frac{Z_B}{|-\r'+\bm{R}_A|}
	%
	\qquad
	(\bm{R}_A=-\bm{R}_B) \\
%
%
&=
	-
	\frac{Z_A}{|\r'-\bm{R}_B|}
	-
	\frac{Z_B}{|\r'-\bm{R}_A|} \\
%
%
&=
	-
	\frac{Z_B}{|\r'-\bm{R}_B|}
	-
	\frac{Z_A}{|\r'-\bm{R}_A|}
	%
	\qquad
	(Z_A=Z_B)
\end{align}
であるので$h(\r)=h(\r')$が成立する。
よって、核-1電子ハミルトニアン行列の要素$H^{\rm core}_{21}$は
\begin{align}
	H^{\rm core}_{21}
&=
	\int\d\r
		\conju{\phi_2}(\r)
		h(\r)
		\phi_1(\r) \\
%
%
&=
	\int\d\r
		\conju{\phi_1}(\r')
		h(\r')
		\phi_2(\r') \\
%
%
&=
	\int_{-\infty}^{\infty} \d x
	\int_{-\infty}^{\infty} \d y
	\int_{-\infty}^{\infty} \d z
		\conju{\phi_1}(\r')
		h(\r')
		\phi_2(\r') \\
%
%
&=
	\int_{-\infty}^{\infty} \d x'
	\int_{-\infty}^{\infty} \d y'
	\int_{-\infty}^{\infty} \d z'
		\conju{\phi_1}(\r')
		h(\r')
		\phi_2(\r') \\
%
%
&=
	\int\d\r'
		\conju{\phi_1}(\r')
		h(\r')
		\phi_2(\r') \\
%
%
&=
	H^{\rm core}_{12}
\end{align}
同様に
\begin{align}
	H^{\rm core}_{11}
&=
	H^{\rm core}_{22}
\end{align}
が成立する。
	%
	\footnote{
		$H^{\rm core}_{21}$については
		こんな面倒なことを考えなくとも、
		$h(1)$がエルミート演算子であるから
		$\bm{H}^{\rm core}$はエルミート行列であり、
		基底関数として実関数を用いているので
		$\bm{H}^{\rm core}$は対称行列であると言えばいい。
	}
	%

よって、Roothaan方程式より
\begin{align}
&
	\left[
	\begin{array}{cc}
		H^{\rm core}_{11} & H^{\rm core}_{12} \\
		%
		H^{\rm core}_{12} & H^{\rm core}_{11}
	\end{array}
	\right]
	\left[
	\begin{array}{cc}
		\left[2(1+S_{12})\right]^{-1/2} & \left[2(1-S_{12})\right]^{-1/2} \\
		%
		\left[2(1+S_{12})\right]^{-1/2} & -\left[2(1-S_{12})\right]^{-1/2}
	\end{array}
	\right] \nonumber\\
&\qquad\qquad
=
	\left[
	\begin{array}{cc}
		1 & S_{12} \\
		%
		S_{12} & 1
	\end{array}
	\right]
	\left[
	\begin{array}{cc}
		\left[2(1+S_{12})\right]^{-1/2} & \left[2(1-S_{12})\right]^{-1/2} \\
		%
		\left[2(1+S_{12})\right]^{-1/2} & -\left[2(1-S_{12})\right]^{-1/2}
	\end{array}
	\right]
	\left[
	\begin{array}{cc}
		\epsilon_1 & 0 \\
		%
		0 & \epsilon_2
	\end{array}
	\right] \\[3mm]
%
%
&
	\left[
	\begin{array}{cc}
		\left[2(1+S_{12})\right]^{-1/2}
		(H^{\rm core}_{11} +H^{\rm core}_{12})
		&
		\left[2(1-S_{12})\right]^{-1/2}
		(H^{\rm core}_{11} -H^{\rm core}_{12}) \\
		%
		\left[2(1+S_{12})\right]^{-1/2}
		(H^{\rm core}_{11} +H^{\rm core}_{12})
		&
		\left[2(1-S_{12})\right]^{-1/2}
		(H^{\rm core}_{12} -H^{\rm core}_{11})
	\end{array}
	\right] \nonumber\\
&\qquad\qquad
=
	\left[
	\begin{array}{cc}
		\left[2(1+S_{12})\right]^{-1/2}
		(1+S_{12})
		&
		\left[2(1-S_{12})\right]^{-1/2}
		(1-S_{12}) \\
		%
		\left[2(1+S_{12})\right]^{-1/2}
		(1+S_{12})
		&
		\left[2(1-S_{12})\right]^{-1/2}
		(S_{12}-1)
	\end{array}
	\right]
	\left[
	\begin{array}{cc}
		\epsilon_1 & 0 \\
		0 & \epsilon_2
	\end{array}
	\right]
\end{align}
\begin{align}
&
	\therefore
	\left\{
	\begin{array}{r@{\,}l}
			\left[2(1+S_{12})\right]^{-1/2}
			(H^{\rm core}_{11} +H^{\rm core}_{12})
		&=
			\epsilon_1
			\left[2(1+S_{12})\right]^{-1/2}
			(1+S_{12}) \\[2mm]
		%
			\left[2(1-S_{12})\right]^{-1/2}
			(H^{\rm core}_{11} -H^{\rm core}_{12})
		&=
			\epsilon_2
			\left[2(1-S_{12})\right]^{-1/2}
			(1-S_{12})
	\end{array}
	\right. \\
%
%
&\quad
	\left\{
	\begin{array}{r@{\,}l}
			\epsilon_1
		&=
			\dfrac{
				H^{\rm core}_{11} +H^{\rm core}_{12}
			}{
				1 +S_{12}
			} \\[3mm]
		%
			\epsilon_2
		&=
			\dfrac{
				H^{\rm core}_{11} -H^{\rm core}_{12}
			}{
				1 -S_{12}
			}
	\end{array}
	\right.
\end{align}
となる。

本書より最小基底関数(STO-3G)で$R=\SI{1.4}{\atomic}$のとき
\begin{align}
	H^{\rm core}_{11}
&=
	-1.1204 &
%
	H^{\rm core}_{12}
&=
	-0.9584 &
%
	S_{12}
&=
	0.6593
\end{align}
であるので、
\begin{align}
	\epsilon_1
&=
	-1.2528 &
%
	\epsilon_2
&=
	-0.4755
\end{align}
である。従って、軌道エネルギーはそれぞれ\SI{-1.2528}{\atomic}、\SI{-0.4755}{\atomic}である。