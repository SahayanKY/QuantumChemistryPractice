%ファイルID
%2020/07/29 14:04
%->2007291404SahayanKY(ファイル作成者)
\subsection{問}
スピン軌道$\chi_i$の軌道エネルギー$\epsilon_i$は
\begin{align}
	\epsilon_i
&=
	\braket{\chi_i|h|\chi_i}
	+
	\sum_b^N
		\antitwo{\chi_i\chi_b}{\chi_i\chi_b}
\end{align}
である。これを用いて、
閉殻系における空間軌道を使った表式
\begin{align}
	\epsilon_i
&=
	\spaceone{\psi_i}{h}{\psi_i}
	+
	\sum_b^{N/2}\left(
		2\spacetwo{ii}{bb}
		-
		\spacetwo{ib}{bi}
	\right)
=
	h_{ii}
	+
	\sum_b^{N/2}\left(
		2J_{ib}
		-
		K_{ib}
	\right)
\end{align}
に書き換えよ。

\subsection{解}
スピン軌道に制限付きの条件を加え、
\begin{align}
	\left\{
	\begin{array}{l@{}l}
			\chi_{2i-1}
		&=
			\psi_i(\r) \alpha(\omega) \\
		%
			\chi_{2i}
		&=
			\psi_i(\r) \beta(\omega)
	\end{array}
	\right.
\end{align}
とする。

このとき、
\begin{align}
	\braket{\chi_{2i-1}|h|\chi_{2i-1}}
&=
	\int\d\r\d\omega
		\conju{\psi_i}(\r) \conju{\alpha}(\omega)
		h(\r)
		\psi_i(\r) \alpha(\omega) \\
%
%
&=
	\int\d\r
		\conju{\psi_i}(\r)
		h(\r)
		\psi_i(\r) \\
%
%
&=
	\spaceone{\psi_i}{h}{\psi_i} \\[2mm]
%
%
	\braket{\chi_{2i}|h|\chi_{2i}}
&=
	\spaceone{\psi_i}{h}{\psi_i}
\end{align}
である。
また、クーロン積分については$\chi_{2i-1}$では
\begin{align}
	\sum_b^N
		\phystwo{\chi_{2i-1}\chi_b}{\chi_{2i-1}\chi_b}
&=
	\sum_b^N
		\int\d\x_1\d\x_2
			\conju{\chi_{2i-1}}(\x_1)
			\conju{\chi_b}(\x_2)
			r_{12}^{-1}
			\chi_{2i-1}(\x_1)
			\chi_b(\x_2) \\
%
%
&=
	\sum_j^{N/2}
		\int\d\x_1\d\x_2
			\conju{\chi_{2i-1}}(\x_1)
			\conju{\chi_{2j-1}}(\x_2)
			r_{12}^{-1}
			\chi_{2i-1}(\x_1)
			\chi_{2j-1}(\x_2) \nonumber \\&\qquad
	+
	\sum_j^{N/2}
		\int\d\x_1\d\x_2
			\conju{\chi_{2i-1}}(\x_1)
			\conju{\chi_{2j}}(\x_2)
			r_{12}^{-1}
			\chi_{2i-1}(\x_1)
			\chi_{2j}(\x_2) \\
%
%
&=
	\sum_j^{N/2}
		\int\d\r_1\d\r_2
			\conju{\psi_i}(\r_1)
			\conju{\psi_j}(\r_2)
			r_{12}^{-1}
			\psi_i(\r_1)
			\psi_j(\r_2) \nonumber \\&\qquad
	+
	\sum_j^{N/2}
		\int\d\r_1\d\r_2
			\conju{\psi_i}(\r_1)
			\conju{\psi_j}(\r_2)
			r_{12}^{-1}
			\psi_i(\r_1)
			\psi_j(\r_2) \\
%
%
&=
	2
	\sum_j^{N/2}
		\int\d\r_1\d\r_2
			\conju{\psi_i}(\r_1)
			\psi_i(\r_1)
			r_{12}^{-1}
			\conju{\psi_j}(\r_2)
			\psi_j(\r_2) \\
%
%
&=
	2
	\sum_j^{N/2}
		\spacetwo{\psi_i\psi_i}{\psi_j\psi_j} \\
%
%
&=
	2
	\sum_j^{N/2}
		\spacetwo{ii}{jj}
\end{align}
であり、$\chi_{2i}$では、同様にして
\begin{align}
	\sum_b^N
		\phystwo{\chi_{2i}\chi_b}{\chi_{2i}\chi_b}
&=
	2
	\sum_j^{N/2}
		\spacetwo{ii}{jj}
\end{align}
る。また、同様に交換積分については、$\chi_{2i-1}$では
\begin{align}
	\sum_b^N
		\phystwo{\chi_{2i-1}\chi_b}{\chi_b\chi_{2i-1}}
&=
	\sum_b^N
		\int\d\x_1\d\x_2
			\conju{\chi_{2i-1}}(\x_1)
			\conju{\chi_b}(\x_2)
			r_{12}^{-1}
			\chi_b(\x_1)
			\chi_{2i-1}(\x_2) \\
%
%
&=
	\sum_j^{N/2}
		\int\d\x_1\d\x_2
			\conju{\chi_{2i-1}}(\x_1)
			\conju{\chi_{2j-1}}(\x_2)
			r_{12}^{-1}
			\chi_{2j-1}(\x_1)
			\chi_{2i-1}(\x_2) \nonumber \\&\qquad
	+
	\sum_j^{N/2}
		\int\d\x_1\d\x_2
			\conju{\chi_{2i-1}}(\x_1)
			\conju{\chi_{2j}}(\x_2)
			r_{12}^{-1}
			\chi_{2j}(\x_1)
			\chi_{2i-1}(\x_2) \\
%
%
&=
	\sum_j^{N/2}
		\int\d\r_1\d\r_2
			\conju{\psi_i}(\r_1)
			\conju{\psi_j}(\r_2)
			r_{12}^{-1}
			\psi_j(\r_1)
			\psi_i(\r_2) \nonumber \\&\qquad
	+
	\sum_j^{N/2}
		0 \\
%
%
&=
	\sum_j^{N/2}
		\int\d\r_1\d\r_2
			\conju{\psi_i}(\r_1)
			\psi_j(\r_1)
			r_{12}^{-1}
			\conju{\psi_j}(\r_2)
			\psi_i(\r_2) \\
%
%
&=
	\sum_j^{N/2}
		\spacetwo{\psi_i\psi_j}{\psi_j\psi_i} \\
%
%
&=
	\sum_j^{N/2}
		\spacetwo{ij}{ji}
\end{align}
であり、同様にして$\chi_{2i}$では
\begin{align}
	\sum_b^N
		\phystwo{\chi_{2i}\chi_b}{\chi_b\chi_{2i}}
&=
	\sum_j^{N/2}
		\spacetwo{ij}{ji}
\end{align}
である。

従って、軌道エネルギーは
\begin{align}
	\epsilon_{2i-1}
=
	\epsilon_{2i}
=
	\spaceone{\psi_i}{h}{\psi_i}
	+
	\sum_j^{N/2}\left(
		2\spacetwo{ii}{jj}
		-
		\spacetwo{ij}{ji}
	\right)
\end{align}
となる。