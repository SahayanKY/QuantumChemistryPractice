%ファイルID
%2020/09/15 1356
%->2009151356SahayanKY(ファイル作成者)
\subsection{問}
最小基底関数における\ce{H2}の密度行列が
\begin{align}
	\bm{P}
&=
	\left[
	\begin{array}{cc}
		(1+S_{12})^{-1} & (1+S_{12})^{-1} \\
		%
		(1+S_{12})^{-1} & (1+S_{12})^{-1}
	\end{array}
	\right]
\end{align}
となることを導け。

また、\ce{H2+}での密度行列を求めよ。

\subsection{解}
閉殻系における密度行列は、本書より次の通りである。
\begin{align}
	P_{\mu\nu}
&=
	2
	\sum_{a}^{N/2}
		C_{\mu a} \conju{C_{\nu a}}
\end{align}
また、\ce{H2}の係数行列は既に出ている通り
\begin{align}
	\bm{C}
&=
	\left[
	\begin{array}{cc}
		\left[2(1+S_{12})\right]^{-1/2}
		&
		\left[2(1-S_{12})\right]^{-1/2} \\
		%
		\left[2(1+S_{12})\right]^{-1/2}
		&
		-
		\left[2(1-S_{12})\right]^{-1/2}
	\end{array}
	\right]
\end{align}
である。
従って、\ce{H2}(閉殻系)の密度行列は
\begin{align}
	\bm{P}
&=
	\left[
	\begin{array}{>{\displaystyle}c>{\displaystyle}c}
		2\sum_a^1 C_{1a} \conju{C_{1a}}
		&
		2\sum_a^1 C_{1a} \conju{C_{2a}} \\[2mm]
		%
		2\sum_a^1 C_{2a} \conju{C_{1a}}
		&
		2\sum_a^1 C_{2a} \conju{C_{2a}}
	\end{array}
	\right] \\
%
%
&=
	2
	\left[
	\begin{array}{cc}
		C_{11} \conju{C_{11}}
		&
		C_{11} \conju{C_{21}} \\
		%
		C_{21} \conju{C_{11}}
		&
		C_{21} \conju{C_{21}}
	\end{array}
	\right] \\
%
%
&=
	2
	\left[
	\begin{array}{cc}
		\left[2(1+S_{12})\right]^{-1}
		&
		\left[2(1+S_{12})\right]^{-1} \\
		%
		\left[2(1+S_{12})\right]^{-1}
		&
		\left[2(1+S_{12})\right]^{-1}
	\end{array}
	\right] \\
%
%
&=
	\left[
	\begin{array}{cc}
		(1+S_{12})^{-1}
		&
		(1+S_{12})^{-1} \\
		%
		(1+S_{12})^{-1}
		&
		(1+S_{12})^{-1}
	\end{array}
	\right]
\end{align}
となる。

次に\ce{H2+}の密度行列を考える。
今回は開殻系であるので上記の式は適用できない。
そこで、電荷密度を分子軌道を使って表現するところから考える。
問題3.11より
\begin{align}
	\rho(\r)
&=
	\braket{\Psi_0|\hat{\rho}(\r)|\Psi_0} \\
%
%
&=
	\int\d\r_1\d\omega_1
		\conju{\psi_1}(\r_1)
		\conju{\alpha}(\omega_1)
		\delta(\r_1-\r)
		\psi_1(\r_1)
		\alpha(\omega_1) \\
%
%
&=
	\int\d\r_1
		\conju{\psi_1}(\r_1)
		\delta(\r_1-\r)
		\psi_1(\r_1) \\
%
%
&=
	\conju{\psi_1}(\r)
	\psi_1(\r)
\end{align}
これを最小基底関数で展開すると
\begin{align}
	\rho(\r)
&=
	\conju[1]{
		C_{11} \phi_1(\r)
		+
		C_{21} \phi_2(\r)
	}
	\left(
		C_{11} \phi_1(\r)
		+
		C_{21} \phi_2(\r)
	\right) \\
%
%
&=
	C_{11} \conju{C_{11}} \phi_1 \conju{\phi_1}
	+
	C_{11} \conju{C_{21}} \phi_1 \conju{\phi_2}
	+
	C_{21} \conju{C_{11}} \phi_2 \conju{\phi_1}
	+
	C_{21} \conju{C_{21}} \phi_2 \conju{\phi_2}
\end{align}
となる。従って、密度行列は
\begin{align}
	\bm{P}
&=
	\left[
	\begin{array}{cc}
		C_{11} \conju{C_{11}}
		&
		C_{11} \conju{C_{21}} \\
		%
		C_{21} \conju{C_{11}}
		&
		C_{21} \conju{C_{21}}
	\end{array}
	\right] \\
%
%
&=
	\frac{1}{2}
	\left[
	\begin{array}{cc}
		(1+S_{12})^{-1}
		&
		(1+S_{12})^{-1} \\
		%
		(1+S_{12})^{-1}
		&
		(1+S_{12})^{-1}
	\end{array}
	\right]
\end{align}
である。
